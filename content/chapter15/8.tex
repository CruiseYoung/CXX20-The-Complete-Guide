
到目前为止,我们只给co\_await传递了awaiter。例如,通过使用标准的awaiter:

\begin{cpp}
co_await std::suspend_always{};
\end{cpp}

或者,另一个例子,co\_await接受用户定义的awaiter:

\begin{cpp}
co_await CoroPrio{CoroPrioRequest::less}; // SUSPEND with lower prio
\end{cpp}

然而,co\_await expr是一个操作符,可以作为更大表达式的一部分,并接受不具有awaiter类型的值。例如:

\begin{cpp}
co_await 42;
co_await (x + y);
\end{cpp}

co\_await操作符与sizeof或new具有相同的优先级。由于这个原因,不能跳过上面例子中的圆括号而不改变其含义。该声明

\begin{cpp}
co_await x + y;
\end{cpp}

将按以下方式进行计算:

\begin{cpp}
(co_await x) + y;
\end{cpp}

注意,x + y,或者只是x,在这里不一定是一个附加项。co\_await需要一个awaitable对象,而awaiter只是它们的一种(常规的)实现。事实上,co\_await接受任何类型的值,只要有一个到awaiter的API的映射。对于映射,C++标准提供了两种方法:

\begin{itemize}
\item 
promise的成员函数await\_transform()

\item 
co\_await()操作符
\end{itemize}

这两种方法都允许协程将任何类型的值传递给co\_await,并隐式或间接地指定一个awaiter。


\mySubsubsection{15.8.1}{await\_transform()}

若co\_await表达式出现在协程中,编译器首先查找是否存在由协程的promise提供的成员函数await\_transform()。若是这样,则调用await\_transform()并产生一个等待器,然后使用该等待器挂起协程。

例如,这意味着:

\begin{cpp}
class CoroTask {
	struct promise_type {
		...
		auto await_transform(int val) {
			return MyAwaiter{val};
		}
	};
	...
};

CoroTask coro()
{
	co_await 42;
}
\end{cpp}

与如下代码具有相同的效果:

\begin{cpp}
class CoroTask {
	...
};

CoroTask coro()
{
	co_await MyAwaiter{42};
}
\end{cpp}

也可以使用它来启用协程在使用(标准)awaiter之前向promise传递一个值:

\begin{cpp}
class CoroTask {
	struct promise_type {
		...
		auto await_transform(int val) {
			... // process val
			return std::suspend_always{};
		}
	};
	...
};

CoroTask coro()
{
	co_await 42; // let 42 be processed by the promise and suspend
}
\end{cpp}

\mySamllsection{使用值让co\_await更新正在运行的协程}

还记得使用awaiter来更改协程优先级的示例吗?在这里,我们使用CoroPrio类型的awaiter使协程能够请求新的优先级,如下所示:

\begin{cpp}
co_await CoroPrio{CoroPrioRequest::less}; // SUSPEND with lower prio
\end{cpp}

也可以只允许传递新的优先级:

\begin{cpp}
co_await CoroPrioRequest::less; // SUSPEND with lower prio
\end{cpp}

只需要让协程接口promise为这种类型的值提供await\_transform()成员函数:

\begin{cpp}
class CoroPrioTask {
public:
	struct promise_type;
	using CoroHdl = std::coroutine_handle<promise_type>;
private:
	CoroHdl hdl; // native coroutine handle
	friend class CoroPrioScheduler; // give access to the handle
public:
	struct promise_type {
		CoroPrioScheduler* schedPtr = nullptr; // each task knows its scheduler:
		...
		auto await_transform(CoroPrioRequest); // deal with co_await CoroPrioRequest
	};
	...
};
\end{cpp}

await\_transform()的实现如下:

\begin{cpp}
inline auto CoroPrioTask::promise_type::await_transform(CoroPrioRequest pr) {
	auto hdl = CoroPrioTask::CoroHdl::from_promise(*this);
	schedPtr->changePrio(hdl, pr);
	return std::suspend_always{};
}
\end{cpp}

这里,再次使用协程句柄from\_promise()中的静态成员函数来获取句柄,因为changePrio()需要该句柄作为其第一个参数。

这样的话,我们就可以跳过整个awaiter CoroPrio了。完整代码请参见coro/coropriosched2.hpp(以及coro/coroprio2.cpp和coro/coroprio2.hpp)。

\mySamllsection{让co\_await像co\_yield一样运行}

记住co\_yield的例子。要处理要产生的值,并做了以下的操作:

\begin{cpp}
struct promise_type {
	int coroValue = 0; // last value from co_yield
	auto yield_value(int val) { // reaction to co_yield
		coroValue = val; // - store value locally
		return std::suspend_always{}; // - suspend coroutine
	}
	...
};

co_yield val; // calls yield_value(val) on promise
\end{cpp}

可以用下面的方法得到同样的效果:

\begin{cpp}
 can get the same effect with the following:
struct promise_type {
	int coroValue = 0; // last value from co_yield
	auto await_transform(int val) {
		coroValue = val; // - store value locally
		return std::suspend_always{};
	}
	...
};

co_await val; // calls await_transform(val) on promise
\end{cpp}

事实上,

\begin{cpp}
co_yield val;
\end{cpp}

等于

\begin{cpp}
co_await prm.yield_value(val);
\end{cpp}

其中prm是封闭协程的promise。

\mySubsubsection{15.8.2}{co\_await()操作符}

让co\_await处理(几乎)任何类型的任何值的另一个选择是为该类型实现操作符co\_await(),所以必须传递一个类的值。

假设已经为某些类型MyType实现了操作符co\_await():

\begin{cpp}
class MyType {
	auto operator co_await() {
		return std::suspend_always{};
	}
};
\end{cpp}

然后,对该类型的对象调用co\_await:

\begin{cpp}
CoroTask coro()
{
	...
	co_await MyType{};
}
\end{cpp}

呼叫接线员并使用返回的awaiter来处理暂停。这里,操作符co\_await()产生std::suspend\_always{},则可得到:

\begin{cpp}
CoroTask coro()
{
	...
	co_await std::suspend_always{};
}
\end{cpp}

也可以将参数传递给MyType\{\},并将其传递给一个awaiter,以便将值传递给挂起的协程,立即或稍后恢复。

另一个例子是将协程传递给co\_await,然后可以调用,包括实现准备工作,例如:更改线程或将协程调度到线程池中。













