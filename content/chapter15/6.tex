

协程需要内存来保存它们的状态。因为协程可能会切换上下文,所以通常使用堆内存。本节将讨论如何做到这一点,以及如何更改。


\mySubsubsection{15.6.1}{协程如何分配内存}

协程需要内存来存储它们从挂起到恢复的状态,但由于恢复可能发生在非常不同的上下文中,因此通常只能在堆上分配内存。事实上,C++标准规定:

\begin{quote}
实现可能需要为协程分配额外的存储空间。这种存储称为协程状态,通过调用非数组分配函数获得。
\end{quote}

这里重要的词是“可能”,编译器可以优化掉对堆内存的需求。当然,编译器需要足够的信息,代码必须非常简单。事实上,最有可能优化的是

\begin{itemize}
\item 
协程的生命周期保持在调用者的生命周期内。

\item 
使用内联函数使编译器至少可以看到计算帧大小的所有内容。

\item 
final\_suspend()返回std::suspend\_always{},否则生命周期管理将变得过于复杂。
\end{itemize}

然而,在撰写本章时(2022年3月),只有Clang编译器为协程提供了相应的分配省略。[Visual C++提供了一个优化选项\texttt{/await:heapelide},但目前似乎关闭了。]

例如,考虑以下协程:

\begin{cpp}
CoroTask coro(int max)
{
	for (int val = 1; val <= max; ++val) {
		std::cout << "coro(" << max << "): " << val << '\n';
		co_await std::suspend_always{};
	}
}

CoroTask coroStr(int max, std::string s)
{
	for (int val = 1; val <= max; ++val) {
		std::cout << "coroStr(" << max << ", " << s << "): " << '\n';
		co_await std::suspend_always{};
	}
}
\end{cpp}

其不同之处在于第二个协程的附加字符串参数。假设只是将一些临时对象传递给每个参数:

\begin{cpp}
coro(3); // create and destroy temporary coroutine
coroStr(3, "hello"); // create and destroy temporary coroutine
\end{cpp}

若没有优化和跟踪分配,可能会得到如下输出:

\begin{shell}
::new #1 (36 Bytes) => 0x8002ccb8
::delete (no size) at 0x8002ccb8

::new #2 (60 Bytes) => 0x8004cd28
::delete (no size) at 0x8004cd28
\end{shell}

这里,第二个协程需要额外的字符串参数24个字节。

有关使用coro/tracknews.hpp查看何时分配或释放堆内存的完整示例,请参阅coro/coromem.cpp。

\mySubsubsection{15.6.2}{避免堆内存分配}

协程的promise类型允许开发者更改为协程分配内存的方式,只需要提供以下成员:

\begin{itemize}
\item 
\texttt{void* operator new(std::size\_t sz)}

\item 
\texttt{void operator delete(void* ptr, std::size\_t sz)}
\end{itemize}

\mySamllsection{确保没有使用堆内存}

首先,可以使用new()和delete()操作符来确定协程是否在堆上分配了内存。简单地声明new操作符而不实现它:

\begin{cpp}
class CoroTask {
	...
	public:
	struct promise_type {
		...
		// find out whether heap memory is allocated:
		void* operator new(std::size_t sz); // declared, but not implemented
	};
	...
};
\end{cpp}


\mySamllsection{让协程不使用堆内存}

这些操作符的另一个用途是改变协程分配内存的方式。例如,可以将对堆内存的调用,映射到堆栈上或程序的数据段中已经分配的一些内存。

下面是一个具体的例子:

\filename{coro/corotaskpmr.hpp}

\begin{cpp}
#include <coroutine>
#include <exception> // for terminate()
#include <cstddef> // for std::byte
#include <array>
#include <memory_resource>

// coroutine interface to deal with a simple task
// - providing resume() to resume it
class [[nodiscard]] CoroTaskPmr {
	// provide 200k bytes as memory for all coroutines:
	inline static std::array<std::byte, 200'000> buf;
	inline static std::pmr::monotonic_buffer_resource
		monobuf{buf.data(), buf.size(), std::pmr::null_memory_resource()};
	inline static std::pmr::synchronized_pool_resource mempool{&monobuf};

public:
	struct promise_type;
	using CoroHdl = std::coroutine_handle<promise_type>;
private:
	CoroHdl hdl; // native coroutine handle
	
public:
	struct promise_type {
		auto get_return_object() { // init and return the coroutine interface
			return CoroTaskPmr{CoroHdl::from_promise(*this)};
		}
		auto initial_suspend() { // initial suspend point
			return std::suspend_always{}; // - suspend immediately
		}
		void unhandled_exception() { // deal with exceptions
			std::terminate(); // - terminate the program
		}
		void return_void() { // deal with the end or co_return;
		}
		auto final_suspend() noexcept { // final suspend point
			return std::suspend_always{}; // - suspend immediately
		}
		
		// define the way memory is allocated:
		void* operator new(std::size_t sz) {
			return mempool.allocate(sz);
		}
		void operator delete(void* ptr, std::size_t sz) {
			mempool.deallocate(ptr, sz);
		}
	};
	// constructor and destructor:
	CoroTaskPmr(auto h) : hdl{h} { }
	~CoroTaskPmr() { if (hdl) hdl.destroy(); }
	
	// don’t copy or move:
	CoroTaskPmr(const CoroTaskPmr&) = delete;
	CoroTaskPmr& operator=(const CoroTaskPmr&) = delete;
	
	// API to resume the coroutine
	// - returns whether there is still something to process
	bool resume() const {
		if (!hdl || hdl.done()) {
			return false; // nothing (more) to process
		}
		hdl.resume(); // RESUME (blocks until suspended again or end)
		return !hdl.done();
	}
};
\end{cpp}

这里,我们使用多态内存资源,这是C++17引入的一个特性,通过提供标准化的内存池来简化内存管理。本例中,我们将200kb的数据段传递给内存池monobuf,使用null\_memory\_resource()作为回退确保在内存不足时抛出std::bad\_alloc异常。在最上面,放置了同步内存池mempool,其用很少的碎片来管理这个缓冲区:[详细信息请参阅我的书《C++17——完整指南》]。

\begin{cpp}
class [[nodiscard]] CoroTaskPmr {
	// provide 200k bytes as memory for all coroutines:
	inline static std::array<std::byte, 200’000> buf;
	inline static std::pmr::monotonic_buffer_resource
		monobuf{buf.data(), buf.size(), std::pmr::null_memory_resource()};
	inline static std::pmr::synchronized_pool_resource mempool{&monobuf};
	...
public:
	struct promise_type {
		...
			// define the way memory is allocated:
		void* operator new(std::size_t sz) {
			return mempool.allocate(sz);
		}
		void operator delete(void* ptr, std::size_t sz) {
			mempool.deallocate(ptr, sz);
		}
	};
	...
};
\end{cpp}

可以像往常一样使用这个协程接口:

\filename{coro/coromempmr.cpp}

\begin{cpp}
#include <iostream>
#include <string>
#include "corotaskpmr.hpp"
#include "tracknew.hpp"

CoroTaskPmr coro(int max)
{
	for (int val = 1; val <= max; ++val) {
		std::cout << "    coro(" << max << "): " << val << '\n';
		co_await std::suspend_always{};
	}
}

CoroTaskPmr coroStr(int max, std::string s)
{
	for (int val = 1; val <= max; ++val) {
		std::cout << "    coroStr(" << max << ", " << s << "): " << '\n';
		co_await std::suspend_always{};
	}
}

int main()
{
	TrackNew::trace();
	TrackNew::reset();
	coro(3); // initialize temporary coroutine
	coroStr(3, "hello"); // initialize temporary coroutine
	
	auto coroTask = coro(3); // initialize coroutine
	std::cout << "coro() started\n";
	while (coroTask.resume()) { // RESUME
		std::cout << "coro() suspended\n";
	}
	std::cout << "coro() done\n";
}
\end{cpp}

不再为这些协程分配堆内存。

注意,还可以提供一个特定的操作符new,在size参数之后接受协程参数。例如,对于接受int型和string型参数的协程,可以提供:

\begin{cpp}
class CoroTaskPmr {
	public:
	struct promise_type {
		...
		void* operator new(std::size_t sz, int, const std::string&) {
			return mempool.allocate(sz);
		}
	};
	...
};
\end{cpp}

\mySubsubsection{15.6.3}{get\_return\_object\_on\_allocation\_failure()}

若promise有一个静态成员get\_return\_object\_on\_allocation\_failure(),则假定内存分配从不抛出。默认情况下,其效果是使用new操作符

\begin{cpp}
::operator new(std::size_t sz, std::nothrow_t)
\end{cpp}

这种情况下,用户定义的操作符new必须为noexcept,失败时返回nullptr。

然后,可以使用该函数来实现变通方法,例如创建不引用协程的协程句柄:

\begin{cpp}
class CoroTask {
	...
	public:
	struct promise_type {
		...
		static auto get_return_object_on_allocation_failure() {
			return CoroTask{nullptr};
		}
	};
	...
};
\end{cpp}





























