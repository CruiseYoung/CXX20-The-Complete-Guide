
协程具有以下属性和限制:

\begin{itemize}
\item
协程不允许有返回语句。

\item
协程不能是constexpr或consteval。

\item
协程不能有返回类型auto或其他占位符类型。

\item
main()不能是协程。

\item
构造函数或析构函数不可为协程。
\end{itemize}

协程可以是静态的。协程若不是构造函数或析构函数,可以是成员函数。

协程甚至可以是Lambda,但在这种情况下,必须谨慎使用。

\mySubsubsection{15.1.1}{Lambda协程}

协程可以是Lambda,也可以像这样实现第一个协程示例:

\begin{cpp}
auto coro = [] (int max) -> CoroTask {
				std::cout << "          CORO " << max << " start\n";
				for (int val = 1; val <= max; ++val) {
					// print next value:
					std::cout << " CORO " << val << '/' << max << '\n';
					co_await std::suspend_always{}; // SUSPEND
				}
				std::cout << "         CORO " << max << " end\n";
};
\end{cpp}

但请注意协程Lambda不应该捕获任何内容。这是因为Lambda是定义函数对象的快捷方式,该对象在定义Lambda的作用域中创建。当离开该作用域时,协程Lambda可能仍然会恢复,即使这时Lambda对象已销毁了。

下面是一个导致致命运行时错误的代码示例:

\begin{cpp}
auto getCoro()
{
	string str = "      CORO ";
	auto coro = [str] (int max) -> CoroTask {
				std::cout << str << max << " start\n";
				for (int val = 1; val <= max; ++val) {
					std::cout << str << val << '/' << max << '\n';
					co_await std::suspend_always{}; // SUSPEND
				}
				std::cout << str << max << " end\n";
			};
	return coro;
}

auto coroTask = getCoro()(3); // initialize coroutine
// OOPS: lambda destroyed here
coroTask.resume(); // FATAL RUNTIME ERROR
\end{cpp}

通常,返回一个按值捕获的Lambda没有生命周期问题。

可以像下面这样使用协程Lambda:

\begin{cpp}
auto coro = getCoro(); // initialize coroutine lambda
auto coroTask = coro(3); // initialize coroutine
coroTask.resume(); // OK
\end{cpp}

还可以通过引用将str传递给getCoro(),并确保只要使用协程,str就存在。

但不能将两个初始化组合到一个语句中,因为很容易出现更微妙的生命周期问题,所以强烈建议不要在协程Lambda中使用捕获。

