

C++20引入了几个新的扩展来捕获Lambda中的值和对象。

\mySubsubsection{17.5.1}{捕获this和*this}

若在成员函数中定义Lambda,问题是如何访问调用该成员函数对象的数据。C++20前,有以下规则:

\begin{cpp}
class MyType {
	std::string name;
	...
	void foo() {
		int val = 0;
		...
		auto l0 = [val] { bar(val, name); }; // ERROR: member name not captured
		auto l1 = [val, name=name] { bar(val, name); }; // OK, capture val and name by value
		
		auto l2 = [&] { bar(val, name); }; // OK (val and name by reference)
		auto l3 = [&, this] { bar(val, name); }; // OK (val and name by reference)
		auto l4 = [&, *this] { bar(val, name); }; // OK (val by reference, name by value)
		
		auto l5 = [=] { bar(val, name); }; // OK (val by value, name by reference)
		auto l6 = [=, this] { bar(val, name); }; // ERROR before C++20
		auto l7 = [=, *this] { bar(val, name); }; // OK (val and name by value)
		...
	}
};
\end{cpp}

自C++20起,可使用以下规则:

\begin{cpp}
class MyType {
	std::string name;
	...
	void foo() {
		int val = 0;
		...
		auto l0 = [val] { bar(val, name); }; // ERROR: member name not captured
		auto l1 = [val, name=name] { bar(val, name); }; // OK, capture val and name by value
		
		auto l2 = [&] { bar(val, name); }; // deprecated (val and name by ref.)
		auto l3 = [&, this] { bar(val, name); }; // OK (val and name by reference)
		auto l4 = [&, *this] { bar(val, name); }; // OK (val by reference, name by value)
		
		auto l5 = [=] { bar(val, name); }; // deprecated (val by value, name by ref.)
		auto l6 = [=, this] { bar(val, name); }; // OK (val by value, name by reference)
		auto l7 = [=, *this] { bar(val, name); }; // OK (val and name by value)
		...
	}
};
\end{cpp}

C++20起,有如下的变化:

\begin{itemize}
\item 
[=, this]现在允许作为Lambda捕获(一些编译器以前允许,尽管在形式上无效)。

\item 
不可隐式捕获*this。
\end{itemize}

\mySubsubsection{17.5.2}{捕获结构化绑定}

C++20起,允许捕获结构化绑定(C++17引入):

\begin{cpp}
std::map<int, std::string> mymap;
...

for (const auto& [key,val] : mymap) {
	auto l = [key, val] { // OK since C++20
				...
			};
	...
}
\end{cpp}

一些编译器以前确实允许捕获结构化绑定,尽管它在形式上无效。

\mySubsubsection{17.5.3}{捕获可变模板的参数包}

若有一个可变的模板,可以像下面这样捕获参数包:

\begin{cpp}
template<typename... Args>
void foo(Args... args)
{
	auto l1 = [&] {
					bar(args...); // OK
				};
	auto l2 = [args...] { // or [=]
					bar(args...); // OK
				};
	...
}
\end{cpp}

然而,若想返回为以后使用而创建的Lambda,就会出现问题:

\begin{itemize}
\item 
使用[\&],将返回一个lambda,该Lambda会引用已销毁的参数包。

\item 
使用[args…]或[=],则复制传递的参数包。
\end{itemize}

在捕获对象时,可以使用init-capturing来使用移动语义:

\begin{cpp}
template<typename T>
void foo(T arg)
{
	auto l3 = [arg = std::move(arg)] { // OK since C++14
				bar(arg); // OK
			};
	...
}
\end{cpp}

但是,没有提供与参数包一起使用init-capture的语法。

C++20引入了相应的语法:

\begin{cpp}
template<typename... Args>
void foo(Args... args)
{
	auto l4 = [...args = std::move(args)] { // OK since C++20
				bar(args...); // OK
			};
	...
}
\end{cpp}

还可以通过引用来初始化捕获参数包。例如,可以将Lambda的参数名更改为:

\begin{cpp}
template<typename... Args>
void foo(Args... args)
{
	auto l4 = [&...fooArgs = args] { // OK since C++20
				bar(fooArgs...); // OK
			};
	...
}
\end{cpp}

\mySamllsection{使用实例获取可变模板的参数包}

创建并返回一个按值捕获可变数量参数的Lambda泛型函数,可以这样:

\begin{cpp}
template<typename Callable, typename... Args>
auto createToCall(Callable op, Args... args)
{
	return [op, ...args = std::move(args)] () -> decltype(auto) {
				return op(args...);
			};
}
\end{cpp}

使用简化函数模板的新语法,代码如下所示:

\begin{cpp}
auto createToCall(auto op, auto... args)
{
	return [op, ...args = std::move(args)] () -> decltype(auto) {
				return op(args...);
			};
}
\end{cpp}

下面是一个完整的例子:

\filename{lang/capturepack.cpp}

\begin{cpp}
#include <iostream>
#include <string_view>

auto createToCall(auto op, auto... args)
{
	return [op, ...args = std::move(args)] () -> decltype(auto) {
		return op(args...);
	};
}

void printWithGAndNoG(std::string_view s)
{
	std::cout << s << "g " << s << '\n';
}

int main()
{
	auto printHero = createToCall(printWithGAndNoG, "Zhan");
	...
	printHero();
}
\end{cpp}


\mySubsubsection{17.5.4}{Lambda作为协程}

Lambda也可以是协程,这是在C++20引入的。但请注意,在这种情况下,Lambda不该捕获任何内容,因为协程可能比本地创建的Lambda对象存在的时间更长。













