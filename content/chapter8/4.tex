
本节讨论C++20中用于创建生成值的视图的所有特性(不引用视图外部的元素或值)。

\mySubsubsection{8.4.1}{iota视图}

% Please add the following required packages to your document preamble:
% \usepackage{longtable}
% Note: It may be necessary to compile the document several times to get a multi-page table to line up properly
\begin{longtable}[c]{|l|l|}
\hline
\textbf{类型:}                 & std::ranges::iota\_view\textless{}\textgreater{} \\ \hline
\endfirsthead
%
\endhead
%
\textbf{内容:}              & 值递增序列的生成器  \\ \hline
\textbf{工厂:}              & std::views::iota()                               \\ \hline
\textbf{元素类型:}         & 值                                           \\ \hline
\textbf{类别:}        & 随机访问的输入(取决于开始值的类型) \\ \hline
\textbf{是否是长度范围:}       & 若用结束值初始化                 \\ \hline
\textbf{是否是通用范围:} & 若是有限的,end和值的类型相同         \\ \hline
\textbf{是否是租借范围:}    & 总是                                           \\ \hline
\textbf{缓存:}               & 无                                          \\ \hline
\textbf{常量可迭代:}       & 总是                                           \\ \hline
\textbf{传播常量性:} & 从不(但元素不是左值)              \\ \hline
\end{longtable}

类模板std::ranges::iota\_view<>是一个生成一系列值的视图。这些值可以是整型的,例如

\begin{itemize}
\item
1, 2, 3 ...

\item
‘a’, ‘b’, ‘c’ ...
\end{itemize}

也可以使用操作符++生成指针或迭代器序列。

这个序列可能是有限的,也可能是无限的。

iota视图的主要用例是提供在一系列值上迭代的视图。例如:

\begin{cpp}
std::ranges::iota_view v1{1, 100}; // view with values: 1, 2, ... 99
for (auto val : v1) {
	std::cout << val << '\n'; // print these values
}
\end{cpp}

\mySamllsection{iota视图的范围工厂}

iota视图也可以(通常应该)使用范围工厂创建:

\begin{cpp}
std::views::iota(val)
std::views::iota(val, endVal)
\end{cpp}

例如:

\begin{cpp}
for (auto val : std::views::iota(1, 100)) { // iterate over values 1, 2, ... 99
	std::cout << val << '\n'; // print these values
}
\end{cpp}

下面是一个使用iota视图的完整示例程序:

\filename{ranges/iotaview.cpp}

\begin{cpp}
#include <iostream>
#include <string>
#include <vector>
#include <ranges>

void print(std::ranges::input_range auto&& coll)
{
	int num = 0;
	for (const auto& elem : coll) {
		std::cout << elem << ' ';
		if (++num > 30) { // print only up to 30 values:
			std::cout << "...";
			break;
		}
	}
	std::cout << '\n';
}

int main()
{
	std::ranges::iota_view<int> iv0; // 0 1 2 3 ...
	print(iv0);
	
	std::ranges::iota_view iv1{-2}; // -2 -1 0 1 ...
	print(iv1);
	
	std::ranges::iota_view iv2{10, 20}; // 10 11 12 ... 19
	print(iv2);
	
	auto iv3 = std::views::iota(1); // -2 -1 0 1 ...
	print(iv3);
	
	auto iv4 = std::views::iota('a', 'z'+1); // a b c ... z
	print(iv4);
	
	std::vector coll{0, 8, 15, 47, 11};
	for (auto p : std::views::iota(coll.begin(), coll.end())) { // sequence of iterators
		std::cout << *p << ' '; // 0 8 15 47 11
	}
	std::cout << '\n';
}
\end{cpp}

该程序有以下输出:

\begin{shell}
0 1 2 3 4 5 6 7 8 9 10 11 12 13 14 15 16 17 18 19 20 21 22 23 24 25 26 27 28 29 30 ...
-2 -1 0 1 2 3 4 5 6 7 8 9 10 11 12 13 14 15 16 17 18 19 20 21 22 23 24 25 26 27 28 ...
10 11 12 13 14 15 16 17 18 19
1 2 3 4 5 6 7 8 9 10 11 12 13 14 15 16 17 18 19 20 21 22 23 24 25 26 27 28 29 30 31 ...
a b c d e f g h i j k l m n o p q r s t u v w x y z
0 8 15 47 11	
\end{shell}

请注意,从'a'到'z'的输出取决于平台的字符集。

\mySamllsection{iota视图的特殊特征}

因为迭代器持有当前值,所以iota视图是租借范围(其迭代器的生命周期不依赖于视图)。若iota视图是有限的,并且结束值与开始值具有相同的类型,则该视图是一个通用范围。

若iota视图无限大,或者结束的类型不同于开始的类型,则它不是通用范围,可能必须使用一个通用视图来协调开始迭代器和哨兵(结束迭代器)的类型。

\mySamllsection{iota视图的接口}

“类std::ranges::iota\_view<>操作”表列出了iota视图的API。


% Please add the following required packages to your document preamble:
% \usepackage{longtable}
% Note: It may be necessary to compile the document several times to get a multi-page table to line up properly
\begin{longtable}[c]{|l|l|}
\hline
\textbf{Operation}       & \textbf{Effect}                                                    \\ \hline
\endfirsthead
%
\endhead
%
ioat\_view\textless{}Type\textgreater v & 创建从默认值Type开始的无限序列               \\ \hline
iota\_view v\{begVal\}   & 创建一个以begVal开头的无限序列                 \\ \hline
iota\_view v\{begVal, endVal\}          & 创建一个从begVal开始直到endVal之前的值的序列               \\ \hline
iota\_view v\{beg, end\} & 启用子视图的辅助构造函数                             \\ \hline
v.begin()                & 生成begin迭代器(指向起始值)         \\ \hline
v.end()                  & 产生哨兵(end迭代器),若无限制,则为std::unreachable\_sentinel \\ \hline
v.empty()                & 表示v是否为空                                          \\ \hline
if (v)                   & 若v不为空,则为true                                             \\ \hline
v.size()                 & 产生元素的数量(若有限且可计算,则可用) \\ \hline
v.front()                & 生成第一个元素                                           \\ \hline
v.back()                 & 生成最后一个元素(若有限且通用,则可用)           \\ \hline
v{[}idx{]}               & 生成的第n个元素                                            \\ \hline
\end{longtable}

\begin{center}
表8.8 类std::ranges::iota\_view<>操作
\end{center}

当声明iota视图时,如下所示:

\begin{cpp}
std::ranges::iota_view v1{1, 100}; // range with values: 1, 2, ... 99
\end{cpp}

v1的类型由std::ranges::iota\_view<int, int>推导而出,会创建一个提供基本API的对象,并使用begin()和end()得到这些值的迭代器。迭代器在内部存储当前值,并在迭代器加1时递增当前值:

\begin{cpp}
std::ranges::iota_view v1{1, 100}; // range with values: 1, 2, ... 99
auto pos = v1.begin(); // initialize iterator with value 1
std::cout << *pos; // print current value (here, value 1)
++pos; // increment to next value (i.e., increment value to 2)
std::cout << *pos; // print current value (here, value 2)
\end{cpp}

因此,值的类型必须支持自增操作符++,所以bool或std::string这样的类型就不可能了。

注意,此迭代器的类型取决于iota\_view的实现,所以在使用时必须使用auto。或者,可以用std::ranges::iterator\_t<>声明pos:

\begin{cpp}
std::ranges::iterator_t<decltype(v1)> pos = v1.begin();
\end{cpp}

至于迭代器的范围,值的范围是半开范围,不包括结束值。要包含end值,需要对end加1:

\begin{cpp}
std::ranges::iota_view letters{'a', 'z'+1}; // values: ’a’, ’b’, ... ’z’
for (auto c : letters) {
	std::cout << c << ' '; // print these values/letters
}
\end{cpp}

注意,这个循环的输出取决于字符集。只有当小写字母具有连续的值(如ASCII或ISO-Latin-1或UTF-8的情况)时,视图才只迭代小写字符。通过使用char8\_t,可以确保这是可移植的UTF-8字符:

\begin{cpp}
std::ranges::iota_view letters{u8'a', u8'z'+1}; // UTF-8 values from a to z
\end{cpp}

若没有传递end值,视图将是无限的,并生成一个无限长的值序列:

\begin{cpp}
std::ranges::iota_view v2{10L}; // unlimited range with values: 10L, 11L, 12L, ...
std::ranges::iota_view<int> v3; // unlimited range with values: 0, 1, 2, ...
\end{cpp}

v3的类型是std::ranges::iota\_view<int, std::unreachable\_sentinel\_t>,所以end()产生std::unreachable\_sentinel。

无限视图是无尽的。当迭代器表示最大值并迭代到下一个值时,调用自增操作符++,这在形式上是未定义的行为(实践中,执行通常会溢出,以便下一个值是值类型的最小值)。若视图有一个永远不匹配的结束值,其行为是一样的。

\mySamllsection{使用iota视图对指针和迭代器进行迭代}

可以用迭代器或指针初始化iota视图,可使用第一个值作为begin,最后一个值作为end,但是元素是迭代器/指针,而非值。

可以使用它来处理指向范围中所有元素的迭代器。

\begin{cpp}
std::list<int> coll{2, 4, 6, 8, 10, 12, 14};
...
// pass iterator to each element to foo():
for (const auto& itorElem : std::views::iota(coll.begin(), coll.end())) {
	std::cout << *itorElem << '\n';
}
\end{cpp}

也可以使用这个特性初始化一个带有迭代器的容器,该迭代器指向集合coll的所有元素:

\begin{cpp}
std::ranges::iota_view itors{coll.begin(), coll.end()};
std::vector<std::ranges::iterator_t<decltype(coll)>> refColl{itors.begin(),
	itors.end()};
\end{cpp}

注意,若跳过refColl元素类型的说明,则必须使用括号。否则,可以用两个迭代器元素初始化vector:

\begin{cpp}
std::vector refColl(itors.begin(), itors.end());
\end{cpp}

提供这个构造函数的主要原因是,为了支持创建iota视图子视图的泛型代码,比如drop视图:

\begin{cpp}
// generic function to drop the first element from a view:
auto dropFirst = [] (auto v) {
						return decltype(v){++v.begin(), v.end()};
					};
std::ranges::iota_view v1{1, 9}; // iota view with elems from 1 to 8
auto v2 = dropFirst(v1); // iota view with elems from 2 to 8
\end{cpp}

成员函数size()和operator[]的可用性取决于,传递的范围内对这些操作符的支持情况。

\mySubsubsection{8.4.2}{单视图}

% Please add the following required packages to your document preamble:
% \usepackage{longtable}
% Note: It may be necessary to compile the document several times to get a multi-page table to line up properly
\begin{longtable}[c]{|l|l|}
\hline
\textbf{类型:}                 & std::ranges::single\_view\textless{}\textgreater{} \\ \hline
\endfirsthead
%
\endhead
%
\textbf{内容:}              & 只有一个元素的范围生成器         \\ \hline
\textbf{工厂:}              & std::views::single()                               \\ \hline
\textbf{元素类型:}         & 引用                                          \\ \hline
\textbf{类别:}             & 连续的                                         \\ \hline
\textbf{是否是长度范围:}       & 长度总为1                                 \\ \hline
\textbf{是否是通用范围:}      & 总是                                             \\ \hline
\textbf{是否是租借范围:}    & 从不                                              \\ \hline
\textbf{缓存:}               & 无                                            \\ \hline
\textbf{常量可迭代:}       & 总是                                             \\ \hline
\textbf{传播常量性:} & 总是                                             \\ \hline
\end{longtable}

类模板std::ranges::single\_view<>是一个拥有一个元素的视图。除非值类型是const,否则可以对值进行。

总体效果是,单视图的行为大致类似于一个元素的低成本集合,不分配堆上的内存。

单视图的主要用例是调用泛型代码来操作只有一个元素/值的视图:

\begin{cpp}
std::ranges::single_view<int> v1{42}; // single view
for (auto val : v1) {
	... // called once
}
\end{cpp}

这可以用于测试用例,或者用于代码必须在特定上下文中提供视图的情况,该视图必须只有一个元素。

\mySamllsection{单视图的范围工厂}

单视图也可以(通常应该)通过范围工厂创建:

\begin{cpp}
std::views::single(val)
\end{cpp}

例如:

\begin{cpp}
for (auto val : std::views::single(42)) { // iterate over the single int value 42
	... // called once
}
\end{cpp}

下面是一个使用单视图的完整示例程序:

\filename{ranges/singleview.cpp}

\begin{cpp}
#include <iostream>
#include <string>
#include <ranges>

void print(std::ranges::input_range auto&& coll)
{
	for (const auto& elem : coll) {
		std::cout << elem << ' ';
	}
	std::cout << '\n';
}

int main()
{
	std::ranges::single_view<double> sv0; // single view double 0.0
	std::ranges::single_view sv1{42}; // single_view<int> with int 42
	
	auto sv2 = std::views::single('x'); // single_view<char>
	auto sv3 = std::views::single("ok"); // single_view<const char*>
	std::ranges::single_view<std::string> sv4{"ok"}; // single view with string "ok"
	
	print(sv0);
	print(sv1);
	print(sv2);
	print(sv3);
	print(sv4);
	print(std::ranges::single_view{42});
}
\end{cpp}

该程序有以下输出:

\begin{shell}
0
42
x
ok
ok
42
\end{shell}

\mySamllsection{单视图的特殊特性}

单视图为以下概念建模:

\begin{itemize}
\item
std::ranges::contiguous\_range

\item
std::ranges::sized\_range
\end{itemize}

视图始终是一个通用范围,而不是一个租借范围(其迭代器的生命周期不依赖于视图)。

\mySamllsection{单视图的接口}

“类std::ranges::single\_view<>的操作”表列出了单视图的API。

% Please add the following required packages to your document preamble:
% \usepackage{longtable}
% Note: It may be necessary to compile the document several times to get a multi-page table to line up properly
\begin{longtable}[c]{|l|l|}
\hline
\textbf{操作} & \textbf{效果}                                         \\ \hline
\endfirsthead
%
\endhead
%
single\_view\textless{}Type\textgreater v                                    & 创建具有type类型的一个值初始化元素的视图   \\ \hline
single\_view v\{val\}                                                        & 创建具有其类型的一个值为val的元素的视图         \\ \hline
single\_view\textless{}Type\textgreater v\{std::in\_place, arg1, arg2, ...\} & 创建一个视图,其中一个元素初始化为arg1, arg2,… \\ \hline
v.begin()          & 生成指向元素的原始指针                     \\ \hline
v.end()            & 生成指向元素后面位置的原始指针 \\ \hline
v.empty()          & 总为false                                            \\ \hline
if(v)              & 总为true                                             \\ \hline
v.size()           & 生成1                                                \\ \hline
v.front()          & 生成对当前值的引用                 \\ \hline
v.back()           & 生成对当前值的引用                 \\ \hline
v{[}idx{]}         & 生成idx 0的当前值                      \\ \hline
r.data()           & 产生指向元素的原始指针                     \\ \hline
\end{longtable}

\begin{center}
表8.9 类std::ranges::single\_view<>的操作
\end{center}

若调用构造函数时没有传递初始值,则单视图中的元素将初始化。所以要么使用其Type的默认构造函数,要么用0、false或nullptr初始化值。

要用需要多个初始化器的对象初始化单视图时,有两个选择:

\begin{itemize}
\item
传递初始化的对象:

\begin{cpp}
std::ranges::single_view sv1{std::complex{1,1}};
auto sv2 = std::views::single(std::complex{1,1});
\end{cpp}

\item
在参数std::in\_place后传递初始值作为参数:

\begin{cpp}
std::ranges::single_view<std::complex<int>> sv4{std::in_place, 1, 1};
\end{cpp}
\end{itemize}

注意,对于非const单视图,可以修改“element”的值:

\begin{cpp}
std::ranges::single_view v2{42};
std::cout << v2.front() << '\n'; // prints 42
++v2.front(); // OK, modifies value of the view
std::cout << v2.front() << '\n'; // prints 43
\end{cpp}

可以通过将元素类型声明为const来防止这种情况:

\begin{cpp}
std::ranges::single_view<const int> v3{42};
++v3.front(); // ERROR
\end{cpp}

接受值的构造函数复制或移动该值到视图中,所以视图不能引用初始值(将元素类型声明为引用不会编译)。

因为begin()和end()会生成相同的值,所以视图总是一个通用范围。因为迭代器引用存储在视图中的值,以便能对其进行修改,所以单视图不是一个租借范围(其迭代器的生命周期确实取决于视图)。

\mySubsubsection{8.4.3}{空视图}

% Please add the following required packages to your document preamble:
% \usepackage{longtable}
% Note: It may be necessary to compile the document several times to get a multi-page table to line up properly
\begin{longtable}[c]{|l|l|}
\hline
\textbf{类型:}                 & std::ranges::empty\_view\textless{}\textgreater{} \\ \hline
\endfirsthead
%
\endhead
%
\textbf{内容:}              & 无元素范围的生成器               \\ \hline
\textbf{工厂:}              & std::views::empty\textless{}\textgreater{}        \\ \hline
\textbf{元素类型:}         & 引用                                         \\ \hline
\textbf{类别:}             & 连续的                                        \\ \hline
\textbf{是否是长度范围:}       & 总是0                                \\ \hline
\textbf{是否是通用范围:}      & 总是                                            \\ \hline
\textbf{是否是租借范围:}    & 总是                                            \\ \hline
\textbf{缓存:}               & 无                                           \\ \hline
\textbf{常量可迭代:}       & 总是                                            \\ \hline
\textbf{传播常量性:} & 从不                                             \\ \hline
\end{longtable}

类模板std::ranges::empty\_view<>是一个没有元素的视图,但必须指定元素类型。

空视图的主要用例是调用具有低成本视图的泛型代码,该视图没有元素,并且类型系统知道它没有任何元素:

\begin{cpp}
std::ranges::empty_view<int> v1; // empty view
for (auto val : v1) {
	... // never called
}
\end{cpp}

这可以用于测试用例,或者用于代码必须提供在特定上下文中,其中视图不能有元素。

\mySamllsection{空视图的范围工厂}

空视图也可以(通常应该)用范围工厂创建,其为一个变量模板,所以用类型的模板参数声明它,但没有调用参数:

\begin{cpp}
std::views::empty<type>
\end{cpp}

例如:

\begin{cpp}
for (auto val : std::views::empty<int>) { // iterate over no int values
	... // never called
}
\end{cpp}

下面是一个使用空视图的完整示例程序:

\filename{ranges/emptyview.cpp}

\begin{cpp}
#include <iostream>
#include <string>
#include <ranges>

void print(std::ranges::input_range auto&& coll)
{
	if (coll.empty()) {
		std::cout << "<empty>\n";
	}
	else {
		for (const auto& elem : coll) {
			std::cout << elem << ' ';
		}
		std::cout << '\n';
	}
}

int main()
{
	std::ranges::empty_view<double> ev0; // empty view of double
	auto ev1 = std::views::empty<int>; // empty view of int
	
	print(ev0);
	print(ev1);
}
\end{cpp}

该程序有以下输出:

\begin{shell}
<empty>
<empty>
\end{shell}

\mySamllsection{空视图的特殊特性}

空视图的行为大致类似于空vector,它模拟了以下概念:

\begin{itemize}
\item
std::ranges::contiguous\_range(暗示std::ranges::random\_access\_range)

\item
std::ranges::sized\_range
\end{itemize}

begin()和end()总是产生nullptr,所以该范围既是通用范围又是租借范围(其迭代器的生命周期不依赖于视图)。

\mySamllsection{空视图的接口}

“类std::ranges::empty\_view<>的操作”表列出了空视图的API。

读者们可能想知道为什么空视图提供front()、back()和operator[],其行为总是未定义的,调用它们总会触发致命的运行时错误。然而,泛型代码在调用front()、back()或[]运算符之前总是必须检查(或知道)是否存在元素,所以泛型代码永远不会调用这些成员函数。即使是空视图,这样的代码也可以编译。

% Please add the following required packages to your document preamble:
% \usepackage{longtable}
% Note: It may be necessary to compile the document several times to get a multi-page table to line up properly
\begin{longtable}[c]{|l|l|}
\hline
\textbf{操作} & \textbf{效果}                                                       \\ \hline
\endfirsthead
%
\endhead
%
empty\_view\textless{}Type\textgreater v & 创建不包含Type类型元素的视图                          \\ \hline
v.begin()                                & 生成一个原始指针,指向用nullptr初始化的元素类型 \\ \hline
v.end()            & 生成一个原始指针,指向用nullptr初始化的元素类型 \\ \hline
v.empty()          & 生成true                                                           \\ \hline
if (v)             & 总是false                                                          \\ \hline
v.size()           & 生成0                                                              \\ \hline
v.front()          & 总是未定义的行为(致命的运行时错误)                        \\ \hline
v.back()           & 总是未定义的行为(致命的运行时错误)                        \\ \hline
v{[}idx{]}         & 总是未定义的行为(致命的运行时错误)                        \\ \hline
r.data             & 生成一个原始指针,指向用nullptr初始化的元素类型 \\ \hline
\end{longtable}

\begin{center}
表8.10 类std::ranges::empty\_view<>的操作
\end{center}

例如,可以这样传递一个空视图:

\begin{cpp}
void foo(std::ranges::random_access_range auto&& rg)
{
	std::cout << "sortFirstLast(): \n";
	std::ranges::sort(rg);
	if (!std::ranges::empty(rg)) {
		std::cout << " first: " << rg.front() << '\n';
		std::cout << " last: " << rg.back() << '\n';
	}
}

foo(std::ranges::empty_view<int>{}); // OK
\end{cpp}

\mySubsubsection{8.4.4}{istream视图}

% Please add the following required packages to your document preamble:
% \usepackage{longtable}
% Note: It may be necessary to compile the document several times to get a multi-page table to line up properly
\begin{longtable}[c]{|l|l|}
\hline
\textbf{类型:} &
\begin{tabular}[c]{@{}l@{}}std::ranges::basic\_istream\_view\textless{}\textgreater\\ std::ranges::istream\_view\textless{}\textgreater\\ std::ranges::wistream\_view\textless{}\textgreater{}\end{tabular} \\ \hline
\endfirsthead
%
\endhead
%
\textbf{内容:}              & 从流中读取元素的范围的生成器 \\ \hline
\textbf{工厂:}              & std::views::istream\textless{}\textgreater{}()        \\ \hline
\textbf{元素类型:}         & 引用                                             \\ \hline
\textbf{类别:}             & 输入                                                 \\ \hline
\textbf{是否是长度范围:}       & 从不                                                 \\ \hline
\textbf{是否是通用范围:}      & 从不                                                 \\ \hline
\textbf{是否是租借范围:}    & 从不                                                 \\ \hline
\textbf{缓存:}               & 无                                               \\ \hline
\textbf{常量可迭代:}       & 从不                                                 \\ \hline
\textbf{传播常量性:} & --                                                    \\ \hline
\end{longtable}

类模板std::ranges::basic\_istream\_view<>是一个从输入流(如标准输入、文件或字符串流)中读取元素的视图。

与通常的流类型一样,该类型是字符类型的泛型,并提供char和wchar\_t的特化:

\begin{itemize}
\item
类模板std::ranges::istream\_view<>是一个视图,使用char类型的字符从输入流中读取元素。

\item
类模板std::ranges::wistream\_view<>是一个视图,使用wchar\_t类型的字符从输入流中读取元素。
\end{itemize}

例如:

\begin{cpp}
std::istringstream myStrm{"0 1 2 3 4"};

for (const auto& elem : std::ranges::istream_view<int>{myStrm}) {
	std::cout << elem << '\n';
}
\end{cpp}

\mySamllsection{istream视图的范围工厂}

Istream视图也可以(通常应该)使用范围工厂创建,工厂使用传入范围的字符类型将其参数传递给std::ranges::basic\_istream\_view构造函数:

\begin{cpp}
std::views::istream<Type>(rg)
\end{cpp}

例如:

\begin{cpp}
std::istringstream myStrm{"0 1 2 3 4"};

for (const auto& elem : std::views::istream<int>(myStrm)) {
	std::cout << elem << '\n';
}
\end{cpp}

或:

\begin{cpp}
std::wistringstream mywstream{L"1.1 2.2 3.3 4.4"};
auto vw = std::views::istream<double>(mywstream);
\end{cpp}

vw的初始化相当于以下两个初始化:

\begin{cpp}
auto vw2 = std::ranges::basic_istream_view<double, wchar_t>{myStrm};
auto vw3 = std::ranges::wistream_view<double>{myStrm};
\end{cpp}

下面是一个使用istream视图的完整示例程序:

\filename{ranges/istreamview.cpp}

\begin{cpp}
#include <iostream>
#include <string>
#include <sstream>
#include <ranges>

void print(std::ranges::input_range auto&& coll)
{
	for (const auto& elem : coll) {
		std::cout << elem << " ";
	}
	std::cout << '\n';
}

int main()
{
	std::string s{"2 4 6 8 Motorway 1977 by Tom Robinson"};
	
	std::istringstream mystream1{s};
	std::ranges::istream_view<std::string> vs{mystream1};
	print(vs);
	
	std::istringstream mystream2{s};
	auto vi = std::views::istream<int>(mystream2);
	print(vi);
}
\end{cpp}

该程序有以下输出:

\begin{shell}
2 4 6 8 Motorway 1977 by Tom Robinson
2 4 6 8
\end{shell}

程序在用字符串s初始化的字符串流上迭代两次:

\begin{itemize}
\item
istream视图vs遍历从输入字符串流mystream1读取的所有字符串,打印s的所有子字符串。

\item
istream视图vi遍历从输入字符串流mystream2中读取的所有整型数,读数结束时,到达Motorway。
\end{itemize}

\mySamllsection{istream视图和const}

注意,不能迭代const istream视图。例如:

\begin{cpp}
void printElems(const auto& coll) {
	for (const auto elem& e : coll) {
		std::cout << elem << '\n';
	}
}

std::istringstream myStrm{"0 1 2 3 4"};

printElems(std::views::istream<int>(myStrm)); // ERROR
\end{cpp}

问题是begin()仅为非const istream视图提供,因为值分两步处理(begin()/++和operator*),同时值存储在视图中。

下面的例子演示了修改视图的原因,通过使用底层的方法来迭代视图的“元素”,从istream视图中读取字符串:

\begin{cpp}
std::istringstream myStrm{"stream with very-very-very-long words"};

auto v = std::views::istream<std::string>(myStrm);

for (auto pos = v.begin(); pos != v.end(); ++pos) {
	std::cout << *pos << '\n';
}
\end{cpp}

代码有以下输出:

\begin{shell}
stream
with
very-very-very-long
words
\end{shell}

当迭代器pos遍历视图va时,使用begin()和++从输入流myStrm中读取字符串。这些字符串必须存储在内部,以便可以通过操作符*使用它们。若将字符串存储在迭代器中,迭代器可能必须为该字符串保存内存,复制迭代器必须复制该内存,复制迭代器的代价不应该太高,所以视图将字符串存储在视图中,其效果是在我们迭代时修改视图。

因此,必须在泛型代码中使用通用/转发引用来支持此视图:

\begin{cpp}
void printElems(auto&& coll) {
	...
}
\end{cpp}

\mySamllsection{istream视图的接口}

“类std::ranges::basic\_istream\_view<>的操作”表列出了istream视图的API。

% Please add the following required packages to your document preamble:
% \usepackage{longtable}
% Note: It may be necessary to compile the document several times to get a multi-page table to line up properly
\begin{longtable}[c]{|l|l|}
\hline
\textbf{操作} & \textbf{效果}                                     \\ \hline
\endfirsthead
%
\endhead
%
istream\_view\textless{}Type\textgreater v\{strm\} & Creates an istream view of type Type reading from strm \\ \hline
v.begin()          & 读取第一个值并生成begin迭代器 \\ \hline
v.end()            & 产生std::default\_sentinel作为结束迭代器       \\ \hline
\end{longtable}

\begin{center}
表8.11 类std::ranges::basic\_istream\_view<>的操作
\end{center}

所能做的就是使用迭代器逐值读取,直到到达流的末尾。没有提供其他成员函数(如empty()或front())。

请注意,C++20标准中istream视图的原始规范与其他视图有一些不一致,这些不一致后来修复了(参见\url{http://wg21.link/p2432})。通过这个修复,有了当前的行为:

\begin{itemize}
\item
可以完全指定所有模板参数:

\begin{cpp}
std::ranges::basic_istream_view<int, char, std::char_traits<char>> v1{myStrm};
\end{cpp}

\item
最后一个模板参数有一个默认值,所以可以使用:

\begin{cpp}
std::ranges::basic_istream_view<int, char> v2{myStrm}; // OK
\end{cpp}

\item
但是,不能跳过太多参数:

\begin{cpp}
std::ranges::basic_istream_view<int> v3{myStrm}; // ERROR
\end{cpp}

\item
istream视图遵循通常的约定,即为没有basic\_前缀的char和wchar\_t流有特殊类型:

\begin{cpp}
std::ranges::istream_view<int> v4{myStrm}; // OK for char streams

std::wistringstream mywstream{L"0 1 2 3 4"};
std::ranges::wistream_view<int> v5{mywstream}; // OK for wchar_t streams
\end{cpp}

\item
并提供了相应的范围工厂:

\begin{cpp}
auto v6 = std::views::istream<int>(myStrm); // OK
\end{cpp}
\end{itemize}

\mySubsubsection{8.4.5}{字符视图}

% Please add the following required packages to your document preamble:
% \usepackage{longtable}
% Note: It may be necessary to compile the document several times to get a multi-page table to line up properly
\begin{longtable}[c]{|l|l|}
\hline
\textbf{类型:} &
\begin{tabular}[c]{@{}l@{}}std::basic\_string\_view\textless{}\textgreater\\ std::string\_view\\ std::u8string\_view\\ std::u16string\_view\\ std::u32string\_view\\ std::wstring\_view\end{tabular} \\ \hline
\endfirsthead
%
\endhead
%
\textbf{内容:}              & 字符序列中的所有字符 \\ \hline
\textbf{工厂:}              & --                                     \\ \hline
\textbf{元素类型:}         & const引用                       \\ \hline
\textbf{类别:}             & 连续                             \\ \hline
\textbf{是否是长度范围:}       & 总是                                 \\ \hline
\textbf{是否是通用范围:}      & 总是                                 \\ \hline
\textbf{是否是租借范围:}    & 总是                                 \\ \hline
\textbf{缓存:}               & 无                                \\ \hline
\textbf{常量可迭代:}       & 总是                                 \\ \hline
\textbf{传播常量性:} & 元素总是const              \\ \hline
\end{longtable}

类模板std::basic\_string\_view<>,及其特化std::string\_view、std::u16string\_view、std::u32string\_view和std::wstring\_view是C++17标准库中仅有的视图类型。然而,C++20为UTF-8字符增加了特化:std::u8string\_view。

字符串视图不遵循视图具有的几个约定:

\begin{itemize}
\item
视图定义在命名空间std中(而不是std::ranges)。

\item
视图有自己的头文件<string\_view>。

\item
视图没有范围适配器/工厂来创建视图对象。

\item
视图只提供对元素/字符的读访问。

\item
视图提供了cbegin()和cend()成员函数。

\item
视图不支持bool转换。
\end{itemize}

例如:

\begin{cpp}
for (char c : std::string_view{"hello"}) {
	std::cout << c << ' ';
}
std::cout << '\n';
\end{cpp}

这个循环有以下输出:

\begin{shell}
h e l l o
\end{shell}

\mySamllsection{字符视图的特殊特性}

迭代器不引用这个视图。相反,其引用底层字符序列,所以字符串视图是一个租借范围。但是,当底层字符序列不再存在时,迭代器仍然可以悬空。

\mySamllsection{字符串视图特定于视图的接口}

“类std::basic\_string\_view<>视图的操作”表列出了字符串视图API中与视图相关的部分。

% Please add the following required packages to your document preamble:
% \usepackage{longtable}
% Note: It may be necessary to compile the document several times to get a multi-page table to line up properly
\begin{longtable}[c]{|l|l|}
\hline
\textbf{操作}   & \textbf{效果}                                                       \\ \hline
\endfirsthead
%
\endhead
%
string\_view sv      & 创建一个没有元素的字符串视图                                \\ \hline
string\_view sv\{s\} & 使用s创建一个字符串视图                                            \\ \hline
v.begin()            & 生成一个原始指针,指向用nullptr初始化的元素类型 \\ \hline
v.end()              & 生成一个原始指针,指向用nullptr初始化的元素类型 \\ \hline
sv.empty()           & 生成的sv是否为空                                            \\ \hline
sv.size()            & 生成字符的数量                                       \\ \hline
sv.front()           & 生成的第一个字符                                            \\ \hline
sv.back()            & 生成的最后一个字符                                             \\ \hline
sv{[}idx{]}          & 生成的第n个字符                                              \\ \hline
sv.data()            & 产生指向字符或nullptr的原始指针                     \\ \hline
...                  & 为只读字符串提供的其他几个操作               \\ \hline
\end{longtable}

\begin{center}
表8.12 类std::basic\_string\_view<>视图的操作
\end{center}

除了通常的视图接口之外,该类型还提供了所有字符串只读操作的完整API。

要了解更多细节,请查看我的另一本书《C++17-完整指南》(参见\url{http://www.cppstd17.com})。

\mySubsubsection{8.4.6}{span}

% Please add the following required packages to your document preamble:
% \usepackage{longtable}
% Note: It may be necessary to compile the document several times to get a multi-page table to line up properly
\begin{longtable}[c]{|l|l|}
\hline
\textbf{类型:}                 & std::span\textless{}\textgreater{}           \\ \hline
\endfirsthead
%
\endhead
%
\textbf{内容:}              & 连续存储器中一个范围的所有元素 \\ \hline
\textbf{工厂:}              & std::views::counted()                        \\ \hline
\textbf{元素类型:}         & 引用                                    \\ \hline
\textbf{要求:}             & 连续的范围                             \\ \hline
\textbf{类别:}             & 连续                                   \\ \hline
\textbf{是否是长度范围:}       & Always                                       \\ \hline
\textbf{是否是通用范围:}      & 总是                                       \\ \hline
\textbf{是否是租借范围:}    & 总是                                       \\ \hline
\textbf{缓存:}               & 无                                      \\ \hline
\textbf{常量可迭代:}       & 总是                                       \\ \hline
\textbf{传播常量性:} & 从不                                        \\ \hline
\end{longtable}

类模板std::span<>是存储在连续内存中的元素序列的视图,其在单独的章节中有详细的描述。这里,仅将span的属性作为视图呈现。

span的主要优点是,支持在引用范围的中间或末尾引用n个元素的子范围。例如:

\begin{cpp}
std::vector<std::string> vec{"New York", "Tokyo", "Rio", "Berlin", "Sydney"};

// sort the three elements in the middle:
std::ranges::sort(std::span{vec}.subspan(1, 3));

// print last three elements:
print(std::span{vec}.last(3));
\end{cpp}

这个例子将在后面关于span的章节中讨论。

请注意,span不遵循视图通常具有的几个约定:

\begin{itemize}
\item
视图定义在命名空间std中,而不是std::ranges中。

\item
视图有自己的头文件<span>。

\item
视图不支持bool转换。
\end{itemize}

\mySamllsection{span的范围工厂}

没有范围工厂用于从begin(迭代器)和end(哨兵)初始化span、,但有一个range工厂可以创建一个begin和count的span:

\begin{cpp}
std::views::counted(beg, sz)
\end{cpp}

std::views::counted()创建一个动态大小的跨度,从迭代器beg开始,用连续范围的前sz个元素初始化(对于非连续范围,counts()创建子范围)。

例如:

\begin{cpp}
std::vector<int> vec{1, 2, 3, 4, 5, 6, 7, 8, 9};

auto v = std::views::counted(vec.begin()+1, 3); // span with 2nd to 4th elem of vec
\end{cpp}

当std::views::counted()创建一个span时,span具有动态范围(其大小可以变化)。

\mySamllsection{span的特点}

迭代器不指向span。相反,其引用底层元素,所以跨度是一个租借范围。但请注意,当底层字符序列不再存在时,迭代器仍然可以悬空。

\mySamllsection{span的特定视图接口}

“类std::span<>的视图操作”表列出了span的API中与视图相关的部分。

% Please add the following required packages to your document preamble:
% \usepackage{longtable}
% Note: It may be necessary to compile the document several times to get a multi-page table to line up properly
\begin{longtable}[c]{|l|l|}
\hline
\textbf{操作} & \textbf{效果}                                   \\ \hline
\endfirsthead
%
\endhead
%
span sp            & 创建一个没有元素的span                   \\ \hline
span sp\{s\}       & 为s创建一个span视图                             \\ \hline
sp.begin()         & 生成begin迭代器                         \\ \hline
sp.end()           & 生成哨兵(end迭代器)                \\ \hline
sp.empty()         & 生成的sv是否为空                        \\ \hline
sp.size()          & 生成字符的数量                   \\ \hline
sp.front()         & 生成的第一个字符                        \\ \hline
sp.back()          & 生成的最后一个字符                         \\ \hline
sp{[}idx{]}        & 生成的第n个字符                         \\ \hline
sp.data()          & 生成指向字符或nullptr的原始指针 \\ \hline
...                & 请参阅关于span操作的部分            \\ \hline
\end{longtable}

\begin{center}
表8.13 类std::span<>的视图操作
\end{center}

有关更多详细信息,请参阅有关span操作的部分。









