为了处理范围,C++20引入了两个视图类型,这些视图类型不会在自己的内存中存储元素,只引用存储在其他范围或视图中的元素。std::span<>类模板就是这种视图。

历史上看,span是C++17中引入的字符串视图的泛化。span指的是任何元素类型的数组,作为原始指针和大小的组合,提供了通常的集合接口,用于读取和写入存储在连续内存中的元素。

通过要求span只能引用连续内存中的元素,迭代器可以是原始指针,这使得其很廉价。该集合提供随机访问(以便跳转到范围内的任何位置),因此可以使用此视图对元素进行排序,或者可以使用生成位于底层范围中间或末尾的n个元素的子序列的操作。

使用span既廉价又快速(应该按值传递),但也有潜在的危险。与原始指针一样,在使用span时,需要由开发者来确保引用序列的有效性。此外,span支持写访问的情况,可能会破坏const的正确性(或者不能按期望的方式工作)。