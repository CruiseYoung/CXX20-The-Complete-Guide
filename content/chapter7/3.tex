
为了(更好地)支持范围和视图,C++20引入了几个新的迭代器和哨兵类型:

\begin{itemize}
\item
std::counted\_iterator用于迭代器,该迭代器本身有一个计数来指定范围的结束

\item
std::common\_iterator用于公共迭代器类型,可用于具有不同类型的两个迭代器

\item
std::default\_sentinel\_t用于结束迭代器,强制迭代器检查其结束

\item
std::unreachable\_sentinel\_t表示永远无法到达的end迭代器,表示无限范围

\item
std::move\_sentinel用于将副本映射到move的end迭代器
\end{itemize}

\mySubsubsection{7.3.1}{std::counted\_iterator}

计数迭代器是一种迭代器,其有一个计数器,表示要迭代的最大元素数。

有两种方法可以使用这样的迭代器:

\begin{itemize}
\item
遍历并检查还剩下多少个元素:

\begin{cpp}
for (std::counted_iterator pos{coll.begin(), 5}; pos.count() > 0; ++pos) {
	std::cout << *pos << '\n';
}
std::cout << '\n';
\end{cpp}

\item
迭代并与默认的哨兵进行比较:

\begin{cpp}
for (std::counted_iterator pos{coll.begin(), 5};
pos != std::default_sentinel; ++pos) {
	std::cout << *pos << '\n';
}
\end{cpp}

当视图适配器std::ranges:: counts()为非随机访问迭代器生成子范围时,就会使用此功能。
\end{itemize}

类std:: counts \_iterator<>的表操作列出了计数迭代器的API。

\begin{longtable}[c]{|l|l|}
\hline
\textbf{操作} &
\textbf{效果} \\ \hline
\endfirsthead
%
\endhead
%
\begin{tabular}[c]{@{}l@{}}countedItor pos\{\}\\ countedItor pos\{pos2, num\}\\ countedItor pos\{pos2\}\\ pos.count()\\ pos.base()\\ ...\\ pos == std::default\_sentinel\\ pos != std::default\_sentinel\end{tabular} &
\begin{tabular}[c]{@{}l@{}}创建一个不引用任何元素的计数迭代器(count为0)\\ 创建以pos2开头的num个元素的计数迭代器\\ 创建一个有计数迭代器,作为有计数迭代器pos2的(类型转换)副本\\ 返回剩余元素的数量(0表示已到结束)\\ 生成底层迭代器的(副本)\\ 底层迭代器类型的所有标准迭代器操作\\ 返回迭代器是否在末尾\\ 返回迭代器是否在末尾\end{tabular} \\ \hline
\end{longtable}

\begin{center}
表7.2 std::counted\_iterator<>类的操作
\end{center}

开发者则负责确保:

\begin{itemize}
\item
初始计数不高于初始传递的迭代器计数

\item
计数迭代器的增量不会超过count次

\item
计数迭代器不访问第n个元素以外的元素(通常,最后一个元素后面的位置是有效的)
\end{itemize}

\mySubsubsection{7.3.2}{std::common\_iterator}

类型std::common\_iterator<>用于协调两个迭代器的类型,其包装两个迭代器,以便从外部看,两个迭代器具有相同的类型。在内部,存储两种类型之一的值(为此,迭代器通常使用std::variant<>)。

例如,接受开始迭代器和结束迭代器的传统算法要求这些迭代器具有相同的类型。若有不同类型的迭代器,可以使用这个类型函数来调用这些算法:

\begin{cpp}
algo(beg, end); // if this is an error due to different types

algo(std::common_iterator<decltype(beg), decltype(end)>{beg}, // OK
std::common_iterator<decltype(beg), decltype(end)>{end});
\end{cpp}

注意,若传递给common\_iterator<>的类型相同,则会导致编译时错误,所以在泛型代码中,必须调用:

\begin{cpp}
template<typename BegT, typename EndT>
void callAlgo(BegT beg, EndT end)
{
	if constexpr(std::same_as<BegT, EndT>) {
		algo(beg, end);
	}
	else {
		algo(std::common_iterator<decltype(beg), decltype(end)>{beg},
		std::common_iterator<decltype(beg), decltype(end)>{end});
	}
}
\end{cpp}

要达到同样的效果,更方便的方法是使用common()范围适配器:

\begin{cpp}
template<typename BegT, typename EndT>
void callAlgo(BegT beg, EndT end)
{
	auto v = std::views::common(std::ranges::subrange(beg,end));
	algo(v.begin(), v.end());
}
\end{cpp}

\mySubsubsection{7.3.3}{std::default\_sentinel}

默认哨兵是一个不提供任何操作的迭代器。C++20提供了一个相应的对象std::default\_sentinel,其类型为std::default\_sentinel\_t,其在<iterator>中提供。该类型没有成员:

\begin{cpp}
namespace std {
	class default_sentinel_t {
	};
	inline constexpr default_sentinel_t default_sentinel{};
}
\end{cpp}

提供类型和值作为伪哨兵(结束迭代器),当迭代器无需查看结束迭代器就知道其结束时。对于这些迭代器,定义了与std::default\_sentinel的比较,但从未使用默认的哨兵。相反,迭代器在内部检查它是否在结束处(或距离结束处有多远)。

例如,计数迭代器为自己定义了一个操作符==,并带有一个默认的哨兵来检查其是否位于末尾:

\begin{cpp}
namespace std {
	template<std::input_or_output_iterator I>
	class counted_iterator {
		...
		friend constexpr bool operator==(const counted_iterator& p,
		std::default_sentinel_t) {
			... // returns whether p is at the end
		}
	};
}
\end{cpp}

代码可以这样写:

\begin{cpp}
// iterate over the first five elements:
for (std::counted_iterator pos{coll.begin(), 5};
pos != std::default_sentinel;
++pos) {
	std::cout << *pos << '\n';
}
\end{cpp}

该标准用默认的哨兵定义了以下操作:

\begin{itemize}
\item
std::counted\_iterator:

\begin{itemize}
\item
使用操作符==和!=进行比较

\item
用运算符-计算距离
\end{itemize}

\item
std::views:: counts()可以创建一个被计数迭代器的子范围和一个默认的哨兵

\item
std::istream\_iterator:

\begin{itemize}
\item
默认哨兵可以用作初始值,其效果与默认构造函数相同

\item
使用操作符==和!=进行比较
\end{itemize}

\item
std::istreambuf\_iterator:

\begin{itemize}
\item
默认哨兵可以用作初始值,其效果与默认构造函数相同

\item
使用操作符==和!=进行比较
\end{itemize}

\item
std::ranges::basic\_istream\_view<>:

\begin{itemize}
\item
Istream视图生成std::default\_sentinel作为end()

\item
Istream视图迭代器可以使用==和!=进行比较
\end{itemize}

\item
std::ranges::take\_view<>:

\begin{itemize}
\item
Take视图可能产生std::default\_sentinel作为end()
\end{itemize}

\item
std::ranges::split\_view<>:

\begin{itemize}
\item
Split视图可能产生std::default\_sentinel作为end()

\item
Split视图迭代器可以使用==和!=进行比较
\end{itemize}
\end{itemize}

\mySubsubsection{7.3.4}{std::unreachable\_sentinel}

std::unreachable\_sentinel\_t类型的值std::unreachable\_sentinel是在C++20中引入的,用于指定不可达的哨兵(范围的结束迭代器),其实际上是在说:“不要和我比较。”此类型和值可用于指定无限范围。

使用时,它可以优化生成的代码,因为编译器可以检测到它与另一个迭代器进行比较时永远不会返回true,所以它可能会跳过对末尾的任何检查。

例如,若知道一个值42存在于一个集合中,就可以这样进行搜索:

\begin{cpp}
auto pos42 = std::ranges::find(coll.begin(), std::unreachable_sentinel,
					42);
\end{cpp}

通常,该算法将同时比较42和coll.end()(或作为结束/前哨传递的任何内容)。因为使用了unreachable\_sentinel(与迭代器的任何比较总是产生false),编译器可以优化代码,只与42进行比较,所以开发者必须确保42是存在的。

iota视图提供了一个对无限范围使用unreachable\_sentinel的示例。

\mySubsubsection{7.3.5}{std::move\_sentinel}

C++11起,C++标准库提供了一个迭代器类型std::move\_iterator,可用于将迭代器的行为从复制映射到移动值,C++20引入了相应的哨兵类型。

此类型只能用于使用操作符==和!=比较移动哨兵和移动迭代器,并计算移动迭代器和移动哨兵之间的差异(若支持)。

若有一个形成有效范围的迭代器和一个哨兵(满足std::sentinel\_for概念),可以将它们转换为一个移动迭代器和一个移动哨兵,以获得仍然满足sentinel\_for的有效范围。

可以像下面这样移动哨兵:

\begin{cpp}
std::list<std::string> coll{"tic", "tac", "toe"};
std::vector<std::string> v1;
std::vector<std::string> v2;

// copy strings into v1:
for (auto pos{coll.begin()}; pos != coll.end(); ++pos) {
	v1.push_back(*pos);
}

// move strings into v2:
for (std::move_iterator pos{coll.begin()};
pos != std::move_sentinel(coll.end()); ++pos) {
	v2.push_back(*pos);
}
\end{cpp}

注意,这样做的效果是迭代器和哨兵具有不同的类型,所以不能直接初始化vector,因为它要求begin和end类型相同。在这种情况下,还必须为end使用move迭代器:

\begin{cpp}
std::vector<std::string> v3{std::move_iterator{coll.begin()},
							std::move_sentinel{coll.end()}}; // ERROR
std::vector<std::string> v4{std::move_iterator{coll.begin()},
							std::move_iterator{coll.end()}}; // OK
\end{cpp}















