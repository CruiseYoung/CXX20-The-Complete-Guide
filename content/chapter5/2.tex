

This section lists the concepts that apply to objects and types in general.



\subsubsection*{\zihao{3} 5.2.1\hspace{0.2cm}Arithmetic Concepts}
\addcontentsline{toc}{subsubsection}{5.2.2\hspace{0.2cm}Arithmetic Concepts}



\noindent
\hspace*{\fill} \\ %插入空行
\textbf{std::integral<T>}

\begin{itemize}
\item
Guarantees that type T is an integral type (including bool and all character types).

\item
Requires:
\begin{itemize}
\item
The type trait std::is\_integral\_v<T> yields true
\end{itemize}
\end{itemize}


\noindent
\hspace*{\fill} \\ %插入空行
\textbf{std::signed\_integral<T>}

\begin{itemize}
\item
Guarantees that type T is a signed integral type (including signed character types, which char might be).

\item
Requires:
\begin{itemize}
\item
std::integral<T> is satisfied

\item
The type trait std::is\_signed\_v<T> yields true
\end{itemize}
\end{itemize}

\noindent
\hspace*{\fill} \\ %插入空行
\textbf{std::unsigned\_integral<T>}

\begin{itemize}
\item
Guarantees that type T is an unsigned integral type (including bool and unsigned character types, which char might be).

\item
Requires:
\begin{itemize}
\item
std::integral<T> is satisfied

\item
The type trait std::is\_signed\_v<T> yields true
\end{itemize}
\end{itemize}

\noindent
\hspace*{\fill} \\ %插入空行
\textbf{std::floating\_point<T>}

\begin{itemize}
\item
Guarantees that type T is a floating-point type (float, double, or long double).

\item
The concept was introduced to enable the definition of the mathematical constants.

\item
Requires:
\begin{itemize}
\item
The type trait std::is\_floating\_point\_v<T> yields true
\end{itemize}
\end{itemize}


\subsubsection*{\zihao{3} 5.2.2\hspace{0.2cm}Object Concepts}
\addcontentsline{toc}{subsubsection}{5.2.2\hspace{0.2cm}Object Concepts}

For objects (types that are neither references nor functions nor void) there is a hierarchy of required basic operations.

\noindent
\hspace*{\fill} \\ %插入空行
\textbf{std::movable<T>}

\begin{itemize}
\item
Guarantees that type T is movable and swappable. That is, you can move construct, move assign, and swap with another object of the type.

\item
Requires:
\begin{itemize}
\item
std::move\_constructible<T> is satisfied

\item
std::assignable\_from<T\&, T> is satisfied

\item
std::swappable<T> is satisfied

\item
T is neither a reference nor a function nor void
\end{itemize}
\end{itemize}

\noindent
\hspace*{\fill} \\ %插入空行
\textbf{std::copyable<T>}

\begin{itemize}
\item
Guarantees that type T is copyable (which implies movable and swappable).

\item
Requires:
\begin{itemize}
\item
std::movable<T> is satisfied

\item
std::copy\_constructible<T> is satisfied

\item
std::assignable\_from for any T, T\&, const T, and const T\& to T\&

\item
std::swappable<T> is satisfied

\item
T is neither a reference nor a function nor void
\end{itemize}
\end{itemize}

\noindent
\hspace*{\fill} \\ %插入空行
\textbf{std::semiregular<T>}

\begin{itemize}
\item
Guarantees that type T is semiregular (can default initialize, copy, move, and swap).

\item
Requires:
\begin{itemize}
\item
std::copyable<T> is satisfied

\item
std::default\_initializable<T> is satisfied

\item
std::movable<T> is satisfied

\item
std::copy\_constructible<T> is satisfied

\item
std::assignable\_from for any T, T\&, const T, and const T\& to T\&

\item
std::swappable<T> is satisfied

\item
T is neither a reference nor a function nor void
\end{itemize}
\end{itemize}

\noindent
\hspace*{\fill} \\ %插入空行
\textbf{std::regular<T>}

\begin{itemize}
\item
Guarantees that type T is regular (can default initialize, copy, move, swap, and check for equality).

\item
Requires:
\begin{itemize}
\item
std::semiregular<T> is satisfied

\item
std::equality\_comparable<T> is satisfied

\item
std::copyable<T> is satisfied

\item
std::default\_initializable<T> is satisfied

\item
std::movable<T> is satisfied

\item
std::copy\_constructible<T> is satisfied

\item
std::assignable\_from for any T, T\&, const T, and const T\& to T\&

\item
std::swappable<T> is satisfied

\item
T is neither a reference nor a function nor void
\end{itemize}
\end{itemize}

\subsubsection*{\zihao{3} 5.2.3\hspace{0.2cm}Concepts for Relationships between Types}
\addcontentsline{toc}{subsubsection}{5.2.3\hspace{0.2cm}Concepts for Relationships between Types}

\noindent
\hspace*{\fill} \\ %插入空行
\textbf{std::same\_as<T1, T2>}

\begin{itemize}
\item
Guarantees that the types T1 and T2 are the same.

\item
The concept calls the std::is\_same\_v type trait twice to ensure that the order of the parameters does not matter.

\item
Requires:
\begin{itemize}
\item
The type trait std::is\_same\_v<T1, T2> yields true

\item
The order of T1 and T2 does not matter
\end{itemize}
\end{itemize}


\noindent
\hspace*{\fill} \\ %插入空行
\textbf{std::convertible\_to<From, To>}

\begin{itemize}
\item
Guarantees that objects of type From are both implicitly and explicitly convertible to objects of type To.

\item
Requires:
\begin{itemize}
\item
The type trait std::is\_convertible\_v<From, To> yields true

\item
A static\_cast from From to To is supported

\item
You can return an object of type From as a To
\end{itemize}
\end{itemize}

\noindent
\hspace*{\fill} \\ %插入空行
\textbf{std::derived\_from<D, B>}

\begin{itemize}
\item
Guarantees that type D is publicly derived from type B (or D and B are the same) so that any pointer of type D can be converted to a pointer of type B. In other words: the concept guarantees that a D reference/pointer can be used as a B reference/pointer.

\item
Requires:
\begin{itemize}
\item
The type trait std::is\_base\_of\_v<B, D> yields true

\item
The type trait std::is\_convertible\_v for a const pointer of type D to B yields true
\end{itemize}
\end{itemize}

\noindent
\hspace*{\fill} \\ %插入空行
\textbf{std::constructible\_from<T, Args...>}

\begin{itemize}
\item
Guarantees that you can initialize an object of type T with parameters of types Args...

\item
Requires:
\begin{itemize}
\item
std::destructible<T> is satisfied

\item
The type trait std::is\_constructible\_v<T, Args...> yields true
\end{itemize}
\end{itemize}

\noindent
\hspace*{\fill} \\ %插入空行
\textbf{std::assignable\_from<To, From>}

\begin{itemize}
\item
Guarantees that you can move or copy assign a value of type From to a value of type To.

The assignment also has to yield the original To object.

\item
Requires:
\begin{itemize}
\item
To has to be an lvalue reference

\item
std::common\_reference\_with<To, From> is satisfied for const lvalue references of the types

\item
The operator = has to be supported and yield the same type as To
\end{itemize}
\end{itemize}

\noindent
\hspace*{\fill} \\ %插入空行
\textbf{std::swappable\_with<T1, T2>}

\begin{itemize}
\item
Guarantees that values of types T1 and T2 can be swapped.

\item
Requires:
\begin{itemize}
\item
std::common\_reference\_with<T1, T2> is satisfied

\item
Any two objects of types T1 and T2 can swap values with each other by using std::ranges::swap()
\end{itemize}
\end{itemize}

\noindent
\hspace*{\fill} \\ %插入空行
\textbf{std::common\_with<T1, T2>}

\begin{itemize}
\item
Guarantees that types T1 and T2 share a common type to which they can be explicitly converted.

\item
Requires:
\begin{itemize}
\item
The type trait std::common\_type\_t<T1, T2> yields a type

\item
A static\_cast is supported to their common type

\item
References of both types share a common\_reference type

\item
The order of T1 and T2 does not matter
\end{itemize}
\end{itemize}

\noindent
\hspace*{\fill} \\ %插入空行
\textbf{std::common\_reference\_with<T1, T2>}

\begin{itemize}
\item
Guarantees that types T1 and T2 share a common\_reference type to which they can be explicitly and implicitly converted.

\item
Requires:
\begin{itemize}
\item
The type trait std::common\_reference\_t<T1, T2> yields a type

\item
Both types are std::convertible\_to their common reference type

\item
The order of T1 and T2 does not matter
\end{itemize}
\end{itemize}

\subsubsection*{\zihao{3} 5.2.4\hspace{0.2cm}Comparison Concepts}
\addcontentsline{toc}{subsubsection}{5.2.4\hspace{0.2cm}Comparison Concepts}

\noindent
\hspace*{\fill} \\ %插入空行
\textbf{std::equality\_comparable<T>}

\begin{itemize}
\item
Guarantees that objects of type T are comparable with the operators == and !=. The order should not matter.

\item
The operator == for T should be symmetric and transitive:
\begin{itemize}
\item
t1==t2 is true if and only if t2==t1

\item
If t1==t2 and t2==t3 are true, then t1==t3 is true
\end{itemize}

However, this a semantic constraint that cannot be checked at compile time.

\item
Requires:
\begin{itemize}
\item
Both == and != are supported and yield a value convertible to bool
\end{itemize}
\end{itemize}


\noindent
\hspace*{\fill} \\ %插入空行
\textbf{std::equality\_comparable\_with<T1, T2>}

\begin{itemize}
\item
Guarantees that objects of types T1 and T2 are comparable using the operators == and !=.

\item
Requires:
\begin{itemize}
\item
== and != are supported for all comparisons where objects of T1 and/or T2 are involved and yield a value of the same type convertible to bool
\end{itemize}
\end{itemize}



\noindent
\hspace*{\fill} \\ %插入空行
\textbf{std::totally\_ordered<T>}

\begin{itemize}
\item
Guarantees that objects of type T are comparable with the operators ==, !=, <, <=, >, and >= so that two values are always either equal to, or less than, or greater than each other.

\item
The concept does not claim that type T has a total order for all values. In fact, the expression std::totally\_ordered<double> yields true even though floating-point values do not have a total order:

\begin{lstlisting}[style=styleCXX]
std::totally_ordered<double> // true
std::totally_ordered<std::pair<double, double>> // true
std::totally_ordered<std::complex<int>> // false
\end{lstlisting}

This concept is therefore provided to check the formal requirements to be able to sort elements. It is used by std::ranges::less, which is the default sorting criterion for sorting algorithms. That way, sorting algorithms do not fail to compile if types do not have a total order for all values. Supporting all six basic comparison operators is enough. The values to be sorted should have a total or weak order. However, this a semantic constraint that cannot be checked at compile time.

\item
Requires:
\begin{itemize}
\item
std::equality\_comparable<T> is satisfied

\item
All comparisons with the operators ==, !=, <, <=, >, and >= yield a value convertible to bool
\end{itemize}
\end{itemize}

\noindent
\hspace*{\fill} \\ %插入空行
\textbf{std::totally\_ordered\_with<T>}

\begin{itemize}
\item
Guarantees that objects of types T1 and T2 are comparable with the operators ==, !=, <, <=, >, and >= so that two values are always either equal to, or less than, or greater than each other.

\item
Requires:
\begin{itemize}
\item
==, !=, <, <=, >, and >= are supported for all comparisons where objects of T1 and/or T2 are involved and yield a value of the same type convertible to bool
\end{itemize}
\end{itemize}


\noindent
\hspace*{\fill} \\ %插入空行
\textbf{std::three\_way\_comparable<T>, std::three\_way\_comparable<T, Cat>}


\begin{itemize}
\item
Guarantees that objects of type T are comparable with the operators ==, !=, <, <=, >, >=, and the operator <=> (and have at least the comparison category type Cat). If no Cat is passed, std::partial\_ordering is required.

\item
This concept is defined in the header file <compare>.

\item
Note that this concept does not imply and subsume std::equality\_comparable because the latter requires that the operator == yields true only for two objects that are equal. With a weak or partial order, this might not be the case.

\item
Note that this concept does not imply and subsume std::totally\_ordered because the latter requires that the comparison category is std::strong\_ordering or std::weak\_ordering.

\item
Requires:
\begin{itemize}
\item
All comparisons with the operators ==, !=, <, <=, >, and >= yield a value convertible to bool

\item
Any comparison with the operator <=> yields a comparison category (which is at least Cat)
\end{itemize}
\end{itemize}

\noindent
\hspace*{\fill} \\ %插入空行
\textbf{std::three\_way\_comparable\_with<T1, T2>, 	std::three\_way\_comparable\_with<T1, T2, Cat>}

\begin{itemize}
\item
Guarantees that any two objects of types T1 and T2 are comparable with the operators ==, !=, <, <=, >, >=, and the operator <=> (and have at least the comparison category type Cat). If no Cat is passed, std::partial\_ordering is required.

\item
This concept is defined in the header file <compare>.

\item
Note that this concept does not imply and subsume std::equality\_comparable\_with because the latter requires that the operator == yields true only for two objects that are equal. With a weak or partial order, this might not be the case.

\item
Note that this concept does not imply and subsume std::totally\_ordered\_with because the latter requires that the comparison category is std::strong\_ordering or std::weak\_ordering.

\item
Requires:
\begin{itemize}
\item
std::three\_way\_comparable is satisfied for values and common references of T1 and T2 (and Cat)

\item
All comparisons with the operators ==, !=, <, <=, >, and >= yield a value convertible to bool

\item
Any comparison with the operator <=> yields a comparison category (which is at least Cat)

\item
The order of T1 and T2 does not matter
\end{itemize}
\end{itemize}




