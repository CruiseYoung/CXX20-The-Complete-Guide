
如前所述,视图是轻量级范围,可以用作构建块来处理其他范围和视图的所有或部分元素的(修改过的)值。

C++20提供了几个标准视图,可以用来将一个范围转换为一个视图,或者将一个视图转换为一个范围/视图,其中元素可以以各种方式修改:

\begin{itemize}
\item
过滤元素

\item
生成转换后的元素值

\item
修改迭代元素的顺序

\item
拆分或合并范围
\end{itemize}

此外,还有一些视图本身产生值。

对于几乎每一种视图类型,都有一个相应的定制点对象(通常是函数对象),其允许开发者通过调用函数来创建视图。若函数从传递的范围中创建视图,则这样的函数对象称为范围适配器。若函数在不传递现有范围的情况下创建视图,则称为范围工厂。大多数情况下,这些函数对象具有不带\_view后缀的视图名,但一些更通用的函数对象可能会根据传递的参数创建不同的视图。这些函数对象都定义在特殊的命名空间std::views中,这命名空间std::ranges::views的别名。

表“源视图”列出了C++ 0中从外部资源创建视图或生成值的标准视图,这些视图可以作为视图管道中的起始构建块。还可以查看哪些范围适配器或工厂可以创建它们(若有的话),若可用,更推荐使用它们,而非原始视图类型。

若没有另行指定,适配器和工厂在名称空间std::视图中可用,视图类型在名称空间std::范围中可用。std::string\_view已经在C++17中引入。所有其他视图都是在C++20中引入的,通常以\_view结尾。唯一不以\_view结尾的视图类型是std::subange和std::span。

表“适配视图”列出了C++20中处理范围和其他视图的范围适配器和标准视图,可以作为视图管道中的任何地方的构建块,包括在视图的开头。同样,更推荐使用它们。

所有视图都提供了具有恒定复杂性的移动(和可选的复制)操作(这些操作所需的时间与元素的数量无关)[原始C++20标准还要求视图具有具有恒定复杂性的默认构造函数和析构函数,但这些要求后来被\url{http://wg21.link/P2325R3}和\url{http://wg21.link/p2415r2}删除了]。概念std::ranges::视图检查相应的需求。

范围工厂/适配器all()、counts()和common()将在一个特殊的章节中描述,所有视图类型以及其他适配器和工厂的详细信息将在视图类型细节一章中描述。

\begin{longtable}[c]{|l|l|l|}
\hline
\textbf{适配器/工厂} &
\textbf{类型} &
\textbf{效果} \\ \hline
\endfirsthead
%
\endhead
%
all(rg) &
\begin{tabular}[c]{@{}l@{}}种类:\\ rg的类型\\ ref\_view\\ owning\_view\end{tabular} &
\begin{tabular}[c]{@{}l@{}}生成的范围rg是视图\\ - 若已经是视图,则返回rg\\ - 若rg是左值,则返回ref\_view\\ - 若rg是右值,则返回一个owning\_view\end{tabular} \\ \hline
counted(beg, sz) &
\begin{tabular}[c]{@{}l@{}}种类:\\ std::span\\ 子范围\end{tabular} &
\begin{tabular}[c]{@{}l@{}}从begin迭代器和计数产生一个视图\\ - 若rg是连续且常规的,则生成一个span\\ - 否则产生子范围(若有效)\end{tabular} \\ \hline
iota(val) &
iota\_view &
产生一个无限视图,其中包含以val开头的值的递增序列 \\ \hline
iota(val, endVal) &
iota\_view &
生成一个视图,其值序列从val递增到(但不包括)endVal \\ \hline
single(val) &
signle\_view &
生成一个只有val元素的视图 \\ \hline
\begin{tabular}[c]{@{}l@{}}empty\textless{}T\textgreater\\ -\end{tabular} &
\begin{tabular}[c]{@{}l@{}}empty\_view\\ basic\_istream\_view\end{tabular} &
\begin{tabular}[c]{@{}l@{}}生成类型为T的元素的空视图\\ 生成读取T类型的T个元素的视图\end{tabular} \\ \hline
\begin{tabular}[c]{@{}l@{}}istream\textless{}T\textgreater{}(s)\\ -\\ -\\ -\\ -\end{tabular} &
\begin{tabular}[c]{@{}l@{}}istream\_view\\ wistream\_view\\ std::basic\_string\_view\\ std::span\\ subrange\end{tabular} &
\begin{tabular}[c]{@{}l@{}}产生从字符流s读取Ts的视图\\ 产生一个从wchar\_t流s中读取Ts的视图\\ 生成字符数组的只读视图\\ 生成连续内存中元素的视图\\ 生成begin迭代器和哨兵的视图\end{tabular} \\ \hline
\end{longtable}

\begin{center}
表6.3 源视图
\end{center}

\begin{longtable}[c]{|l|l|l|}
\hline
\textbf{适配器} & \textbf{类型}   & \textbf{效果}                                       \\ \hline
\endfirsthead
%
\endhead
%
take(num)        & 变化          & 第一个(最多)num个元素                         \\ \hline
take\_while(pred)                    & take\_while\_view & 匹配谓词的所有首元素                        \\ \hline
drop(num)        & 变化          & 除了第一个num元素之外的所有元素                     \\ \hline
drop\_while(pred)                    & drop\_while\_view & 除首元素外的所有匹配谓词的元素                 \\ \hline
filter(pred)     & filter\_view    & 匹配谓词的所有元素                  \\ \hline
transform(func)  & transform\_view & 转换后的值的所有元素                \\ \hline
elements\textless{}idx\textgreater{} & elements\_view    & 所有元素的idxth成员/属性                         \\ \hline
keys             & elements\_view  & 所有元素的第一个成员                      \\ \hline
values           & elements\_view  & 所有元素的第二个成员                     \\ \hline
reverse          & 变化          & 所有元素按倒序排列                         \\ \hline
join             & join\_view      & 多个范围中的所有元素            \\ \hline
split(sep)       & split\_view     & 一个范围的所有元素拆分为多个范围    \\ \hline
lazy\_split(sep)                     & lazy\_split\_view & 输入或常量范围的所有元素拆分为多个范围 \\ \hline
common           & 变化          & 迭代器和哨兵中所有类型相同的元素 \\ \hline
\end{longtable}

\begin{center}
表6.4 适配器视图
\end{center}

\mySubsubsection{6.3.1}{视图的范围性}

容器和字符串不是视图,因为它们不够轻量级:没有提供低成本的复制构造函数,所以必须复制元素。

然而,可以很容易地使用容器作为视图:

\begin{itemize}
\item
通过将容器传递给范围适配器std::views::all(),可以显式地将容器转换为视图。

\item
通过将begin迭代器和end(哨兵)或大小传递给std::ranges::subrange或std::views::counts(),可以显式地将容器的元素转换为视图。

\item
可以通过将容器传递给其中一个自适应视图来隐式地将其转换为视图,这些视图通常通过将容器隐式地转换为视图来获取容器。
\end{itemize}

通常,应该使用后一种选项,有多种方法可以实现此选项。例如,有以下选项来传递一个范围coll给获取视图:

\begin{itemize}
\item
可以将范围作为参数传递给视图的构造函数:

\begin{cpp}
std::ranges::take_view first4{coll, 4};
\end{cpp}

\item
可以把这个范围作为参数传递给相应的适配器:

\begin{cpp}
auto first4 = std::views::take(coll, 4);
\end{cpp}

\item
可以通过管道将范围传入相应的适配器:

\begin{cpp}
auto first4 = coll | std::views::take(4);
\end{cpp}
\end{itemize}

任何情况下,视图first4只迭代coll的前4个元素(若元素不够,迭代次数会更少)。然而,这里发生了什么取决于coll是什么类型:

\begin{itemize}
\item
若coll已经是一个视图,take()会按原样获取视图。

\item
若coll是一个容器,take()使用一个到容器的视图,该视图是用适配器std::views::all()自动创建的。适配器会产生一个ref\_view,若容器通过名称传递(作为左值),将引用容器的所有元素。

\item
若传递了一个右值(一个临时范围,比如函数返回的容器或带有std::move()标记的容器),则该范围将被移动到owning\_view中,然后owning\_view直接保存传递类型的范围,其中包含所有移动的元素。[C++20发布\url{http://wg21.link/p2415r2}后增加了对视图临时对象(右值)的支持。]
\end{itemize}

例如:

\begin{cpp}
std::vector<std::string> coll{"just", "some", "strings", "to", "deal", "with"};

auto v1 = std::views::take(coll, 4); // iterates over a ref_view to coll

auto v2 = std::views::take(std::move(coll), 4); // iterates over an owning_view
// to a local vector<string>

auto v3 = std::views::take(v1, 2); // iterates over v1
\end{cpp}

std::views::take()会创建一个新的获取视图,最终迭代在coll中初始化的值,但结果类型和确切的行为有如下不同:

\begin{itemize}
\item
v1是take\_view<ref\_view<vector<string>{}>{}>。

将容器coll作为左值(命名对象)传递,所以获取视图通过一个ref视图迭代到容器。

\item
v2是take\_view<owning\_view<vector<string>{}>{}>.

将coll作为右值(临时对象或用std::move()标记的对象)传递,所以获取视图遍历拥有视图,该视图拥有用传递的集合初始化的字符串vector。

\item
v3是take\_view<take\_view<ref\_view<vector<string>{}>{}>{}>.

因为传递了视图v1, 获取视图遍历这个视图。最终遍历了coll(其中的元素已经被第二条语句移除了,所以第二条语句之后不需要再遍历了)。
\end{itemize}

在内部,初始化使用演绎指南和类型工具std::views::all\_t<>,后面将详细解释。

注意,这种行为允许基于范围的for循环迭代临时范围:

\begin{cpp}
for (const auto& elem : getColl() | std::views::take(5)) {
	std::cout << "- " << elem << '\n';
}

for (const auto& elem : getColl() | std::views::take(5) | std::views::drop(2)) {
	std::cout << "- " << elem << '\n';
}
\end{cpp}

值得注意的,一般来说,使用对临时对象的引用作为基于范围的for循环迭代的集合是一个致命的运行时错误(这是一个C++标准委员会多年来一直不愿意修复的错误,参见\url{http://wg21.link/p2012})。由于传递临时范围对象(rvalue)将范围移动到owning\_view中,因此视图不会引用外部容器,从而不会出现运行时错误。

\mySubsubsection{6.3.2}{惰性计算}

理解什么时候处理视图是很重要的。视图在定义后不会开始处理,其会按需运行:

\begin{itemize}
\item
若需要视图的下一个元素,则通过执行必要的迭代来计算其是哪一个。

\item
若需要一个视图元素的值,测通过执行定义的转换来计算其值。
\end{itemize}

考虑下面的代码:

\filename{ranges/filttrans.cpp}

\begin{cpp}
#include <iostream>
#include <vector>
#include <ranges>
namespace vws = std::views;

int main()
{
	std::vector<int> coll{ 8, 15, 7, 0, 9 };
	
	// define a view:
	auto vColl = coll
				| vws::filter([] (int i) {
					std::cout << " filter " << i << '\n';
					return i % 3 == 0;
				})
				| vws::transform([] (int i) {
					std::cout << " trans " << i << '\n';
					return -i;
				});
				
	// and use it:
	std::cout << "*** coll | filter | transform:\n";
	for (int val : vColl) {
		std::cout << "val: " << val << "\n\n";
	}
}
\end{cpp}

这里定义了一个视图vColl,只过滤和转换范围coll的元素:

\begin{itemize}
\item
使用std::views::filter(),只处理那些是3的倍数的元素。

\item
使用std::views::transform(),对每个值求反。
\end{itemize}

该程序有以下输出:

\begin{shell}
*** coll | filter | transform:
filter 8
filter 15
trans 15
val: -15

filter 7
filter 0
trans 0

val: 0
filter 9
trans 9
val: -9
\end{shell}

首先,在定义视图vColl时或之后,都不会调用filter()和transform()。当使用视图时,处理就开始了(这里:迭代vColl)。视图使用延迟求值,只是对处理的描述,当需要下一个元素或值时才进行处理。

假设对vColl进行更多的手动迭代,调用begin()和++来获取下一个值,使用操作符*来获取它的值:

\begin{cpp}
std::cout << "pos = vColl.begin():\n";
auto pos = vColl.begin();
std::cout << "*pos:\n";
auto val = *pos;
std::cout << "val: " << val << "\n\n";

std::cout << "++pos:\n";
++pos;
std::cout << "*pos:\n";
val = *pos;
std::cout << "val: " << val << "\n\n";
\end{cpp}

对于这段代码,可以得到以下输出:

\begin{shell}
pos = vColl.begin():
filter 8
filter 15
*pos:
trans 15
val: -15

++pos:
filter 7
filter 0
*pos:
trans 0
val: 0
\end{shell}

一步一步来看看会发生什么:

\begin{itemize}
\item
当调用begin()时,会发生以下情况:

\begin{itemize}
\item
获取vColl的第一个元素的请求被传递给转换视图,转换视图将其传递给过滤器视图,过滤器视图将其传递给coll,后者生成迭代器的第一个元素8。

\item
过滤器视图查看第一个元素的值并拒绝它,并通过调用++向coll请求下一个元素。过滤器获取第二个元素的位置15,并将其位置传递给变换视图。

\item
因此,pos初始化为指向第二个元素的位置的迭代器。
\end{itemize}

\item
当调用*pos时,会发生以下情况:

\begin{itemize}
\item
因为需要这个值,所以现在为当前元素调用转换视图,并生成其负值。

\item
用当前元素的负值初始化val。
\end{itemize}

\item
当调用++pos时,同样的情况再次发生:

\begin{itemize}
\item
获取下一个元素的请求被传递给过滤器,过滤器将请求传递给coll,直到找到合适的元素(或者到达coll的末尾)。

\item
因此,pos获取第四个元素的位置。
\end{itemize}

\item
通过再次调用*pos,执行转换并为循环生成下一个值。
\end{itemize}

这个迭代会一直持续下去,直到范围或其中一个视图表示已经到达终点。

这种pull模型有一个很大的好处:不处理不需要的元素。例如,使用视图来查找第一个结果值为0:

\begin{cpp}
std::ranges::find(vColl, 0);
\end{cpp}

那么输出将只有:

\begin{shell}
filter 8
filter 15
trans 15
filter 7
filter 0
trans 0
\end{shell}

pull模型的另一个好处是视图的序列或管道甚至可以在无限范围内操作。不会在不知道使用了多少的情况下计算无限个值,从而根据视图的用户请求计算尽可能多的值。

\mySubsubsection{6.3.3}{缓存视图}

假设想要多次迭代一个视图。若一次又一次地计算第一个有效元素,似乎是对性能的浪费。实际上,跳过的元素视图会在调用begin()后进行缓存。

修改上面的程序,使其在视图vColl的元素上迭代两次:

\filename{ranges/filttrans2.cpp}

\begin{cpp}
#include <iostream>
#include <vector>
#include <ranges>
namespace vws = std::views;

int main()
{
	std::vector<int> coll{ 8, 15, 7, 0, 9 };
	// define a view:
	auto vColl = coll
	| vws::filter([] (int i) {
		std::cout << " filter " << i << '\n';
		return i % 3 == 0;
	})
	| vws::transform([] (int i) {
		std::cout << " trans " << i << '\n';
		return -i;
	});
	
	// and use it:
	std::cout << "*** coll | filter | transform:\n";
	for (int val : vColl) {
		...
	}
	std::cout << "-------------------\n";
	
	// and use it again:
	std::cout << "*** coll | filter | transform:\n";
	for (int val : vColl) {
		std::cout << "val: " << val << "\n\n";
	}
}
\end{cpp}

输出如下所示:

\begin{shell}
*** coll | filter | transform:
filter 8
filter 15
trans 15
filter 7
filter 0
trans 0
filter 9
trans 9
-------------------
*** coll | filter | transform:
trans 15
val: -15

filter 7
filter 0
trans 0
val: 0

filter 9
trans 9
val: -9
\end{shell}

在第二次使用vcol .begin()时,不再尝试查找第一个元素,因为与过滤器元素的第一次迭代一起进行了缓存。

注意,begin()的缓存有好有坏,可能还有意想不到的后果。首先,最好初始化一次缓存视图并使用它两次:

\begin{cpp}
// better:
auto v1 = coll | std::views::drop(5);
check(v1);
process(v1);
\end{cpp}

然后初始化并使用两次:

\begin{cpp}
// worse:
check(coll | std::views::drop(5));
process(coll | std::views::drop(5));
\end{cpp}

此外,修改range的前导元素(改变它们的值或插入/删除元素)可能会使视图失效,当且仅当修改之前已经调用了begin()。

这意味着:

\begin{itemize}
\item
若在修改之前不调用begin(),这个视图通常是有效的,以后使用时也能正常工作:

\begin{cpp}
std::list coll{1, 2, 3, 4, 5};
auto v = coll | std::views::drop(2);
coll.push_front(0); // coll is now: 0 1 2 3 4 5
print(v); // initializes begin() with 2 and prints: 2 3 4 5
\end{cpp}

\item
但是,若在修改之前调用begin()(例如,打印元素),就很容易得到错误的元素。例如:

\begin{cpp}
std::list coll{1, 2, 3, 4, 5};
auto v = coll | std::views::drop(2);
print(v); // init begin() with 3
coll.push_front(0); // coll is now: 0 1 2 3 4 5
print(v); // begin() is still 3, so prints: 3 4 5
\end{cpp}

这里,begin()缓存为一个迭代器,若向范围中添加或删除新元素,视图就不再对基础范围中的所有元素进行操作。

也可能得到无效的值。例如:

\begin{cpp}
std::vector vec{1, 2, 3, 4};

auto biggerThan2 = [](auto v){ return v > 2; };
auto vVec = vec | std::views::filter(biggerThan2);

print(vVec); // OK: 3 4

++vec[1];
vec[2] = 0; // vec becomes 1 3 0 4

print(vVec); // OOPS: 0 4;
\end{cpp}

\end{itemize}

注意,这是一个迭代,即使它只是读,也可以算作写访问,若视图的引用范围在此期间被修改,那么迭代视图的元素可能会使以后的使用无效。

其效果取决于何时,以及如何进行缓存。请参阅关于缓存视图特定部分的注释:

\begin{itemize}
\item
Filter视图,将begin()缓存为迭代器或偏移量

\item
Drop视图,将begin()缓存为迭代器或偏移量

\item
Drop-while视图,将begin()缓存为迭代器或偏移量

\item
反向(Reverse)视图,将begin()缓存为迭代器或偏移量
\end{itemize}

这里,看看C++对性能的关心:

\begin{itemize}
\item
若根本不遍历视图的元素,那么在初始化时进行缓存将会带来不必要的性能开销。

\item
若在视图的元素上迭代一秒或更多次,完全不缓存将会带来不必要的性能成本(某些情况下,在drop-while视图上应用反向视图甚至可能具有二次复杂度)。
\end{itemize}

由于缓存,使用非临时视图可能会产生非常令人惊讶的结果。所以,在修改视图使用的范围时必须小心。

另一个结果是,缓存可能要求视图在遍历其元素时不能是const。其后果更为严重,稍后再讨论。

\mySubsubsection{6.3.4}{过滤器的性能问题}

The pull model also has its drawbacks. To demonstrate that, let us change the order of the two views involved (above) so that we first call transform() and then filter():

\filename{ranges/transfilt.cpp}

\begin{cpp}
#include <iostream>
#include <vector>
#include <ranges>
namespace vws = std::views;

int main()
{
	std::vector<int> coll{ 8, 15, 7, 0, 9 };
	
	// define a view:
	auto vColl = coll
	| vws::transform([] (int i) {
		std::cout << " trans: " << i << '\n';
		return -i;
	})
	| vws::filter([] (int i) {
		std::cout << " filt: " << i << '\n';
		return i % 3 == 0;
	});
	
	// and use it:
	std::cout << "*** coll | transform | filter:\n";
	for (int val : vColl) {
	std::cout << "val: " << val << "\n\n";
	}
}
\end{cpp}

Now, the output of the program is as follows:

\begin{shell}
*** coll | transform | filter:
trans: 8
filt: -8
trans: 15
filt: -15
trans: 15
val: -15

trans: 7
filt: -7
trans: 0
filt: 0
trans: 0
val: 0

trans: 9
filt: -9
trans: 9
val: -9
\end{shell}

We have additional calls of the transform view:

\begin{itemize}
\item
We now call transform() for each element.

\item
For elements that pass the filter, we even perform the transformation twice.
\end{itemize}

Calling the transformation for each element is necessary because the filter view comes afterwards and now looks at the transformed values. In this case, the negation does not impact the filter; therefore, putting it to the front looks better.

However, why do we call the transformations for elements that pass the filter twice? The reason lies in the nature of a pipeline that uses the pull model and the fact that we use iterators.

\begin{itemize}
\item
First, the filter needs the transformed value to check it; therefore, the leading transformation has to be performed before the filter can use the resulting value.
	
\item
Remember that ranges and views iterate in two steps over the elements of a range: they first compute the position/iterator (with begin() and ++), and then as a separate step, use * to get the value. This means that for each filter, we have to perform all previous transformations once to be able check the value of the element. However, if true, the filter delivers only the position of the element, not its value. Therefore, when a user of the filter needs the value, it has to perform the transformations once again.
\end{itemize}

In fact, each filter adds one more call of all leading transformations on each element that pass the filter and for which the value is used afterwards.

Given the following pipeline of transformations t1, t2, t3 and filters f1, f2:

\begin{cpp}
t1 | t2 | f1 | t3 | f2
\end{cpp}

we have the following behavior:

\begin{itemize}
\item
For elements where f1 yields false, we call:

\begin{cpp}
t1 t2 f1
\end{cpp}

\item
For elements where f1 yields true but f2 yields false, we call:

\begin{cpp}
t1 t2 f1 t1 t2 t3 f2
\end{cpp}

\item
For elements where f1 and f2 yield true, we call:

\begin{cpp}
t1 t2 f1 t1 t2 t3 f2 t1 t2 t3
\end{cpp}
\end{itemize}

See ranges/viewscalls.cpp for a complete example.

If this makes you worry about the performance of pipelines, take the following into account: yes, you should avoid expensive transformations before you use a filter. However, remember that all views provide functionality that is usually optimized in such a way that only the expressions inside the transformations and filters remain. In trivial cases such as the one here, where a filter checks for a multiple of 3 and the transformation negates, the resulting difference in behavior is effectively like calling

\begin{cpp}
if (-x % 3 == 0) return -x; // first transformation then filter
\end{cpp}

instead of

\begin{cpp}
if (x % 3 == 0) return -x; // first filter then transformation
\end{cpp}

For more details about filters, see the section about the filter view.
