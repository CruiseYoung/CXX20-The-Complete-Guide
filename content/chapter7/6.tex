
With the support to pass ranges as one argument and be able to process sentinels (end iterators) with a different type, new ways of calling algorithms are now available in C++20.

However, they come with some restrictions and not all algorithms support passing ranges as a whole yet.

\subsubsection*{\zihao{3} 7.6.1\hspace{0.2cm}Benefits and Restrictions for Range Algorithms}
\addcontentsline{toc}{subsubsection}{7.6.1\hspace{0.2cm}Benefits and Restrictions for Range Algorithms}

For some algorithms, there are no range versions that allow programmers to pass ranges as a single argument yet:

\begin{itemize}
\item
Range algorithms do not support parallel execution (introduced with C++17). There are no APIs that take range parameters as single objects and an execution policy.

\item
There are no numeric algorithms for single-object ranges yet. To call algorithms like std::accumulate(), you still have to pass both the begin and the end of the range. C++23 will probably provide numeric range algorithms.
\end{itemize}

If there is range support for algorithms, they use concepts to find possible errors at compile time:

\begin{itemize}
\item
Concepts for iterators and ranges ensure that you pass valid iterators and ranges.

\item
Concepts for callables ensure that you pass valid helper functions.
\end{itemize}

In addition, return types may differ because:

\begin{itemize}
\item
They support and might return iterators of different types, for which special return types are defined. These types are listed in table New return types for range algorithms.

\item
They might return borrowed iterators, signaling that the iterator is no valid iterator because a temporary range (rvalue) was passed.
\end{itemize}


\begin{table}[H]
\centering
\begin{tabular}{|l|l|l|}
	\hline
	\textbf{Type} &
	\textbf{Meaning} &
	\textbf{Members} \\ \hline
	\begin{tabular}[c]{@{}l@{}}std::ranges::in\_in\_result\\ std::ranges::in\_out\_result\\ std::ranges::in\_in\_out\_result\\ std::ranges::in\_out\_out\_result\\ std::ranges::in\_fun\_result\\ std::ranges::min\_max\_result\\ std::ranges::in\_found\_result\end{tabular} &
	\begin{tabular}[c]{@{}l@{}}For the positions of two input ranges\\ For one position of an input range and one position of an output range\\ For the positions of two input ranges and one position of an output range\\ For one position of an input range and the position of two output ranges\\ For one position of an input range and a function\\ For one maximum and one minimum position/value\\ For one position of an input range and a Boolean value\end{tabular} &
	\begin{tabular}[c]{@{}l@{}}in1, in2\\ in, out\\ in1, in2, out\\ in, out1, out2\\ in, out\\ min, max\\ in, found\end{tabular} \\ \hline
\end{tabular}
\end{table}

\begin{center}
Table 7.11. New return types for range algorithms
\end{center}

The following program demonstrates how to use these return types:

\noindent
\hspace*{\fill} \\ %插入空行
\textit{ranges/results.cpp}

\begin{lstlisting}[style=styleCXX]
#include <iostream>
#include <string_view>
#include <vector>
#include <algorithm>

void print(std::string_view msg, auto beg, auto end)
{
	std::cout << msg;
	for(auto pos = beg; pos != end; ++pos) {
		std::cout << ' ' << *pos;
	}
	std::cout << '\n';
}

int main()
{
	std::vector inColl{1, 2, 3, 4, 5, 6, 7};
	std::vector outColl{1, 2, 3, 4, 5, 6, 7, 8, 9, 10};
	
	auto result = std::ranges::transform(inColl, outColl.begin(),
	[] (auto val) {
		return val*val;
	});
	
	print("processed in:", inColl.begin(), result.in);
	print("rest of in: ", result.in, inColl.end());
	print("written out: ", outColl.begin(), result.out);
	print("rest of out: ", result.out, outColl.end());
}
\end{lstlisting}

The program has the following output:

{\footnotesize
\begin{tcblisting}{commandshell={}}
processed in: 1 2 3 4 5 6 7
rest of in:
written out: 1 4 9 16 25 36 49
rest of out: 8 9 10
\end{tcblisting}
}

\subsubsection*{\zihao{3} 7.6.2\hspace{0.2cm}Algorithm Overview}
\addcontentsline{toc}{subsubsection}{7.6.2\hspace{0.2cm}Algorithm Overview}

The situation of all algorithms becomes more and more complex. Some of them can be used in parallel, some not. Some of them are supported by the ranges library, some not. This section gives an overview of which algorithms are available in which form (without going into detail about what each algorithm does).

The last three columns have the following meaning:

\begin{itemize}
\item
Ranges signals whether the algorithm is supported by the range library

\item
\_result signals whether and which \_result types are used (e.g., in\_out stands for in\_out\_result)

\item
Borrowed signals whether the algorithm returns borrowed iterators
\end{itemize}

Table Non-modifying algorithms lists the available non-modifying standard algorithms of C++20.

\begin{table}[H]
\centering
\begin{tabular}{|l|l|l|l|l|l|}
	\hline
	\textbf{Name}                          & \textbf{Since} & \textbf{Parallel} & \textbf{Ranges} & \textbf{\_result} & \textbf{Borrowed} \\ \hline
	for\_each()                & C++98 & yes & yes       & in\_fun  & yes \\ \hline
	for\_each\_n()             & C++17 & yes & yes       & in\_fun  &     \\ \hline
	count()                    & C++98 & yes & yes       &          &     \\ \hline
	count\_if()                & C++98 & yes & yes       &          &     \\ \hline
	min\_element()             & C++98 & yes & yes       & yes      &     \\ \hline
	max\_element()             & C++98 & yes & yes       & yes      &     \\ \hline
	minmax\_element()          & C++11 & yes & yes       & min\_max & yes \\ \hline
	min()                      & C++20 & no  & yes(only) &          &     \\ \hline
	max()                      & C++20 & no  & yes(only) &          &     \\ \hline
	minmax()                   & C++20 & no  & yes(only) & min\_max &     \\ \hline
	find()                     & C++98 & yes & yes       &          & yes \\ \hline
	find\_if()                 & C++98 & yes & yes       &          & yes \\ \hline
	find\_if\_not()            & C++11 & yes & yes       &          & yes \\ \hline
	search()                   & C++98 & yes & yes       &          & yes \\ \hline
	search\_n()                & C++98 & yes & yes       &          & yes \\ \hline
	find\_end()                & C++98 & yes & yes       &          & yes \\ \hline
	find\_first\_of()          & C++98 & yes & yes       &          & yes \\ \hline
	adjacent\_find()           & C++98 & yes & yes       &          & yes \\ \hline
	equal()                    & C++98 & yes & yes       &          & yes \\ \hline
	is\_permutation()          & C++11 & no  & yes       &          &     \\ \hline
	mismatch()                 & C++98 & yes & yes       & in\_in   & yes \\ \hline
	lexicographical\_compare() & C++98 & yes & yes       &          &     \\ \hline
	lexicographical\_compare\_three\_way() & C++20          & no                & no              &                   &                   \\ \hline
	is\_sorted()               & C++11 & yes & yes       &          &     \\ \hline
	is\_sorted\_until()        & C++11 & yes & yes       &          & yes \\ \hline
	is\_partitioned()          & C++11 & yes & yes       &          &     \\ \hline
	partition\_point()         & C++11 & no  & yes       &          &     \\ \hline
	is\_heap()                 & C++11 & yes & yes       &          &     \\ \hline
	is\_heap\_until()          & C++11 & yes & yes       &          & yes \\ \hline
	all\_of()                  & C++11 & yes & yes       &          &     \\ \hline
	any\_of()                  & C++11 & yes & yes       &          &     \\ \hline
	none\_of()                 & C++11 & yes & yes       &          &     \\ \hline
\end{tabular}
\end{table}

\begin{center}
Table 7.12. Non-modifying algorithms
\end{center}

Note that the algorithms std::ranges::min(), std::ranges::max(), and std::ranges::minmax() have only counterparts in std that take two values or a std::initializer\_list<>.

Table Modifying algorithms lists the available modifying standard algorithms in C++20.

\begin{table}[H]
\centering
\begin{tabular}{|l|l|l|l|l|l|}
	\hline
	\textbf{Name}              & \textbf{Since} & \textbf{Parallel} & \textbf{Ranges} & \textbf{\_result} & \textbf{Borrowed} \\ \hline
	for\_each()         & C++98 & yes & yes & in\_fun & yes \\ \hline
	for\_each\_n()      & C++17 & yes & yes & in\_fun &     \\ \hline
	copy()              & C++98 & yes & yes & in\_out & yes \\ \hline
	copy\_if()          & C++11 & yes & yes & in\_out &     \\ \hline
	copy\_n()           & C++11 & yes & yes & in\_out & yes \\ \hline
	copy\_backward()    & C++11 & no  & yes & in\_out & yes \\ \hline
	move()              & C++11 & yes & yes & in\_out & yes \\ \hline
	move\_backward()    & C++11 & no  & yes & in\_out & yes \\ \hline
	transform() for one range  & C++98          & yes               & yes             & in\_out           & yes               \\ \hline
	transform() for two ranges & C++98          & yes               & yes             & in\_in\_out       & yes               \\ \hline
	merge()             & C++98 & yes & yes &         &     \\ \hline
	swap\_ranges()      & C++98 & yes & yes & in\_in  & yes \\ \hline
	fill()              & C++98 & yes & yes &         & yes \\ \hline
	fill\_n()           & C++98 & yes & yes &         &     \\ \hline
	generate()          & C++98 & yes & yes &         & yes \\ \hline
	generate\_n()       & C++98 & yes & yes &         &     \\ \hline
	iota()              & C++11 & no  & no  &         &     \\ \hline
	replace()           & C++98 & yes & yes &         & yes \\ \hline
	replace\_if()       & C++98 & yes & yes &         & yes \\ \hline
	replcae\_copy()     & C++98 & yes & yes & in\_out & yes \\ \hline
	replace\_copy\_if() & C++98 & yes & yes & in\_out & yes \\ \hline
\end{tabular}
\end{table}

\begin{center}
Table 7.13. Modifying algorithms
\end{center}

Table Removing algorithms lists the available “removing” standard algorithms in C++20. Note that algorithms never really remove elements. This still applies when they take whole ranges as single arguments. They change the order so that the elements not removed are packed to the front and return the new end.


\begin{table}[H]
\centering
\begin{tabular}{|l|l|l|l|l|l|}
	\hline
	\textbf{Name} & \textbf{Since} & \textbf{Parallel} & \textbf{Ranges} & \textbf{\_result} & \textbf{Borrowed} \\ \hline
	remove()           & C++98 & yes & yes &         & yes \\ \hline
	remove\_if()       & C++98 & yes & yes &         & yes \\ \hline
	remove\_copy()     & C++98 & yes & yes & in\_out & yes \\ \hline
	remove\_copy\_if() & C++98 & yes & yes & in\_out & yes \\ \hline
	unique()           & C++98 & yes & yes &         & yes \\ \hline
	unique\_copy()     & C++98 & yes & yes & in\_out & yes \\ \hline
\end{tabular}
\end{table}

\begin{center}
Table 7.14. Removing algorithms
\end{center}

Table Mutating algorithms lists the available mutating standard algorithms in C++20.



\begin{table}[H]
\centering
\begin{tabular}{|l|l|l|l|l|l|}
	\hline
	\textbf{Name} & \textbf{Since} & \textbf{Parallel} & \textbf{Ranges} & \textbf{\_result} & \textbf{Borrowed} \\ \hline
	reverse()           & C++98 & yes & yes &           & yes \\ \hline
	reverse\_copy()     & C++98 & yes & yes & in\_out   & yes \\ \hline
	rotate()            & C++98 & yes & yes &           & yes \\ \hline
	rotate\_copy()      & C++98 & yes & yes & in\_out   & yes \\ \hline
	shift\_left()       & C++20 & yes & no  &           &     \\ \hline
	shift\_right()      & C++20 & yes & no  &           &     \\ \hline
	sample()            & C++17 & no  & yes &           &     \\ \hline
	next\_permutation() & C++98 & no  & yes & in\_found & yes \\ \hline
	prev\_permutation() & C++98 & no  & yes & in\_found & yes \\ \hline
	shuffle()           & C++11 & no  & yes &           & yes \\ \hline
	random\_shuffle()   & C++98 & no  & no  &           &     \\ \hline
	partition()         & C++98 & yes & yes &           &     \\ \hline
	stable\_partition() & C++98 & yes & yes &           &     \\ \hline
	partition\_copy()   & C++11 & yes & yes &           &     \\ \hline
\end{tabular}
\end{table}

\begin{center}
Table 7.15. Mutating algorithms
\end{center}

Table Sorting algorithms lists the available sorting standard algorithms in C++20.


\begin{table}[H]
\centering
\begin{tabular}{|l|l|l|l|l|l|}
	\hline
	\textbf{Name} & \textbf{Since} & \textbf{Parallel} & \textbf{Ranges} & \textbf{\_result} & \textbf{Borrowed} \\ \hline
	sort()                & C++98 & yes & yes &              & yes \\ \hline
	stable\_sort()        & C++98 & yes & yes &              & yes \\ \hline
	partial\_sort()       & C++98 & yes & yes &              & yes \\ \hline
	partial\_sort\_copy() & C++98 & yes & yes & in\_out      & yes \\ \hline
	nth\_element()        & C++98 & yes & yes &              & yes \\ \hline
	partition()           & C++98 & yes & yes &              & yes \\ \hline
	stable\_partition()   & C++98 & yes & yes &              & yes \\ \hline
	partition\_copy()     & C++11 & yes & yes & in\_out\_out & yes \\ \hline
	make\_heap()          & C++98 & no  & yes &              & yes \\ \hline
	push\_heap()          & C++98 & no  & yes &              & yes \\ \hline
	pop\_heap()           & C++98 & no  & yes &              & yes \\ \hline
	sort\_heap()          & C++98 & no  & yes &              & yes \\ \hline
\end{tabular}
\end{table}

\begin{center}
Table 7.16. Sorting algorithms
\end{center}

Table Algorithms for sorted ranges lists the available standard algorithms for sorted ranges in C++20.

\begin{table}[H]
\centering
\begin{tabular}{|l|l|l|l|l|l|}
	\hline
	\textbf{Name}                & \textbf{Since} & \textbf{Parallel} & \textbf{Ranges} & \textbf{\_result} & \textbf{Borrowed} \\ \hline
	binary\_search()    & C++98 & no  & yes &             &     \\ \hline
	includes()          & C++98 & yes & yes &             &     \\ \hline
	lower\_bound()      & C++98 & no  & yes &             & yes \\ \hline
	upper\_bound()      & C++98 & no  & yes &             & yes \\ \hline
	equal\_range()      & C++98 & no  & yes &             & yes \\ \hline
	merge()             & C++98 & yes & yes & in\_in\_out & yes \\ \hline
	inplace\_merge()    & C++98 & yes & yes &             & yes \\ \hline
	set\_union()        & C++98 & yes & yes & in\_in\_out & yes \\ \hline
	set\_intersection() & C++98 & yes & yes & in\_in\_out & yes \\ \hline
	set\_difference()   & C++98 & yes & yes & in\_out     & yes \\ \hline
	set\_symmetric\_difference() & C++98          & yes               & yes             & in\_in\_out       & yes               \\ \hline
	partition\_point()  & C++11 & no  & yes &             & yes \\ \hline
\end{tabular}
\end{table}


\begin{center}
Table 7.17. Algorithms for sorted ranges
\end{center}

Table Numeric algorithms lists the available numeric standard algorithms in C++20.


\begin{table}[H]
\centering
\begin{tabular}{|l|l|l|l|l|l|}
\hline
\textbf{Name} & \textbf{Since} & \textbf{Parallel} & \textbf{Ranges} & \textbf{\_result} & \textbf{Borrowed} \\ \hline
accumulate()                 & C++98 & no  & no &  &  \\ \hline
reduce()                     & C++17 & yes & no &  &  \\ \hline
transform\_reduce()          & C++17 & yes & no &  &  \\ \hline
inner\_product()             & C++98 & no  & no &  &  \\ \hline
adjacent\_difference()       & C++98 & no  & no &  &  \\ \hline
partial\_sum()               & C++98 & no  & no &  &  \\ \hline
inclusive\_scan()            & C++17 & yes & no &  &  \\ \hline
exclusive\_scan()            & C++17 & yes & no &  &  \\ \hline
transform\_inclusive\_scan() & C++17 & yes & no &  &  \\ \hline
transform\_exclusive\_scan() & C++17 & yes & no &  &  \\ \hline
iota                         & C++11 & no  & no &  &  \\ \hline
\end{tabular}
\end{table}

\begin{center}
Table 7.18. Numeric algorithms
\end{center}



