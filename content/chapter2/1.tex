C++14起,Lambda可以使用auto等占位符来声明/定义其参数:

\begin{cpp}
auto printColl = [] (const auto& coll) { // generic lambda
	for (const auto& elem : coll) {
		std::cout << elem << '\n';
	}
}
\end{cpp}

只要支持Lambda内部的操作,占位符允许传递任何类型的参数:

\begin{cpp}
std::vector coll{1, 2, 4, 5};
...
printColl(coll); // compiles the lambda for vector<int>

printColl(std::string{"hello"}); // compiles the lambda for std::string
\end{cpp}

C++20起,可以对所有函数(包括成员函数和操作符)使用auto等占位符:

\begin{cpp}
void printColl(const auto& coll) // generic function
{
	for (const auto& elem : coll) {
		std::cout << elem << '\n';
	}
}
\end{cpp}

声明只是声明模板的快捷方式:

\begin{cpp}
template<typename T>
void printColl(const T& coll) // equivalent generic function
{
	for (const auto& elem : coll) {
		std::cout << elem << '\n';
	}
}
\end{cpp}

使用auto替代模板参数T,所以该特性也称为“函数模板化”的语法。

因为带有auto的函数是函数模板,所以适用于函数模板的所有规则。从而不能在一个翻译单元(CPP文件)中实现带有自动参数的函数,并在另一个翻译单元中调用。对于具有auto参数的函数,实现存在于头文件,以便可以在多个CPP文件中使用(否则,必须在一个翻译单元中显式地实例化函数)。另一方面,因为函数模板总是内联,所以不需要声明为内联。

另外,可以显式地指定模板参数:

\begin{cpp}
void print(auto val)
{
	std::cout << val << '\n';
}

print(64); // val has type int
print<char>(64); // val has type char
\end{cpp}

\mySubsubsection{2.1.1}{auto用于成员函数的参数}

可以使用这个特性来定义成员函数:

\begin{cpp}
class MyType {
	...
	void assign(const auto& newVal);
};
\end{cpp}

该声明等价于(不同之处在于没有定义类型T):

\begin{cpp}
class MyType {
	...
	template<typename T>
	void assign(const T& newVal);
};
\end{cpp}

注意,模板不能在函数内部声明。对于使用auto形参的成员函数,不能再在函数内部局部定义类或数据结构:

\begin{cpp}
void foo()
{
	struct Data {
		void mem(auto); // ERROR can’t declare templates inside functions
	};
}
\end{cpp}

有关带有auto的成员==操作符的示例,请参阅sentinel1.cpp。
