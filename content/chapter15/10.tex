所有示例中,协程接口都是由协程返回的。但也可以实现协程,以便它们使用在其他地方创建的接口。考虑以下示例(第一个协程示例的修改版本):

\begin{cpp}
void coro(int max, CoroTask&)
{
	std::cout << "CORO " << max << " start\n";

	for (int val = 1; val <= max; ++val) {
		// print next value:
		std::cout << "CORO " << val << '/' << max << '\n';

		co_await std::suspend_always{}; // SUSPEND
	}

	std::cout << "CORO " << max << " end\n";
}
\end{cpp}

协程将CoroTask类型的协程接口作为参数,但必须指定将参数用作协程接口。通过模板std::coroutine\_traits<>的特化完成,模板参数必须指定签名(返回类型和参数类型),并且在内部,必须将类型成员promise\_type映射到协程接口参数的类型成员:

\begin{cpp}
template<>
struct std::coroutine_traits<void, int, CoroTask&>
{
	using promise_type = CoroTask::promise_type;
};
\end{cpp}

现在唯一需要确保的是,可以在没有协程的情况下创建协程接口,并在以后引用协程。为此,需要一个带有构造函数的promise,该构造函数接受与协程相同的参数

\begin{cpp}
class CoroTask {
	public:
	// required type for customization:
	struct promise_type {
		promise_type(int, CoroTask& ct) { // init passed coroutine interface
			ct.hdl = CoroHdl::from_promise(*this);
		}
		void get_return_object() { // nothing to do anymore
		}
		...
	};

private:
	// handle to allocate state (can be private):
	using CoroHdl = std::coroutine_handle<promise_type>;
	CoroHdl hdl;

public:
	// constructor and destructor:
	CoroTask() : hdl{} { // enable coroutine interface without handle
	}
	...
};
\end{cpp}

现在使用协程的代码可能如下所示:

\begin{cpp}
CoroTask coroTask; // create coroutine interface without coroutine

// start coroutine:
coro(3, coroTask); // init and store coroutine in the interface created

// loop to resume the coroutine until it is done:
while (coroTask.resume()) { // resume
	std::this_thread::sleep_for(500ms);
}
\end{cpp}

可以在coro/corotraits.cpp中找到完整的示例。

promise构造函数获取传递给协程的所有参数,由于promise通常只需要协程接口,可以将接口作为第一个参数传递,然后使用auto\&\&…作为最后一个参数。










