
C++20 provides a couple of improvements for aggregates. For generic code we now have:

\begin{itemize}
\item 
Using class template argument deduction (CTAD) for aggregates

\item 
Aggregates can be used as non-type template parameters (NTTP)
\end{itemize}

The former is described in this section.

Note that there are other new features for aggregates:

\begin{itemize}
\item 
Designated initializers (initial values for specific members) are (partially) supported

\item 
You can initialize aggregates with parentheses

\item 
The fixed definition of aggregates and the consequence for std::is\_default\_constructible<>
\end{itemize}


\mySubsubsection{22.2.1}{Class Template Argument Deduction (CTAD) for Aggregates}

Since C++17, constructors can be used to deduce template parameters of class templates. For example:

\begin{cpp}
template<typename T>
class Type {
	T value;
public:
	Type(T val)
	: value{val} {
	}
	...
};

Type<int> t1{42};
Type t2{42}; // deduces Type<int> since C++17
\end{cpp}

However, even for simple aggregates, a similar deduction according to the way objects are initialized was not supported:

\begin{cpp}
template<typename T>
struct Aggr {
	T value;
};

Aggr<int> a1{42}; // OK
Aggr a2{42}; // ERROR before C++20
\end{cpp}

You had to provide a deduction guide:

\begin{cpp}
template<typename T>
struct Aggr {
	T value;
};

template<typename T>
Aggr(T) -> Aggr<T>;

Aggr<int> a1{42}; // OK
Aggr a2{42}; // OK since C++17
\end{cpp}

Since C++20, there is no longer any need for a deduction guide, meaning that the following is enough:

\begin{cpp}
template<typename T>
struct Aggr {
	T value;
};

Aggr<int> a1{42}; // OK
Aggr a2{42}; // OK since C++20 even without deduction guide
\end{cpp}

Note that this feature also works when using parentheses to initialize aggregates:

\begin{cpp}
Aggr a3(42); // OK since C++20 even without deduction guide
\end{cpp}

The deduction rules can be subtle, as the following example demonstrates:

\begin{cpp}
template<typename T>
struct S {
	T x;
	T y;
};

template<typename T>
struct C {
	S<T> s;
	T x;
	T y;
};

C c1 = {{1, 2}, 3, 4}; // OK, C<int> deduced
C c2 = {{1, 2}, 3}; // OK, C<int> deduced (y is 0)
C c3 = {{1, 2}, 3.3, 4.4}; // OK, C<double> deduced

C c4 = {{1, 2}, 3, 4.4}; // ERROR: int and double deduced for T
C c5 = {{1, 2}}; // ERROR: T only indirectly deduced
C c6 = {1, 2, 3}; // ERROR: don’t know how many values S<T> needs
C<int> c7 = {1, 2, 3}; // OK
\end{cpp}

Note that class template argument deduction even works for aggregates with a variadic number of elements:

\begin{cpp}
// aggregate with variadic number of base types:
template<typename... T>
struct C : T... {
};

struct Base1 {
};
struct Base2 {
};

// aggregate initialized with two elements of types Base1 and Base2:
C c1{Base1{}, Base2{}};

// aggregate initialized with three lambda elements:
C c2{[] { f1(); },
	[] { f2(); },
	[] { f3(); }
};
\end{cpp}
















