
之前已经介绍了awaiter,让我们更详细的对其进行了解。

\mySubsubsection{15.7.1}{awaiter接口的详细信息}

“awaiter的特殊成员函数”表再次列出了awaiter必须提供的关键操作。

% Please add the following required packages to your document preamble:
% \usepackage{longtable}
% Note: It may be necessary to compile the document several times to get a multi-page table to line up properly
\begin{longtable}[c]{|l|l|}
\hline
\textbf{操作}                & \textbf{效果}                                           \\ \hline
\endfirsthead
%
\endhead
%
Constructor              & 初始化awaiter                          \\ \hline
await\_ready()           & 是否(当前)禁用挂起 \\ \hline
await\_suspend(awaitHdl) & 处理挂起                                \\ \hline
await\_resume()          & 处理恢复                               \\ \hline
\end{longtable}

\begin{center}
表15.3 awaiter的特殊成员函数
\end{center}

详细看看awaiter的接口:

\begin{itemize}
\item
\textbf{构造函数}

构造函数允许协程(或任何创建awaiter的地方)向awaiter传递参数,这些参数可用于影响或更改侍者处理挂起和恢复的方式。协程优先级请求的awaiter就是一个例子。

\item
\textbf{bool await\_ready()}

在调用此函数的协程挂起前调用。

可以用来(暂时)完全关闭挂起。若它返回true,就“准备好”立即从请求返回挂起并继续执行协程,而不挂起它。

通常,这个函数只返回false(“不,不要避免/阻止任何挂起”),但可能会有条件地产生true(例如,若挂起依赖于可用的某些数据)。

通过其返回类型,await\_suspend()也可以发出不接受协程挂起的信号(请注意,true和false在这里具有相反的含义:通过在await\_suspend()中返回true,接受挂起)。不使用await\_ready()接受挂起的好处是,程序完全节省了启动协程挂起的成本。

在这个函数内部,调用它的协程还没有挂起。不要在这里(间接)调用resume()或destroy(),甚至可以在这里调用更复杂的业务逻辑,只要能够确保逻辑不会为这里挂起的协程调用resume()或destroy()。

\item
\textbf{auto await\_suspend(awaitHdl)}

在调用此函数的协程挂起后立即调用。awaitHdl是请求挂起的协程(带有co\_await),其具有等待协程句柄的类型std::coroutine\_handle<PromiseType>。

可以指定下一步要做什么,包括立即恢复挂起的协程或等待的协程,并安排另一个延迟恢复,可以使用特殊的返回类型(这将在下面讨论)。

甚至可以在这里销毁协程,但必须确保协程没有在其他任何地方使用(例如:调用协程接口中的done())。

\item
\textbf{auto await\_resume()}

当调用此函数的协程在成功挂起后,恢复时调用(即,若协程句柄调用resume())。

await\_resume()的返回类型是导致暂停的co\_await或co\_yield表达式产生的值的类型。若不是空的,协程的上下文可以返回一个值给恢复的协程。
\end{itemize}

await\_suspend()是这里的关键函数,其参数和返回值可以有如下变化:

\begin{itemize}
\item
await\_suspend()的参数可以是:

\begin{itemize}
\item
使用协程句柄的类型:
\begin{cpp}
std::coroutine_handle<PrmType>
\end{cpp}

\item
使用可用于所有协程句柄的基本类型:
\begin{cpp}
std::coroutine_handle<void> (or just std::coroutine_handle<>)
\end{cpp}

这时,无法访问promise。

\item
可以使用auto让编译器确定类型。
\end{itemize}

\item
await\_suspend()的返回类型可以是:
\begin{itemize}
\item
void,在执行await\_suspend()中的语句后继续执行挂起,并返回到协程的调用者。

\item
bool,来表明暂停是否真的应该发生,false表示“不再挂起”(这与await\_ready()的布尔返回值相反)。

\item
std::coroutine\_handle<>用于恢复另一个协程。

await\_suspend()的这种使用称为对称传输(symmetric transfer),稍后将详细描述。

这时,可以使用noop协程来发出不恢复任何协程的信号(与函数返回false的方式相同)。
\end{itemize}
\end{itemize}

此外,请注意以下事项:

\begin{itemize}
\item
成员函数通常是const,除非awaiter有一个可修改的成员(比如将协程句柄存储在await\_suspend()中,以便在恢复时可用)。

\item
成员函数通常是noexcept(这是允许在final\_suspend()中使用的必要条件)。

\item
成员函数可以是constexpr。
\end{itemize}

\mySubsubsection{15.7.2}{让co\_await更新正在运行的协程}

因为awaiter可以在挂起时执行代码,所以可以使用它们来改变处理协程的系统的行为。让我们看一个例子,让协程改变其优先级。

假设在调度器中管理所有的协程,并使用co\_await更改其优先级。为此,首先定义一个默认优先级,并为co\_await定义一些参数来改变优先级:

\begin{cpp}
int CoroPrioDefVal = 10;
enum class CoroPrioRequest {same, less, more, def};
\end{cpp}

协程代码可能如下所示:

\filename{coro/coroprio.hpp}

\begin{cpp}
#include "coropriosched.hpp"
#include <iostream>
CoroPrioTask coro(int max)
{
	std::cout << " coro(" << max << ")\n";
	for (int val = 1; val <= max; ++val) {
		std::cout << " coro(" << max << "): " << val << '\n';
		co_await CoroPrio{CoroPrioRequest::less}; // SUSPEND with lower prio
	}
	std::cout << " end coro(" << max << ")\n";
}
\end{cpp}

通过使用特殊的协程接口CoroPrioTask,协程可以使用类型为CoroPrio的特殊awaiter,允许协程传递更改优先级的请求。在挂起时,可以用它来改变请求的当前优先级:

\begin{cpp}
co_await CoroPrio{CoroPrioRequest::less}; // SUSPEND with lower prio
\end{cpp}

为此,主程序创建一个调度器并调用start()将每个协程传递给调度器:

\filename{coro/coroprio.cpp}

\begin{cpp}
#include "coroprio.hpp"
#include <iostream>

int main()
{
	std::cout << "start main()\n";
	CoroPrioScheduler sched;

	std::cout << "schedule coroutines\n";
	sched.start(coro(5));
	sched.start(coro(1));
	sched.start(coro(4));

	std::cout << "loop until all are processed\n";
	while (sched.resumeNext()) {
	}
	std::cout << "end main()\n";
}
\end{cpp}

类CoroPrioScheduler和CoroPrioTask交互如下:

\begin{itemize}
\item
调度器按优先级顺序存储所有协程,并提供成员来存储新协程、恢复下一个协程和更改协程的优先级:

\begin{cpp}
class CoroPrioScheduler
{
	std::multimap<int, CoroPrioTask> tasks; // all tasks sorted by priority
	...
	public:
	void start(CoroPrioTask&& task);
	bool resumeNext();
	bool changePrio(CoroPrioTask::CoroHdl hdl, CoroPrioRequest pr);
};
\end{cpp}

\item
调度器的成员函数start()将传递的协程存储在列表中,并将调度器存储在每个任务的promise中:

\begin{cpp}
class CoroPrioScheduler
{
	...
	public:
	void start(CoroPrioTask&& task) {
		// store scheduler in coroutine state:
		task.hdl.promise().schedPtr = this;
		// schedule coroutine with a default priority:
		tasks.emplace(CoroPrioDefVal, std::move(task));
	}
	...
};
\end{cpp}

\item
类CoroPrioTask赋予类CoroPrioScheduler对句柄的访问权,在promise类型中提供指向调度程序的成员,并允许协程使用CoroPrioRequest的co\_await:

\begin{cpp}
class CoroPrioTask {
	...
	friend class CoroPrioScheduler; // give access to the handle
	struct promise_type {
		...
		CoroPrioScheduler* schedPtr = nullptr; // each task knows its scheduler:
		auto await_transform(CoroPrioRequest); // deal with co_await a CoroPrioRequest
	};
	...
}
\end{cpp}

对于start(),还可以以提供移动语义。

\end{itemize}

协程和调度器之间的接口是CoroPrio类型的awaiter,其构造函数接受优先级请求,并在挂起时(当我们获得协程句柄时)将其存储在promise中,所以构造函数将请求存储在await\_suspend()中:

\begin{cpp}
class CoroPrio {
private:
	CoroPrioRequest prioRequest;
public:
	CoroPrio(CoroPrioRequest pr)
	: prioRequest{pr} { // deal with co_await a CoroPrioRequest
	}
	...
	void await_suspend(CoroPrioTask::CoroHdl h) noexcept {
		h.promise().schedPtr->changePrio(h, prioRequest);
	}
	...
};
\end{cpp}

其余部分只是通常的协同程序接口的样板代码加上优先级处理。完整的代码请参见coro/coroprioschedule.hpp。

主程序调度三个协程并循环,直到所有协程处理完:

\begin{cpp}
sched.start(coro(5));
sched.start(coro(1));
sched.start(coro(4));

while (sched.resumeNext()) {
}
\end{cpp}

输出结果如下所示:

\begin{shell}
start main()
schedule coroutines
loop until all are processed
	coro(5)
	coro(5): 1
	coro(1)
	coro(1): 1
	coro(4)
	coro(4): 1
	coro(5): 2
	end coro(1)
	coro(4): 2
	coro(5): 3
	coro(4): 3
	coro(5): 4
	coro(4): 4
	coro(5): 5
	end coro(4)
	end coro(5)
end main()
\end{shell}

最初,启动的三个协程具有相同的优先级。由于协程的优先级在每次挂起时都会降低,因此其他协程会连续运行,直到第一个协程再次具有最高优先级。

\mySubsubsection{15.7.3}{具有等待器的对称传输}

await\_suspend()的返回类型可以是协程句柄,通过立即恢复返回的协程来挂起协程,这种技术称为对称传输。

引入对称传输是为了提高性能并避免协程调用其他协程时堆栈溢出。通常,当使用resume()恢复协程时,程序需要为新协程提供一个新的堆栈帧。若要在await\_suspend()中或在它被调用之后恢复协程,是要付出代价的。通过返回要调用的协程,当前协程的堆栈帧只是替换它的协程,因此不需要新的堆栈帧。

\mySamllsection{用协程实现对称传输}

这种技术的典型应用是通过使用处理协程的final await来实现的。[这项技术在Lewis Baker和Michael Eiler的文章和电子邮件中得到了极大的帮助。\url{http://lewissbaker.github.io/2020/05/11/understanding_symmetric_transfer}特别提供了详细的动机和解释。]

假设有一个协程结束了,然后应该继续使用另一个后续协程,而不将控制权交还给调用者,会遇到以下问题:在final\_suspend()中,协程尚未处于挂起状态。必须等待,直到await\_suspend()调用,以返回可等待对象来处理恢复。这可能看起来如下代码所示:

\begin{cpp}
class CoroTask
{
public:
	struct promise_type;
	using CoroHdl = std::coroutine_handle<promise_type>;
private:
	CoroHdl hdl; // native coroutine handle
public:
	struct promise_type {
		std::coroutine_handle<> contHdl = nullptr; // continuation (if there is one)
		...
		auto final_suspend() noexcept {
			// the coroutine is not suspended yet, use awaiter for continuation
			return FinalAwaiter{};
		}
	};
	...
};
\end{cpp}

返回一个awaiter,可用于在协程暂停后处理协程。这里,返回的awaiter的类型是FinalAwaiter,可能如下所示:

\begin{cpp}
struct FinalAwaiter {
	bool await_ready() noexcept {
		return false;
	}
	std::coroutine_handle<> await_suspend(CoroTask::CoroHdl h) noexcept {
		// the coroutine is now suspended at the final suspend point
		// - resume its continuation if there is one
		if (h.promise().contHdl) {
			return h.promise().contHdl; // return the next coro to resume
		}
		else {
			return std::noop_coroutine(); // no next coro => return to caller
		}
	}
	void await_resume() noexcept {
	}
};
\end{cpp}

因为await\_suspend()返回协程句柄,所以在挂起时自动恢复返回的协程,函数std::noop\_coroutine()发出不恢复其他协程的信号,所以挂起的协程会返回给调用者。

std::noop\_coroutine()会返回一个sstd::noop\_coroutine\_handle,是std::coroutine\_handle<std::noop\_coroutine\_promise>的别名类型。该类型的协程在调用resume()或destroy()时不起作用,在调用address()时返回nullptr,并且在调用done()时始终返回false。

std::noop\_coroutine()及其返回类型是为await\_suspend()可能可选地返回要继续的协程的情况提供的。通过返回类型std::coroutine\_handle<>, await\_suspend()可以返回一个noop协程,以表示不自动恢复另一个协程。

注意,std::noop\_coroutine()的两个不同的返回值并不一定相等。

因此,下面的代码不可移植:

\begin{cpp}
std::coroutine_handle<> coro = std::noop_coroutine();
...
if (coro == std::noop_coroutine()) { // OOPS: does not check whether coro has initial value
	...
	return coro;
}
\end{cpp}

应该使用nullptr:

\begin{cpp}
std::coroutine_handle<> coro = nullptr;
...
if (coro) { // OK (checks whether coro has initial value)
	...
	return std::noop_coroutine();
}
\end{cpp}

处理线程池中的协程演示了这种技术。













