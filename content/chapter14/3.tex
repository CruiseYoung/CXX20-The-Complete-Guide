
介绍了使用co\_await的示例之后,还应该介绍协程的其他两个关键字:

\begin{itemize}
\item 
co\_yield允许协程在每次挂起时产生一个值。

\item 
co\_return允许协程在其结束时返回一个值。
\end{itemize}

\mySubsubsection{14.3.1}{使用co\_yield}

通过使用co\_yield,协程可以在挂起时产生中间结果。

一个明显的例子是“生成”值的协程。作为示例的一个变体,可以循环一些值直到最大值,并将它们输出给协程的调用者,而不是仅仅输出它们。为此,协程看起来如下所示:

\filename{coro/coyield.hpp}

\begin{cpp}
#include <iostream>
#include "corogen.hpp" // for CoroGen

CoroGen coro(int max)
{
	std::cout << "            CORO " << max << " start\n";
	
	for (int val = 1; val <= max; ++val) {
		// print next value:
		std::cout << "              CORO " << val << '/' << max << '\n';
		
		// yield next value:
		co_yield val; // SUSPEND with value
	}
	std::cout << "            CORO " << max << " end\n";
}
\end{cpp}

通过使用co\_yield,协程产生中间结果。当协程到达co\_yield时,会挂起协程,并提供co\_yield后面的表达式的值:

\begin{cpp}
co_yield val; // calls yield_value(val) on promise
\end{cpp}

协程帧将此映射到对yield\_value()的调用,用于协程的承诺,可以定义如何处理此中间结果。我们的例子中,将值存储在promise的成员中,这使得它可以在协程接口中使用:

\begin{cpp}
struct promise_type {
	int coroValue = 0; // last value from co_yield
	
	auto yield_value(int val) { // reaction to co_yield
	coroValue = val; // - store value locally
	return std::suspend_always{}; // - suspend coroutine
	}
	...
};
\end{cpp}

将值存储在promise中之后,返回std::suspend\_always\{\},这实际上是挂起协程。可以在这里编写不同的行为,以便协同程序(有条件地)继续。

协程可以这样使用:

\filename{coro/coyield.hpp}

\begin{cpp}
#include "coyield.hpp"
#include <iostream>

int main()
{
	// start coroutine:
	auto coroGen = coro(3); // initialize coroutine
	std::cout << "coro() started\n";
	
	// loop to resume the coroutine until it is done:
	while (coroGen.resume()) { // RESUME
		auto val = coroGen.getValue();
		std::cout << "coro() suspended with " << val << '\n';
	}
	
	std::cout << "coro() done\n";
}
\end{cpp}

同样,通过调用coro(3),初始化协程计数到3。每当我们为返回的协程接口调用resume()时,协程就会“计算”并产生下一个值。

然而,resume()并不返回产生的值(仍然返回是否有必要再次恢复协程)。要访问下一个值,提供并使用getValue(),所以程序有如下输出:

\begin{shell}
coro() started
         CORO 3 start
         CORO 1/3
coro() suspended with 1
         CORO 2/3
coro() suspended with 2
         CORO 3/3
coro() suspended with 3
         CORO 3 end
coro() done
\end{shell}

协程接口必须处理产生的值,并为调用者提供稍微不同的API,所以使用不同的类型名称:CoroGen。这个名字表明,没有执行任务,而是有一个协程,在每次挂起时生成值。CoroGen类型可以定义如下:

\filename{coro/corogen.hpp}

\begin{cpp}
#include <coroutine>
#include <exception> // for terminate()

class [[nodiscard]] CoroGen {
public:
	// initialize members for state and customization:
	struct promise_type;
	using CoroHdl = std::coroutine_handle<promise_type>;
private:
	CoroHdl hdl; // native coroutine handle
public:
	struct promise_type {
		int coroValue = 0; // recent value from co_yield
		
		auto yield_value(int val) { // reaction to co_yield
			coroValue = val; // - store value locally
			return std::suspend_always{}; // - suspend coroutine
		}
		
		// the usual members:
		auto get_return_object() { return CoroHdl::from_promise(*this); }
		auto initial_suspend() { return std::suspend_always{}; }
		void return_void() { }
		void unhandled_exception() { std::terminate(); }
		auto final_suspend() noexcept { return std::suspend_always{}; }
	};
	
	// constructors and destructor:
	CoroGen(auto h) : hdl{h} { }
	~CoroGen() { if (hdl) hdl.destroy(); }
	
	// no copying or moving:
	CoroGen(const CoroGen&) = delete;
	CoroGen& operator=(const CoroGen&) = delete;
	
	// API:
	// - resume the coroutine:
	bool resume() const {
		if (!hdl || hdl.done()) {
			return false; // nothing (more) to process
			}
			hdl.resume(); // RESUME
			return !hdl.done();
		}
		
	// - yield value from co_yield:
	int getValue() const {
		return hdl.promise().coroValue;
	}
};
\end{cpp}

通常,这里使用的协程接口的定义遵循协程接口的一般原则。有两点不同:

\begin{itemize}
\item 
promise提供了成员yield\_value(),在到达co\_yield时调用。

\item
对于协程接口的外部API,CoroGen提供了getValue(),其返回存储在promise中的最后一个生成值:

\begin{cpp}
class CoroGen {
	public:
	...
	int getValue() const {
		return hdl.promise().coroValue;
	}
};
\end{cpp}
\end{itemize}

\mySamllsection{迭代协程产生的值}

也可以使用一个协程接口,像一个范围一样使用(提供一个API来迭代挂起产生的值):

\filename{coro/cororange.cpp}

\begin{cpp}
#include "cororange.hpp"
#include <iostream>
#include <vector>

int main()
{
	auto gen = coro(3); // initialize coroutine
	std::cout << "--- coro() started\n";
	
	// loop to resume the coroutine for the next value:
	for (const auto& val : gen) {
		std::cout << " val: " << val << '\n';
	}
	
	std::cout << "--- coro() done\n";
}
\end{cpp}

只需要协程返回一个稍微不同的生成器(参见coro/cororange.hpp):

\begin{cpp}
Generator<int> coro(int max)
{
	std::cout << "CORO " << max << " start\n";
	...
}
\end{cpp}

如您所见,为传递的类型(本例中为int)的生成器值使用了泛型协程接口。一个非常简单的实现(它显示了这一点,但有一些缺陷)可能如下所示:

\filename{coro/generator.hpp}

\begin{cpp}
#include <coroutine>
#include <exception> // for terminate()
#include <cassert> // for assert()

template<typename T>
class [[nodiscard]] Generator {
	public:
	// customization points:
	struct promise_type {
		T coroValue{}; // last value from co_yield
		
		auto yield_value(T val) { // reaction to co_yield
			coroValue = val; // - store value locally
			return std::suspend_always{}; // - suspend coroutine
		}
		
		auto get_return_object() {
			return std::coroutine_handle<promise_type>::from_promise(*this);
		}
		auto initial_suspend() { return std::suspend_always{}; }
		void return_void() { }
		void unhandled_exception() { std::terminate(); }
		auto final_suspend() noexcept { return std::suspend_always{}; }
	};
private:
	std::coroutine_handle<promise_type> hdl; // native coroutine handle
	
public:
	
	// constructors and destructor:
	Generator(auto h) : hdl{h} { }
	~Generator() { if (hdl) hdl.destroy(); }
	
	// no copy or move supported:
	Generator(const Generator&) = delete;
	Generator& operator=(const Generator&) = delete;
	
	// API to resume the coroutine and access its values:
	// - iterator interface with begin() and end()
	struct iterator {
		std::coroutine_handle<promise_type> hdl; // nullptr on end
		iterator(auto p) : hdl{p} {
		}
		void getNext() {
			if (hdl) {
				hdl.resume(); // RESUME
				if (hdl.done()) {
					hdl = nullptr;
				}
			}
		}
		int operator*() const {
			assert(hdl != nullptr);
			return hdl.promise().coroValue;
		}
		iterator operator++() {
			getNext(); // resume for next value
			return *this;
		}
		bool operator== (const iterator& i) const = default;
	};
	
	iterator begin() const {
		if (!hdl || hdl.done()) {
			return iterator{nullptr};
		}
		iterator itor{hdl}; // initialize iterator
		itor.getNext(); // resume for first value
		return itor;
	}
	
	iterator end() const {
		return iterator{nullptr};
	}
};
\end{cpp}

关键的方法是协程接口提供了成员begin()和end(),以及迭代器来迭代值:

\begin{itemize}
\item 
begin()在恢复第一个值后生成迭代器。迭代器在内部存储协程句柄,以便知道协程的状态。

\item 
operator++() 迭代器会返回下一个值。

\item 
end() 生成一个表示范围结束的状态,它的hdl是null。这也是迭代器在没有更多值时的状态。

\item 
operator==() 默认通过比较两个迭代器的句柄来比较状态。
\end{itemize}

C++23可能会在其标准库中提供这样一个协程接口,如std::generator<>,其具有更加复杂和健壮的API(参见\url{http://wg21.link/p2502})。

\mySubsubsection{14.3.2}{使用co\_return}

通过使用co\_return,协程可以在其结束时向调用者返回结果。

考虑下面的例子:

\filename{coro/coreturn.cpp}

\begin{cpp}
#include <iostream>
#include <vector>
#include <ranges>
#include <coroutine> // for std::suspend_always{}
#include "resulttask.hpp" // for ResultTask

ResultTask<double> average(auto coll)
{
	double sum = 0;
	for (const auto& elem : coll) {
		std::cout << " process " << elem << '\n';
		sum = sum + elem;
		co_await std::suspend_always{}; // SUSPEND
	}
	co_return sum / std::ranges::ssize(coll); // return resulting average
}

int main()
{
	std::vector values{0, 8, 15, 47, 11, 42};
	// start coroutine:
	auto task = average(std::views::all(values));
	
	// loop to resume the coroutine until all values have been processed:
	std::cout << "resume()\n";
	while (task.resume()) { // RESUME
		std::cout << "resume() again\n";
	}
	
	// print return value of coroutine:
	std::cout << "result: " << task.getResult() << '\n';
}
\end{cpp}

这个程序中,main()启动协程average(),遍历传递的集合中的元素,并将它们的值添加到初始和中。处理完每个元素后,协程挂起。

最后,协例程通过将总和除以元素数量返回平均值。

注意,需要co\_return来返回协程的结果,不允许使用return。

协程接口是在ResultTask类中定义的,该类是根据返回值的类型参数化的。这个接口提供resume(),以便在协程挂起时恢复它。此外,还提供了getResult()来请求协程完成后的返回值。

协同程序接口ResultTask<>如下所示:

\filename{coro/resulttask.hpp}

\begin{cpp}
#include <coroutine>
#include <exception> // for terminate()
template<typename T>
class [[nodiscard]] ResultTask {
public:
	// customization points:
	struct promise_type {
		T result{}; // value from co_return
		
		void return_value(const auto& value) { // reaction to co_return
			result = value; // - store value locally
		}
		
		auto get_return_object() {
			return std::coroutine_handle<promise_type>::from_promise(*this);
		}
		
		auto initial_suspend() { return std::suspend_always{}; }
		void unhandled_exception() { std::terminate(); }
		auto final_suspend() noexcept { return std::suspend_always{}; }
	};
	
private:
	std::coroutine_handle<promise_type> hdl; // native coroutine handle
	
public:
	// constructors and destructor:
	// - no copying or moving is supported
	ResultTask(auto h) : hdl{h} { }
	~ResultTask() { if (hdl) hdl.destroy(); }
	ResultTask(const ResultTask&) = delete;
	ResultTask& operator=(const ResultTask&) = delete;
	
	// API:
	// - resume() to resume the coroutine
	bool resume() const {
		if (!hdl || hdl.done()) {
			return false; // nothing (more) to process
		}
		hdl.resume(); // RESUME
		return !hdl.done();
	}
	
	// - getResult() to get the last value from co_yield
	T getResult() const {
		return hdl.promise().result;
	}
};
\end{cpp}

同样,该定义遵循协程接口的一般原则。

但这次我们支持返回值,所以在promise类型中,不再提供自定义点return\_void()。相反,提供了return\_value(),当协程到达co\_return表达式时调用:

\begin{cpp}
template<typename T>
class ResultTask {
	public:
	struct promise_type {
		T result{}; // value from co_return
		void return_value(const auto& value) { // reaction to co_return
			result = value; // - store value locally
		}
		...
	};
	...
};
\end{cpp}

当调用getResult()时,协程接口只返回这个值:

\begin{cpp}
template<typename T>
class ResultTask {
	public:
	...
	T getResult() const {
		return hdl.promise().result;
	}
};
\end{cpp}

\mySamllsection{return\_void() 和 return\_value()}

请注意,若协程以有时可能返回值,有时可能不返回值的方式实现,则这是未定义的行为。则这个协程是无效的:

\begin{cpp}
ResultTask<int> coroUB( ... )
{
	if ( ... ) {
		co_return 42;
	}
}
\end{cpp}












