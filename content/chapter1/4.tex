最后,介绍一下如何使用重写支持的比较操作符,对表达式进行求值。


\mySamllsection{相等操作符}

编译

\begin{cpp}
x != y
\end{cpp}

编译器现在可能会尝试以下的操作:

\begin{cpp}
x.operator!=(y) // calling member operator!= for x
operator!=(x, y) // calling a free-standing operator!= for x and y

!x.operator==(y) // calling member operator== for x
!operator==(x, y) // calling a free-standing operator== for x and y

!x.operator==(y) // calling member operator== generated by operator<=> for x

!y.operator==(x) // calling member operator== generated by operator<=> for y
\end{cpp}

最后一种形式试图支持第一个操作数的隐式类型转换,这要求操作数是一个参数。

通常,编译器会尝试调用:

\begin{itemize}
\item
操作符!=:操作符!=(x, y)或成员操作符!=:x.operator!=(y)

同时定义两个操作符!=会导致歧义错误。

\item
操作符==:!operator==(x, y)或成员operator==: !x.operator==(y)

注意,成员==操作符可以使用默认操作符<=>成员生成。

同样,同时定义两个操作符==会导致歧义错误,这也适用于成员函数。

==操作符是由于默认操作符<=>生成。
\end{itemize}

当需要对第一个操作数v进行隐式类型转换时,编译器还会尝试对操作数重新排序:

\begin{cpp}
42 != y // 42 implicitly converts to the type of y
\end{cpp}

这种情况下,编译器会按顺序调用:

\begin{itemize}
\item
独立或成员操作符!=

\item
独立的或成员操作符==(注意,成员操作符==可以使用默认的<=>成员生成)[最初的C++20标准在\url{http://wg21.link/p2468r2}中进行了修复。]
\end{itemize}

注意,重写的表达式永远不会尝试调用成员操作符!=。

\mySamllsection{关系操作符}

对于关系操作符,有类似的行为,只是重写的语句返回到新的操作符<=>并将结果与0进行比较。操作符的行为类似于一个三方比较函数,表示less时返回负值,表示equal时返回0,表示greater时返回正值(返回值不是数值,只是一个支持相应比较的值)。

例如,编译

\begin{cpp}
x <= y
\end{cpp}

编译器现在会尝试以下的操作:

\begin{cpp}
x.operator<=(y) // calling member operator<= for x
operator<=(x, y) // calling a free-standing operator<= for x and y

x.operator<=>(y) <= 0 // calling member operator<=> for x
operator<=>(x, y) <= 0 // calling a free-standing operator<=> for x and y

0 <= y.operator<=>(x) // calling member operator<=> for y
\end{cpp}

同样,最后一种形式试图支持第一个操作数的隐式类型转换,所以相应的值必须成为第一个操作数的参数。






