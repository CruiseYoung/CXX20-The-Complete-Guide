每个C++版本都会逐步引入编译器支持的各种语言和库特性。出于这个原因,知道编译器通常支持哪个C++版本通常是不够的;对于可移植代码,了解某个特定特性是否可用很重要。

为此,C++20正式引入了特性测试宏。对于每个新语言和库特性,都可以使用一个宏来表示该特性是否可用。宏甚至可以提供关于支持哪个版本的特性的信息。

例如,以下源代码将根据是否可用(以及以何种形式)使用不同的代码:

\begin{cpp}
#ifdef __cpp_generic_lambdas
#if __cpp_generic_lambdas >= 201707
... // generic lambdas with template parameters can be used
#else
... // generic lambdas can be used
#endif
#else
... // no generic lambdas can be used
#endif
\end{cpp}

所有用于语言特性的特性测试宏都以\_\_cpp开头。

作为另一个例子,若std::as\_const()还不可用,下面的代码提供并使用了一个解决方案:

\begin{cpp}
#ifndef __cpp_lib_as_const
template<typename T>
const T& asConst(T& t) {
	return t;
}
#endif

#ifdef __cpp_lib_as_const
	auto printColl = [&coll = std::as_const(coll)] {
#else
	auto printColl = [&coll = asConst(coll)] {
#endif
	...
	};
\end{cpp}

库特性的所有特性测试宏都以\_\_cpp\_lib开头。

语言特性的特性测试宏由编译器定义,库特性的特性测试宏由新头文件<version>提供。

参见\_\_cpp\_char8\_t的用法,这是另一个使代码在C++20之前和之后都可移植UTF-8字符的例子。





















