
通过定义概念,可以为一个或多个约束引入名称。

模板(函数、类、变量和别名模板)可以使用概念来约束其能力(通过requires子句或作为模板参数的直接类型约束),但概念也是布尔编译时表达式(类型谓词),可以在需要检查类型的地方使用(if constexpr条件中)。

\mySubsubsection{4.5.1}{定义概念}

概念定义如下:

\begin{cpp}
template< ... >
concept name = ... ;
\end{cpp}

等号是必需的(不能在没有定义的情况下声明一个概念,也不能在这里使用大括号)。等号之后,可以指定可转换为true或false的编译时表达式。

概念很像bool类型的constexpr变量模板,但没有显式指定类型:

\begin{cpp}
template<typename T>
concept MyConcept = ... ;

std::is_same<MyConcept< ... >, bool> // yields true
\end{cpp}

无论在编译时还是在运行时,都可以在需要布尔表达式值的地方使用一个概念。但不接受地址,因为其后面没有对象(地址是一个纯右值)。

模板参数可能没有约束(不能使用概念来定义概念)。

不能在函数中定义概念(所有模板都是如此)。

\mySubsubsection{4.5.2}{概念的特殊能力}

概念具有特殊的能力。

例如,考虑以下概念:

\begin{cpp}
template<typename T>
concept IsOrHasThisOrThat = ... ;
\end{cpp}

与布尔变量模板的定义(定义类型特征的常用方式)相比:

\begin{cpp}
template<typename T>
inline constexpr bool IsOrHasThisOrThat = ... ;
\end{cpp}

有以下不同:

\begin{itemize}
\item
概念并不表示代码,没有类型、存储、生命周期或与对象相关的其他属性。

通过在编译时为特定模板参数实例化它们,实例化只会变为true或false,所以可以在使用true或false的地方使用,可以得到这些字面量的所有属性。

\item
概念不必声明为内联的,其隐式内联。

\item
概念可以用作类型约束:

\begin{cpp}
template<IsOrHasThisOrThat T>
...
\end{cpp}

变量模板不能这样使用。

\item
概念是给约束命名的唯一方法,所以需要其来决定一个约束是否是另一个约束的特殊情况。

\item
包含的概念。为了让编译器决定一个约束是否决定另一个约束(因此是特殊的),必须将约束公式化为概念。
\end{itemize}

\mySubsubsection{4.5.3}{非类型模板参数的概念}

概念也可以应用于非类型模板参数(NTTP)。例如:

\begin{cpp}
template<auto Val>
concept LessThan10 = Val < 10;

template<int Val>
requires LessThan10<Val>
class MyType {
	...
};
\end{cpp}

作为一个可用的例子,可以使用一个概念将非类型模板形参的值约束为2的幂:

\filename{lang/conceptnttp.cpp}

\begin{cpp}
#include <bit>

template<auto Val>
concept PowerOf2 = std::has_single_bit(static_cast<unsigned>(Val));

template<typename T, auto Val>
requires PowerOf2<Val>
class Memory {
	...
};

int main()
{
	Memory<int, 8> m1; // OK
	Memory<int, 9> m2; // ERROR
	Memory<int, 32> m3; // OK
	Memory<int, true> m4; // OK
	...
}
\end{cpp}

概念PowerOf2接受一个值而不是类型作为模板参数(这里使用auto,不需要指定类型):

\begin{cpp}
template<auto Val>
concept PowerOf2 = std::has_single_bit(static_cast<unsigned>(Val));
\end{cpp}

当新的标准函数std::has\_single\_bit()对传递的值产生true(只设置一个位意味着值是2的幂)时,这个概念就满足了,std::has\_single\_bit()要求我们有一个无符号整型值。通过强制转换为unsigned,开发者可以传递有符号整型值,并拒绝不能转换为无符号整型值的类型。

接下来,使用这个概念要求Memory类只接受2的幂次大小和类型:

\begin{cpp}
template<typename T, auto Val>
requires PowerOf2<Val>
class Memory {
	...
};
\end{cpp}

注意,不能这样写:

\begin{cpp}
template<typename T, PowerOf2 auto Val>
class Memory {
	...
};
\end{cpp}

这就对Val类型提出了要求,但概念PowerOf2不约束其类型,并限制了值。








