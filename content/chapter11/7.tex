
C++20现在支持两种时钟,本节讨论它们之间的区别,以及如何使用特殊时钟。

\mySubsubsection{11.7.1}{具有指定epoch的时钟}

C++20现在提供了以下与epoch相关联的时钟(以便定义唯一的时间点):

\begin{itemize}
\item
系统时钟是操作系统的时钟。从C++20开始,其指定为Unix Time[有关Unix时间的详细信息,例如,\url{http://en.wikipedia.org/wiki/Unix_time}.],计算从UTC时间1970年1月1日00:00:00开始的时间。

闰秒的处理方式使得某些秒可能会花费更长的时间,所以永远不会有61秒的小时,而所有365天的年都有相同的31,536,000秒。

\item
UTC时钟是代表协调世界时的时钟,通常称为GMT(格林威治标准时间)或祖鲁时间。本地时间与UTC的差异在于您所在时区的UTC偏移量。

它使用与系统时钟相同的历元(1970年1月1日00:00:00 UTC)。

闰秒的处理使得某些分钟可能有61秒。例如,有一个时间点1972-06-30 23:59:60,因为在1972年6月的最后一分钟增加了一个闰秒。

因此,365天的年份有时可能有31,536,001,甚至31,536,002秒。

\item
GPS时钟使用的是全球定位系统的时间,是GPS地面控制站和卫星上的原子钟实现的原子时标。GPS时间从1980年1月6日00:00:00 UTC的epoch开始。

每分钟有60秒,但GPS通过提前切换到下一个小时来考虑闰秒,所以GPS比UTC快越来越多的秒(或者在1980年之前比UTC慢越来越多的秒)。例如,时间点“20121-01-01 00:00:00 UTC”表示为GPS时间点“20121-01-01 00:00:18 GPS”。在2021年,当我写这本书的时候,GPS时间点领先了18秒。

所有具有365天的GPS年(GPS日期的午夜与一年后GPS日期的午夜之间的差值)都有相同的31,536,000秒,但可能比“真正的”年短一到两秒。

\item
TAI时钟使用国际原子时,这是基于SI秒连续计数的国际原子时标。TAI时间从UTC时间1958年1月1日00:00:00开始。

就像GPS时间一样,每分钟有60秒,而闰秒会通过提前切换到下一个小时来考虑,所以TAI比UTC快越来越多的秒,但与GPS总是有19秒的恒定偏移。例如,时间点“2021-01-01 00:00:00 UTC”表示为TAI时间点“2021-01-01 00:00:37 TAI”。在2021年,当我写这本书的时候,TAI的时间点提前了37秒。
\end{itemize}

\mySubsubsection{11.7.2}{伪时钟local\_t}

如前所述,有一个类型为std::chrono::local\_t的特殊时钟。这个时钟允许指定没有时区(甚至还没有UTC)的本地时间点。其epoch解释为“本地时间”,所以必须将它与时区结合起来才能知道代表的是哪个时间点。

local\_t是一个“伪时钟”,它不满足时钟的所有要求。事实上,它现在没有提供成员函数now():

\begin{cpp}
auto now1 = std::chrono::local_t::now(); // ERROR: now() not provided
\end{cpp}

相反,若需要一个系统时钟时间点和一个时区,可以使用时区(如当前时区或UTC)将时间点转换为本地时间点:

\begin{cpp}
auto sysNow = chr::system_clock::now(); // NOW as UTC timepoint
...
chr::local_time now2
	= chr::current_zone()->to_local(sysNow); // NOW as local timepoint
chr::local_time now3
	= chr::locate_zone("Asia/Tokyo")->to_local(sysNow); // NOW as Tokyo timepoint
\end{cpp}

另一种获得相同结果的方法是调用get\_local\_time()来获取分区时间(具有关联时区的时间点):

\begin{cpp}
chr::local_time now4 = chr::zoned_time{chr::current_zone(),
										sysNow}.get_local_time();
\end{cpp}

另一种方法是将字符串解析为本地时间点:

\begin{cpp}
chr::local_seconds tp; // time_point<local_t, seconds>
std::istringstream{"2021-1-1 18:30"} >> chr::parse(std::string{"%F %R"}, tp);
\end{cpp}

记住本地时间点与其他时间点之间的细微差别:

\begin{itemize}
\item
系统/UTC/GPS/TAI时间点表示一个特定的时间点,将应用于时区将转换它所表示的时间值。

\item
本地时间点表示本地时间。当与时区结合,全局时间点就会变得清晰。
\end{itemize}

例如:

\begin{cpp}
auto now = chr::current_zone()->to_local(chr::system_clock::now());
std::cout << now << '\n';
std::cout << "Berlin: " << chr::zoned_time("Europe/Berlin", now) << '\n';
std::cout << "Sydney: " << chr::zoned_time("Australia/Sydney", now) << '\n';
std::cout << "Cairo: " << chr::zoned_time("Africa/Cairo", now) << '\n';
\end{cpp}

这里,将当前的本地时间应用于三个不同的时区:

\begin{shell}
2021-04-14 08:59:31.640004000
Berlin: 2021-04-14 08:59:31.640004000 CEST
Sydney: 2021-04-14 08:59:31.640004000 AEST
Cairo:  2021-04-14 08:59:31.640004000 EET
\end{shell}

当使用本地时间点时,不能对时区使用转换说明符:

\begin{cpp}
chr::local_seconds tp; // time_point<local_t, seconds>
...
std::cout << std::format("{:%F %T %Z}\n", tp); // ERROR: invalid format
std::cout << std::format("{:%F %T}\n", tp); // OK
\end{cpp}


\mySubsubsection{11.7.3}{处理闰秒}

前面对具有指定epoch的时钟的讨论已经介绍了处理闰秒的基本方式。

为了更清楚地处理闰秒,我们使用不同的时钟迭代闰秒的时间点:

\filename{lib/chronoclocks.cpp}

\begin{cpp}
#include <iostream>
#include <chrono>

int main()
{
	using namespace std::literals;
	namespace chr = std::chrono;

	auto tpUtc = chr::clock_cast<chr::utc_clock>(chr::sys_days{2017y/1/1} - 1000ms);
	for (auto end = tpUtc + 2500ms; tpUtc <= end; tpUtc += 200ms) {
		auto tpSys = chr::clock_cast<chr::system_clock>(tpUtc);
		auto tpGps = chr::clock_cast<chr::gps_clock>(tpUtc);
		auto tpTai = chr::clock_cast<chr::tai_clock>(tpUtc);
		std::cout << std::format("{:%F %T} SYS ", tpSys);
		std::cout << std::format("{:%F %T %Z} ", tpUtc);
		std::cout << std::format("{:%F %T %Z} ", tpGps);
		std::cout << std::format("{:%F %T %Z}\n", tpTai);
	}
}
\end{cpp}

该程序有以下输出:

\begin{shell}
2016-12-31 23:59:59.000 SYS 2016-12-31 23:59:59.000 UTC 2017-01-01 00:00:16.000 GPS 2017-01-01 00:00:35.000 TAI
2016-12-31 23:59:59.200 SYS 2016-12-31 23:59:59.200 UTC 2017-01-01 00:00:16.200 GPS 2017-01-01 00:00:35.200 TAI
2016-12-31 23:59:59.400 SYS 2016-12-31 23:59:59.400 UTC 2017-01-01 00:00:16.400 GPS 2017-01-01 00:00:35.400 TAI
2016-12-31 23:59:59.600 SYS 2016-12-31 23:59:59.600 UTC 2017-01-01 00:00:16.600 GPS 2017-01-01 00:00:35.600 TAI
2016-12-31 23:59:59.800 SYS 2016-12-31 23:59:59.800 UTC 2017-01-01 00:00:16.800 GPS 2017-01-01 00:00:35.800 TAI
2016-12-31 23:59:59.999 SYS 2016-12-31 23:59:60.000 UTC 2017-01-01 00:00:17.000 GPS 2017-01-01 00:00:36.000 TAI
2016-12-31 23:59:59.999 SYS 2016-12-31 23:59:60.200 UTC 2017-01-01 00:00:17.200 GPS 2017-01-01 00:00:36.200 TAI
2016-12-31 23:59:59.999 SYS 2016-12-31 23:59:60.400 UTC 2017-01-01 00:00:17.400 GPS 2017-01-01 00:00:36.400 TAI
2016-12-31 23:59:59.999 SYS 2016-12-31 23:59:60.600 UTC 2017-01-01 00:00:17.600 GPS 2017-01-01 00:00:36.600 TAI
2016-12-31 23:59:59.999 SYS 2016-12-31 23:59:60.800 UTC 2017-01-01 00:00:17.800 GPS 2017-01-01 00:00:36.800 TAI
2017-01-01 00:00:00.000 SYS 2017-01-01 00:00:00.000 UTC 2017-01-01 00:00:18.000 GPS 2017-01-01 00:00:37.000 TAI
2017-01-01 00:00:00.200 SYS 2017-01-01 00:00:00.200 UTC 2017-01-01 00:00:18.200 GPS 2017-01-01 00:00:37.200 TAI
2017-01-01 00:00:00.400 SYS 2017-01-01 00:00:00.400 UTC 2017-01-01 00:00:18.400 GPS 2017-01-01 00:00:37.400 TAI
\end{shell}

这里的闰秒是写这本书时的最后一个闰秒(事先不知道闰秒在未来什么时候会发生),输出带有相应时区的时间点(对于系统时间点,这里输出的是SYS,而非其默认时区UTC)。可以观察到以下几点:

\begin{itemize}
\item
闰秒时:

\begin{itemize}
\item
UTC时间的秒可到60。

\item
系统时钟使用在插入闰秒之前最后一个可表示的sys\_time值,行为由C++标准保证。
\end{itemize}

\item
闰秒前:

\begin{itemize}
\item
GPS时间比UTC时间快17秒。

\item
TAI时间比UTC时间早36秒(一如既往,比GPS时间早19秒)。
\end{itemize}

\item
闰秒后:

\begin{itemize}
\item
GPS时间比UTC时间快18秒。

\item
TAI时间比UTC时间早37秒(仍比GPS时间早19秒)。
\end{itemize}
\end{itemize}

\mySubsubsection{11.7.4}{时钟间的转换}

可以在时钟间转换时间点,前提是这种转换有意义。为此,chrono库提供了一个clock\_cast<>,只能在具有指定稳定epoch (sys\_time<>、utc\_time<>、gps\_time<>、tai\_time<>)的时钟时间点和文件系统时间点间进行转换。

强制转换需要目标时钟,可以选择传递不同的时间段。

下面的程序创建一个UTC闰秒到其他几个时钟的输出:

\filename{lib/chronoconv.cpp}

\begin{cpp}
#include <iostream>
#include <sstream>
#include <chrono>

int main()
{
	namespace chr = std::chrono;

	// initialize a utc_time<> with a leap second:
	chr::utc_time<chr::utc_clock::duration> tp;
	std::istringstream{"2015-6-30 23:59:60"}
		>> chr::parse(std::string{"%F %T"}, tp);

	// convert it to other clocks and print that out:
	auto tpUtc = chr::clock_cast<chr::utc_clock>(tp);
	std::cout << "utc_time: " << std::format("{:%F %T %Z}", tpUtc) << '\n';
	auto tpSys = chr::clock_cast<chr::system_clock>(tp);
	std::cout << "sys_time: " << std::format("{:%F %T %Z}", tpSys) << '\n';
	auto tpGps = chr::clock_cast<chr::gps_clock>(tp);
	std::cout << "gps_time: " << std::format("{:%F %T %Z}", tpGps) << '\n';
	auto tpTai = chr::clock_cast<chr::tai_clock>(tp);
	std::cout << "tai_time: " << std::format("{:%F %T %Z}", tpTai) << '\n';
	auto tpFile = chr::clock_cast<chr::file_clock>(tp);
	std::cout << "file_time: " << std::format("{:%F %T %Z}", tpFile) << '\n';
}
\end{cpp}

该程序有以下输出:

\begin{shell}
utc_time: 2015-06-30 23:59:60.0000000 UTC
sys_time: 2015-06-30 23:59:59.9999999 UTC
gps_time: 2015-07-01 00:00:16.0000000 GPS
tai_time: 2015-07-01 00:00:35.0000000 TAI
file_time: 2015-06-30 23:59:59.9999999 UTC
\end{shell}

对于所有这些时钟,都为格式化的输出提供了伪时区。

转换规则如下:

\begin{itemize}
\item
从本地时间点进行的任何转换都只添加epoch,时间值保持不变。

\item
UTC, GPS和TAI时间点之间的转换增加或减少必要的偏移量。

\item
UTC和系统时间之间的转换不改变时间值,但对于UTC闰秒的时间点,使用前面的最后一个时间点作为系统时间。
\end{itemize}

还支持从本地时间点到其他时钟的转换,大门要转换为本地时间点,必须使用to\_local()(在转换为系统时间点之后)。

不支持与steady\_clock的时间点之间的转换。

\mySamllsection{内部时钟的转换}

在内部,clock\_cast<>是一个“双轮毂辐条和车轮系统”。这两个中心是system\_clock和utc\_clock,每个可转换时钟都必须转换到这些中心中的一个(但不是两个)。到中心的转换是通过时钟的to\_sys()或to\_utc()静态成员函数完成的。对于来自中心的转换,提供了from\_sys()或from\_utc()。clock\_cast<>将这些成员函数串在一起,用于将一种时钟转换为其他时钟的表述。

不能处理闰秒的时钟应该转换为/从system\_clock。例如,utc\_clock和local\_t提供给to\_sys()。

可以以某种方式处理闰秒的时钟(这并不一定说明其具有闰秒值)应该转换为/从utc\_clock。这适用于gps\_clock和tai\_clock,即使gps和tai没有闰秒,也有唯一到UTC闰秒的双向映射。

对于file\_clock,是否提供与system\_clock或utc\_clock之间的转换取决于具体实现。


\mySubsubsection{11.7.5}{处理文件时钟}

时钟std::chrono::file\_clock是文件系统库用于文件系统(文件、目录等)的时间点的时钟,是一种特定于实现的时钟类型,反映文件系统时间值的分辨率和范围。

例如,可以使用文件时钟来更新最后访问文件的时间:

\begin{cpp}
// touch file with path p (update last write access to file):
std::filesystem::last_write_time(p,
								 std::chrono::file_clock::now());
\end{cpp}

还可以使用C++17中的文件系统时钟

\begin{cpp}
std::filesystem::file_time_type:
	std::filesystem::last_write_time(p,
									std::filesystem::file_time_type::clock::now());
\end{cpp}

从C++20开始,文件系统类型名称file\_time\_type定义如下:

\begin{cpp}
namespace std::filesystem {
	using file_time_type = chrono::time_point<chrono::file_clock>;
}
\end{cpp}

C++17中,只能使用未指定的普通时钟。

对于文件系统的时间点,现在也定义了类型file\_time:

\begin{cpp}
namespace std::chrono {
	template<typename Duration>
	using file_time = time_point<file_clock, Duration>;
}
\end{cpp}

没有定义像file\_seconds这样的类型(就像其他时钟一样)。

file\_time类型的新定义现在允许开发者可移植地将文件系统时间点转换为系统时间点。例如,可以输出最后一次访问传入文件的时间:

\begin{cpp}
void printFileAccess(const std::filesystem::path& p)
{
	std::cout << "\"" << p.string() << "\":\n";

	auto tpFile = std::filesystem::last_write_time(p);
	std::cout << std::format(" Last write access: {0:%F} {0:%X}\n", tpFile);

	auto diff = std::chrono::file_clock::now() - tpFile;
	auto diffSecs = std::chrono::round<std::chrono::seconds>(diff);
	std::cout << std::format(" It is {} old\n", diffSecs);
}
\end{cpp}

这段代码可能的输出:

\begin{shell}
"chronoclocks.cpp":
 Last write access: 2021-07-12 16:50:08
 It is 18s old
\end{shell}

若要使用默认的时间点输出操作符,其会根据文件时钟的粒度输出亚秒:

\begin{cpp}
std::cout << " Last write access: " << diffSecs << '\n';
\end{cpp}

输出可能如下所示:

\begin{shell}
Last write access: 2021-07-12 16:50:08.3680536
\end{shell}

要将文件访问时间作为系统时间或本地时间处理,必须使用clock\_cast<>()(内部可能会将静态file\_clock成员函数调用to\_sys()或to\_utc())。例如:

\begin{cpp}
auto tpFile = std::filesystem::last_write_time(p);
auto tpSys = std::chrono::file_clock::to_sys(tpFile);
auto tpSys = std::chrono::clock_cast<std::chrono::system_clock>(tpFile);
\end{cpp}











