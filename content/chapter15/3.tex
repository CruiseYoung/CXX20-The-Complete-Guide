
“协程promise的特殊成员函数”表列出了promise类型中为协程句柄提供的所有操作。

% Please add the following required packages to your document preamble:
% \usepackage{longtable}
% Note: It may be necessary to compile the document several times to get a multi-page table to line up properly
\begin{longtable}[c]{|l|l|}
\hline
\textbf{操作}       & \textbf{效果}                                          \\ \hline
\endfirsthead
%
\endhead
%
Constructor              & 初始化promise对象                                  \\ \hline
get\_return\_object() &
定义协程返回的对象(通常为协程接口类型) \\ \hline
initial\_suspend()       & 初始挂起点(让协程惰性启动) \\ \hline
yield\_value(val)        & 处理co\_yield的值                        \\ \hline
unhandled\_exception()   & 对协程内未处理异常的反应  \\ \hline
return\_void()           & 处理end或不返回任何值的co\_return  \\ \hline
return\_value(val)       & 处理co\_return的返回值                \\ \hline
final\_suspend()         & 最终挂起点(让协程惰性结束)     \\ \hline
await\_transform(...)    & 将值从co\_await映射到awaiter                   \\ \hline
operator new(sz)         & 定义协程分配内存的方式           \\ \hline
operator delete(ptr, sz) & 定义协程释放内存的方式               \\ \hline
\begin{tabular}[c]{@{}l@{}}get\_return\_object\_on\_...\\ ...allocation\_failure()\end{tabular} &
定义内存分配失败时的反应 \\ \hline
\end{longtable}

\begin{center}
表15.1 协程promise的特殊成员函数
\end{center}

其中一些函数是强制性的,一些函数取决于协程是产生中间结果还是最终结果,还有一些函数是可选的。

\mySubsubsection{15.3.1}{强制性promise操作}

对于协程promise,需要以下成员函数。否则,代码就会无法编译,或者会出现未定义行为。

\mySamllsection{构造函数}

协程promise的构造函数在协程初始化时由协程框架调用。

编译器可以使用构造函数用一些参数初始化协程状态,因此构造函数的签名必须与调用协程时传递给协程的参数相匹配。这种技术特别适用于协同程序特征。

\mySamllsection{get\_return\_object()}

get\_return\_object()由协程框架用于创建协程的返回值。

该函数通常必须返回协程接口,该接口由协程句柄初始化,而协程句柄通常from\_promise()中的静态协程句柄成员函数初始化,该函数本身由promise初始化。

例如:

\begin{cpp}
class CoroTask {
	public:
	// native coroutine handle and its promise type:
	struct promise_type;
	using CoroHdl = std::coroutine_handle<promise_type>;
	CoroHdl hdl;

	struct promise_type {
		auto get_return_object() { // init and return the coroutine interface
			return CoroTask{CoroHdl::from_promise(*this)};
		}
		...
	}
	...
};
\end{cpp}

原则上,get\_return\_object()可以有不同的返回类型:

\begin{itemize}
\item
get\_return\_object()通过使用协程句柄显式初始化协程接口(如上所述)来返回协程接口。

\item
相反,get\_return\_object()也可以返回协程句柄,将隐式地创建协程接口。这要求接口构造函数不是显式的。

目前还不清楚在这种情况下何时创建协程接口(参见\url{http://wg21.link/cwg2563}),所以在调用initial\_suspend()时,协程接口对象可能存在,也可能不存在,很可能出现问题。

因此,最好避免使用这种方法。

\item
极少数情况下,甚至可以不返回任何东西,并将返回类型指定为void。使用协程特征就是一个例子。
\end{itemize}

\mySamllsection{initial\_suspend()}

initial\_suspend()定义协程的初始挂起点,主要用于指定协程是在初始化后自动挂起(惰性启动),还是立即启动(急切启动)。

协程的promise prm会使用该函数,如下所示:

\begin{cpp}
co_await prm.initial_suspend();
\end{cpp}

因此,initial\_suspend()应该返回一个awaiter。

通常,initial\_suspend()会返回

\begin{itemize}
\item
std::suspend\_always\{\},若协程体较晚/惰性地启动,或者

\item
std::suspend\_never\{\},若协程体是立即/急切启动的
\end{itemize}

例如:

\begin{cpp}
class CoroTask {
	public:
	...
	struct promise_type {
		...
		auto initial_suspend() { // suspend immediately
			return std::suspend_always{}; // - yes, always
		}
		...
	}
	...
};
\end{cpp}

然而,也可以让立即启动或延迟启动的决定取决于某些业务逻辑,或者使用此函数来执行协程主体范围的一些初始化。

\mySamllsection{final\_suspend() noexcept}

final\_suspend()定义协程的最终挂起点,并像initial\_suspend()一样调用协程的promise prm,如下所示:

\begin{cpp}
co_await prm.final_suspend();
\end{cpp}

这个函数在调用return\_void()、return\_value()或unhandled\_exception()后由try块外的协程框架调用,try块包含了协程体。因为final\_suspend()在try代码块之外,所以必须是noexcept。

这个成员函数的名称有一点误导,因为它给人的印象是,也可以在这里返回std::suspend\_never\{\}来强制在到达协程末尾后再次执行。然而,在最终挂起点恢复真正挂起的协程是未定义的行为。对于挂起的协程,唯一能做的就是销毁它。

因此,这个成员函数的真正目的是执行一些逻辑,例如发布结果、发出完成信号或在其他地方恢复执行。

相反,建议构建协程,以便在可能的情况下暂停。一个原因是,它使编译器更容易确定协程框架的生命周期何时嵌套在协程的调用者中,这使得编译器更有可能忽略协程框架的堆内存分配。

因此,除非有很好的理由不这样做,否则final\_suspend()应该始终返回std::suspend\_always\{\}。例如:

\begin{cpp}
class CoroTask {
	public:
	...
	struct promise_type {
		...
		auto final_suspend() noexcept { // suspend at the end
			...
			return std::suspend_always{}; // - yes, always
		}
		...
	}
	...
};
\end{cpp}

\mySamllsection{unhandled\_exception()}

或unhandled\_exception()定义协程抛出异常时的反应,该函数在协程框架的catch子句中调用。

对异常的可能反应如下所示:

\begin{itemize}
\item
忽略异常

\item
本地处理异常

\item
结束或终止程序(例如,通过调用std::terminate())

\item
使用std::current\_exception()存储异常以供以后使用
\end{itemize}

稍后将在单独的小节中讨论实现这些反应的方法。

在不结束程序的情况下调用此函数后,将直接调用final\_suspend(),并挂起协程。

若在unhandled\_exception()中抛出或重新抛出异常,协程也会挂起。

\mySubsubsection{15.3.2}{promise操作返回或生成值}

根据是否以及如何使用co\_yield或co\_return,还需要以下一些promise操作。

\mySamllsection{return\_void() 或 return\_value()}

必须实现两个成员函数中的一个来处理return语句和协程的结束。

\begin{itemize}
\item
\textbf{return\_void()}

协程到达终点时调用(到达协程体的终点或到达不带参数的co\_return语句)。

有关详细信息,请参阅第一个协程示例。

\item
\textbf{return\_value(Type)}

当协程到达带有参数的co\_return语句时调用,传递的参数必须或必须转换为指定的类型。

有关详细信息,请参阅带有co\_return的协程示例。
\end{itemize}

若协程的实现方式有时可能返回值,有时可能不返回值,这是未定义行为。考虑下面的例子:

\begin{cpp}
ResultTask<int> coroUB( ... )
{
	if ( ... ) {
		co_return 42;
	}
}
\end{cpp}

这个协程无效,不允许同时拥有return\_void()和return\_value()。[关于是否允许promise在未来的C++标准中同时具有return\_void()和return\_value()的讨论正在进行。]

不幸的是,若只提供return\_value(int)成员函数,这段代码甚至可能会编译并运行,编译器甚至可能不会对此发出警告,但这种情况有望很快改变。

这里可以重载不同类型的return\_value() (void除外)或使其泛型:

\begin{cpp}
struct promise_type {
	...
	void return_value(int val) { // reaction to co_yield for int
		... // - process returned int
	}
	void return_value(std::string val) { // reaction to co_yield for string
		... // - process returned string
	}
};
\end{cpp}

协程可以共同返回不同类型的值。

\begin{cpp}
CoroGen coro()
{
	int value = 0;
	...
	if ( ... ) {
		co_return "ERROR: can't compute value";
	}
	...
	co_return value;
}
\end{cpp}

\mySamllsection{yield\_value(Type)}

若协程到达co\_yield语句,则调用yield\_value(Type)。

有关基本细节,请参阅co\_yield的协程示例。

可以为不同的类型重载yield\_value()或其泛型:

\begin{cpp}
struct promise_type {
	...
	auto yield_value(int val) { // reaction to co_yield for int
		... // - process yielded int
		return std::suspend_always{}; // - suspend coroutine
	}
	auto yield_value(std::string val) { // reaction to co_yield for string
		... // - process yielded string
		return std::suspend_always{}; // - suspend coroutine
	}
};
\end{cpp}

协程可以co\_yield不同类型的值:

\begin{cpp}
CoroGen coro()
{
	while ( ... ) {
		if ( ... ) {
			co_yield "ERROR: can't compute value";
		}
		int value = 0;
		...
		co_yield value;
	}
}
\end{cpp}

\mySubsubsection{15.3.3}{可选的promise操作}

promise也可以用来定义一些可选的操作,这些操作定义了协程的特殊行为,通常会使用一些默认行为。

\mySamllsection{await\_transform()}

可以定义await\_transform()将值从co\_await映射到awaiter。

\mySamllsection{new()和delete()操作符}

new()和delete()操作符允许开发者为协程状态分配内存定义不同的方式。

这些函数还可以用于确保协程不会意外地使用堆内存。

\mySamllsection{get\_return\_object\_on\_allocation\_failure()}

get\_return\_object\_on\_allocation\_failure()允许开发者定义,如何对协程内存分配的无异常失败做出反应。








