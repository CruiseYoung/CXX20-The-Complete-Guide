有时,让程序处理当前正在处理的源代码的位置很重要。这尤其用于记录、测试和检查不变量。到目前为止,开发者需要使用C预处理器宏\_\_FILE\_\_、\_\_LINE\_\_和\_\_func\_\_。C++20为此引入了一个类型安全特性,这样就可以用当前源位置初始化对象,并且可以像传递其他对象一样传递该信息。

用法很简单:

\begin{cpp}
#include <source_location>

void foo()
{
	auto sl = std::source_location::current();
	...
	std::cout << "file: " << sl.file_name() << '\n';
	std::cout << "function: " << sl.function_name() << '\n';
	std::cout << "line/col: " << sl.line() << '/' << sl.column() << '\n';
}
\end{cpp}

静态consteval函数std::source\_location::current()为std::source\_location类型的当前源位置生成一个对象,其接口如下所示:

\begin{itemize}
\item 
file\_name()可产生文件的名称。

\item 
function\_name()生成函数的名称(若在函数之外调用则为空)。

\item 
line()生成行号(当行号未知时可能为0)。

\item 
column()生成行中的列(当列号未知时可能为0)。
\end{itemize}

函数名的确切格式和列的确切位置等细节可能会有所不同。例如,使用gcc库,输出可能如下所示:

\begin{shell}
file:     sourceloc.cpp
function: void foo()
line/col: 8/42
\end{shell}

而Visual C++的输出可能如下所示:

\begin{shell}
file:     sourceloc.cpp
function: foo
line/col: 8/35
\end{shell}

注意,通过在形参声明中使用std::source\_location::current()作为默认实参,将获得函数调用的位置。例如:

\begin{cpp}
void bar(std::source_location sl = std::source_location::current())
{
	...
	std::cout << "file: " << sl.file_name() << '\n';
	std::cout << "function: " << sl.function_name() << '\n';
	std::cout << "line/col: " << sl.line() << '/' << sl.column() << '\n';
}

int main()
{
	...
	bar();
	...
}
\end{cpp}

输出可能是这样的:

\begin{shell}
file:     sourceloc.cpp
function: int main()
line/col: 34/6
\end{shell}

或:

\begin{shell}
file:     sourceloc.cpp
function: main
line/col: 34/3
\end{shell}

因为std::source\_location是一个对象,所以可以将它存储在容器中并四处传递:

\begin{cpp}
std::source_location myfunc()
{
	auto sl = std::source_location::current();
	...
	return sl;
}

int main()
{
	std::vector<std::source_location> locs;
	...
	locs.push_back(myfunc());
	...
	for (const auto& loc : locs) {
		std::cout << "called: " << loc.function_name() << '\n';
	}
}
\end{cpp}

输出可能是:

\begin{shell}
called: ’std::source_location myfunc()’
\end{shell}

或:

\begin{shell}
called: ’myfunc’
\end{shell}

完整的示例请参见lib/sourceloc.cpp。


















