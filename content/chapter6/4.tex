
视图通常具有引用语义。通常,指的是存在于自身之外的范围,所以必须小心,因为只有当底层范围存在并且存储在视图或其迭代器中的对它们的引用有效时,才能使用视图。

\mySubsubsection{6.4.1}{视图及其范围之间的生命周期依赖关系}

所有对作为左值传递的范围(作为第一个构造函数参数或使用管道)进行操作的视图,都在内部存储对传递范围的引用。

所以在使用视图时,底层范围仍然必须存在。这样的代码会导致未定义的行为:

\begin{cpp}
auto getValues()
{
	std::vector coll{1, 2, 3, 4, 5};
	...
	return coll | std::views::drop(2); // ERROR: return reference to local range
}
\end{cpp}

这里返回的是按值放视图。在内部,其引用coll,在getValues()结束时销毁。

这段代码与返回指向局部对象的引用或指针一样糟糕,可能会意外工作或导致致命的运行时错误。不幸的是,编译器(目前)不会对此发出警告。

在右值范围对象上使用视图很好,可以将一个视图返回给一个临时范围对象:

\begin{cpp}
auto getValues()
{
	...
	return std::vector{1, 2, 3, 4, 5} | std::views::drop(2); // OK
}
\end{cpp}

或者可以用std::move()标记底层范围:

\begin{cpp}
auto getValues()
{
	std::vector coll{1, 2, 3, 4, 5};
	...
	return std::move(coll) | std::views::drop(2); // OK
}
\end{cpp}

\mySubsubsection{6.4.2}{具有写访问权限的视图}

视图不应该修改传递的参数或为其调用非const操作。这样,视图及其副本对于相同的输入具有相同的行为。

对于使用辅助函数检查或转换值的视图,这些辅助函数永远不应该修改元素。理想情况下,应该按值或按const引用取值。若修改作为非const引用传递的实参,就会有未定义的行为:

\begin{cpp}
coll | std::views::transform([] (auto& val) { // better declare val as const&
	++val; // ERROR: undefined behavior
})

coll | std::views::drop([] (auto& val) { // better declare val as const&
	return ++val > 0; // ERROR: undefined behavior
})
\end{cpp}

注意,编译器不能检查辅助函数或谓词是否修改了传递的值。视图要求传递的函数或谓词是std::regular\_invocable的(这是std::predicate的隐式要求)。但不修改值是语义约束,在编译时不能总是检查这一点,所以要由开发者来做。

但是,支持使用视图来限制要修改的元素子集。例如:

\begin{cpp}
// assign 0 to all but the first five elements of coll:
for (auto& elem : coll | vws::drop(5)) {
	elem = 0;
}
\end{cpp}

\mySubsubsection{6.4.3}{更改范围的视图}

缓存begin()的视图若没有特别使用,并且底层范围发生了变化,可能会遇到严重的问题(请注意,我写的是在范围和调用begin()仍然有效时发生的问题)。

考虑下面的程序:

\filename{ranges/viewslazy.cpp}

\begin{cpp}
#include <iostream>
#include <vector>
#include <list>
#include <ranges>

void print(auto&& coll)
{
	for (const auto& elem : coll) {
		std::cout << ' ' << elem;
	}
	std::cout << '\n';
}

int main()
{
	std::vector vec{1, 2, 3, 4, 5};
	std::list lst{1, 2, 3, 4, 5};
	
	auto over2 = [](auto v) { return v > 2; };
	auto over2vec = vec | std::views::filter(over2);
	auto over2lst = lst | std::views::filter(over2);
	
	std::cout << "containers and elements over 2:\n";
	print(vec); // OK: 1 2 3 4 5
	print(lst); // OK: 1 2 3 4 5
	print(over2vec); // OK: 3 4 5
	print(over2lst); // OK: 3 4 5
	
	// modify underlying ranges:
	vec.insert(vec.begin(), {9, 0, -1});
	lst.insert(lst.begin(), {9, 0, -1});
	
	std::cout << "containers and elements over 2:\n";
	print(vec); // vec now: 9 0 -1 1 2 3 4 5
	print(lst); // lst now: 9 0 -1 1 2 3 4 5
	print(over2vec); // OOPS: -1 3 4 5
	print(over2lst); // OOPS: 3 4 5
	
	// copying might eliminate caching:
	auto over2vec2 = over2vec;
	auto over2lst2 = over2lst;
	std::cout << "elements over 2 after copying the view:\n";
	print(over2vec2); // OOPS: -1 3 4 5
	print(over2lst2); // OK: 9 3 4 5
	}
\end{cpp}

输出如下:

\begin{shell}
containers and elements over 2:
1 2 3 4 5
1 2 3 4 5
3 4 5
3 4 5
containers and elements over 2:
9 0 -1 1 2 3 4 5
9 0 -1 1 2 3 4 5
-1 3 4 5
3 4 5
elements over 2 after copying the view:
-1 3 4 5
9 3 4 5
\end{shell}

问题在于过滤器视图的缓存。在第一次迭代中,两个视图都缓存视图的开头。注意,这里不同的方式:

\begin{itemize}
\item
对于像vector这样的随机访问范围,视图将偏移量缓存到第一个元素,当再次使用视图时,使begin()失效的重新分配就不会导致未定义的行为。

\item
对于其他范围(如列表),视图实际上会缓存调用begin()的结果。若缓存的元素被删除,则缓存的begin()不再有效,这可能是一个严重的问题。但若前面插入新的元素,也会造成混乱。
\end{itemize}

缓存的总体效果是,对底层范围的进一步修改可能会以各种方式使视图无效:

\begin{itemize}
\item
缓存的begin()可能不再有效。

\item
缓存的偏移量可能在基础范围的末尾之后。

\item
符合谓词的新元素可能不会在以后的迭代中使用。

\item
不适合的新元素可能会被后面的迭代使用。
\end{itemize}

\mySubsubsection{6.4.4}{复制视图可能会改变行为}

最后,请注意,前面的例子演示了复制视图有时可能会使缓存无效:

\begin{itemize}
\item
缓存了begin()的视图有如下输出:

\begin{shell}
-1 3 4 5
3 4 5
\end{shell}

\item
这些视图的副本有不同的输出:

\begin{shell}
-1 3 4 5
9 3 4 5
\end{shell}

\end{itemize}

复制后,视图到vector的缓存偏移量仍然使用,而视图到列表(begin())的缓存begin()则被删除。

缓存有一个额外的后果,即视图的副本可能与其源的状态不同。由于这个原因,在复制视图之前应该三思而行(尽管一个设计目标是通过值来降低复制的成本)。

若这种行为有一个后果,就是:特别使用视图(在视图定义之后立即使用)。

这有点可悲,因为视图的惰性求值。原则上,会允许一些难以置信的用例,这在实践中无法实现,因为代码的行为很难预测,若视图不是特别使用,底层范围可能会改变。

\mySamllsection{过滤器视图的写访问}

当使用过滤器视图时,写访问有一些重要的附加限制:

\begin{itemize}
\item
首先,如前所述,由于过滤器视图缓存,修改后的元素可能会通过过滤器,即使不应该通过。

\item
此外,必须确保修改后的值仍然满足传递给过滤器的谓词。否则,将得到未定义的行为(尽管有时会产生正确的结果)。有关详细信息和示例,请参见过滤器视图的描述。
\end{itemize}

