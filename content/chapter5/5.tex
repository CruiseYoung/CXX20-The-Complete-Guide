

介绍一些标准化的概念,这些概念主要用于实现其他概念。通常,不应该在应用程序代码中使用。

\mySubsubsection{5.5.1}{特定类型属性的概念}

\mySamllsection{std::default\_initializable<T>}

\begin{itemize}
\item
保证类型T支持默认构造(没有初始值的声明/构造)。

\item
要求:
\begin{itemize}
\item
满足std::constructible\_from<T>

\item
满足std::destructible<T>
\item
T\{\};是可用的

\item
T x;是可用的
\end{itemize}
\end{itemize}

\mySamllsection{std::move\_constructible<T>}

\begin{itemize}
\item
保证类型T的对象可以用其类型的右值初始化。

\item
以下操作是有效的(尽管它可以复制而不是移动):

\begin{cpp}
T t2{std::move(t1)} // for any t1 of type T
\end{cpp}

之后,t2的值应该是t1,这是一个在编译时无法检查的语义约束。

\item
要求:
\begin{itemize}
\item
满足std::constructible\_from<T, T>

\item
满足std::convertible<T, T>

\item
满足std::destructible<T>
\end{itemize}
\end{itemize}

\mySamllsection{std::copy\_constructible<T>}

\begin{itemize}
\item
保证类型T的对象可以用其类型的左值初始化。

\item
以下操作是有效的:

\begin{cpp}
T t2{t1} // for any t1 of type T
\end{cpp}

之后,t2的值应该是t1,这是一个在编译时无法检查的语义约束。

\item
要求:
\begin{itemize}
\item
满足std::move\_constructible<T>

\item
std::constructible\_from和std::convertible\_to适用于任何T、T\&、const T和const T\&转换成的T

\item
满足std::destructible<T>
\end{itemize}
\end{itemize}

\mySamllsection{std::destructible<T>}

\begin{itemize}
\item
保证类型T的对象可析构而不抛出异常。

注意,即使是实现的析构函数也会自动保证不抛出,除非显式地用noexcept(false)标记。

\item
要求:
\begin{itemize}
\item
类型特性std::is\_nothrow\_destructible\_v<T>为true
\end{itemize}
\end{itemize}

\mySamllsection{std::swappable<T>}

\begin{itemize}
\item
保证可以交换两个T类型对象的值。

\item
要求:
\begin{itemize}
\item
std::ranges::swap()可以为两个类型为T的对象调用
\end{itemize}
\end{itemize}

\mySubsubsection{5.5.2}{可增量类型的概念}

\mySamllsection{std::weakly\_incrementable<T>}

\begin{itemize}
\item
保证类型T支持自增操作符。

\item
注意,这个概念并不要求相同值的两次增量产生相同的结果,所以这只是单向迭代的要求。

\item
与std::incrementable相比,若满足以下条件,这个概念也能得到满足:

\begin{itemize}
\item
该类型不是默认可构造的、不可复制的或不可相等比较的

\item
后置自增运算符返回void(或任何其他类型)。

\item
相同值的两次增量会产生不同的结果
\end{itemize}

\item
注意,与概念std::incrementable的区别纯粹是语义上的区别,所以对于那些增量产生不同结果的类型,可能在技术上仍然满足增量概念。要针对这种语义差异实现不同的行为,应该使用迭代器概念。

\item
要求:
\begin{itemize}
\item
满足std::default\_initializable<T>

\item
满足std::movable<T>

\item
std::iter\_difference\_t<T>是一个有效的有符号整型
\end{itemize}
\end{itemize}

\mySamllsection{std::incrementable<T>}

\begin{itemize}
\item
保证类型T是一个可递增的类型,这样您就可以对相同的值序列进行多次迭代。

\item
与std::weak\_incrementable不同,这个概念要求:

\begin{itemize}
\item
相同值的两次增量产生相同的结果(就像前向迭代器的情况一样)

\item
类型T是默认可构造的、可复制的和可比较的

\item
后置自增操作符返回迭代器的副本(返回类型为T)。
\end{itemize}

\item
注意,与概念std::weak \_incrementable的区别纯粹是语义上的区别,所以于增量产生不同结果的类型,在技术上可能仍然满足概念incrementable。要针对这种语义差异实现不同的行为,应该使用迭代器概念。

\item
要求:
\begin{itemize}
\item
满足std::weakly\_incrementable<T>

\item
满足std::regular<T>,使得该类型是默认可构造的、可复制的和可比较的
\end{itemize}
\end{itemize}
