

需求表达式(不同于requires子句)提供了一种简单而灵活的语法,用于在一个或多个模板参数上指定多个需求:

\begin{itemize}
\item
必需的类型定义

\item
表达式必须有效

\item
对表达式产生类型的要求
\end{itemize}

表达式以requires开头,后跟一个可选的参数列表,然后是一个需求块(都以分号结束)。例如:

\begin{cpp}
template<typename Coll>
... requires {
		typename Coll::value_type::first_type; // elements/values have first_type
		typename Coll::value_type::second_type; // elements/values have second_type
	}
\end{cpp}

可选参数列表允许引入一组“虚拟变量”,可用于在表达式的主体中表达需求:

\begin{cpp}
template<typename T>
... requires(T x, T y) {
		x + y; // supports +
		x - y; // supports -
	}
\end{cpp}

这些形参永远不会由实参取代,所以通过值或引用声明都可以。

参数还允许引入子类型(参数):

\begin{cpp}
template<typename Coll>
... requires(Coll::value_type v) {
		std::cout << v; // supports output operator
	}
\end{cpp}

要求检查Coll::value\_type是否有效,以及该类型的对象是否支持输出操作符。

此参数列表中的类型成员不必用typename限定。

当使用此方法检查Coll::value\_type是否有效时,在需求块的主体中不需要任何内容,但不能为空,所以可以简单地使用true:

\begin{cpp}
template<typename Coll>
... requires(Coll::value_type v) {
		true; // dummy requirement because the block cannot be empty
	}
\end{cpp}

\mySubsubsection{4.4.1}{简单的需求}

简单的需求就是必须格式良好的表达式,必须进行编译。调用不会执行,从而操作产生的结果并不重要。

\begin{cpp}
template<typename T1, typename T2>
... requires(T1 val, T2 p) {
		*p; // operator* has to be supported for T2
		p[0]; // operator[] has to be supported for int as index
		p->value(); // calling a member function value() without arguments has to be possible
		*p > val; // support the comparison of the result of operator* with T1
		p == nullptr; // support the comparison of a T2 with a nullptr
	}
\end{cpp}

最后一个调用不要求p是nullptr(要做到这一点,必须检查T2是否是类型std::nullptr\_t)。相反,可以要求将T2类型的对象与nullptr类型的对象进行比较。

通常使用运算符||没什么意义。一个简单的需求,例如

\begin{cpp}
*p > val || p == nullptr;
\end{cpp}

不要求左子表达式或右子表达式可能成立,其规定了可以用运算符||组合两个子表达式结果的要求。

要满足这两个子表达式中的一个,必须使用:

\begin{cpp}
template<typename T1, typename T2>
... requires(T1 val, T2 p) {
		*p > val; // support the comparison of the result of operator* with T1
	}
	|| requires(T2 p) { // OR
		p == nullptr; // support the comparison of a T2 with nullptr
	}
\end{cpp}

注意,这个概念并不要求T是整型:

\begin{cpp}
template<typename T>
... requires {
		std::integral<T>; // OOPS: does not require T to be integral
		...
	};
\end{cpp}

概念只要求表达式std::integral<T>有效,这是所有类型的情况。T是整型的要求必须这样表述:

\begin{cpp}
template<typename T>
... std::integral<T> && // OK, does require T to be integral
	requires {
		...
	};
\end{cpp}

或者:

\begin{cpp}
template<typename T>
... requires {
		requires std::integral<T>; // OK, does require T to be integral
		...
	};
\end{cpp}

\mySubsubsection{4.4.2}{类型的需求}

类型需求是在使用类型名称时必须格式良好的表达式,所以名称必须定义为有效类型。

\begin{cpp}
template<typename T1, typename T2>
... requires {
	typename T1::value_type; // type member value_type required for T1
	typename std::ranges::iterator_t<T1>; // iterator type required for T1
	typename std::common_type_t<T1, T2>; // T1 and T2 have to have a common type
}
\end{cpp}

对于所有类型需求,若类型存在但为空,则满足需求。

只能检查给定类型的名称(类名、枚举类型的名称,来自typedef或using),不能使用类型检查其他类型声明:

\begin{cpp}
template<typename T>
... requires {
	typename int; // ERROR: invalid type requirement
	typename T&; // ERROR: invalid type requirement
}
\end{cpp}

测试后者的方法是声明相应的形参:

\begin{cpp}
template<typename T>
... requires(T&) {
	true; // some dummy requirement
};
\end{cpp}

同样,需求检查使用传递的类型来定义另一个类型是否有效:

\begin{cpp}
template<std::integral T>
class MyType1 {
	...
};

template<typename T>
requires requires {
	typename MyType1<T>; // instantiation of MyType1 for T would be valid
}
void mytype1(T) {
	...
}

mytype1(42); // OK
mytype1(7.7); // ERROR
\end{cpp}

因此,以下需求不检查类型T是否存在标准哈希函数:

\begin{cpp}
template<typename T>
concept StdHash = requires {
	typename std::hash<T>; // does not check whether std::hash<> is defined for T
};
\end{cpp}

需要标准哈希函数的方法是尝试创建或使用:

\begin{cpp}
template<typename T>
concept StdHash = requires {
	std::hash<T>{}; // OK, checks whether we can create a standard hash function for T
};
\end{cpp}

注意,简单的需求只检查需求是否有效,而不检查需求是否满足。出于这个原因:

\begin{itemize}
\item
使用总是产生值的类型函数没有意义:

\begin{cpp}
template<typename T>
... requires {
	std::is_const_v<T>; // not useful: always valid (doesn’t matter what it yields)
}
\end{cpp}

要检查const性,需要使用:

\begin{cpp}
template<typename T>
... std::is_const_v<T> // ensure that T is const
\end{cpp}

requires表达式中,可以使用嵌套的需求(见下文)。

\item
使用产生类型的类型函数没有意义:

\begin{cpp}
template<typename T>
... requires {
	typename std::remove_const_t<T>; // not useful: always valid (yields a type)
}
\end{cpp}

需求只检查类型表达式是否产生类型,这总是正确的。
\end{itemize}

使用可能具有未定义行为的类型函数也没有意义。类型特性std::make\_unsigned<>要求传递的参数是整型,而非bool型。若传递的类型不是整型,则有未定义行为,所以不应该使用std::make\_unsigned<>作为需求,而不限制调用其类型:

\begin{cpp}
template<typename T>
... requires {
	std::make_unsigned<T>::type; // not useful as type requirement (valid or undefined behavior)
}
\end{cpp}

这种情况下,需求只能满足或导致未定义行为(这可能需求仍然满足)。相反,应该约束可以使用嵌套需求的类型T:

\begin{cpp}
template<typename T>
... requires {
	requires (std::integral<T> && !std::same_as<T, bool>);
	std::make_unsigned<T>::type; // OK
}
\end{cpp}

\mySubsubsection{4.4.3}{复合需求}

复合需求允许将简单需求和类型需求结合起来,可以指定一个表达式(大括号内),然后添加以下一个或两个:

\begin{itemize}
\item
noexcept要求表达式保证不抛出异常

\item
-> type-constraint将概念应用于表达式的求值
\end{itemize}

下面是一些例子:

\begin{cpp}
template<typename T>
... requires(T x) {
	{ &x } -> std::input_or_output_iterator;
	{ x == x };
	{ x == x } -> std::convertible_to<bool>;
	{ x == x }noexcept;
	{ x == x }noexcept -> std::convertible_to<bool>;
}
\end{cpp}

->之后的类型约束将结果类型作为其第一个模板参数:

\begin{itemize}
\item
第一个需求中,需求在对类型为T的对象使用运算符\&时满足std::input\_or\_output\_iterator的概念(std::input\_or\_output\_iterator<decltype(\&x)>产生true)。

也可以这样指定:

\begin{cpp}
{ &x } -> std::is_pointer_v<>;
\end{cpp}

\item
最后一个需求中,需求可以将两个T类型对象的==操作符的结果用作bool(当将两个T类型对象和bool类型对象的==操作符的结果作为参数传递时,满足std::convertible\_to的概念)。
\end{itemize}

需求表达式还可以表达对关联类型的需求:

\begin{cpp}
template<typename T>
... requires(T coll) {
	{ *coll.begin() } -> std::convertible_to<T::value_type>;
}
\end{cpp}

但不能使用嵌套类型指定类型需求,例如:不能使用它们来要求使用解引用操作符的返回值产生整数值。其中的问题是,返回值是一个引用,必须先解引用:

\begin{cpp}
std::integral<std::remove_reference_t<T>>
\end{cpp}

而且不能在requires表达式的结果中,使用带有类型特性的嵌套表达式:

\begin{cpp}
template<typename T>
concept Check = requires(T p) {
	{ *p } -> std::integral<std::remove_reference_t<>>; // ERROR
	{ *p } -> std::integral<std::remove_reference_t>; // ERROR
};
\end{cpp}

要么先定义相应的概念:

\begin{cpp}
template<typename T>
concept UnrefIntegral = std::integral<std::remove_reference_t<T>>;

template<typename T>
concept Check = requires(T p) {
	{ *p } -> UnrefIntegral; // OK
};
\end{cpp}

要么使用嵌套需求。

\mySubsubsection{4.4.4}{嵌套需求}

嵌套需求可用于在requires表达式中指定附加约束。以requires开头,后跟一个编译时布尔表达式,该表达式本身也可能是或使用requires表达式。嵌套需求的好处是,可以确保编译时表达式(使用所需表达式的参数或子表达式)产生特定的结果,而不是仅仅确保表达式有效。

例如,考虑一个概念,必须确保对于给定类型,解引用和[](下标)操作符产生相同的类型。通过使用嵌套需求,可以这样指定:

\begin{cpp}
template<typename T>
concept DerefAndIndexMatch = requires (T p) {
								requires std::same_as<decltype(*p),
										decltype(p[0])>;
							};
\end{cpp}

好的方面是,对于“假设有一个T类型的对象”,这里有一个简单的语法,不必在这里使用requires表达式,但代码必须使用std::declval<>():

\begin{cpp}
template<typename T>
concept DerefAndIndexMatch = std::same_as<decltype(*std::declval<T>()),
			 decltype(std::declval<T>()[0])>;
\end{cpp}

另一个例子,可以使用嵌套需求来解决刚才,在表达式上指定复杂类型需求的问题:

\begin{cpp}
template<typename T>
concept Check = requires(T p) {
	requires std::integral<std::remove_cvref_t<decltype(*p)>>;
};
\end{cpp}

请注意requires表达式中的区别:

\begin{cpp}
template<typename T>
... requires {
	!std::is_const_v<T>; // OOPS: checks whether we can call is_const_v<>
	requires !std::is_const_v<T>; // OK: checks whether T is not const
}
\end{cpp}

这里,使用的类型特性是is\_const\_v<>带/不带requires,而只有第二个需求有用:

\begin{itemize}
\item
第一个表达式只要求检查const性并对结果取反有效,这个需求总是可满足(即使T是const int)。因为做这个检查总是有效的,所以这个需求毫无价值。

\item
第二个需求表达式必须满足。若T是int则满足要求,但若T是const int则不满足。
\end{itemize}

