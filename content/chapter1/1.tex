首先了解一下C++20处理比较的新方式和新操作符<=>的动机。

\mySubsubsection{1.1.1}{C++20前如何定义比较运算符}

C++20之前,必须为一个类型定义六个操作符,以提供对象所有比较的支持。

例如,若要比较Value类型的对象(具有整型ID),则须实现以下操作:

\begin{cpp}
class Value {
private:
	long id;
	...
public:
	...
	// equality operators:
	bool operator== (const Value& rhs) const {
		return id == rhs.id; // basic check for equality
	}
	bool operator!= (const Value& rhs) const {
		return !(*this == rhs); // derived check
	}
	// relational operators:
	bool operator< (const Value& rhs) const {
		return id < rhs.id; // basic check for ordering
	}
	bool operator<= (const Value& rhs) const {
		return !(rhs < *this); // derived check
	}
	bool operator> (const Value& rhs) const {
		return rhs < *this; // derived check
	}
	bool operator>= (const Value& rhs) const {
		return !(*this < rhs); // derived check
	}
};
\end{cpp}

当使用另一个Value(作为参数rhs传递)调用Value(为其定义操作符的对象)的六个比较操作符中的一个。例如:

\begin{cpp}
Value v1, v2;
... ;
if (v1 <= v2) { // calls v1.operator<=(v2)
	...
}
\end{cpp}

操作符也可以间接调用(例如,使用sort()):

\begin{cpp}
std::vector<Value> coll;
... ;
std::sort(coll.begin(), coll.end()); // uses operator < to sort
\end{cpp}

C++20开始,可以使用range:

\begin{cpp}
std::ranges::sort(coll); // uses operator < to sort
\end{cpp}

问题是,尽管大多数操作符都是根据其他操作符定义的(都基于operator ==或operator <),但这些定义很繁琐,而且会增加很多阅读上的混乱。

此外,对于实现良好的类型,可能需要更多声明:

\begin{itemize}
\item
若操作符不能抛出,就用noexcept声明

\item
若操作符可以在编译时使用,则使用constexpr声明

\item
若构造函数不是显式的,则将操作符声明为“隐藏友元”(在类结构中与友元一起声明,以便两个操作数都成为参数,并支持隐式类型转换)

\item
声明带有[[nodiscard]]的操作符,以便在返回值未使用时强制发出警告
\end{itemize}

例如:

\begin{cpp}
// lang/valueold.hpp

class Value {
private:
	long id;
	...
public:
	constexpr Value(long i) noexcept // supports implicit type conversion
	: id{i} {
	}
	...
	// equality operators:
	[[nodiscard]] friend constexpr
	bool operator== (const Value& lhs, const Value& rhs) noexcept {
		return lhs.id == rhs.id; // basic check for equality
	}
	[[nodiscard]] friend constexpr
	bool operator!= (const Value& lhs, const Value& rhs) noexcept {
		return !(lhs == rhs); // derived check for inequality
	}
	// relational operators:
	[[nodiscard]] friend constexpr
	bool operator< (const Value& lhs, const Value& rhs) noexcept {
		return lhs.id < rhs.id; // basic check for ordering
	}
	[[nodiscard]] friend constexpr
	bool operator<= (const Value& lhs, const Value& rhs) noexcept {
		return !(rhs < lhs); // derived check
	}
	[[nodiscard]] friend constexpr
	bool operator> (const Value& lhs, const Value& rhs) noexcept {
		return rhs < lhs; // derived check
	}
	[[nodiscard]] friend constexpr
	bool operator>= (const Value& lhs, const Value& rhs) noexcept {
		return !(lhs < rhs); // derived check
	}
};
\end{cpp}

\mySubsubsection{1.1.2}{C++20后如何定义比较运算符}

C++20后,关于比较操作符的定义发生了一些变化。

\mySamllsection{==与!=}

为了检查是否相等,现在定义==操作符就够了。

当编译器找不到表达式的匹配声明a!=b时,编译器会重写表达式并查找!(a==b)。若这不起作用,编译器也会尝试改变操作数的顺序,所以它也会尝试!(b==a):

\begin{cpp}
a != b // tries: a!=b, !(a==b), and !(b==a)
\end{cpp}

因此,对于TypeA的a和TypeB的b,编译器将能够识别并编译

\begin{cpp}
a != b
\end{cpp}

若需要的话,可以这样做

\begin{itemize}
\item
一个独立的operator!=(TypeA, TypeB)

\item
一个独立的operator==(TypeA, TypeB)

\item
一个独立的operator==(TypeB, TypeA)

\item
一个成员函数TypeA::operator!=(TypeB)

\item
一个成员函数TypeA::operator==(TypeB)

\item
一个成员函数TypeB::operator==(TypeA)
\end{itemize}

直接调用已定义的operator!=是首选(但类型的顺序必须合适),更改操作数的顺序为最低的优先级。同时拥有独立函数和成员函数是一种歧义错误。

因此,

\begin{cpp}
bool operator==(const TypeA&, const TypeB&);
\end{cpp}

或

\begin{cpp}
class TypeA {
public:
	...
	bool operator==(const TypeB&) const;
};
\end{cpp}

编译器可以进行编译:

\begin{cpp}
MyType a;
MyType b;
...
a == b; // OK: fits perfectly
b == a; // OK, rewritten as: a == b
a != b; // OK, rewritten as: !(a == b)
b != a; // OK, rewritten as: !(a == b)
\end{cpp}

当重写将操作数转换为已定义成员函数的参数时,也可以对第一个操作数进行隐式类型转换。

请参阅示例sentinel1.cpp,了解如何在调用具有不同顺序的操作数的!=时仅定义成员操作符==。

\mySamllsection{操作符<=>}

对于所有的关系操作符,没有等价的规则说定义operator<就足够了。但现在,只需要定义新的操作符<=>即可。

事实上,以下内容足以让开发者使用所有可能的比较操作符:

\begin{cpp}
// lang/value20.hpp

#include <compare>
class Value {
private:
	long id;
	...
public:
	constexpr Value(long i) noexcept
	: id{i} {
	}
	...
	// enable use of all equality and relational operators:
	auto operator<=> (const Value& rhs) const = default;
};
\end{cpp}

通常,operator==可以通过定义==和!=来处理对象的相等性,而operator <=>通过定义关系操作符来处理对象的顺序。若通过=default声明操作符<=>,则可以使用了一个特殊的规则,即默认成员操作符<=>:

\begin{cpp}
class Value {
	...
	auto operator<=> (const Value& rhs) const = default;
};
\end{cpp}

生成对应的成员操作符==,从而得到:

\begin{cpp}
class Value {
	...
	auto operator<=> (const Value& rhs) const = default;
	auto operator== (const Value& rhs) const = default; // implicitly generated
};
\end{cpp}

结果是两个操作符都使用了默认实现,逐个成员对象进行比较,所以类中成员的顺序很重要。

因此,

\begin{cpp}
class Value {
	...
	auto operator<=> (const Value& rhs) const = default;
};
\end{cpp}

至此,我们得到了能够使用所有六个比较操作符所需的一切。

此外,即使将操作符声明为成员函数,也适用于生成的操作符:

\begin{itemize}
\item
若比较成员不抛出异常,则是noexcept

\item
若可以在编译时比较成员,则是constexpr

\item
因为重写,还可以支持第一个操作数的隐式类型转换
\end{itemize}

所以,通常情况下,operator==和operator<=>处理不同但相关的事情:

\begin{itemize}
\item
operator==定义相等性,可由相等操作符==和!=使用。

\item
operator<=>定义了排序,可以由关系操作符<、<=、>和>=使用。
\end{itemize}

注意,当默认或使用操作符<=>时,必须包含头文件<compare>。

\begin{cpp}
#include <compare>
\end{cpp}

不过,大多数标准类型(字符串、容器、<utility>)的头文件都会包含这个头文件。

\mySamllsection{操作符<=>的实现}

为了更好地控制生成的比较操作符,可以自己定义operator==和operator<=>。例如:

\begin{cpp}
// lang/value20def.hpp

#include <compare>
class Value {
private:
	long id;
	...
public:
	constexpr Value(long i) noexcept
	: id{i} {
	}
	...
	// for equality operators:
	bool operator== (const Value& rhs) const {
		return id == rhs.id; // defines equality (== and !=)
	}
	// for relational operators:
	auto operator<=> (const Value& rhs) const {
		return id <=> rhs.id; // defines ordering (<, <=, >, and >=)
	}
};
\end{cpp}

可以指定哪个成员以哪个顺序重要或实现特殊行为。

这些基本操作符的工作方式是,若表达式使用其中一个比较操作符,并且没有找到匹配的直接定义,则重写表达式,以便使用这些操作符。

与重写相等操作符调用相对应,重写也可能改变关系操作数的顺序,从而可能对第一个操作数启用隐式类型转换。例如:

\begin{cpp}
x <= y
\end{cpp}

没有找到操作符<=的匹配定义,可以重写为

\begin{cpp}
(x <=> y) <= 0
\end{cpp}

甚至

\begin{cpp}
0 <= (y <=> x)
\end{cpp}

通过重写可以看到,操作符<=>执行一个三向比较,生成一个可以与0比较的值:

\begin{itemize}
\item
若x<=>y的值等于0,则x和y等于或相等。

\item
若x<=>y小于0,则x小于y.

\item
若x<=>y大于0,则x大于y。
\end{itemize}

但请注意,操作符<=>的返回类型不是整数值。返回类型是表示比较类别的类型,可以是强排序、弱排序或部分排序。不过,这些类型支持与0进行比较。










