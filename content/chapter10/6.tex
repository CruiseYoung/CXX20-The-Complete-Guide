
格式化库可以为用户定义的类型定义格式化。需要定义一个格式化器,也很容易实现。

\mySubsubsection{10.6.1}{基本格式器的API}

格式器是类型的类模板std::formatter<>的特化。在格式化程序内部,必须定义两个成员函数:

\begin{itemize}
\item
parse()实现如何解析类型的格式字符串说明符

\item
format()为自定义类型的对象/值执行实际格式化
\end{itemize}

来看一下第一个最小的例子(我们将逐步改进它),指定了如何格式化具有固定值的对象/值。假设类型是这样定义的(见format/always40.hpp):

\begin{cpp}
class Always40 {
	public:
	int getValue() const {
		return 40;
	}
};
\end{cpp}

对于这种类型,可以定义第一个格式化器(应该改进)如下:

\filename{format/formatalways40.hpp}

\begin{cpp}
#include "always40.hpp"
#include <format>
#include <iostream>

template<>
struct std::formatter<Always40>
{
	// parse the format string for this type:
	constexpr auto parse(std::format_parse_context& ctx) {
		return ctx.begin(); // return position of } (hopefully there)
	}
	
	// format by always writing its value:
	auto format(const Always40& obj, std::format_context& ctx) const {
		return std::format_to(ctx.out(), "{}", obj.getValue());
	}
};
\end{cpp}

这已经足够好,可以进行如下工作:

\begin{cpp}
Always40 val;
std::cout << std::format("Value: {}\n", val);
std::cout << std::format("Twice: {0} {0}\n", val);
\end{cpp}

输出为:

\begin{shell}
Value: 40
Twice: 40 40
\end{shell}

我们将格式化器定义为类型std::formatter<>的特化,用于类型Always40:

\begin{cpp}
template<>
struct std::formatter<Always40>
{
	...
};
\end{cpp}

因为只有public成员,所以使用struct,而非class。


\mySamllsection{解析格式字符串}

在parse()中,我们实现了解析格式字符串的函数:

\begin{cpp}
// parse the format string for this type:
constexpr auto parse(std::format_parse_context& ctx) {
	return ctx.begin(); // return position of } (hopefully there)
}
\end{cpp}

该函数接受一个std::format\_parse\_context,提供了一个API来遍历传递的格式字符串的剩余字符。Ctx.begin()指向要解析的值的格式说明符的第一个字符,若没有说明符,则指向\}:

\begin{itemize}
\item
若格式字符串为 "Value: \{:7.2f\}"
ctx.begin()指向: "7.2f\}"

\item
若格式字符串为 "Twice: \{0\} \{0\}"
ctx.begin()指向: "\} \{0\}"
第一次调用时

\item
若格式字符串为 "\{\}\verb|\|n"
ctx.begin()指向: "\}\verb|\|n"
\end{itemize}

还有一个ctx.end(),它指向整个格式字符串的末尾。开始的\{已经解析过了,所以必须解析所有的字符,直到对应的结束的\}。

对于格式字符串"Val: \{1:\_>20\}cm \verb|\|n",ctx.begin()是\_的位置,ctx.end()是\verb|\|n之后整个格式字符串的末尾。parse()的任务是解析传递的参数的指定格式,所以必须解析字符\_>20,然后返回格式说明符末尾的位置,即字符0后面的末尾\}。

我们的实现中,还不支持任何格式说明符,所以可以简单地返回得到的第一个字符的位置,只有当下一个字符确实是\}时才会起作用(在结束\}之前处理字符是必须改进的第一件事)。使用任何指定的格式字符调用std::format()将不起作用:

\begin{cpp}
Always40 val;
std::format("'{:7}", val) // ERROR
\end{cpp}

注意,parse()成员函数应该是constexpr,以支持格式字符串的编译时计算,并且代码必须接受constexpr函数的所有限制(C++20放宽了这些限制)。

但可以看到这个API如何允许程序员解析我们为其类型指定的任何格式。例如,用于支持chrono库的格式化输出。当然,我们应该遵循标准说明符的约定,以避免混淆。

\mySamllsection{执行格式化}

在format()中,实现了格式化传递值的函数:

\begin{cpp}
// format by always writing its value:
auto format(const Always40& value, std::format_context& ctx) const {
	return std::format_to(ctx.out(), "{}", value.getValue());
}
\end{cpp}

该函数接受两个形参:

\begin{itemize}
\item
作为参数传递给std::format()(或类似函数)的值。

\item
一个std::format\_context,提供API来编写格式化的结果字符(根据已解析的格式)
\end{itemize}

format上下文最重要的函数是out(),它会产生一个对象,可以将其传递给std::format\_to()来编写格式化的实际字符。该函数必须返回新的位置以获得进一步的输出,该输出由std::format\_to()返回。

注意,格式化器的format()成员函数应该是const。根据最初的C++20标准,这是不需要的(参见\url{http://wg21.link/lwg3636}了解详细信息)。

\mySubsubsection{10.6.2}{改进解析}

改进一下前面看到的例子。首先,应该确保解析器,以更好的方式处理格式说明符:

\begin{itemize}
\item
应该处理到结尾\}之前的所有字符。

\item
当指定非法格式化参数时,应该抛出异常。

\item
应该处理有效的格式化参数(比如:指定的字段的宽度)。
\end{itemize}

来看看对前一个格式化器的改进版本(这次处理的类型值总是为41):

\filename{format/formatalways41.hpp}

\begin{cpp}
#include "always41.hpp"
#include <format>
template<>
class std::formatter<Always41>
{
	int width = 0; // specified width of the field
public:
	// parse the format string for this type:
	constexpr auto parse(std::format_parse_context& ctx) {
		auto pos = ctx.begin();
		while (pos != ctx.end() && *pos != '}') {
		if (*pos < '0' || *pos > '9') {
			throw std::format_error{std::format("invalid format '{}'", *pos)};
		}
		width = width * 10 + *pos - '0'; // new digit for the width
		++pos;
		}
		return pos; // return position of }
	}
	
	// format by always writing its value:
	auto format(const Always41& obj, std::format_context& ctx) const {
		return std::format_to(ctx.out(), "{:{}}", obj.getValue(), width);
	}
};
\end{cpp}

格式化器现在有了一个成员来存储指定字段的宽度:

\begin{cpp}
template<>
class std::formatter<Always41>
{
	int width = 0; // specified width of the field
	...
};
\end{cpp}

字段的宽度初始化为0,但可以通过格式字符串指定。

解析器现在有一个循环,处理所有字符,直到结尾的\}:

\begin{cpp}
constexpr auto parse(std::format_parse_context& ctx) {
	auto pos = ctx.begin();
	while (pos != ctx.end() && *pos != '}') {
		...
		++pos;
	}
	return pos; // return position of }
}
\end{cpp}

注意,循环必须检查是否还有一个字符,以及它是否是末尾的\},因为调用std::format()的开发者可能忘记了末尾的\}。

在循环中,将当前宽度乘以数字字符的整数值:

\begin{cpp}
width = width * 10 + *pos - '0'; // new digit for the width
\end{cpp}

若字符不是数字,std::format()初始化抛出std::format异常:

\begin{cpp}
if (*pos < '0' || *pos > '9') {
	throw std::format_error{std::format("invalid format '{}'", *pos)};
}
\end{cpp}

注意,这里不能使用std::isdigit(),其不是一个可以在编译时调用的函数。

可以测试这样的格式化器:format/always.cpp

该程序有以下输出:

\begin{shell}
41
Value: 41
Twice: 41 41
With width: ’      41’
Format Error: invalid format ’f’
\end{shell}

该值是右对齐的,这是整数值的默认对齐方式。

\mySubsubsection{10.6.3}{使用标准格式化程序用于自定义格式器}

我们仍然可以改进上面实现的格式化器:

\begin{itemize}
\item
可以使用对齐说明符。

\item
可以支持填充字符。
\end{itemize}

幸运的是,我们不必自己实现完整的解析。相反,可以使用标准格式化器,从它们支持的格式说明符中获益。

实际上,有两种方法:

\begin{itemize}
\item
可以将工作委托给本地标准格式化器。

\item
可以从标准格式化器继承。
\end{itemize}

\mySamllsection{将格式化委托给标准格式化器}

要将格式化委托给标准格式化器,必须

\begin{itemize}
\item
声明一个本地标准格式化器

\item
将parse()函数将工作委托给标准格式化器

\item
将format()函数将工作委托给标准格式化器
\end{itemize}

一般来说,应该如下所示:

\filename{format/formatalways42ok.hpp}

\begin{cpp}
#include "always42.hpp"
#include <format>

// *** formatter for type Always42:
template<>
struct std::formatter<Always42>
{
	// use a standard int formatter that does the work:
	std::formatter<int> f;
	
	// delegate parsing to the standard formatter:
	constexpr auto parse(std::format_parse_context& ctx) {
		return f.parse(ctx);
	}
	
	// delegate formatting of the value to the standard formatter:
	auto format(const Always42& obj, std::format_context& ctx) const {
		return f.format(obj.getValue(), ctx);
	}
};
\end{cpp}

像往常一样,我们为Always42类型声明std::formatter<>的特化。但这一次,使用int类型的本地标准格式化程序来完成这项工作,将解析和格式化都委托给它。实际上,我们使用getValue()从类型中提取值,并使用标准int格式化器完成其余的格式化工作。

可以用下面的程序测试格式化器:

\filename{format/always42.cpp}

\begin{cpp}
#include "always42.hpp"
#include "formatalways42.hpp"
#include <iostream>

int main()
{
	try {
		Always42 val;
		std::cout << val.getValue() << '\n';
		std::cout << std::format("Value: {}\n", val);
		std::cout << std::format("Twice: {0} {0}\n", val);
		std::cout << std::format("With width: '{:7}'\n", val);
		std::cout << std::format("With all: '{:.^7}'\n", val);
	}
	catch (std::format_error& e) {
		std::cerr << "Format Error: " << e.what() << std::endl;
	}
}
\end{cpp}

该程序有以下输出:

\begin{shell}
42
Value: 42
Twice: 42 42
With width: ’     42’
With all:   ’..42...’
\end{shell}

注意,默认情况下,该值仍然是右对齐的,这是int的默认对齐方式。

实践中,可能需要对这段代码进行一些修改,稍后将详细讨论:

\begin{itemize}
\item
将format()声明为const可能无法编译,除非将格式化器声明为mutable。

\item
将parse()声明为constexpr可能无法编译。
\end{itemize}

\mySamllsection{继承标准格式器}

通常,可以从标准格式化器派生,以便格式化器成员及其parse()函数隐式可用:

\filename{format/formatalways42inherit.hpp}

\begin{cpp}
#include "always42.hpp"
#include <format>

// *** formatter for type Always42:
// - use standard int formatter
template<>
struct std::formatter<Always42> : std::formatter<int>
{
	auto format(const Always42& obj, std::format_context& ctx) {
		// delegate formatting of the value to the standard formatter:
		return std::formatter<int>::format(obj.getValue(), ctx);
	}
};
\end{cpp}

但请注意,在实践中,可能还需要对这段代码进行一些修改:

\begin{itemize}
\item
将format()声明为const可能无法编译。
\end{itemize}

\mySamllsection{在实践中使用标准格式化器}

在实践中,C++20的标准化存在一些问题,之后必须澄清一些事情:

\begin{itemize}
\item
最初标准化的C++20并不要求格式化器的format()成员函数必须是const(参见\url{http://wg21.link/lwg3636}了解详细信息)。为了支持不将format()声明为const成员函数的C++标准库实现,必须将其声明为非const函数或将本地格式化器声明为mutable。

\item
现有的实现可能还不支持parse()成员函数为constexpr的编译时解析,因为编译时解析是在C++20标准化之后添加的(参见\url{http: //wg21.link/p2216r3})。这种情况下,不能将编译时解析委托给标准格式化器。
\end{itemize}

在实践中,类型Always42的格式化程序可能必须如下所示:

\filename{format/formatalways42.hpp}

\begin{cpp}
#include "always42.hpp"
#include <format>

// *** formatter for type Always42:
template<>
struct std::formatter<Always42>
{
	// use a standard int formatter that does the work:
#if __cpp_lib_format < 202106
	mutable // in case the standard formatters have a non-const format()
#endif
	std::formatter<int> f;
	
	// delegate parsing to the standard int formatter:
#if __cpp_lib_format >= 202106
	constexpr // in case standard formatters don’t support constexpr parse() yet
#endif
	auto parse(std::format_parse_context& ctx) {
		return f.parse(ctx);
	}
	
	// delegate formatting of the int value to the standard int formatter:
	auto format(const Always42& obj, std::format_context& ctx) const {
		return f.format(obj.getValue(), ctx);
	}
};
\end{cpp}

如您所见,

\begin{itemize}
\item
若采用了相应的修复,则仅使用constexpr声明parse()

\item
可以使用mutable声明本地格式化器,以便const format()成员函数可以调用非const的标准format()函数
\end{itemize}

对于两者,实现都使用了一个特性测试宏,该宏表示支持编译时解析(期望采用也使标准格式化器的format()成员函数为const)。

\mySubsubsection{10.6.4}{为字符串使用标准格式化器}

若要格式化更复杂的类型,一种常见的方法是创建一个字符串,然后使用字符串的标准格式化程序(若只使用字符串字面值,则使用std::string或std::string\_view)。

例如,可以定义枚举类型和格式化器,如下所示:

\filename{format/color.hpp}

\begin{cpp}
#include <format>
#include <string>

enum class Color { red, green, blue };

// *** formatter for enum type Color:
template<>
struct std::formatter<Color> : public std::formatter<std::string>
{
	auto format(Color c, format_context& ctx) const {
		// initialize a string for the value:
		std::string value;
		switch (c) {
			using enum Color;
			case red:
				value = "red";
				break;
			case green:
				value = "green";
				break;
			case blue:
				value = "blue";
				break;
			default:
				value = std::format("Color{}", static_cast<int>(c));
				break;
		}
		// and delegate the rest of formatting to the string formatter:
		return std::formatter<std::string>::format(value, ctx);
	}
};
\end{cpp}

通过从字符串格式化器继承格式化器,继承了它的parse()函数,所以支持字符串具有的所有格式说明符。在format()函数中,执行到字符串的映射,然后让标准格式化器完成其余的格式化工作。

可以这样使用格式化器:

\filename{format/color.cpp}

\begin{cpp}
#include "color.hpp"
#include <iostream>
#include <string>
#include <format>

int main()
{
	for (auto val : {Color::red, Color::green, Color::blue, Color{13}}) {
		// use user-provided formatter for enum Color:
		std::cout << std::format("Color {:_>8} has value {:02}\n",
								  val, static_cast<int>(val));
	}
}
\end{cpp}

该程序有以下输出:

\begin{shell}
Color _____red has value 00
Color ___green has value 01
Color ____blue has value 02
Color _Color13 has value 13
\end{shell}

若不引入自定义的格式说明符,这种方法通常可以很好地工作。若只使用字符串字面值作为可能的值,甚至可以使用std::string\_view的格式化器。




