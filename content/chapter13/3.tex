

C++20引入了两个新的原子类型(处理引用和共享指针)和原子类型的新特性。另外,我们现在有了一个类型std::atomic<char8\_t>。

\mySubsubsection{13.3.1}{原子引用std::atomic\_ref<>}

C++11起,标准库提供了类模板std::atomic<>来为普通可复制类型提供原子API。

C++20现在引入了类模板std::atomic\_ref<>,为普通的可复制引用类型提供原子API,允许为通常不是原子的现有对象提供临时原子API。一种应用程序是初始化对象而不关心并发性,然后在不同的线程中使用它。

之所以给这个类型指定一个唯一的名称atomic\_ref,而不只是为引用类型提供std::atomic<>,是因为用户应该看到这个对象可能提供非原子访问,而且比std::atomic<>的保障更弱。

\mySamllsection{使用原子引用的例子}

下面的程序演示了如何使用原子引用:

\filename{lib/atomicref.cpp}

\begin{cpp}
#include <iostream>
#include <array>
#include <algorithm> // for std::fill_n()
#include <vector>
#include <format>
#include <random>
#include <thread>
#include <atomic> // for std::atomic_ref<>
using namespace std::literals; // for duration literals

int main()
{
	// create and initialize an array of integers with the value 100:
	std::array<int, 1000> values;
	std::fill_n(values.begin(), values.size(), 100);
	
	// initialize a common stop token for all threads:
	std::stop_source allStopSource;
	std::stop_token allStopToken{allStopSource.get_token()};
	
	// start multiple threads concurrently decrementing the value:
	std::vector<std::jthread> threads;
	for (int i = 0; i < 9; ++i) {
		threads.push_back(std::jthread{
			[&values] (std::stop_token st) {
				// initialize random engine to generate an index:
				std::mt19937 eng{std::random_device{}()};
				std::uniform_int_distribution distr{0, int(values.size()-1)};
				
				while (!st.stop_requested()) {
					// compute the next index:
					int idx = distr(eng);
					
					// enable atomic access to the value with the index:
					std::atomic_ref val{values[idx]};
					
					// and use it:
					--val;
					if (val <= 0) {
						std::cout << std::format("index {} is zero\n", idx);
					}
				}
			},
			allStopToken // pass the common stop token
		});
	}
	
	// after a while/event request to stop all threads:
	std::this_thread::sleep_for(0.5s);
	std::cout << "\nSTOP\n";
	allStopSource.request_stop();
	...
}
\end{cpp}

首先创建并初始化一个包含1000个整数值的数组,不使用原子:

\begin{cpp}
std::array<int, 1000> values;
std::fill_n(values.begin(), values.size(), 100);
\end{cpp}

稍后启动多个线程并发地递减这些值。关键是,只有在这种上下文中,才将值用作原子整数。为此,需要初始化std::atomic\_ref<>:

\begin{cpp}
std::atomic_ref val{values[idx]};
\end{cpp}

注意,由于类模板实参推导,不必指定所引用对象的类型。

这个初始化的效果是,当使用原子引用时,对值的访问是原子地进行的:

\begin{itemize}
\item 
--val以原子的方式递减

\item 
val <= 0以原子方式加载值,将其与0进行比较(表达式在读取值后使用隐式类型转换到基础类型)
\end{itemize}

可以考虑将后者实现为val.load() <= 0,以使用原子接口。

注意,这个程序的不同线程不使用相同的atomic\_ref<>对象。std::atomic\_ref<>保证通过为特定对象创建的任何atomic\_ref<>对该对象的所有并发访问都是同步的。[如何确保这一点取决于标准库的实现。]若需要互斥锁或锁,则可以使用全局哈希表,其中为包装对象的每个地址存储一个关联的锁。这与为用户定义类型实现std::atomic<>的方式没有什么不同。

\mySamllsection{原子引用的特性}

原子引用的头文件也是<atomic>。对于原始指针、整型和浮点型(C++20起)的std::atomic<>的特化也在该头文件中:

\begin{cpp}
namespace std {
	template<typename T> struct atomic_ref; // primary template
	
	template<typename T> struct atomic_ref<T*>; // partial specialization for pointers
	
	template<> struct atomic_ref<integralType>; // full specializations for integral types
	
	template<> struct atomic_ref<floatType>; // full specializations for floating-point types
}
\end{cpp}

与std::atomic<>相比,原子引用有以下限制:

\begin{itemize}
\item 
不支持volatile

\item
对象是指可能需要对齐,通常需要大于对齐的基础类型。静态成员std::atomic\_ref<type>::required\_alignment提供了这种最小对齐方式。
\end{itemize}

与std::atomic<>相比,原子引用有以下扩展:

\begin{itemize}
\item 
复制构造函数,用于创建对相同基础对象的另一个引用(但提供赋值操作符仅用于赋值基础值)。

\item 
常量不会传播到包装对象,所以可以给const std::atomic\_ref<>赋一个新值:

\begin{cpp}
MyType x, y;
const std::atomic_ref cr{x};
cr = y; // OK (would not be OK for const std::atomic<>)
\end{cpp}

\item 
线程同步支持使用wait()、notify\_one()和notify\_all(),就像现在为所有原子类型提供的那样。
\end{itemize}

其他方面,提供了与std::atomic<>相同的特性:

\begin{itemize}
\item 
该类型既提供具有内存屏障的高级API,也提供禁用它们的底层API。

\item 
静态成员is\_always\_lock\_free,非静态成员函数is\_lock\_free()表示原子支持是否无锁,这可能仍然取决于所使用的对齐方式。
\end{itemize}

\mySamllsection{原子引用的详情}

原子引用提供与相应原子类型相同的API。对于普通的可复制类型、指针类型、整型类型或浮点类型T, c++标准库提供了std::atomic\_ref<T>与std::atomic<T>相同的原子API。原子引用类型也在头文件<atomic>中提供。

所以,使用静态成员is\_always\_lock\_free()或非静态成员函数is\_lock\_free()时,可以检查原子类型是否在内部使用锁作为原子。若没有,则有对原子操作的本机硬件支持(这是在信号处理程序中使用原子的先决条件)。类型的检查如下所示:

\begin{cpp}
if constexpr(std::atomic<int>::is_always_lock_free) {
	...
}
else {
	... // special handling if locks are used
}
\end{cpp}

对特定对象的检查(是否无锁可能取决于对齐)如下所示:

\begin{cpp}
std::atomic_ref val{values[idx]};
if (val.is_lock_free()) {
	...
}
else {
	... // special handling if locks are used
}
\end{cpp}

注意,对于类型T,尽管类型std::atomic<T>是无锁的,std::atomic\_ref<T>可能不是无锁的。

由原子引用引用的对象可能还必须满足特定于架构的约束,例如:对象可能需要在内存中正确对齐,或者可能不允许缓存在GPU寄存器内存中。对于所需的对齐,有一个静态成员:

\begin{cpp}
namespace std {
	template<typename T> struct atomic_ref {
		static constexpr size_t required_alignment;
		...
	};
}
\end{cpp}

该值至少为alignof(T)。但该值可以是,例如2*alignof(double),以支持std::complex<double>的无锁操作。

\mySubsubsection{13.3.2}{原子共享指针}

C++11引入了带有可选原子接口的共享指针。使用像atomic\_load()、atomic\_store()和atomic\_exchange()这样的函数,可以并发访问共享指针引用的值。然而,问题是可以将这些共享指针与非原子接口一起使用,这将破坏共享指针的所有原子使用。

C++20现在为共享指针和弱指针提供了偏特化:

\begin{itemize}
\item 
std::atomic<std::shared\_ptr<T>{}>

\item 
std::atomic<std::weak\_ptr<T>{}>
\end{itemize}

以前用于共享指针的原子API现在已弃用。

注意,原子共享/弱指针不提供原子原始指针提供的额外操作,提供的API与std::atomic<>一般为普通可复制类型T提供的API相同。

提供了wait()、notify()\_one()和notify\_all()的线程同步支持。还支持具有使用内存顺序参数选项的低层原子接口。

\mySamllsection{使用原子共享指针的例子}

下面的示例演示了原子共享指针的用法,原子共享指针用作共享值链表的头。

\filename{lib/atomicshared.cpp}

\begin{cpp}
#include <iostream>
#include <thread>
#include <memory> // includes <atomic> now
using namespace std::literals; // for duration literals

template<typename T>
class AtomicList {
	private:
	struct Node {
		T val;
		std::shared_ptr<Node> next;
	};
	std::atomic<std::shared_ptr<Node>> head;
public:
	AtomicList() = default;
	
	void insert(T v) {
		auto p = std::make_shared<Node>();
		p->val = v;
		p->next = head;
		while (!head.compare_exchange_weak(p->next, p)) { // atomic update
		}
	}
	
	void print() const {
		std::cout << "HEAD";
		for (auto p = head.load(); p; p = p->next) { // atomic read
			std::cout << "->" << p->val;
		}
		std::cout << std::endl;
	}
};

int main()
{
	AtomicList<std::string> alist;
	
	// populate list with elements from 10 threads:
	{
		std::vector<std::jthread> threads;
		for (int i = 0; i < 100; ++i) {
			threads.push_back(std::jthread{[&, i]{
					for (auto s : {"hi", "hey", "ho", "last"}) {
						alist.insert(std::to_string(i) + s);
						std::this_thread::sleep_for(5ns);
					}
			}});
		}
	} // wait for all threads to finish
	
	alist.print(); // print resulting list
	}
\end{cpp}

该程序可能有以下输出:

\begin{shell}
HEAD->94last->94ho->76last->68last->57last->57ho->60last->72last-> ... ->1hey->1hi
\end{shell}

和通常的原子操作一样,也可以这样写:

\begin{cpp}
for (auto p = head.load(); p; p = p->next)
\end{cpp}

而非:

\begin{cpp}
for (auto p = head.load(); p.load(); p = p->next)
\end{cpp}

\mySamllsection{使用原子弱指针的例子}

下面的例子演示了原子弱指针的用法:

\filename{lib/atomicweak.cpp}

\begin{cpp}
#include <iostream>
#include <thread>
#include <memory> // includes <atomic> now
using namespace std::literals; // for duration literals

int main()
{
	std::atomic<std::weak_ptr<int>> pShared; // pointer to current shared value (if one exists)
	
	// loop to set shared value for some time:
	std::atomic<bool> done{false};
	std::jthread updates{[&] {
							for (int i = 0; i < 10; ++i) {
								{
									auto sp = std::make_shared<int>(i);
									pShared.store(sp); // atomic update
									std::this_thread::sleep_for(0.1s);
								}
								std::this_thread::sleep_for(0.1s);
							}
							done.store(true);
					}};
				
	// loop to print shared value (if any):
	while (!done.load()) {
		if (auto sp = pShared.load().lock()) { // atomic read
			std::cout << "shared: " << *sp << '\n';
		}
		else {
			std::cout << "shared: <no data>\n";
		}
		std::this_thread::sleep_for(0.07s);
	}
}
\end{cpp}

注意,在这个程序中,不必使共享指针原子化,因为它只让一个线程使用。唯一的问题是两个线程并发地更新或使用弱指针。

该程序可能有以下输出:

\begin{shell}
shared: <no data>
shared: 0
shared: <no data>
shared: 1
shared: <no data>
shared: <no data>
shared: 2
shared: <no data>
shared: <no data>
shared: 3
shared: <no data>
shared: 4
shared: 4
shared: <no data>
shared: 5
shared: 5
shared: <no data>
shared: 6
shared: <no data>
shared: <no data>
shared: 7
shared: <no data>
shared: 8
shared: 8
shared: <no data>
shared: 9
shared: 9
shared: <no data>
\end{shell}

和通常的原子操作一样,也可以这样写:

\begin{cpp}
pShared = sp;
\end{cpp}

而非:

\begin{cpp}
pShared.store(sp);
\end{cpp}

\mySubsubsection{13.3.3}{浮点原子类型}

std::atomic<>和std::atomic\_ref<>现在都为float、double和long double类型提供了完整的特化。

与任意可复制类型的主模板相比,其提供了额外的原子操作来添加和减去值[与整型的特化相反,整型的特化还提供了对自增/自减值的原子支持,并可执行按位修改]:

\begin{itemize}
\item 
fetch\_add(), fetch\_sub()

\item 
operator+=, operator-=
\end{itemize}

因此,现在可以做到以下几点:

\begin{cpp}
std::atomic<double> d{0};
...
d += 10.3; // OK since C++20
\end{cpp}

\mySubsubsection{13.3.4}{使用原子类型的线程同步}

所有原子类型(std::atomic<>、std::atomic\_ref<>和std::atomic\_flag)现在都提供了一个简单的API,让线程阻塞并等待其他线程对其值的更改。

因此,对于原子值:

\begin{cpp}
std::atomic<int> aVal{100};
\end{cpp}

或者原子引用:

\begin{cpp}
int value = 100;
std::atomic_ref<int> aVal{value};
\end{cpp}

可以定义想要等待的值,直到引用的值已经改变:

\begin{cpp}
int lastValue = aVal.load();
aVal.wait(lastValue); // block unless/until value changed (and notified)
\end{cpp}

若引用对象的值与传递的参数不匹配,则立即返回。否则,会阻塞,直到为原子值或引用调用了notify\_one()或notify\_all():

\begin{cpp}
--aVal; // atomically modify the (referenced) value
aVal.notify_all(); // notify all threads waiting for a change
\end{cpp}

但对于条件变量,wait()可能会由于伪唤醒而结束(因此没有调用通知)。因此您应该在wait()之后再次检查该值。

等待特定原子值的代码如下所示:

\begin{cpp}
while ((int val = aVal.load()) != expectedVal) {
	aVal.wait(val);
	// here, aVal may or may not have changed
}
\end{cpp}

请注意,不能保证会将获得所有更新。考虑下面的程序:

\filename{lib/atomicwait.cpp}

\begin{cpp}
#include <iostream>
#include <thread>
#include <atomic>
using namespace std::literals;
int main()
{
	std::atomic<int> aVal{0};
	// reader:
	std::jthread tRead{[&] {
						int lastX = aVal.load();
						while (lastX >= 0) {
							aVal.wait(lastX);
							std::cout << "=> x changed to " << lastX << std::endl;
							lastX = aVal.load();
						}
						std::cout << "READER DONE" << std::endl;
				}};

	// writer:
	std::jthread tWrite{[&] {
						for (int newVal : { 17, 34, 3, 42, -1}) {
							std::this_thread::sleep_for(5ns);
							aVal = newVal;
							aVal.notify_all();
						}
				}};
	...
}
\end{cpp}

输出可能是:

\begin{shell}
=> x changed to 17
=> x changed to 34
=> x changed to 3
=> x changed to 42
=> x changed to -1
READER DONE
\end{shell}

或:

\begin{shell}
=> x changed to 17
=> x changed to 3
=> x changed to -1
READER DONE
\end{shell}

或者是:

\begin{shell}
READER DONE
\end{shell}

注意,通知函数是const成员函数。

\mySamllsection{使用原子通知的公平票务系统}

使用原子wait()和通知的一个应用是像使用互斥锁一样使用,因为使用互斥锁的成本可能会高得多。

下面是一个例子,使用原子来实现队列中值的公平处理(与使用信号量的不公平版本相比)。虽然可能有多个线程等待,但只有有限数量的线程可以运行。通过使用一个售票系统,确保队列中的元素按顺序处理:[这个例子的想法是基于Bryce Adelstein Lelbach在CppCon 2029上的演讲《C++20同步库》中的一个例子(参见\url{http://youtu.be/Zcqwb3CWqs4?t=1810})。]

\filename{lib/atomicticket.cpp}

\begin{cpp}
#include <iostream>
#include <queue>
#include <chrono>
#include <thread>
#include <atomic>
#include <semaphore>
using namespace std::literals; // for duration literals
int main()
{
	char actChar = 'a'; // character value iterating endless from ’a’ to ’z’
	std::mutex actCharMx; // mutex to access actChar
	// limit the availability of threads with a ticket system:
	std::atomic<int> maxTicket{0}; // maximum requested ticket no
	std::atomic<int> actTicket{0}; // current allowed ticket no
	// create and start a pool of numThreads threads:
	constexpr int numThreads = 10;
	std::vector<std::jthread> threads;
	for (int idx = 0; idx < numThreads; ++idx) {
		threads.push_back(std::jthread{[&, idx] (std::stop_token st) {
										while (!st.stop_requested()) {
											// get next character value:
											char val;
											{
												std::lock_guard lg{actCharMx};
												val = actChar++;
												if (actChar > 'z') actChar = 'a';
												}
												
												// request a ticket to process it and wait until enabled:
												int myTicket{++maxTicket};
												int act = actTicket.load();
												while (act < myTicket) {
												actTicket.wait(act);
												act = actTicket.load();
												}
												
												// print the character value 10 times:
												for (int i = 0; i < 10; ++i) {
												std::cout.put(val).flush();
												auto dur = 20ms * ((idx % 3) + 1);
												std::this_thread::sleep_for(dur);
												}
												
												// done, so enable next ticket:
												++actTicket;
												actTicket.notify_all();
											}
										}});
	}
	
	// enable and disable threads in the thread pool:
	auto adjust = [&, oldNum = 0] (int newNum) mutable {
					actTicket += newNum - oldNum; // enable/disable tickets
					if (newNum > 0) actTicket.notify_all(); // wake up waiting threads
					oldNum = newNum;
					};
					
	for (int num : {0, 3, 5, 2, 0, 1}) {
		std::cout << "\n====== enable " << num << " threads" << std::endl;
		adjust(num);
		std::this_thread::sleep_for(2s);
	}
	
	for (auto& t : threads) { // request all threads to stop (join done when leaving scope)
		t.request_stop();
	}
}
\end{cpp}

每个线程请求下一个字符值,然后请求一个票据来处理这个值:

\begin{cpp}
int myTicket{++maxTicket};
\end{cpp}

然后线程等待,直到票据启用:

\begin{cpp}
int act = actTicket.load();
while (act < myTicket) {
	actTicket.wait(act);
	act = actTicket.load();
}
\end{cpp}

每次线程唤醒时,都会仔细检查它的票据是否启用(actTicket至少是myTicket)。通过增加actTicket的值并通知所有等待线程来启用新票据。

这发生在线程完成处理时:

\begin{cpp}
++actTicket;
actTicket.notify_all();
\end{cpp}

或者当启用了新数量的票时:

\begin{cpp}
actTicket += newNum - oldNum; // enable/disable tickets
if (newNum > 0) actTicket.notify_all(); // wake up waiting threads
\end{cpp}

该示例可能有以下输出:

\begin{shell}
====== enable 0 threads

====== enable 3 threads
acbabacabacbaacbabacbacdbdbdcdbdcdecdeddcedegfgegfegegfgegfgegfegfhhfhfhhfhfhijhjjhijhji
jkjkijkjkijkkliklkilkkilmmlimlmilmmlnmlmnml
====== enable o5 threads
pmpnoqrpropnqpronpqorppnorqppornqsrosnqrsosnrqosrsntqstustvqstuvstqtvwuwtvxwtxuvwtxwtxvu
wyxwvyxuwvxwyxuvwyxzxvuyzabyuzbabyzubyabzybczabyzbcabdzbdcazdeedczadedecfafdefcedaefdedf
caefgfecghfigfichfgijhcjgijhgkijjkgihjkgjihjkgij
====== enable khg2 threads
ijkihkhkhkkllmllmlmlllmllmnnmnnmnnmnnmnmnoopoopoopoopoopqpqqpqpqpqpqqrrqrqrrsrsrrsrsrsts
tsts
====== enable 0 threads
ststttttt
====== enable 1 threads
uuuuuuuuuuvvvvvvvvvvwwwwwwwwwwxxxxxxxxxxyyyyyyyyyyzzzzzzzzzzaaaaaaaaaabbbbbbbbbbcccccccc
ccddddddddddeeeeeeeeeeffffffffffgggggggggghhhhhhhhhh
\end{shell}

使用同步输出流来确保输出没有交错字符。

\mySubsubsection{13.3.5}{std::atomic\_flag的扩展}

C++20之前,若不设置std::atomic\_flag,就无法检查它的值,因此C++20添加了全局函数和成员函数来检查当前值:

\begin{itemize}
\item 
atomic\_flag\_test(const atomic\_flag*) noexcept;

\item 
atomic\_flag\_test\_explicit(const atomic\_flag*, memory\_order) noexcept;

\item 
atomic\_flag::test() const noexcept;

\item 
atomic\_flag::test(memory\_order) const noexcept;
\end{itemize}