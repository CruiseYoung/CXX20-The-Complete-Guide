
C++20提供了大量不同的视图。其中一些封装了现有源的元素,其中一些本身生成值,还有许多对元素或值进行操作(过滤或转换)。

本章简要概述了C++ 20中所有可用的视图类型。请注意,对于几乎所有这些类型,都有辅助的范围适配器/工厂,允许开发者通过调用函数或管道来创建视图。通常,应该使用这些适配器和工厂,而不是直接初始化视图,因为适配器和工厂执行额外的优化(例如在不同的视图中选择最佳的),可以在更多的情况下工作,对需求进行双重检查,并且更容易使用。

\mySubsubsection{8.1.1}{概述包装和生成视图}

表“包装和生成视图”列出了只能作为管道源元素的标准视图。

这些观点可能是

\begin{itemize}
\item
包装视图,对外部资源的元素序列进行操作(例如:从输入流读取的容器或元素)

\item
工厂视图,自己生成元素
\end{itemize}


\begin{longtable}[c]{|l|l|l|}
\hline
\textbf{类型}             & \textbf{适配器/工厂}                                              & \textbf{效果}                                \\ \hline
\endfirsthead
%
\endhead
%
std::ranges::red\_view    & all(rg)                                                               & 引用一个范围                           \\ \hline
std::ranges::owning\_view & all(rg)                                                               & 包含范围的视图                        \\ \hline
std::ranges::subrange     & counted(beg, sz)                                                      & 开始迭代器和哨兵                    \\ \hline
std::span                 & counted(beg, sz)                                                      & 指向连续内存和大小的开始迭代器 \\ \hline
std::ranges::iota\_view   & \begin{tabular}[c]{@{}l@{}}iota(val)\\ iota(val, endVal)\end{tabular} & 递增值的生成器                \\ \hline
std::ranges::single\_view & single(val)                                                           & 视图只拥有一个值                  \\ \hline
std::ranges::empty\_view  & empty\textless{}T\textgreater{}                                       & 没有元素的视图                          \\ \hline
\begin{tabular}[c]{@{}l@{}}std::ranges::basic\_istream\_view\\   std::ranges::istream\_view\textless{}\textgreater\\   std::ranges::wistream\_view\textless{}\textgreater{}\end{tabular} &
\begin{tabular}[c]{@{}l@{}}istream\textless{}T\textgreater{}(strm)\\ -\end{tabular} &
\begin{tabular}[c]{@{}l@{}}从流中读取元素\\   从char流\\   从wchar\_t流\end{tabular} \\ \hline
\begin{tabular}[c]{@{}l@{}}std::basic\_string\_view\\   std::string\_view\\   std::u8string\_view\\   std::u16string\_view\\   std::u32string\_view\\   std::wstring\_view\end{tabular} &
\begin{tabular}[c]{@{}l@{}}-\\ -\\ -\\ -\\ -\\ -\end{tabular} &
\begin{tabular}[c]{@{}l@{}}字符的只读视图\\   转换成char序列\\   转换为char8\_t序列\\   转换为char16\_t序列\\   转换为char32\_t序列\\   转换为wchar\_t序列\end{tabular} \\ \hline
\end{longtable}

\begin{center}
表 8.1 包装和生成视图
\end{center}

该表列出了视图的类型和可用于创建视图的范围适配器/工厂的名称(若有的话)。

对于字符串视图,在C++17中作为std::basic\_string\_view<>类型引入,C++提供了别名类型std::string\_view, std::u8string\_view, std::u16string\_view, std::u32string\_view和std::wstring\_view。

对于istream视图,其类型std::basic\_istream\_view<>,C++提供了别名std::istream\_view和std::wistream\_view。

注意,视图类型使用不同的命名空间:

\begin{itemize}
\item
span和string\_view在命名空间std中。

\item
所有范围适配器和工厂都在命名空间std::views中提供(std::ranges::views的别名)。

\item
所有其他视图类型都在命名空间std::ranges中提供。
\end{itemize}

一般来说,最好使用范围适配器和工厂。例如,你应该总是倾向于使用适配器std::views::all(),而不是直接使用类型std::ranges::ref\_view<>和std::ranges::owning\_view<>。然而,在某些情况下,没有提供适配器/工厂,必须直接使用视图类型。最重要的例子是当您必须从一对迭代器创建视图时。在这种情况下,必须直接初始化std::ranges::子范围。

\mySubsubsection{8.1.2}{自适应视图概述}

表“适应视图”列出了以某种方式适应给定范围的元素的标准视图(过滤掉元素、修改元素的值、改变元素的顺序,或者组合或创建子范围),尤其可以在视图的管道中使用。


\begin{longtable}[c]{|l|l|l|}
\hline
\textbf{类型}                & \textbf{适配器} & \textbf{效果}                               \\ \hline
\endfirsthead
%
\endhead
%
std::ranges::ref\_view       & all(rg)          & 引用一个范围                          \\ \hline
std::ranges::owning\_view    & all(rg)          & 包含范围的视图                       \\ \hline
std::ranges::take\_view      & take(num)        & 第一个(最多)num个元素                \\ \hline
std::ranges::take\_while\_view & take\_while(pred)                    & 匹配谓词的所有前置元素                   \\ \hline
std::ranges::drop\_view      & drop(num)        & 除第一个num元素外的所有元素            \\ \hline
std::ranges::drop\_while\_view & drop\_while(pred)                    & 除了与谓词匹配的前导元素之外的所有元素            \\ \hline
std::ranges::filter\_view    & filter(pred)     & 匹配谓词的所有元素           \\ \hline
std::ranges::transform\_view & transform(func)  & 所有元素转换后的值        \\ \hline
std::ranges::elements\_view    & elements\textless{}idx\textgreater{} & 所有元素的idxth的成员/属性                       \\ \hline
std::ranges::keys\_view      & keys             & 所有元素的第一个成员              \\ \hline
std::ranges::values\_view    & values           & 所有元素的第二个成员             \\ \hline
std::ranges::reverse\_view   & reverse          & 所有元素按倒序排列                 \\ \hline
std::ranges::split\_view     & split(sep)       & 范围中的所有元素被分割为子范围 \\ \hline
std::ranges::lazy\_split\_view & lazy\_split(sep)                     & 输入范围或常量范围的所有元素拆分为子范围 \\ \hline
std::ranges::join\_view      & join             & 多个范围的范围中的所有元素    \\ \hline
std::ranges::common\_view      & common()                             & 迭代器和哨兵中所有类型相同的元素          \\ \hline
\end{longtable}

\begin{center}
表8.2 适应视图
\end{center}

同样,推荐使用范围适配器,而不是直接使用视图类型。例如,应该总是倾向于使用适配器std::views::take(),而不是类型std::ranges::take\_view<>,因为当可以简单地跳转到底层范围的第n个元素时,适配器可能根本不会创建take视图。作为另一个例子,应该总是更喜欢使用适配器std::views::common(),而不是类型std::ranges::common\_view<>,因为只有适配器允许传递已经通用的范围(只是接受它们的原样)。用common\_view包装公共范围会导致编译时错误。




