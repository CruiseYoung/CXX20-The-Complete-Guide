
引入了新的比较规则之后,事实证明C++20引入了一些问题,当从旧C++版本进行切换时,这些问题就会暴露。

\mySubsubsection{1.6.1}{委托独立比较操作符}

下面的例子演示了最典型问题:

\filename{lang/spacecompat.cpp}

\begin{cpp}
#include <iostream>

class MyType {
private:
	int value;
public:
	MyType(int i) // implicit constructor from int:
		: value{i} {
	}
	bool operator==(const MyType& rhs) const {
		return value == rhs.value;
	}
};

bool operator==(int i, const MyType& t) {
	return t == i; // OK with C++17
}
int main()
{
	MyType x = 42;
	if (0 == x) {
		std::cout << "'0 == MyType{42}' works\n";
	}
}
\end{cpp}

有一个简单的类,存储整型值,并有一个隐式构造函数来初始化对象(隐式构造函数是支持使用=初始化的必要条件):

\begin{cpp}
class MyType {
	public:
	MyType(int i); // implicit constructor from int
	...
};

MyType x = 42; // OK
\end{cpp}

类还声明了一个成员函数来比较对象:

\begin{cpp}
class MyType {
	...
	bool operator==(const MyType& rhs) const;
	...
};
\end{cpp}

但C++20之前,这只允许对第二个操作数进行隐式类型转换,所以类或其他代码引入了一个全局操作符来交换参数的顺序:

\begin{cpp}
bool operator==(int i, const MyType& t) {
	return t == i; // OK until C++17
}
\end{cpp}

对于类来说,最好将operator==()定义为“隐藏的友元”(在类结构中使用友元定义它,以便两个操作符都成为参数,可以直接访问成员,并且只有在至少有一个参数类型适合时才执行隐式类型转换),但上面的代码在C++20之前具有相同的效果。

不幸的是,这段代码在C++20中不再可用,[感谢Peter Dimov和Barry Revzin指出了这个问题,并在\url{http://stackoverflow.com/questions/65648897}上进行了讨论]其导致了无尽的递归。在全局函数内部,因为表达式t == i,编译器会试图将调用重写为i == t,所以也可以调用全局operator==()本身:

\begin{cpp}
bool operator==(int i, const MyType& t) {
	return t == i; // finds operator==(i,t) in addition to t.operator(MyType{i})
}
\end{cpp}

因为不需要隐式类型转换,所以重写的语句是更好的匹配项。我们目前还没有解决方案来支持这里的向后兼容性,但编译器已经开始对这样的代码发出警告了。

若的代码只使用C++20,则可以简单地删除独立函数。否则,就只有两个选择:

\begin{itemize}
\item
使用显式转换:

\begin{cpp}
bool operator==(int i, const MyType& t) {
	return t == MyType{i}; // OK until C++17 and with C++20
}
\end{cpp}

\item
新特性可用时,使用特性测试宏来禁用代码:
\end{itemize}

\mySubsubsection{1.6.2}{protected成员的继承}

对于具有默认比较操作符的派生类,若基类将操作符作为protected成员,则可能会出现问题。

注意,默认的比较操作符需要在基类中支持比较,以下操作不起作用:

\begin{cpp}
struct Base {
};

struct Child : Base {
	int i;
	bool operator==(const Child& other) const = default;
};

Child c1, c2;
...
c1 == c2; // ERROR
\end{cpp}

因此,还必须在基类中提供默认操作符==,但若不希望基类为public提供操作符==,那么显而易见的方法是在基类中添加protected的默认操作符==:

\begin{cpp}
struct Base {
	protected:
	bool operator==(const Base& other) const = default;
};
struct 
Child : Base {
	int i;
	bool operator==(const Child& other) const = default;
};
\end{cpp}

但在这种情况下,派生类的默认比较不起作用。其会被当前实现默认操作符的指定行为的编译器拒绝,这里的默认操作符过于严格。这个问题有望很快得到解决(参见\url{http://wg21.link/cwg2568})。

作为一种解决方法,开发者必须自己实现派生操作符。




