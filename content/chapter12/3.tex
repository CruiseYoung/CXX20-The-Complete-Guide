
“jthread类对象的操作”表列出了std::thread的API。列Diff记录了Mod(修改)了行为或与std::thread相比是New的成员函数。

% Please add the following required packages to your document preamble:
% \usepackage{longtable}
% Note: It may be necessary to compile the document several times to get a multi-page table to line up properly
\begin{longtable}[c]{|l|l|l|}
\hline
\textbf{操作}      & \textbf{效果}                                                                                        & \textbf{Diff} \\ \hline
\endfirsthead
%
\endhead
%
jthread t               & 默认构造函数;创建一个不可汇入的线程对象                                               &               \\ \hline
jthread t\{f,...\} &
\begin{tabular}[c]{@{}l@{}}创建一个表示新线程的对象,该线程调用f(带有附加参数)\\或抛出std::system\_error \end{tabular} &
\\ \hline
jthread t\{rv\}         & \begin{tabular}[c]{@{}l@{}}移动构造函数;创建一个新的线程对象,该对象获取rv的状\\态,并使rv不可连接  \end{tabular}   &               \\ \hline
t.$\sim$jthread()       & 析构函数;若对象是可连接的,调用request\_stop()和join()                                 & Mod           \\ \hline
t = rv                  & \begin{tabular}[c]{@{}l@{}}移动赋值;将rv的状态移动赋值给t(若t是可连接的,则调\\用request\_stop()和join())) \end{tabular} & Mod          \\ \hline
t.joinable()            & 若有关联的线程(可连接)则返回true                                                &               \\ \hline
t.join() &
\begin{tabular}[c]{@{}l@{}}等待相关线程完成并使对象不可汇入(若线程不可汇入则抛\\出std::system\_error) \end{tabular} &
\\ \hline
t.detach() &
\begin{tabular}[c]{@{}l@{}}线程继续运行时释放t与其线程的关联,并使对象不可汇入\\(若线程不可连接则抛出std::system\_error) \end{tabular} &
\\ \hline
t.request\_stop()       & 关联的停止令牌上请求停止                                                           & New           \\ \hline
t.get\_stop\_source()   & 从请求停止生成一个对象来                                                               & New           \\ \hline
t.get\_stop\_token()    & 生成一个对象来检查请求的停止                                                         & New           \\ \hline
t.get\_id()             & \begin{tabular}[c]{@{}l@{}}若可汇入,则返回成员类型ID的唯一线程ID;若不可汇入,则\\返回默认构造ID   \end{tabular}     &               \\ \hline
t.native\_handle() &
\begin{tabular}[c]{@{}l@{}}返回一个平台特定的成员类型native\_handle\_type,用于不可移\\植的线程处理 \end{tabular} &
\\ \hline
t1.swap(t2)             & 交换t1和t2的状态                                                                          &               \\ \hline
swap(t1, t2)            & 交换t1和t2的状态                                                                          &               \\ \hline
hardware\_concurrency() & 带有可能的硬件线程提示的静态函数                                            &               \\ \hline
\end{longtable}

\begin{center}
表12.4 jthread类对象的操作
\end{center}

注意,std::thread和std::jthread的成员类型id和native\_handle\_type相同,所以使用decltype(mythread)::id或std::thread::id都没有关系。这样,就可以直接用std::jthread替换现有代码中的std::thread,就好了。

\mySubsubsection{12.3.1}{对std::jthread使用停止令牌}

除了析构函数连接之外,std::jthread的主要优点是会自动建立停止信号的机制。为此,启动线程的构造函数创建一个停止源,将其存储为线程对象的成员,并将相应的停止令牌传递给被调用的函数,以避免该函数将额外的stop\_token作为第一个参数。

也可以通过线程的成员函数获得停止源和停止令牌:

\begin{cpp}
std::jthread t1{[] (std::stop_token st) {
						...
				}};
...
foo(t1.get_token()); // pass stop token to foo()
...
std::stop_source ssrc{t1.get_stop_source()};
ssrc.request_stop(); // request stop on stop token of t1
\end{cpp}

因为get\_token()和get\_stop\_source()按值返回,所以停止源和停止令牌甚至可以在线程分离时,在后台运行使用。

\mySamllsection{std::jthreads集合中使用停止令牌}

如果启动多个jthread,每个线程都有自己的停止令牌。注意,这可能会导致停止所有线程的时间可能比预期的要长:

\begin{cpp}
{
	std::vector<std::jthread> threads;
	for (int i = 0; i < numThreads; ++i) {
		pool.push_back(std::jthread{[&] (std::stop_token st) {
				while (!st.stop_requested()) {
					...
				}
		}});
	}
	...
} // destructor stops all threads
\end{cpp}

循环结束时,析构函数停止所有正在运行的线程:

\begin{cpp}
for (auto& t : threads) {
	t.request_stop();
	t.join();
}
\end{cpp}

则总是等待一个线程结束,然后向下一个线程发出停止信号。

这样的代码可以为所有线程调用join()之前,请求所有线程停止(通过析构函数)进行改进。

\begin{cpp}
{
	std::vector<std::jthread> threads;
	for (int i = 0; i < numThreads; ++i) {
		pool.push_back(std::jthread{[&] (std::stop_token st) {
				while (!st.stop_requested()) {
					...
				}
		}});
	}
	...
	// BETTER: request stops for all threads before we start to join them:
	for (auto& t : threads) {
		t.request_stop();
	}
} // destructor stops all threads
\end{cpp}

首先请求所有线程停止,线程可能在析构函数调用join()让线程结束之前就结束了。协程线程池的析构函数也有使用这种技术的一个示例。

\mySamllsection{对多个std::jthread使用相同的停止令牌}

可能还需要使用相同的停止令牌为多个线程请求停止。这很容易做到,只需自己创建停止令牌,或者从已经启动的第一个线程中获取停止令牌,并将此停止令牌作为第一个参数启动(其他)线程。例如:

\begin{cpp}
// initialize a common stop token for all threads:
std::stop_source allStopSource;
std::stop_token allStopToken{allStopSource.get_token()};
for (int i = 0; i < 9; ++i) {
	threads.push_back(std::jthread{[] (std::stop_token st) {
			...
			while (!st.stop_requested()) {
				...
			}
		},
		allStopToken // pass token to this thread
	});
}
\end{cpp}

记住,可调用对象通常只接受传递的所有参数。仅当存在未传递参数的附加停止令牌参数时,才使用已启动线程的内部停止令牌。

请参阅lib/atomicref.cpp获得完整的示例。












