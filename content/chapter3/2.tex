

可以使用requirea子句或概念来约束几乎所有形式的泛型代码:

\begin{itemize}
\item
函数模板:
\begin{cpp}
template<typename T>
requires ...
void print(const T&) {
	...
}
\end{cpp}

\item
类模板:
\begin{cpp}
template<typename T>
requires ...
class MyType {
	...
}
\end{cpp}

\item
别名模板

\item
变量模板

\item
(甚至可以约束)成员函数
\end{itemize}

对于这些模板,可以约束类型和值参数。

但这里不能约束概念:

\begin{cpp}
template<std::ranges::sized_range T> // ERROR
concept IsIntegralValType = std::integral<std::ranges::range_value_t<T>>;
\end{cpp}

必须这样指定:

\begin{cpp}
template<typename T>
concept IsIntegralValType = std::ranges::sized_range<T> &&
							std::integral<std::ranges::range_value_t<T>>;
\end{cpp}


\mySubsubsection{3.2.1}{约束别名模板}

下面是一个约束别名模板的例子(使用声明的泛型):

\begin{cpp}
template<std::ranges::range T>
using ValueType = std::ranges::range_value_t<T>;
\end{cpp}

该声明等价于:

\begin{cpp}
template<typename T>
requires std::ranges::range<T>
using ValueType = std::ranges::range_value_t<T>;
\end{cpp}

类型ValueType<>现在只可对范围类型进行定义:

\begin{cpp}
ValueType<int> vt1; // ERROR
ValueType<std::vector<int>> vt2; // int
ValueType<std::list<double>> vt3; // double
\end{cpp}

\mySubsubsection{3.2.2}{约束变量模板}

下面是一个约束变量模板的例子:

\begin{cpp}
template<std::ranges::range T>
constexpr bool IsIntegralValType = std::integral<std::ranges::range_value_t<T>>;
\end{cpp}

同样,这相当于以下内容:

\begin{cpp}
template<typename T>
requires std::ranges::range<T>
constexpr bool IsIntegralValType = std::integral<std::ranges::range_value_t<T>>;
\end{cpp}

类型IsIntegralValType<>现在只可在范围中定义:

\begin{cpp}
bool b1 = IsIntegralValType<int>; // ERROR
bool b2 = IsIntegralValType<std::vector<int>>; // true
bool b3 = IsIntegralValType<std::list<double>>; // false
\end{cpp}

\mySubsubsection{3.2.3}{约束成员函数}

requires子句也可以是成员函数声明的一部分,开发者就可以根据需求和概念指定不同的API。

\filename{lang/valorcoll.hpp}

\begin{cpp}
#include <iostream>
#include <ranges>

template<typename T>
class ValOrColl {
	T value;
	public:
	ValOrColl(const T& val)
	: value{val} {
	}
	ValOrColl(T&& val)
	: value{std::move(val)} {
	}
	
	void print() const {
		std::cout << value << '\n';
	}
	
	void print() const requires std::ranges::range<T> {
		for (const auto& elem : value) {
			std::cout << elem << ' ';
		}
		std::cout << '\n';
	}
};
\end{cpp}

这里,定义了一个类ValOrColl,可以保存一个值或一个值的集合,作为T类型的值。提供了两个print()成员函数,类使用标准概念std::ranges::range来决定使用哪一个:

\begin{itemize}
\item
若类型T是一个集合,则满足约束,所以两个print()成员函数都可用。但重载解析首选第二个print(),将遍历集合的元素,因为该成员函数有约束。

\item
若类型T不是集合,则只有第一个print()可用,因此可以使用。
\end{itemize}

例如,可以这样使用这个类:

\filename{lang/valorcoll.cpp}

\begin{cpp}
#include "valorcoll.hpp"
#include <vector>

int main()
{
	ValOrColl o1 = 42;
	o1.print();
	ValOrColl o2 = std::vector{1, 2, 3, 4};
	o2.print();
}
\end{cpp}

该程序应有以下输出:

\begin{shell}
42
1 2 3 4
\end{shell}

注意,这种方式只能约束模板,不能使用requires来约束普通函数:

\begin{cpp}
void foo() requires std::numeric_limits<char>::is_signed // ERROR
{
	...
}
\end{cpp}

C++标准库中约束成员函数的一个例子是,const视图的begin()的条件可用性。

\mySubsubsection{3.2.4}{约束非类型模板参数}

可以约束的不仅仅是类型,还可以约束作为模板参数的值(非类型模板参数(NTTP))。例如:

\begin{cpp}
template<int Val>
concept LessThan10 = Val < 10;
\end{cpp}

或者更通用的情况:

\begin{cpp}
template<auto Val>
concept LessThan10 = Val < 10;
\end{cpp}

可以这样使用:

\begin{cpp}
template<typename T, int Size>
requires LessThan10<Size>
class MyType {
	...
};
\end{cpp}

稍后将讨论更多的示例。










