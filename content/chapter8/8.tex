
本节讨论处理多个范围的所有视图。

\mySubsubsection{8.8.1}{拆分和惰性拆分视图}

% Please add the following required packages to your document preamble:
% \usepackage{longtable}
% Note: It may be necessary to compile the document several times to get a multi-page table to line up properly
\begin{longtable}[c]{|l|ll|}
\hline
\textbf{类型:} & \multicolumn{1}{l|}{std::ranges::split\_view\textless{}\textgreater{}} & std::range::laze\_split\_view\textless{}\textgreater{} \\ \hline
\endfirsthead
%
\endhead
%
\textbf{内容:}              & \multicolumn{2}{l|}{一个范围的所有元素被分割成多个视图}                       \\ \hline
\textbf{适配器:}              & \multicolumn{1}{l|}{std::views::split()}        & std::views::lazy\_split()                  \\ \hline
\textbf{元素类型:}         & \multicolumn{1}{l|}{集合的引用}   & 集合的引用                   \\ \hline
\textbf{要求:}             & \multicolumn{1}{l|}{至少为前向范围}     & 至少为前向范围                       \\ \hline
\textbf{类别:}             & \multicolumn{1}{l|}{总是前向}             & 输入或前向                           \\ \hline
\textbf{是否是长度范围:}       & \multicolumn{1}{l|}{从不}                      & 从不                                      \\ \hline
\textbf{是否是通用范围:}      & \multicolumn{1}{l|}{若在通用前向范围} & 若在通用前向范围                 \\ \hline
\textbf{是否是租借范围:}    & \multicolumn{1}{l|}{从不}                      & 从不                                      \\ \hline
\textbf{缓存:}               & \multicolumn{1}{l|}{总是缓存begin()}      & 在输入范围内,缓存当前值 \\ \hline
\textbf{常量可迭代:}       & \multicolumn{1}{l|}{从不}                      & 若在可重复的前向范围内         \\ \hline
\textbf{传播常量性:} & \multicolumn{1}{l|}{--}                         & 从不                                      \\ \hline
\end{longtable}

std::ranges::split\_view<>和std::ranges::lazy\_split\_view<>两个类模板都定义了一个视图,该视图引用由分隔符分隔将范围拆分为多个子视图。[std::ranges::lazy\_split\_view<>不是原始C++20标准的一部分,但后来通过\url{http://wg21.link/p2210r2}添加到C++20中。]

split\_view<>和lazy\_split\_view<>的区别如下:

\begin{itemize}
\item
split\_view<>不能迭代const视图;lazy\_split\_view<>可以(引用的范围至少是前向范围)。

\item
split\_view<>只能处理至少具有前向迭代器的范围(必须满足forward\_range的概念)。

\item
split\_view<>元素只是引用范围的迭代器类型的std::ranges::subrange(保持引用范围的类别)。

lazy\_split\_views<>元素是std::ranges::lazy\_split\_view类型的视图,其总是前向范围,甚至不支持size()。

\item
split\_view<>具有更好的性能。
\end{itemize}

因为只使用输入范围或视图被用作const范围,所以应该使用split\_view<>,除非不能这样做。例如:

\begin{cpp}
std::vector<int> rg{1, 2, 3, 4, 5, 6, 7, 8, 9, 10, 11, 12, 13};
...
for (const auto& sub : std::ranges::split_view{rg, 5}) {
	for (const auto& elem : sub) {
		std::cout << elem << ' ';
	}
	std::cout << '\n';
}
\end{cpp}

循环输出:

\begin{shell}
1 2 3 4
6 7 8 9 10 11 12 13
\end{shell}

也就是说,只要在rg中找到值为5的元素,就结束前一个视图并开始一个新视图。

\mySamllsection{用于拆分和延迟拆分视图的范围适配器}

拆分视图和延迟拆分视图也可以(通常应该)使用范围适配器来创建,适配器只是将它们的参数传递给相应的视图构造函数:

\begin{cpp}
std::views::split(rg, sep)
std::views::lazy_split(rg, sep)
\end{cpp}

例如:

\begin{cpp}
for (const auto& sub : std::views::split(rg, 5)) {
	for (const auto& elem : sub) {
		std::cout << elem << ' ';
	}
	std::cout << '\n';
}
\end{cpp}

或:

\begin{cpp}
for (const auto& sub : rg | std::views::split(5)) {
	for (const auto& elem : sub) {
		std::cout << elem << ' ';
	}
	std::cout << '\n';
}
\end{cpp}

创建的视图可能是空的。对于每个前后分隔符,以及每当两个分隔符在彼此后面时,都会创建一个空视图。例如:

\begin{cpp}
std::list<int> rg{5, 5, 1, 2, 3, 4, 5, 6, 5, 5, 4, 3, 2, 1, 5, 5};
for (const auto& sub : std::ranges::split_view{rg, 5}) {
	std::cout << "subview: ";
	for (const auto& elem : sub) {
		std::cout << elem << ' ';
	}
	std::cout << '\n';
}
\end{cpp}

这里,输出如下:[最初的C++20标准指定最后一个分隔符可忽略,所以我们将在最后只得到一个空子视图。这是修复于\url{http://wg21.link/p2210r2}。]

\begin{shell}
subview:
subview:
subview: 1 2 3 4
subview: 6
subview:
subview: 4 3 2 1
subview:
subview:
\end{shell}

除了单值之外,还可以传递一系列值作为分隔符。例如:

\begin{cpp}
std::vector<int> rg{1, 2, 3, 4, 5, 1, 2, 3, 4, 5, 1, 2, 3 };
...
// split separated by a sequence of 5 and 1:
for (const auto& sub : std::views::split(rg, std::list{5, 1})) {
	for (const auto& elem : sub) {
		std::cout << elem << ' ';
	}
	std::cout << '\n';
}
\end{cpp}

这段代码的输出是

\begin{shell}
1 2 3 4
2 3 4
2 3
\end{shell}

传递的元素集合必须是有效的视图和forward\_range,所以在容器中指定子序列时,必须将其转换为视图。例如:

\begin{cpp}
// split with specified pattern of 4 and 5:
std::array pattern{4, 5};
for (const auto& sub : std::views::split(rg, std::views::all(pattern))) {
	...
}
\end{cpp}

可以使用拆分视图来拆分字符串。例如:

\begin{cpp}
std::string str{"No problem can withstand the assault of sustained thinking"};
for (auto sub : std::views::split(str, "th"sv)) { // split by "th"
	std::cout << std::string_view{sub} << '\n';
}
\end{cpp}

每个子字符串sub的类型为std::ranges::subrange<decltype(str.begin())>,这样的代码将不能用于延迟拆分视图。

拆分或延迟拆分视图会在内部存储传入的范围(也可以像all()那样转换为视图),所以只有在传递的范围有效的情况下才有效(除非传递了右值,则在内部使用了归属视图)。

begin迭代器和哨兵(end迭代器)都是特殊的内部辅助类型,若传递的范围是通用的,则是通用的。

这里是一个使用拆分视图的完整示例:

\filename{ranges/splitview.cpp}

\begin{cpp}
#include <iostream>
#include <string>
#include <vector>
#include <array>
#include <ranges>

void print(auto&& obj, int level = 0)
{
	if constexpr(std::ranges::input_range<decltype(obj)>) {
		std::cout << '[';
		for (const auto& elem : obj) {
			print(elem, level+1);
		}
		std::cout << ']';
	}
	else {
		std::cout << obj << ' ';
	}
	if (level == 0) std::cout << '\n';
}

int main()
{
	std::vector coll{1, 2, 3, 4, 1, 2, 3, 4};
	
	print(coll); // [1 2 3 4 1 2 3 4 ]
	print(std::ranges::split_view{coll, 2}); // [[1 ][3 4 1 ][3 4 ]]
	print(coll | std::views::split(3)); // [[1 2 ][4 1 2 ][4 ]]
	print(coll | std::views::split(std::array{4, 1})); // [[1 2 3 ][2 3 4 ]]
}
\end{cpp}

该程序有以下输出:

\begin{shell}
[1 2 3 4 1 2 3 4 ]
[[1 ][3 4 1 ][3 4 ]]
[[1 2 ][4 1 2 ][4 ]]
[[1 2 3 ][2 3 4 ]]
\end{shell}

\mySamllsection{拆分视图和const}

注意,不能在const的拆分视图上迭代。

例如:

\begin{cpp}
std::vector<int> coll{5, 1, 5, 1, 2, 5, 5, 1, 2, 3, 5, 5, 5};
...
const std::ranges::split_view sv{coll, 5};
for (const auto& sub : sv) { // ERROR for const split view
	std::cout << sub.size() << ' ';
}
\end{cpp}

这是一个问题,特别是若将视图传递给一个将参数作为const引用的泛型函数:

\begin{cpp}
void printElems(const auto& coll) {
	...
}
printElems(std::views::split(rg, 5)); // ERROR
\end{cpp}

要在泛型代码中支持这个视图,必须使用通用/转发引用:

\begin{cpp}
void printElems(auto&& coll) {
	...
}
\end{cpp}

或者,可以使用lazy\_split\_view。然后,子视图元素只能用作前向范围,所以不能调用size()、向后迭代或对元素排序的事情:

\begin{cpp}
std::vector<int> coll{5, 1, 5, 1, 2, 5, 5, 1, 2, 3, 5, 5, 5};
...
const std::ranges::lazy_split_view sv{coll, 5};
for (const auto& sub : sv) { // OK for const lazy-split view
	std::cout << sub.size() << ' '; // ERROR
	std::sort(sub); // ERROR
	for (const auto& elem : sub) { // OK
		std::cout << elem << ' ';
	}
}
\end{cpp}

\mySamllsection{拆分和延迟拆分视图的接口}

“类std::ranges::split\_view<>和std::ranges::split\_view<>的操作”表列出了拆分视图或延迟拆分视图的API。

% Please add the following required packages to your document preamble:
% \usepackage{longtable}
% Note: It may be necessary to compile the document several times to get a multi-page table to line up properly
\begin{longtable}[c]{|l|l|}
\hline
\textbf{操作}  & \textbf{效果}                                                 \\ \hline
\endfirsthead
%
\endhead
%
split\_view r\{\} & 创建一个split\_view,它引用一个默认构造的范围                                   \\ \hline
split\_view r\{rg\} & 创建一个split\_view,引用范围rg                   \\ \hline
r.begin()           & 生成begin迭代器                                       \\ \hline
r.end()             & 生成哨兵(end迭代器)                              \\ \hline
r.empty()           & 生成的r是否为空(若范围支持则可用) \\ \hline
if (r)              & 若r不为空,则为true(若定义了empty(),则可用)         \\ \hline
r.size()          & 生成元素的数量(若引用一个长度范围,则可用)                              \\ \hline
r.front()           & 生成的第一个元素(若转发可用)               \\ \hline
r.back()            & 生成的最后一个元素(若范围为双向且通用,则可用)  \\ \hline
r{[}idx{]}          & 生成的第n个元素(若随机访问可用)            \\ \hline
r.data()          & 生成一个指向元素内存的原始指针(若元素在连续内存中可用) \\ \hline
r.base()            & 产生对r所引用或归属范围的引用        \\ \hline
\end{longtable}

\begin{center}
表8.22 类std::ranges::split\_view<>和std::ranges::split\_view<>的操作
\end{center}

\mySubsubsection{8.8.2}{连接视图}

% Please add the following required packages to your document preamble:
% \usepackage{longtable}
% Note: It may be necessary to compile the document several times to get a multi-page table to line up properly
\begin{longtable}[c]{|l|l|}
\hline
\textbf{类型:}           & std::ranges::join\_view\textless{}\textgreater{}                        \\ \hline
\endfirsthead
%
\endhead
%
\textbf{内容:}        & 一个范围或多个范围的所有元素作为一个视图                  \\ \hline
\textbf{适配器:}              & std::views::join()         \\ \hline
\textbf{元素类型:}         & 与传递的范围类型相同 \\ \hline
\textbf{要求:}             & 至少为输入范围       \\ \hline
\textbf{类别:}             & 双向输入     \\ \hline
\textbf{是否是长度范围:}       & 从不                      \\ \hline
\textbf{是否是通用范围:}      & 可变                     \\ \hline
\textbf{是否是租借范围:}    & 从不                      \\ \hline
\textbf{缓存:}               & 无                    \\ \hline
\textbf{常量可迭代:} & 若常量可迭代,范围和元素仍是const的引用 \\ \hline
\textbf{传播常量性:} & 只有在右值范围内    \\ \hline
\end{longtable}

类模板std::ranges::join\_view<>定义了一个遍历多个范围视图的所有元素的视图。

例如:

\begin{cpp}
std::vector<int> rg1{1, 2, 3, 4};
std::vector<int> rg2{0, 8, 15};
std::vector<int> rg3{5, 4, 3, 2, 1, 0};
std::array coll{rg1, rg2, rg3};
...
for (const auto& elem : std::ranges::join_view{coll}) {
	std::cout << elem << ' ';
}
\end{cpp}

循环输出:

\begin{shell}
1 2 3 4 0 8 15 5 4 3 2 1 0
\end{shell}

\mySamllsection{连接视图的范围适配器}

连接视图也可以(通常应该)使用范围适配器创建,适配器只是将其参数传递给std::ranges::join\_view构造函数:

\begin{cpp}
std::views::join(rg)
\end{cpp}

例如:

\begin{cpp}
for (const auto& elem : std::views::join(coll)) {
	std::cout << elem << ' ';
}
\end{cpp}

或:

\begin{cpp}
for (const auto& elem : coll | std::views::join) {
	std::cout << elem << ' ';
}
\end{cpp}

下面是一个使用连接视图的完整示例:

\filename{ranges/joinview.cpp}

\begin{cpp}
#include <iostream>
#include <string>
#include <vector>
#include <array>
#include <ranges>
#include "printcoll.hpp"

int main()
{
	std::vector<std::string> rg1{"he", "hi", "ho"};
	std::vector<std::string> rg2{"---", "|", "---"};
	std::array coll{rg1, rg2, rg1};
	
	printColl(coll); // ranges of ranges of strings
	printColl(std::ranges::join_view{coll}); // range of strings
	printColl(coll | std::views::join); // range of strings
	printColl(coll | std::views::join | std::views::join); // ranges of chars
}
\end{cpp}

使用一个可以递归打印集合的辅助函数:

\filename{ranges/printcoll.cpp}

\begin{cpp}
#include <iostream>
#include <ranges>

template<typename T>
void printColl(T&& obj, int level = 0)
{
	if constexpr(std::same_as<std::remove_cvref_t<T>, std::string>) {
		std::cout << "\"" << obj << "\"";
	}
	else if constexpr(std::ranges::input_range<T>) {
		std::cout << '[';
		for (auto pos = obj.begin(); pos != obj.end(); ++pos) {
			printColl(*pos, level+1);
			if (std::ranges::next(pos) != obj.end()) {
				std::cout << ' ';
			}
		}
		std::cout << ']';
	}
	else {
		std::cout << obj;
	}
	if (level == 0) std::cout << '\n';
}
\end{cpp}

该程序有以下输出:

\begin{shell}
[["he" "hi" "ho"] ["---" "|" "---"] ["he" "hi" "ho"]]
["he" "hi" "ho" "---" "|" "---" "he" "hi" "ho"]
["he" "hi" "ho" "---" "|" "---" "he" "hi" "ho"]
[h e h i h o - - - | - - - h e h i h o]
\end{shell}

结合类型std::ranges::subrange,可以使用连接视图来连接多个数组的元素。

例如:

\begin{cpp}
int arr1[]{1, 2, 3, 4, 5};
int arr2[] = {0, 8, 15};
int arr3[10]{1, 2, 3, 4, 5};
...
std::array<std::ranges::subrange<int*>, 3> coll{arr1, arr2, arr3};
for (const auto& elem : std::ranges::join_view{coll}) {
	std::cout << elem << ' ';
}
\end{cpp}

或者,可以这样声明coll:

\begin{cpp}
std::array coll{std::ranges::subrange{arr1},
				std::ranges::subrange{arr2},
				std::ranges::subrange{arr3}};
\end{cpp}

连接视图是唯一视图在c++中20处理范围的范围,所以它有外部迭代器和内部迭代器,结果视图的属性可能依赖于两者。

注意,内部迭代器不支持迭代器特性。出于这个原因,应该更像std::ranges::next()这样的工具,而非std::next()。否则,代码可能无法编译。

\mySamllsection{连接视图和const}

注意,不能总是迭代const连接视图。若范围不是const可迭代的,或者内部范围产生普通值而不是引用,就会发生这种情况。

对于后一种情况,请考虑以下示例:

\filename{ranges/joinconst.cpp}

\begin{cpp}
#include <vector>
#include <array>
#include <ranges>
#include "printcoll.hpp"

void printConstColl(const auto& coll)
{
	printColl(coll);
}

int main()
{
	std::vector<int> rg1{1, 2, 3, 4};
	std::vector<int> rg2{0, 8, 15};
	std::vector<int> rg3{5, 4, 3, 2, 1, 0};
	std::array coll{rg1, rg2, rg3};
	
	printConstColl(coll);
	printConstColl(coll | std::views::join);
	
	auto collTx = [] (const auto& coll) { return coll; };
	auto coll2values = coll | std::views::transform(collTx);
	
	printConstColl(coll2values);
	printConstColl(coll2values | std::views::join); // ERROR
}
\end{cpp}

当使用连接三个范围数组的元素时,可以调用printConstColl(),将范围作为const引用。会得到以下输出:

\begin{shell}
[[1 2 3 4] [0 8 15] [5 4 3 2 1 0]]
[1 2 3 4 0 8 15 5 4 3 2 1 0]
\end{shell}

然而,当我们将整个数组传递给一个转换视图,该视图按值返回所有内部范围,调用printConstColl()是一个错误。

调用视图的printColl(),其内部范围产生普通值,工作得很好。注意,这要求printColl()使用std::ranges::next(),而非std::next()。否则,即使下面的代码也不能编译:

\begin{cpp}
printColl(coll2values | std::views::join); // ERROR if std::next() used
\end{cpp}

\mySamllsection{连接视图的特殊特征}

若外部范围和内部范围是双向的,并且内部范围是通用范围,则产生的类别是双向的。否则,若外部范围和内部范围至少是前向范围,则生成的类别是前向的。否则,生成的类别为输入范围。

\mySamllsection{连接视图的接口}

“类std::ranges::join\_view<>的操作”表列出了连接视图的API。

% Please add the following required packages to your document preamble:
% \usepackage{longtable}
% Note: It may be necessary to compile the document several times to get a multi-page table to line up properly
\begin{longtable}[c]{|l|l|}
\hline
\textbf{操作} & \textbf{效果}                                                        \\ \hline
\endfirsthead
%
\endhead
%
join\_view r\{\}   & 创建一个引用默认构造范围的join\_view        \\ \hline
join\_view r\{rg\} & 创建一个引用范围rg的join\_view                           \\ \hline
r.begin()          & 生成begin迭代器                                              \\ \hline
r.end()            & 生成哨兵(end迭代器)                                      \\ \hline
r.empty()          & 生成的r是否为空(若该范围支持,则可用)          \\ \hline
if (r)             & 若r不为空,则为true(若定义了empty(),则可用)                \\ \hline
r.size()           & 生成元素的数量(若引用了一个长度范围,则可用) \\ \hline
r.front()          & 生成的第一个元素(若转发可用)                      \\ \hline
r.back()           & 生成的最后一个元素(若范围为双向且通用,则可用)         \\ \hline
r{[}idx{]}         & 生成的第n个元素(若随机访问可用)                    \\ \hline
r.data() & 生成一个指向元素内存的原始指针(若元素在连续内存中可用) \\ \hline
r.base()           & 产生对r所引用或归属范围的引用               \\ \hline
\end{longtable}

\begin{center}
表8.23 类std::ranges::join\_view<>的操作
\end{center}













