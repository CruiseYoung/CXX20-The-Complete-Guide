
Note the following hints about the way I write code and comments in this book

\subsubsection*{\zihao{3} Initializations}
\addcontentsline{toc}{subsubsection}{Initializations}

I usually use the modern form of initialization (introduced in C++11 as uniform initialization) with curly braces or with = in trivial cases:

\begin{cpp}
int i = 42;
std::string s{"hello"};
\end{cpp}

The brace initialization has the following advantages:

\begin{itemize}
\item
It can be used with fundamental types, class types, aggregates, enumeration types, and auto

\item
It can be used to initialize containers with multiple values

\item
It can detect narrowing errors (e.g., initialization of an int by a floating-point value)

\item
It cannot be confused with function declarations or calls
\end{itemize}

If the braces are empty, the default constructors of (sub)objects are called and fundamental data types are guaranteed to be initialized with 0/false/nullptr.


\subsubsection*{\zihao{3} Error Terminology}
\addcontentsline{toc}{subsubsection}{Error Terminology}

I often talk about programming errors. If there is no special hint, the term error or a comment such as

\begin{cpp}
... // ERROR
\end{cpp}

means a compile-time error. The corresponding code should not compile (with a conforming compiler).

If I use the term runtime error, the program might compile but not behave correctly or result in undefined behavior (thus, it might or might not do what is expected)

\subsubsection*{\zihao{3} Code Simplifications}
\addcontentsline{toc}{subsubsection}{Code Simplifications}

I try to explain all features with helpful examples. However, to concentrate on the key aspects to be taught, I might often skip other details that should be part of the code.

\begin{itemize}
\item
Most of the time I use an ellipsis (“...”) to signal additional code that is missing. Note that I do not use code font here. If you see an ellipsis with code font, code must have these three dots as a language feature (such as for “typename...”).

\item
In header files, I usually skip the preprocessor guards. All header files should have something like the following:

\begin{cpp}
#ifndef MYFILE_HPP
#define MYFILE_HPP
...
#endif // MYFILE_HPP
\end{cpp}

Therefore, please beware and fix the code when using these header files in your projects.
\end{itemize}
