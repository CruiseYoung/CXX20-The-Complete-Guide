
当使用模板形参的类型成员时,通常必须使用关键字typename来限定这种用法:

\begin{cpp}
template<typename T>
typename T::value_type getElem(const T& cont, typename T::iterator pos)
{
	using Itor = typename T::iterator;
	typename T::value_type elem;
	...
	return elem;
}
\end{cpp}

C++20之前,类型成员value\_type和T的迭代器的所有限定条件都是必需的。

自C++20起,可以在明确传递类型的上下文中跳过typename,这适用于返回类型和别名声明中使用的类型的规范(其中using为类型引入了一个新名称):

\begin{cpp}
template<typename T>
T::value_type getElem(const T& cont, typename T::iterator pos)
{
	using Itor = T::iterator;
	typename T::value_type elem;
	...
	return elem;
}
\end{cpp}

参数pos和变量elem仍然需要typename,可以跳过typename的地方是:

\begin{itemize}
\item
当声明返回类型时(局部前向声明除外)

\item
在类模板中声明成员时

\item
在类模板中声明成员函数或友元函数的形参时

\item
声明require表达式的参数时

\item
对于别名声明中的类型
\end{itemize}

当声明一个类模板时,可以在大多数情况下跳过typename:

\begin{cpp}
template<typename T>
class MyClass {
	T::value_type val; // no need for typename since C++20
public:
	...
	T::iterator begin() const; // no need for typename since C++20
	T::iterator end() const; // no need for typename since C++20
	void print(T::iterator) const; // no need for typename since C++20
};
\end{cpp}

由于隐式typename的规则在某种程度上非常微妙,因此在使用模板形参的类型成员时(至少在类模板之外),可能仍然使用typename。

\mySubsubsection{22.1.1}{隐式类型名的详情}

自C++20起,以下情况为模板形参使用类型成员时,可以跳过typename:

\begin{itemize}
\item
在别名声明中(即,使用using声明类型名称时);注意,带typedef的类型声明仍然需要typename

\item
当定义或声明函数的返回类型时(除非声明发生在函数或块范围内)

\item
声明尾步返回类型时

\item
当指定static\_cast、const\_cast、reinterpret\_cast或dynamic\_cast的目标类型时

\item
指定类型时

\item
在类中

\begin{itemize}
\item
声明数据成员时

\item
声明成员函数的返回类型时

\item
声明成员函数或友元函数或Lambda的形参(默认实参可能仍然需要)时
\end{itemize}

\item
在require表达式中声明参数类型时

\item
为模板的类型参数声明默认值时

\item
声明非类型模板形参的类型时
\end{itemize}

注意,C++20之前,typename在一些其他情况下是不必要的:

\begin{itemize}
\item
指定继承类的基类型时

\item
在构造函数中将初始值传递给基类时

\item
在类声明中使用类型成员时
\end{itemize}

下面的例子演示了上面的大多数情况(TYPENAME表示自C++20以来可选填的地方):

\begin{cpp}
template<typename T,
auto ValT = typename T::value_type{}> // typename required
class MyClass {
	TYPENAME T::value_type val; // typename optional
	public:
	using iterator = TYPENAME T::iterator; // typename optional
	TYPENAME T::iterator begin() const; // typename optional
	TYPENAME T::iterator end() const; // typename optional
	void print(TYPENAME T::iterator) const; // typename optional
	template<typename T2 = TYPENAME T::value_type> // second typename optional
	void assign(T2);
};

template<typename T>
TYPENAME T::value_type // typename optional
foo(const T& cont, typename T::value_type arg) // typename required
{
	typedef typename T::value_type ValT2; // typename required
	using ValT1 = TYPENAME T::value_type; // typename optional
	typename T::value_type val; // typename required
	typename T::value_type other1(void); // typename required
	auto other2(void) -> TYPENAME T::value_type; // typename optional
	auto l1 = [] (TYPENAME T::value_type) { // typename optional
	};
	auto p = new TYPENAME T::value_type; // typename optional
	val = static_cast<TYPENAME T::value_type>(0); // typename optional
	...
}
\end{cpp}










