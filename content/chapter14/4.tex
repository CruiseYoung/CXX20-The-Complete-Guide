

目前为止,已经看到了如何使用协程接口从外部控制协程(包装协程句柄及其承诺)。然而,协程可以(而且必须)提供另一个配置点:Awaitables(实现方式是使用Awaiter)。

相关术语如下所示:

\begin{itemize}
\item
Awaitables是运算符co\_await需要作为其操作数的术语,所以awaitables是co\_await可以处理的所有对象。

\item
Awaiter是实现Awaitables的一种特定(和典型)方式的术语。其必须提供三个特定的成员函数来处理协程的暂停和恢复。
\end{itemize}

Awaitables在调用co\_await(或co\_yield)时使用,允许提供拒绝请求的代码来挂起(暂时或在某些条件下)或执行一些用于挂起和恢复的逻辑。

我们已经使用过两种Awaiter类型:std::suspend\_always和std::suspend\_never。

\mySubsubsection{14.4.1}{Awaiters}

Awaiters协同程序时可以调用暂停或恢复,"Awaiter的特殊成员函数"表列出了awaiter必须提供的操作。

% Please add the following required packages to your document preamble:
% \usepackage{longtable}
% Note: It may be necessary to compile the document several times to get a multi-page table to line up properly
\begin{longtable}[c]{|l|l|}
\hline
\textbf{操作}       & \textbf{效果}                                   \\ \hline
\endfirsthead
%
\endhead
%
await\_ready()           & 生成是否(当前)禁用挂起 \\ \hline
await\_suspend(awaitHdl) & 处理挂起                                 \\ \hline
await\_resume()          & 处理恢复                                 \\ \hline
\end{longtable}

\begin{center}
表14.1 Awaiters的特殊成员函数
\end{center}

关键函数是await\_suspend()和await\_resume():

\begin{itemize}
\item
await\_ready()

这个函数会在协程被挂起之前调用,提供(暂时)完全关闭挂起。若它返回true,则协程根本不会挂起。

这个函数通常只返回false(“不,不要阻止/忽略任何挂起”)。为了节省挂起的成本,当挂起无意义时(例如,若挂起依赖于可用的某些数据),可能会有条件地产生true。

这个函数中,协程还没有挂起。所以,其不应该用来直接或间接地调用其所调用的协程的resume()或destroy()。

\item
auto await\_suspend(awaitHdl)

在协程挂起后立即为协程调用此函数。参数awaitHdl是被挂起的协程的句柄,其的类型是std::coroutine\_handle<PromiseType>。

这个函数中,可以指定下一步要做什么,包括立即恢复挂起的或等待的协程。不同的返回类型允许以不同的方式指定,还可以通过直接将控制转移到另一个协程来有效地跳过暂停。

甚至可以破坏协程,但请确保不再在其他任何地方使用协程(例如:在协程接口中调用done()或destroy())。

\item
auto await\_resume()

当成功挂起后恢复协程时,将为协程调用此函数。

可以返回一个值,这个值就是co\_await表达式产生的值。
\end{itemize}

考虑下面这个简单的awaiter使用:

\filename{coro/awaiter.hpp}

\begin{cpp}
#include <iostream>

class Awaiter {
	public:
	bool await_ready() const noexcept {
		std::cout << " await_ready()\n";
		return false; // do suspend
	}

	void await_suspend(auto hdl) const noexcept {
		std::cout << " await_suspend()\n";
	}

	void await_resume() const noexcept {
		std::cout << " await_resume()\n";
	}
};
\end{cpp}

这个awaiter中,跟踪该awaiter的哪个函数何时调用。因为await\_ready()返回false,而await\_suspend()不返回任何值,所以awaiter接受挂起(不恢复其他操作)。这是标准的await std::suspend\_always{}结合一些print语句的行为,其他返回类型/值可以提供不同的行为。

协程可以像下面这样使用这个awaiter(完整代码参见coro/awaiter.cpp):

\begin{cpp}
CoroTask coro(int max)
{
	std::cout << "   CORO start\n";
	for (int val = 1; val <= max; ++val) {
		std::cout << "   CORO " << val << '\n';
		co_await Awaiter{}; // SUSPEND with our own awaiter
	}
	std::cout << "   CORO end\n";
}
\end{cpp}

假设这样使用协程:

\begin{cpp}
auto coTask = coro(2);
std::cout << "started\n";

std::cout << "loop\n";
while (coTask.resume()) { // RESUME
	std::cout << " suspended\n";
}
std::cout << "done\n";
\end{cpp}

可得到以下输出:

\begin{shell}
started
loop
  CORO start
  CORO 1
    await_ready()
    await_suspend()
  suspended
    await_resume()
  CORO 2
    await_ready()
    await_suspend()
  suspended
    await_resume()
  CORO end
done
\end{shell}

稍后将讨论更多关于awaiters更多的细节。

\mySubsubsection{14.4.2}{标准Awaiters}

C++标准库提供了已经使用过的两个简单的awaiter:

\begin{itemize}
\item
std::suspend\_always

\item
std::suspend\_never
\end{itemize}

其定义非常简单:

\begin{cpp}
namespace std {
	struct suspend_always {
		constexpr bool await_ready() const noexcept { return false; }
		constexpr void await_suspend(coroutine_handle<>) const noexcept { }
		constexpr void await_resume() const noexcept { }
	};

	struct suspend_never {
		constexpr bool await_ready() const noexcept { return true; }
		constexpr void await_suspend(coroutine_handle<>) const noexcept { }
		constexpr void await_resume() const noexcept { }
	};
}
\end{cpp}

两者都没有暂停或恢复,但await\_ready()的返回值有所不同:

\begin{itemize}
\item
若在await\_ready()中返回false(而在await\_suspend()中没有返回任何值),则suspend\_always接受每个挂起,并将协程返回给其调用者。

\item
若在await\_ready()中返回true,则suspend\_never永远不会接受任何挂起,则协程继续(永远不会调用await\_suspend())。
\end{itemize}

awaiter通常用作co\_await的基本awaiter:

\begin{cpp}
co_await std::suspend_always{};
\end{cpp}

也可以在协程承诺的initial\_suspend()和finally\_suspend()中返回。

\mySubsubsection{14.4.3}{恢复子协程}

通过将协程接口设置为可等待的(提供awaiter API),允许它以一种方式处理子协程,使子协程的挂起点成为主协程的挂起点。

这允许开发者避免使用嵌套循环进行恢复,如示例corocoro.cpp所介绍的那样。

以CoroTask的第一个实现为例,只需要以下修改:

\begin{itemize}
\item
协程接口必须知道它的子协程(若有的话):

\begin{cpp}
class CoroTaskSub {
public:
	struct promise_type;
	using CoroHdl = std::coroutine_handle<promise_type>;
private:
	CoroHdl hdl; // native coroutine handle
public:
	struct promise_type {
		CoroHdl subHdl = nullptr; // sub-coroutine (if there is one)
		...
	}
	...
};
\end{cpp}

\item
协程接口必须提供awaiter的API,以便该接口可以作为co\_await的awaitable对象使用:

\begin{cpp}
class CoroTaskSub {
	public:
	...
	bool await_ready() { return false; } // do not skip suspension
	void await_suspend(auto awaitHdl) {
		awaitHdl.promise().subHdl = hdl; // store sub-coroutine and suspend
	}
	void await_resume() { }
};
\end{cpp}

\item
协程接口必须恢复尚未完成的最深层子协程(若有的话):

\begin{cpp}
class CoroTaskSub {
	public:
	...
	bool resume() const {
		if (!hdl || hdl.done()) {
			return false; // nothing (more) to process
		}
		// find deepest sub-coroutine not done yet:
		CoroHdl innerHdl = hdl;
		while (innerHdl.promise().subHdl) {
			if (innerHdl.promise().subHdl.done()) {
				innerHdl.promise().subHdl = nullptr; // remove sub-coroutine if done
				break;
			}
			innerHdl = innerHdl.promise().subHdl;
		}
		innerHdl.resume(); // RESUME
		return !hdl.done();
	}
	...
};
\end{cpp}
\end{itemize}

coro/corotasksub.hpp中为完整的代码。

有了这个,就可以做以下事情了:

\filename{coro/corocorosub.cpp}

\begin{cpp}
#include <iostream>
#include "corotasksub.hpp" // for CoroTaskSub

CoroTaskSub coro()
{
	std::cout << " coro(): PART1\n";
	co_await std::suspend_always{}; // SUSPEND
	std::cout << " coro(): PART2\n";
}

CoroTaskSub callCoro()
{
	std::cout << " callCoro(): CALL coro()\n";
	co_await coro(); // call sub-coroutine
	std::cout << " callCoro(): coro() done\n";
	co_await std::suspend_always{}; // SUSPEND
	std::cout << " callCoro(): END\n";
}

int main()
{
	auto coroTask = callCoro(); // initialize coroutine
	std::cout << "MAIN: callCoro() initialized\n";

	while (coroTask.resume()) { // RESUME
		std::cout << "MAIN: callCoro() suspended\n";
	}

	std::cout << "MAIN: callCoro() done\n";
}
\end{cpp}

callCoro()内部,可以通过将其传递给co\_await()来调用coro():

\begin{cpp}
CoroTaskSub callCoro()
{
	...
	co_await coro(); // call sub-coroutine
	...
}
\end{cpp}

这里发生了如下的事情:

\begin{itemize}
\item
coro()的调用初始化协程,并产生CoroTaskSub类型的协程接口。

\item
因为这个类型有一个awaiter接口,协程接口可以作为co\_await的awaitable接口使用。

然后用两个操作数调用操作符co\_await:

\begin{itemize}
\item
调用它的等待协程

\item
调用的协程
\end{itemize}

\item
awaiter通常的行为是:

\begin{itemize}
\item
await\_ready()通常询问是否拒绝等待请求。答案是“不”(错误)。

\item
await\_suspend()以等待协程句柄作为参数调用。
\end{itemize}

\item
通过存储子协程的句柄(调用await\_suspend()的对象)作为传递的等待句柄的子协程,callCoro()现在知道其子协程。

\item
通过在await\_suspend()中不返回任何值,挂起最终被接受,这意味着:

\begin{cpp}
co_await coro();
\end{cpp}

挂起callCoro()并将控制流传输回main()。

\item
当main()然后恢复callCoro()时,CoroTaskSub::resume()的实现发现coro()作为其最深层的子协程并恢复它。

\item
每当子协程挂起时,CoroTaskSub::resume()返回给调用者。

\item
这个过程一直持续到子协同程序完成,接下来将恢复callCoro()。
\end{itemize}

最后,程序有以下输出:

\begin{shell}
MAIN: callCoro() initialized
  callCoro(): CALL coro()
MAIN: callCoro() suspended
    coro(): PART1
MAIN: callCoro() suspended
    coro(): PART2
MAIN: callCoro() suspended
  callCoro(): coro() done
MAIN: callCoro() suspended
  callCoro(): END
MAIN: callCoro() done
\end{shell}

\mySamllsection{直接恢复调用的子协程}

注意,前面的CoroTaskSub实现调用

\begin{cpp}
co_await coro(); // call sub-coroutine
\end{cpp}

有一个挂起点。初始化coro(),但在启动它之前,将控制流转移回callCoro()的调用者。

也可以在这里直接启动coro(),只需要对await\_suspend()进行如下修改:

\begin{cpp}
auto await_suspend(auto awaitHdl) {
	awaitHdl.promise().subHdl = hdl; // store sub-coroutine
	return hdl; // and resume it directly
}
\end{cpp}

若await\_suspend()返回协程句柄,则立即恢复该协程。这会将程序的行为改变为以下输出:

\begin{shell}
MAIN: callCoro() initialized
  callCoro(): CALL coro()
    coro(): PART1
MAIN: callCoro() suspended
  callCoro(): coro() done
MAIN: callCoro() suspended
  callCoro(): END
MAIN: callCoro() done
\end{shell}

将另一个协程返回到resume可用于恢复co\_await上的任何其他协程,稍后将使用它进行对称传输。

如前所见,await\_suspend()的返回类型可以有所不同。

通常,若await\_suspend()发出不应该有挂起的信号,则永远不会调用await\_resume()。

\mySubsubsection{14.4.4}{挂起后将值传递回协程}

awaitables和awaiter的另一个应用是允许在挂起后将值传递回协程。考虑以下协程:

\filename{coro/coyieldback.hpp}

\begin{cpp}
#include <iostream>
#include "corogenback.hpp" // for CoroGenBack

CoroGenBack coro(int max)
{
	std::cout << "              CORO " << max << " start\n";

	for (int val = 1; val <= max; ++val) {
		// print next value:
		std::cout << " CORO " << val << '/' << max << '\n';

		// yield next value:
		auto back = co_yield val; // SUSPEND with value with response
		std::cout << "           CORO => " << back << "\n";
	}

	std::cout << "             CORO " << max << " end\n";
}
\end{cpp}

同样,协程迭代到最大值,并在挂起时将当前值返回给调用者。这一次,co\_yield从调用者返回一个值给协程:

\begin{cpp}
auto back = co_yield val; // SUSPEND with value with response
\end{cpp}

为了支持这一点,协程接口提供了一些修改后的常用API:

\filename{coro/corogenback.hpp}

\begin{cpp}
#include "backawaiter.hpp"
#include <coroutine>
#include <exception> // for terminate()
#include <string>

class [[nodiscard]] CoroGenBack {
public:
	struct promise_type;
	using CoroHdl = std::coroutine_handle<promise_type>;
private:
	CoroHdl hdl; // native coroutine handle

public:
	struct promise_type {
		int coroValue = 0; // value TO caller on suspension
		std::string backValue; // value back FROM caller after suspension
		auto yield_value(int val) { // reaction to co_yield
			coroValue = val; // - store value locally
			backValue.clear(); // - reinit back value
			return BackAwaiter<CoroHdl>{}; // - use special awaiter for response
		}

		// the usual members:
		auto get_return_object() { return CoroHdl::from_promise(*this); }
		auto initial_suspend() { return std::suspend_always{}; }
		void return_void() { }
		void unhandled_exception() { std::terminate(); }
		auto final_suspend() noexcept { return std::suspend_always{}; }
	};

	// constructors and destructor:
	CoroGenBack(auto h) : hdl{h} { }
	~CoroGenBack() { if (hdl) hdl.destroy(); }

	// no copying or moving:
	CoroGenBack(const CoroGenBack&) = delete;
	CoroGenBack& operator=(const CoroGenBack&) = delete;

	// API:
	// - resume the coroutine:
	bool resume() const {
	if (!hdl || hdl.done()) {
		return false; // nothing (more) to process
	}
	hdl.resume(); // RESUME
	return !hdl.done();
	}

	// - yield value from co_yield:
	int getValue() const {
		return hdl.promise().coroValue;
	}

	// - set value back to the coroutine after suspension:
	void setBackValue(const auto& val) {
		hdl.promise().backValue = val;
	}
};
\end{cpp}

修改内容如下:

\begin{itemize}
\item
promise\_type有一个新成员backValue。

\item
yield\_value()返回一个特殊的BackAwaiter类型。

\item
协程接口有一个新成员setBackValue(),用于将值发送回协程。
\end{itemize}

让我们详细讨论一下这些变化。

\mySamllsection{promise的数据成员}

通常,协程接口的promise类型是协程与调用者共享和交换数据的最佳位置:

\begin{cpp}
class CoroGenBack {
	...
	public:
	struct promise_type {
		int coroValue = 0; // value TO caller on suspension
		std::string backValue; // value back FROM caller after suspension
		...
	};
	...
};
\end{cpp}

现在这里有两个数据成员:

\begin{itemize}
\item
coroValue从协程传递给调用者的值

\item
backValue从调用者传回给协程的值
\end{itemize}

\mySamllsection{co\_yield的新awaiter}

要通过promise将值传递给调用者,还需要实现yield\_value()。这一次,yield\_value()不会返回std::suspend\_always\{\}。相反,它会返回一个特殊的类型为BackAwaiter的awaiter对象,能够返回调用者返回的值:

\begin{cpp}
class CoroGenBack {
	...
	public:
	struct promise_type {
		int coroValue = 0; // value TO caller on suspension
		std::string backValue; // value back FROM caller after suspension

		auto yield_value(int val) { // reaction to co_yield
			coroValue = val; // - store value locally
			backValue.clear(); // - reinit back value
			return BackAwaiter<CoroHdl>{}; // - use special awaiter for response
		}
		...
	};
	...
};
\end{cpp}

awaiter的定义如下所示:

\filename{coro/backawaiter.hpp}

\begin{cpp}
template<typename Hdl>
class BackAwaiter {
	Hdl hdl = nullptr; // coroutine handle saved from await_suspend() for await_resume()
	public:
	BackAwaiter() = default;

	bool await_ready() const noexcept {
		return false; // do suspend
	}

	void await_suspend(Hdl h) noexcept {
		hdl = h; // save handle to get access to its promise
	}

	auto await_resume() const noexcept {
		return hdl.promise().backValue; // return back value stored in the promise
	}
};
\end{cpp}

awaiter所做的只是将传递给await\_suspend()的协程句柄存储在本地,以便在调用await\_resume()时拥有它。在await\_resume()中,使用此句柄从其承诺(由调用者使用setBackValue()存储)中产生backValue。

yield\_value()的返回值用作co\_yield表达式的值:

\begin{cpp}
auto back = co_yield val; // co_yield yields return value of yield_value()
\end{cpp}

BackAwaiter是通用的,因为它可以用于具有任意类型成员backValue的所有协程句柄。

\mySamllsection{挂起后返回值的协程接口}

通常,必须决定协程接口提供的API,以让调用者返回响应。一个简单的方法是提供成员函数setBackValue():

\begin{cpp}
class CoroGenBack {
	...
	public:
	struct promise_type {
		int coroValue = 0; // value TO caller on suspension
		std::string backValue; // value back FROM caller after suspension
		...
	};
	...
	// - set value back to the coroutine after suspension:
	void setBackValue(const auto& val) {
		hdl.promise().backValue = val;
	}
};
\end{cpp}

也可以提供一个接口,生成一个backValue的引用,以支持对它的直接赋值。

\mySamllsection{使用协程}

协程也可以这样用:

\filename{coro/coyieldback.cpp}

\begin{cpp}
#include "coyieldback.hpp"
#include <iostream>
#include <vector>

int main()
{
	// start coroutine:
	auto coroGen = coro(3); // initialize coroutine
	std::cout << "**** coro() started\n";

	// loop to resume the coroutine until it is done:
	std::cout << "\n**** resume coro()\n";
	while (coroGen.resume()) { // RESUME
		// process value from co_yield:
		auto val = coroGen.getValue();
		std::cout << "**** coro() suspended with " << val << '\n';

		// set response (the value co_yield yields):
		std::string back = (val % 2 != 0 ? "OK" : "ERR");
		std::cout << "\n**** resume coro() with back value: " << back << '\n';
		coroGen.setBackValue(back); // set value back to the coroutine
	}

	std::cout << "**** coro() done\n";
}
\end{cpp}

该程序有以下输出:

\begin{shell}
**** coro() started

**** resume coro()
         CORO 3 start
         CORO 1/3
**** coro() suspended with 1

**** resume coro() with back value: OK
         CORO => OK
         CORO 2/3
**** coro() suspended with 2

**** resume coro() with back value: ERR
         CORO => ERR
         CORO 3/3
**** coro() suspended with 3

**** resume coro() with back value: OK
         CORO => OK
         CORO 3 end
**** coro() done
\end{shell}

协程接口具有以不同方式处理事物的所有灵活性。例如,可以将响应作为resume()的参数传递,或者在一个成员中共享生成值及其响应。









