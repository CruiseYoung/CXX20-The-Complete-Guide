
Let us look at some aspects of abbreviated function templates in detail.

\subsubsection*{\zihao{3} 2.3.1\hspace{0.2cm}Basic Constraints for auto Parameters}
\addcontentsline{toc}{subsubsection}{2.3.1\hspace{0.2cm}Basic Constraints for auto Parameters}

Using auto to declare a function parameter follows the same rules as using it to declare a parameter of a lambda:

\begin{itemize}
\item
For each parameter declared with auto, the function has an implicit template parameter.

\item
The parameters can be a parameter pack:
\begin{lstlisting}[style=styleCXX]
void foo(auto... args);
\end{lstlisting}

This is equivalent to the following (without introducing Types):
\begin{lstlisting}[style=styleCXX]
template<typename... Types>
void foo(Types... args);
\end{lstlisting}

\item
Using decltype(auto) is not allowed.
\end{itemize}

Abbreviated function templates can still be called with (partially) explicitly specified template parameters. The template parameters have the order of the call parameters.

For example:

\begin{lstlisting}[style=styleCXX]
void foo(auto x, auto y)
{
	...
}

foo("hello", 42); // x has type const char*, y has type int
foo<std::string>("hello", 42); // x has type std::string, y has type int
foo<std::string, long>("hello", 42); // x has type std::string, y has type long
\end{lstlisting}

\subsubsection*{\zihao{3} 2.3.2\hspace{0.2cm}Combining Template and auto Parameters}
\addcontentsline{toc}{subsubsection}{2.3.2\hspace{0.2cm}Combining Template and auto Parameters}

Abbreviated function templates can still have explicitly specified template parameters. The generated template parameters for the placeholder types are added after the specified parameters:

\begin{lstlisting}[style=styleCXX]
template<typename T>
void foo(auto x, T y, auto z)
{
	...
}

foo("hello", 42, '?'); // x has type const char*, T and y are int, z is char
foo<long>("hello", 42, '?'); // x has type const char*, T and y are long, z is char
\end{lstlisting}

Therefore, the following declarations are equivalent (except that you have no name for the types where auto is used):

\begin{lstlisting}[style=styleCXX]
template<typename T>
void foo(auto x, T y, auto z);

template<typename T, typename T2, typename T3>
void foo(T2 x, T y, T3 z);
\end{lstlisting}

As we introduce later, by using concepts as type constraints, you can constrain placeholder parameters as well as template parameters. Template parameters may then be used for such a qualification.

For example, the following declaration ensures that the second parameter y has an integral type and that the third parameter z has a type that can be converted to the type of y:

\begin{lstlisting}[style=styleCXX]
template<std::integral T>
void foo(auto x, T y, std::convertible_to<T> auto z)
{
	...
}

foo(64, 65, 'c'); // OK, x is int, T and y are int, z is char
foo(64, 65, "c"); // ERROR: "c" cannot be converted to type int (type of 65)
foo<long,char>(64, 65, 'c'); // NOTE: x is char, T and y are long, z is char
\end{lstlisting}

Note that the last statement specifies the type of the parameters in the wrong order.

The fact that the order of template parameters is not as expected might lead to subtle bugs. Consider the following example:

\noindent
\hspace*{\fill} \\ %插入空行
\textit{lang/tmplauto.cpp}

\begin{lstlisting}[style=styleCXX]
#include <vector>
#include <ranges>

void addValInto(const auto& val, auto& coll)
{
	coll.insert(val);
}

template<typename Coll> // Note: different order of template parameters
requires std::ranges::random_access_range<Coll>
void addValInto(const auto& val, Coll& coll)
{
	coll.push_back(val);
}

int main()
{
	std::vector<int> coll;
	addValInto(42, coll); // ERROR: ambiguous
}
\end{lstlisting}

Due to using auto only for the first parameter in the second declaration of addValInto(), the order of the template parameters differs. Due to \url{http://wg21.link/p2113r0}, which was accepted for C++20, this means that overload resolution does not prefer the second declaration over the first one and we get an ambiguity error.[Not all compilers handle this correctly, yet.]

For this reason, be careful when mixing template and auto parameters. Ideally, make the declarations consistent.











