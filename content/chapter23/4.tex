C++20 introduces constants for the most important mathematical floating-point constants. Table Math constants lists them.

The constants are provided in the header file <numbers> in the namespace std::numbers.

% Please add the following required packages to your document preamble:
% \usepackage{longtable}
% Note: It may be necessary to compile the document several times to get a multi-page table to line up properly
\begin{longtable}[c]{|l|l|}
\hline
\textbf{Constant}        & \textbf{Template}                                    \\ \hline
\endfirsthead
%
\endhead
%
std::number::e           & std::number::e\_v\textless{}\textgreater{}           \\ \hline
std::number::pi          & std::number::pi\_v\textless{}\textgreater{}          \\ \hline
std::number::inv\_pi     & std::number::inv\_pi\_v\textless{}\textgreater{}     \\ \hline
std::number::inv\_sqrtpi & std::number::inv\_sqrtpi\_v\textless{}\textgreater{} \\ \hline
std::number::sqrt2       & std::number::sqrt2\_v\textless{}\textgreater{}       \\ \hline
std::number::sqrt3       & std::number::sqrt3\_v\textless{}\textgreater{}       \\ \hline
std::number::inv\_sqrt3  & std::number::inv\_sqrt3\_v\textless{}\textgreater{}  \\ \hline
std::number::log2e       & std::number::log2e\_v\textless{}\textgreater{}       \\ \hline
std::number::log10e      & std::number::log10e                                  \\ \hline
std::number::ln2         & std::number::ln2\_v\textless{}\textgreater{}         \\ \hline
std::number::ln10        & std::number::ln10\_v                                 \\ \hline
std::number::egamma      & std::number::egamma\_v\textless{}\textgreater{}      \\ \hline
std::number::phi         & std::number::phi\_v\textless{}\textgreater{}         \\ \hline
\end{longtable}

\begin{center}
Table 23.2. Math constants
\end{center}

The constants are specializations for type double of corresponding variable templates that have the suffix \_v. The values are the nearest representable values of the corresponding type. For example:

\begin{cpp}
namespace std::number {
	template<std::floating_point T> inline constexpr T pi_v<T> = ... ;
	inline constexpr double pi = pi_v<double>;
}
\end{cpp}

As you can see, the definitions use the std::floating\_point concept (which was introduced for that reason).

Therefore, you can use them as follows:

\begin{cpp}
#include <numbers>
...

double area1 = rad * rad * std::numbers::pi;

long double area2 = rad * rad * std::numbers::pi_v<long double>;
\end{cpp}






















