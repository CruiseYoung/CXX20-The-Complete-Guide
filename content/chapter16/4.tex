
本节描述使用模块的一些其他细节。

\mySubsubsection{16.4.1}{私有模块}

若在主接口中声明一个模块,有时可能需要一个私有模块。这允许开发者在主接口中拥有声明和定义,这些声明和定义对任何其他模块或翻译单元都是不可见或不可达的。使用私有模块片段的一种方法是禁用导出类或函数的定义,尽管声明是导出的。

例如,考虑下面的主接口:

\begin{cpp}
export module MyMod;

export class C; // class C is exported
export void print(const C& c); // print() is exported

class C { // provides details of the exported class
private:
	int value;
public:
	void print() const;
};

void print(const C& c) { // provides details of the exported function
	c.print();
}
\end{cpp}

首先用export声明类C和函数print():

\begin{cpp}
export module MyMod;

export class C; // class C is exported
export void print(const C& c); // print() is exported
\end{cpp}

export只能在命名空间中引入名称时指定一次,稍后还会导出详细信息。翻译单元都可以导入该模块,并使用C类型的对象:

\begin{cpp}
import MyMod;
...

C c; // OK, definition of class C was exported
print(c); // OK (compiler can replace the function call with its body)
\end{cpp}

若想在模块中封装定义,以便导入代码只看到声明,并且仍然希望在主接口中拥有定义,必须将定义放在私有模块中:

\begin{cpp}
export module MyMod;

export class C; // declaration is exported
export void print(const C& c); // declaration is exported

module :private; // following symbols are not even implicitly exported

class C { // complete class not exported
	private:
	int value;
	public:
	void print() const;
};

void print(const C& c) { // definition not exported
	c.print();
}
\end{cpp}

声明私有模块

\begin{cpp}
module :private;
\end{cpp}

只能发生在主接口中,并且只能发生一次。有了这个声明,文件的其余部分不再隐式导出(甚至不是隐式导出)。之后使用export来导出内容都是错误的。

通过将定义移动到私有模块中,导入代码就不能再使用其中的任何定义,只能使用类C的前向声明(类C是不完整类型)和print()。

例如,不能创建C类型的对象:

\begin{cpp}
import MyMod;
...

C c; // ERROR (C only declared, not defined)
print(c); // OK (compiler can replace the function call with its body)
\end{cpp}

声明还是可以使用C类型的引用和指针:

\begin{cpp}
import MyMod;
...

void foo(const C& c) { // OK
	print(c); // OK
}
\end{cpp}

\mySubsubsection{16.4.2}{详细介绍模块的声明和导出}

模块单元必须以下列之一开头(初始注释和空格之后):

\begin{itemize}
\item
module;

\item
export module name;

\item
module name;

\item
module name:partname;

\item
export module name:partname;
\end{itemize}

若一个模块单元以module开头;要引入全局模块,其他模块声明中的一个必须遵循全局模块中的预处理器命令。

模块中,可以导出所有具有名称的符号:

\begin{itemize}
\item
可以导出具有名称的命名空间,将导出命名空间声明中定义的所有符号。例如:

\begin{cpp}
export namespace MyMod {
	... // exported symbols of namespace MyMod
}

namespace MyMod {
	... // symbols of namespace MyMod not exported
}

export namespace MyMod {
	... // more exported symbols of namespace MyMod
}
\end{cpp}

\item
可以导出类型,这将导出所有成员(若有的话)。例如:

\begin{cpp}
export class MyClass;
export struct MyStruct;
export union MyUnion;
export enum class MyEnum;
export using MyString = std::string;
\end{cpp}

不必导出类成员或枚举值。若导出类型,则自动导出枚举类型的类成员和值。

\item
可以导出对象。例如:

\begin{cpp}
export std::string progname;

namespace MyStream {
	export using std::cout; // export std::cout as MyStream::cout
}

export auto myLambda = [] {};
\end{cpp}

\item
可以导出函数。例如:

\begin{cpp}
export friend std::ostream& operator<< (std::ostream&, const MyType&);
\end{cpp}
\end{itemize}

声明的实体应该导出,以导出作为首次声明。可以稍后再次指定它是导出的,但是不允许在没有导出的情况下声明实体,并在以后使用导出声明/定义实体。[然而,编译器已经接受了这一点。]

在未命名的命名空间、静态对象和私有模块片段中不允许导出。

导出都不需要内联。形式上,模块中的定义总是只存在一次。

即使对象也被另一个模块重新导出,这也适用。

\mySubsubsection{16.4.3}{伞形模块}

模块可以导出所有内容。对于导入的接口分区,也需要导出。

要导出导入的符号,可以使用using:

\begin{cpp}
export module MyMod; // declare module MyMod

// export all symbols from OtherModule as a whole:
export import OtherModule;

// import LogModule to export parts of it:
import LogModule

// export Logger in the namespace LogModule as ::Logger:
export using LogModule::Logger;

// export Logger in the namespace LogModule as LogModule::Logger:
export namespace LogModule {
	using LogModule::Logger;
}

// export global symbol globalLogger:
export using ::globalLogger;

// export global symbol log (e.g., function log()):
export using ::log;
\end{cpp}

\mySubsubsection{16.4.4}{模块导入的详情}

通过import,C++源代码文件都可以导入一个模块来使用导出的函数、类型和对象。

import不是一个普通的关键字,是一个上下文关键字。仍然可以使用标识符导入命名其他组件,尽管不建议这样做。其没有作为关键字引入,所以可能会破坏太多现有的代码。使用import的翻译单元或模块单元,必须在模块预编译之后进行编译。

否则,可能会得到模块未定义的错误,或者更糟的是,可能会针对旧版本的模块进行编译,所以不能循环导入。

\mySubsubsection{16.4.5}{可达符号与可见符号的详情}

让我们看一些更多的详情,并举例说明导出和导入符号的可见性和可达性。

导入模块时,还可以间接导入导出API所使用的所有类型。若没有显式导出这些类型,则可以使用包含其所有成员函数的类型,但不能使用独立函数。

下面的模块将getPerson()导出为一个可见的符号,将Person导出为一个可访问的类:

\filename{modules/person1.cppm}

\begin{cpp}
module;
#include <iostream>
#include <string>

export module ModPerson; // THE module interface

class Person { // note: not exported
	std::string name;
public:
	Person(std::string n)
	: name{std::move(n)} {
	}
	std::string getName() const {
		return name;
	}
};

std::ostream& operator<< (std::ostream& strm, const Person& p)
{
	return strm << p.getName();
}

export Person getPerson(std::string s) {
	return Person{s};
}
\end{cpp}

导入该模块会产生以下结果:

\begin{cpp}
#include <iostream>
import ModPerson; // import module ModPerson
...
Person p1{"Cal"}; // ERROR: Person not visible
Person p2 = getPerson("Kim"); // ERROR: Person not visible
auto p3 = getPerson("Tana"); // OK
std::string s1 = p3.getName(); // ERROR (unless <iostream> includes <string>)
auto s2 = p3.getName(); // OK
std::cout << p3 << '\n'; // ERROR: free-standing operator<< not exported
std::cout << s2 << '\n'; // OK
\end{cpp}

若这段代码还包括字符串的头文件,则s1的声明编译为:

\begin{cpp}
#include <iostream>
#include <string>
import ModPerson; // import module ModPerson
...
Person p1{"Cal"}; // ERROR: Person not visible
Person p2 = getPerson("Kim"); // ERROR: Person not visible
auto p3 = getPerson("Tana"); // OK
std::string s1 = p3.getName(); // OK
auto s2 = p3.getName(); // OK
std::cout << p3 << '\n'; // ERROR: free-standing operator<< not exported
std::cout << s2 << '\n'; // OK
\end{cpp}

若在Person类中声明<{}<操作符作为一个隐藏的友元(应该这样做):

\begin{cpp}
export module ModPerson;

class Person {
	...
	friend std::ostream& operator<< (std::ostream& strm, const Person& p) {
		return strm << p.getName();
	}
};
\end{cpp}

成员操作符变为可访问的:

\begin{cpp}
member operator becomes reachable:
auto p3 = getPerson("Tana"); // OK
std::cout << p3 << '\n'; // OK (operator<< is reachable now)
\end{cpp}

通过使用私有模块,可以限制间接导出符号的可达性。

\mySamllsection{未导出的符号不能冲突}

间接导出的符号不可见,但可访问的行为允许开发者在其导出的接口中,使用相同符号名称的不同模块。例如,有一个模块定义了一个类Person:

\begin{cpp}
export module ModPerson1;

class Person {
	...
	public:
	std::string getName() const {
		return name;
	}
};

export Person getPerson1(std::string s) {
	return Person{s};
}
\end{cpp}

另一个模块也定义了一个不同的类Person:

\begin{cpp}
export module ModPerson2;

class Person {
	...
	public:
	std::string getName() const {
		return name;
	}
};

export Person getPerson2(std::string s) {
	return Person{s};
}
\end{cpp}

程序可以导入两个模块而不会产生任何冲突,因为唯一可见的符号是第一个模块的getPerson1()和第二个模块的getPerson2()。下面的代码可以正常工作:

\begin{cpp}
auto p1 = getPerson1("Tana");
auto s1 = p1.getName();

auto p2 = getPerson2("Tana");
auto s2 = p2.getName();
\end{cpp}

p1和p2的类型名称相同,但实际类型不同:

\begin{cpp}
std::same_as<decltype(p1), decltype(p2)> // yields false
\end{cpp}











