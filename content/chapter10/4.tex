若为格式指定了L,则使用特定于语言环境的表示法:

\begin{itemize}
\item 
对于bool,使用std::numpunct::truename和std::numpunct::falsename上的区域设置字符串。

\item 
对于整数值,使用依赖于语言环境的千位分隔符。

\item 
对于浮点值,使用依赖于语言环境的小数点和千位分隔字符。

\item 
对于chrono库中的几种类型表示法(持续时间、时间点等),使用特定于语言环境的格式。
\end{itemize}

要激活特定于语言环境的表示法,还必须向std::format()传递一个语言环境。

例如:

\begin{cpp}
// initialize a locale for “German in Germany”:
#ifdef _MSC_VER
std::locale locG{"deu_deu.1252"};
#else
std::locale locG{"de_DE"};
#endif

// use it for formatting:
std::format(locG, "{0} {0:L}", 1000.7) // yields 1000.7 1.000,7
\end{cpp}

参见format/formatgerman.cpp获得完整的示例。

注意,只有在使用语言环境说明符L时才会使用语言环境,否则将使用默认的语言环境“C”,它使用美式格式。

或者,可以设置全局语言环境,并使用L说明符:

\begin{cpp}
std::locale::global(locG); // set German locale globally
std::format("{0} {0:L}", 1000.7) // yields 1000.7 1.000,7
\end{cpp}

可能需要创建自己的区域设置(通常基于带有修改facet的现有区域设置)。例如:

\filename{format/formatbool.cpp}

\begin{cpp}
#include <iostream>
#include <locale>
#include <format>

// define facet for German bool names:
class GermanBoolNames : public std::numpunct_byname<char> {
	public:
	GermanBoolNames (const std::string& name)
	: std::numpunct_byname<char>(name) {
	}
	protected:
	virtual std::string do_truename() const {
		return "wahr";
	}
	virtual std::string do_falsename() const {
		return "falsch";
	}
};

int main()
{
	// create locale with German bool names:
	std::locale locBool{std::cin.getloc(),
						new GermanBoolNames{""}};
						
	// use locale to print Boolean values:
	std::cout << std::format(locBool, "{0} {0:L}\n", false); // false falsch
}
\end{cpp}

该程序有以下输出:

\begin{shell}
false falsch
\end{shell}

要用宽字符串打印值(这是Visual C++的一个问题),格式字符串和参数都必须是宽字符串。例如:

\begin{cpp}
std::wstring city = L"K\u00F6ln"; // K¨oln
auto ws1 = std::format("{}", city); // compile-time ERROR
std::wstring ws2 = std::format(L"{}", city); // OK: ws2 is std::wstring
std::wcout << ws2 << '\n'; // OK
\end{cpp}

目前不支持char8\_t (UTF-8字符)、char16\_t和char32\_t类型的字符串。






