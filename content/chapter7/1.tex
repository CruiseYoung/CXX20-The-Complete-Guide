
正如视图及其使用方式所介绍的那样,C++提供了几个范围适配器和范围工厂,以便可以简单地创建具有最佳性能的视图。

其中几个适配器适用于特定的视图类型。然而,根据传递范围的特征,其中一些可能会创建不同的东西。若适配器已经具有结果的特征,可能只生成可传递的范围。

有一些关键的范围适配器和工厂,可以很容易地创建视图,或将范围转换为具有特定特征(独立于内容)的视图:

\begin{itemize}
\item
std::views::all()是将传递的范围转换为视图的主要范围适配器。

\item
std::views::counted()可将传入的begin和count/size转换为视图的主要范围工厂。

\item
std::views::common()是一个范围适配器,将(开始)迭代器和哨兵(结束迭代器)具有不同类型的范围转换为具有统一的开始和结束类型的视图。
\end{itemize}

本节将介绍这些适配器。

\mySubsubsection{7.1.1}{std::views::all()}

范围适配器std::views::all()是将任何还不是视图的范围转换为视图的适配器,可由一个低成本的句柄来处理范围中的元素。

all(rg)的生成以下结果[C++20最初声明all()可以产生子范围,但不能产生所属视图]。但在C++20发布后,这个问题已经修复了(参见\url{http://wg21.link/p2415}):

\begin{itemize}
\item
若rg已经是视图,则为rg的副本

\item
否则,rg的std::ranges::ref\_view是一个左值(有名称的范围对象)

\item
否则,若rg的std::ranges::owned\_view是右值(未命名的临时范围对象或用std::move()标记的范围对象)
\end{itemize}

例如:

\begin{cpp}
std::vector<int> getColl(); // function returning a tmp. container
std::vector coll{1, 2, 3}; // a container
std::ranges::iota_view aView{1}; // a view

auto v1 = std::views::all(aView); // decltype(coll)
auto v2 = std::views::all(coll); // ref_view<decltype(coll)>
auto v3 = std::views::all(std::views::all(coll)); // ref_view<decltype(coll)>
auto v4 = std::views::all(getColl()); // owning_view<decltype(coll)>
auto v5 = std::views::all(std::move(coll)); // owning_view<decltype(coll)>
\end{cpp}

all()适配器通常用于将范围作为轻量级对象传递。将范围转换成视图有两个原因:

\begin{itemize}
\item
性能:

当复制范围时(形参按值接受实参),使用视图要廉价得多。这样做的原因是移动或(若支持的话)复制一个视图保证是廉价的,所以将范围作为视图传递会使调用变得低成本。

例如:

\begin{cpp}
void foo(std::ranges::input_range auto coll) // NOTE: takes range by value
{
	for (const auto& elem : coll) {
		...
	}
}

std::vector<std::string> coll{ ... };

foo(coll); // OOPS: copies coll
foo(std::views::all(coll)); // OK: passes coll by reference
\end{cpp}

这里使用all()有点像使用引用包装器(用std::ref()或std::cref()创建)。

但all()的好处是传递的内容,仍然支持通常的范围接口。

将容器传递给协程(协程通常必须按值接受参数)可能是另一种应用。

\item
视图的支持性要求:

使用all()的另一个原因是为了满足需要视图的约束,这个要求的一个原因可能是为了确保传递参数的代价不高(这就带来了上面提到的好处)。

例如:

\begin{cpp}
void foo(std::ranges::view auto coll) // NOTE: takes view by value
{
	for (const auto& elem : coll) {
		...
	}
}

std::vector<std::string> coll{ ... };

foo(coll); // ERROR
foo(std::views::all(coll)); // OK (passes coll by reference)
\end{cpp}

使用all()可能隐式地发生。

使用all()进行隐式转换的示例是使用容器初始化视图类型。
\end{itemize}

\mySamllsection{std::views::all\_t<>类型}

C++20还定义了类型std::views::all\_t<>作为all()产生的类型,其遵循完全转发规则,若依值类型和右值引用类型都可以用来指定右值的类型:

\begin{cpp}
std::vector<int> v{0, 8, 15, 47, 11, -1, 13};
...
std::views::all_t<decltype(v)> a1{v}; // ERROR
std::views::all_t<decltype(v)&> a2{v}; // ref_view<vector<int>>
std::views::all_t<decltype(v)&&> a3{v}; // ERROR
std::views::all_t<decltype(v)> a4{std::move(v)}; // owning_view<vector<int>>
std::views::all_t<decltype(v)&> a5{std::move(v)}; // ERROR
std::views::all_t<decltype(v)&&> a6{std::move(v)}; // owning_view<vector<int>>
\end{cpp}

接受范围的视图通常使用类型std::views::all\_t<>来确保传递的范围确实是一个视图。因此,若传递的范围还不是视图,则会隐式创建视图。例如:

\begin{cpp}
std::views::take(coll, 3)
\end{cpp}

具有相同的效果:

\begin{cpp}
std::ranges::take_view{std::ranges::ref_view{coll}, 3};
\end{cpp}

其工作方式如下(以视图为例):

\begin{itemize}
\item
视图类型要求传递视图:

\begin{cpp}
namespace std::ranges {
	template<view V>
	class take_view : public view_interface<take_view<V>> {
		public:
		constexpr take_view(V base, range_difference_t<V> count);
		...
	};
}
\end{cpp}

\item
推导指南要求视图的元素类型为std::views::all\_t:

\begin{cpp}
namespace std::ranges {
	// deduction guide to force the conversion to a view:
	template<range R>
	take_view(R&&, range_difference_t<R>) -> take_view<views::all_t<R>>;
}
\end{cpp}

\item
当传递容器时,视图的范围类型必须具有std::views::all\_t<>类型,所以容器会隐式地转换为ref\_view(若是左值)或owned \_view(若是右值)。其效果就好像下面的语句:

\begin{cpp}
std::ranges::ref_view rv{coll}; // convert to a ref_view<>
std::ranges::take_view tv(rv, 3); // and use view and count to initialize the take_view<>
\end{cpp}
\end{itemize}

概念viewable\_range可用于检查类型是否可用于all\_t<>(因此对应的对象可以传递给all())[对于仅移动视图类型的左值,尽管viewable\_range是满足的,但all()仍存在格式错误的问题]:

\begin{cpp}
std::ranges::viewable_range<std::vector<int>> // true
std::ranges::viewable_range<std::vector<int>&> // true
std::ranges::viewable_range<std::vector<int>&&> // true
std::ranges::viewable_range<std::ranges::iota_view<int>> // true
std::ranges::viewable_range<std::queue<int>> // false
\end{cpp}

\mySubsubsection{7.1.2}{std::views::counted()}

范围工厂std::views::counted()提供了从begin迭代器和计数创建视图的最灵活的方式。

\begin{cpp}
std::views::counted(beg, sz)
\end{cpp}

创建一个以beg开头的范围的前sz个元素的视图。

使用视图时,由开发者来确保begin和count是有效的。否则,程序具的行为未定义。

计数可能为0,所以该范围为空。

计数存储在视图中,因此是稳定的。即使在视图引用的范围中插入了新元素,计数也不会改变。例如:

\begin{cpp}
std::list lst{1, 2, 3, 4, 5, 6, 7, 8};

auto c = std::views::counted(lst.begin(), 5);
print(c); // 1 2 3 4 5

lst.insert(++lst.begin(), 0); // insert new second element in lst
print(c); // 1 0 2 3 4
\end{cpp}

若想查看一个范围的前num个元素,获取视图提供了一种更方便、更安全的方式来获取:

\begin{cpp}
std::list lst{1, 2, 3, 4, 5, 6, 7, 8};
auto v1 = std::views::take(lst, 3); // view to first three elements (if they exist)
\end{cpp}

视图检查是否有足够的元素,若没有,则产生更少的元素。

通过使用丢弃视图,甚至可以跳过特定数量的前置元素:

\begin{cpp}
std::list lst{1, 2, 3, 4, 5, 6, 7, 8};
auto v2 = std::views::drop(lst, 2) | std::views::take(3); // 3rd to 5th element (if exist)
\end{cpp}

\mySamllsection{counted()的类型}

counted(beg, sz)根据所调用的范围特征,会产生不同类型的结果:

\begin{itemize}
\item
若传递的开始迭代器是一个contiguous\_iterator(指存储在连续内存中的元素),则会产生一个std::span。这适用于std::vector<>或std::array<>的原始指针、原始数组和迭代器。

\item
否则,若传递的begin迭代器是一个random\_access\_iterator(支持在元素之间来回跳转),就会产生std::ranges::subrange。这适用于std::deque<>的迭代器。

\item
否则,将产生std::ranges::subrange,以std::counts\_iterator作为开始,并以std::default\_sentinel\_t类型的伪哨兵作为结束,所以子范围中的迭代器在迭代时计数。这适用于列表、关联容器和无序容器(哈希表)的迭代器。
\end{itemize}

例如:

\begin{cpp}
std::vector<int> vec{1, 2, 3, 4, 5, 6, 7, 8, 9};
auto vec5 = std::ranges::find(vec, 5);
auto v1 = std::views::counted(vec5, 3); // view to element 5 and two more elems in vec
	// v1 is std::span<int>
std::deque<int> deq{1, 2, 3, 4, 5, 6, 7, 8, 9};
auto deq5 = std::ranges::find(deq, 5);
auto v2 = std::views::counted(deq5, 3); // view to element 5 and two more elems in deq
	// v2 is std::ranges::subrange<std::deque<int>::iterator>
std::list<int> lst{1, 2, 3, 4, 5, 6, 7, 8, 9};
auto lst5 = std::ranges::find(lst, 5);
auto v3 = std::views::counted(lst5, 3); // view to element 5 and two more elems in lst
	// v3 is std::ranges::subrange<std::counted_iterator<std::list<int>::iterator>,
	// std::default_sentinel_t>
\end{cpp}

注意,这段代码有风险,因为若集合中没有5,后面有两个元素,会在创建时出现未定义行为。因此,若不确定情况是否如此,则应该检查此选项。

\mySubsubsection{7.1.3}{std::views::common()}

范围适配器std::views::common()为传递的范围生成一个具有统一类型的视图,用于开始迭代器和哨兵(结束迭代器)。其行为类似于范围适配器std::views::all(),若其迭代器具有不同的类型,则会从传递的参数创建std::ranges::common\_view。

假设想调用传统算法algo(),该算法要求开始迭代器和结束迭代器的类型相同。然后,可以为不同类型的迭代器进行调用,如下所示:

\begin{cpp}
template<typename BegT, typename EndT>
void callAlgo(BegT beg, EndT end)
{
	auto v = std::views::common(std::ranges::subrange(beg,end));
	algo(v.begin(), v.end()); // assume algo() requires iterators with the same type
}
\end{cpp}

common(rg)会生成的类型为

\begin{itemize}
\item
若rg已经是一个具有相同开始和结束迭代器类型的视图,则为rg的副本

\item
否则,若rg是一个具有相同开始迭代器和结束迭代器类型的范围对象,则调用rg的std::ranges::ref\_view

\item
否则,rg为std::ranges::common\_view类型
\end{itemize}

例如:

\begin{cpp}
std::list<int> lst;
std::ranges::iota_view iv{1, 10};
...
auto v1 = std::views::common(lst); // std::ranges::ref_view<decltype(lst)>
auto v2 = std::views::common(iv); // decltype(iv)
auto v3 = std::views::common(std::views::all(vec));
			// std::ranges::ref_view<decltype(lst)>
			
std::list<int> lst {1, 2, 3, 4, 5, 6, 7, 8, 9};
auto vt = std::views::take(lst, 5); // begin() and end() have different types
auto v4 = std::views::common(vt); // std::ranges::common_view<decltype(vt)>
\end{cpp}

注意,std::ranges::common\_view的构造函数和辅助类型std::common:iterator都要求传递的迭代器具有不同的类型,所以若不知道迭代器类型是否不同,则应该使用此适配器。











