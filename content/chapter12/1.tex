
C++11引入了std::thread类型,其与操作系统提供的线程对应,但该类型有一个严重的设计缺陷:不是RAII类型。

让我们看看为什么会有这么一个问题,以及新的线程类型如何解决这个问题。

\mySubsubsection{12.1.1}{std::thread的问题}

std::thread要求在其生命周期结束时,若表示正在运行的线程,则调用join()(等待线程结束)或detach()(让线程在后台运行)。若两者都没有调用,析构函数会立即导致异常的程序终止(在某些系统上导致段错误)。由于这个原因,下面的代码会出现错误(除非不关心程序异常的终止):

\begin{cpp}
void foo()
{
	...
	// start thread calling task() with name and val as arguments:
	std::thread t{task, name, val};
	... // neither t.join() nor t.detach() called
} // std::terminate() called
\end{cpp}

当没有调用join()或detach()就表示正在运行的线程t在析构时,程序会调用std::terminate(),后者调用std::abort()。

即使使用join()来等待正在运行的线程结束,仍然会有一个严重的问题:

\begin{cpp}
void foo()
{
	...
	// start thread calling task() with name and val as arguments:
	std::thread t{task, name, val};
	... // calls std::terminate() on exception
	// wait for tasks to finish:
	t.join();
	...
}
\end{cpp}

这段代码还可能导致程序异常终止,线程开始和调用join()之间的foo()中发生异常时(或者控制流从未到达join()),没有调用t.join()。

代码如下所示:

\begin{cpp}
void foo()
{
	...
	// start thread calling task() with name and val as arguments:
	std::thread t{task, name, val};
	try {
		... // might throw an exception
	}
	catch (...) { // if we have an exception
		// clean up the thread started:
		t.join(); // - wait for thread (blocks until done)
		throw; // - and rethrow the caught exception
	}
	// wait for thread to finish:
	t.join();
	...
}
\end{cpp}

这里,通过确保在离开作用域时调用join()来对异常作出反应,而不解决异常。不幸的是,这可能会导致阻塞(永远)。然而,调用detach()也是一个问题,因为线程在程序的后台继续运行,使用CPU时间和资源,而这些时间和资源现在可能会销毁。

若在更复杂的上下文中使用多个线程,问题会变得更糟,并且会产生非常糟糕的代码。例如,当只启动两个线程时:

\begin{cpp}
void foo()
{
	...
	// start thread calling task1() with name and val as arguments:
	std::thread t1{task1, name, val};
	std::thread t2;
	try {
		// start thread calling task2() with name and val as arguments:
		t2 = std::thread{task2, name, val};
		...
	}
	catch (...) { // if we have an exception
		// clean up the threads started:
		t1.join(); // wait for first thread to end
		if (t2.joinable()) { // if the second thread was started
			t2.join(); // - wait for second thread to end
		}
		throw; // and rethrow the caught exception
	}
	// wait for threads to finish:
	t1.join();
	t2.join();
	...
}
\end{cpp}

启动第一个线程之后,启动第二个线程可能会抛出异常,因此启动第二个线程必须在try子句中进行。另一方面,可在同一范围内使用和join()两个线程。为了满足这两个需求,必须前向声明第二个线程,并在第一个线程的try子句中对其进行赋值。此外,对于异常,必须检查第二个线程是否启动,对没有关联线程的线程对象调用join()会导致异常。

另一个问题是,对两个线程调用join()可能会花费大量时间(甚至永远)。注意,不能“杀死”已经启动的线程。线程不是进程,线程只能通过结束自身或结束整个程序来结束。

因此,在调用join()之前,应该确保等待的线程将取消其执行。不过,对于std::thread,没有这样的机制,必须自己实现取消请求和对它的响应。

\mySubsubsection{12.1.2}{使用std::jthread}

std::jthread解决了这些问题,它是RAII类型。若线程是可汇入的(“j”代表“汇入”),析构函数会自动调用join()。这让上面的复杂代码变简单了:

\begin{cpp}
void foo()
{
	...
	// start thread calling task1() with name and val as arguments:
	std::jthread t1{task1, name, val};
	// start thread calling task2() with name and val as arguments:
	std::jthread t2{task2, name, val};
	...
	// wait for threads to finish:
	t1.join();
	t2.join();
	...
}
\end{cpp}

使用std::jthread就不再存在导致异常程序终止的危险,也不需要异常处理。为了支持尽可能容易地切换到std::jthread类,该类提供了与std::thread相同的API,包括:

\begin{itemize}
\item
使用相同的头文件<thread>

\item
当调用get\_id()时返回std::thread::id (std::jthread::id类型只是一个别名类型)

\item
提供静态成员hardware\_concurrency()
\end{itemize}

只需用std::jthread替换std::thread并重新编译,代码就会变得更安全(若还没有自己实现异常处理的话)。[估计大家想知道为什么不直接修复std::thread,而是引入一个新的std::jthread类型。其原因很简单,即向后兼容性,可能有一些应用希望在离开正在运行的线程的作用域时终止程序。对于接下来讨论的一些新功能,我们也会打破兼容性。]

\mySubsubsection{12.1.3}{停止令牌和停止回调}

类std::jthread做的更多:提供了一种使用停止令牌发出取消信号的机制,这些令牌由jthreads的析构函数在调用join()之前使用,由线程启动的可调用对象(函数、函数对象或Lambda)必须支持此请求:

\begin{itemize}
\item
若可调用对象只为所有传递的参数提供参数,将忽略停止请求:

\begin{cpp}
void task (std::string s, double value)
{
	... // join() waits until this code ends
}
\end{cpp}

\item
为了响应停止请求,可调用对象可以添加一个新的可选第一个类型为std::stop\_token的参数,并不时检查是否请求了停止:

\begin{cpp}
void task (std::stop_token st,
std::string s, double value)
{
	while (!st.stop_requested()) { // stop requested (e.g., by the destructor)?
		... // ensure we check from time to time
	}
}
\end{cpp}
\end{itemize}

所以,std::jthread提供了一种协作机制来表示线程不应该再运行。它是“协作的”,因为该机制不会杀死正在运行的线程(因为C++线程根本不支持杀死线程,所以杀死线程的操作可能很容易使程序处于损坏状态)。为了响应停止请求,已启动的线程必须声明停止令牌作为附加的第一个参数,并使用它不时的检查是否应该继续运行。

也可以手动请求已经启动的jthread停止。例如:

\begin{cpp}
void foo()
{
	...
	// start thread calling task() with name and val as arguments:
	std::jthread t{task, name, val};
	...
	if ( ... ) {
		t.request_stop(); // explicitly request task() to stop its execution
	}
	...
	// wait for thread to finish:
	t.join();
	...
}
\end{cpp}

此外,还有另一种对停止请求作出反应的方法:可以为停止令牌注册回调,该回调将在请求停止时自动调用:

\begin{cpp}
void task (std::stop_token st,
std::string s, double value)
{
	std::stop_callback cb{st, [] {
							   ... // called on a stop request
							  }};
	...
}
\end{cpp}

因此,停止执行task()的线程的请求(无论是显式调用request\_stop()还是由析构函数引起)调用已注册为停止回调的Lambda。请注意,回调通常由请求停止的线程调用。

停止回调函数cb的生命周期结束时,析构函数自动注销回调函数,以便在之后发出停止信号时不再调用该回调函数,也可以通过这种方式注册任意数量的可调用对象(函数、函数对象或Lambda)。

停止机制比最初看起来更灵活:

\begin{itemize}
\item
可以传递句柄来请求停止,并传递令牌来检查请求的停止。

\item
支持条件变量,这样一个有信号的停止就可以中断一个等待。

\item
也可以独立于std::jthread使用这种机制来请求和检查停止。
\end{itemize}

线程、停止源、停止令牌和停止回调之间也没有生命周期约束,存储停止状态的位置在堆上分配。当线程和使用此状态的最后一个停止源、停止令牌或停止回调函数销毁时,用于停止状态的内存将释放。

下面几节中,我将讨论请求停止的机制及其在线程中的应用。之后,会讨论了底层的停止令牌机制。

\mySubsubsection{12.1.4}{停止令牌和条件变量}

当请求停止时,线程可能会因等待条件变量的通知而阻塞(这是避免活动轮询的重要场景)。停止令牌的回调接口也支持这种情况。可以使用传递的停止令牌为条件变量调用wait(),以便在请求停止时暂停等待。由于技术原因,必须使用std::condition\_variable\_any类型作为条件变量。

下面是一个演示在条件变量中使用停止令牌的例子:

\filename{lib/stopcv.cpp}

\begin{cpp}
include <iostream>
#include <queue>
#include <thread>
#include <stop_token>
#include <mutex>
#include <condition_variable>
using namespace std::literals; // for duration literals

int main()
{
	std::queue<std::string> messages;
	std::mutex messagesMx;
	std::condition_variable_any messagesCV;

	// start thread that prints messages that occur in the queue:
	std::jthread t1{[&] (std::stop_token st) {
			while (!st.stop_requested()) {
				std::string msg;
				{
					// wait for the next message:
					std::unique_lock lock(messagesMx);
					if (!messagesCV.wait(lock, st,
										[&] {
											return !messages.empty();
										})) {
						return; // stop requested
					}
					// retrieve the next message out of the queue:
					msg = messages.front();
					messages.pop();
				}

				// print the next message:
				std::cout << "msg: " << msg << std::endl;
			}
	}};

	// store 3 messages and notify one waiting thread each time:
	for (std::string s : {"Tic", "Tac", "Toe"}) {
		std::scoped_lock lg{messagesMx};
		messages.push(s);
		messagesCV.notify_one();
	}

	// after some time
	// - store 1 message and notify all waiting threads:
	std::this_thread::sleep_for(1s);
	{
		std::scoped_lock lg{messagesMx};
		messages.push("done");
		messagesCV.notify_all();
	}

	// after some time
	// - end program (requests stop, which interrupts wait())
	std::this_thread::sleep_for(1s);
}
\end{cpp}

启动一个线程,循环等待消息队列不是空的,若不为空则进行输出:

\begin{cpp}
while (!st.stop_requested()) {
	std::string msg;
	{
		// wait for the next message:
		std::unique_lock lock(messagesMx);
		if (!messagesCV.wait(lock, st,
							[&] {
								return !messages.empty();
							})) {
			return; // stop requested
		}
		// retrieve the next message out of the queue:
		msg = messages.front();
		messages.pop();
	}

	// print the next message:
	std::cout << "msg: " << msg << std::endl;
}
\end{cpp}

条件变量messagesCV的类型为std::condition\_variable\_any:

\begin{cpp}
std::condition_variable_any messagesCV;
\end{cpp}

用停止令牌调用wait(),在这里传递停止令牌信号来停止线程,所以等待现在可能会结束。原因有两个:

\begin{itemize}
\item
有一个通知(队列不再为空)

\item
要求停止
\end{itemize}

wait()的返回值表示是否满足条件。若返回false,则结束wait()的原因是请求停止,则可以做出相应的反应(这里,停止循环)。

表“condition\_variable\_any用于停止令牌的成员函数”列出了使用锁保护lg的std::condition\_variable\_any类型的新成员函数。

% Please add the following required packages to your document preamble:
% \usepackage{longtable}
% Note: It may be necessary to compile the document several times to get a multi-page table to line up properly
\begin{longtable}[c]{|l|l|}
\hline
\textbf{操作} & \textbf{效果} \\ \hline
\endfirsthead
%
\endhead
%
cv.wait(lg, st, pred)            & 等待pred为真或st请求停止的通知                      \\ \hline
cv.wait\_for(lg, dur, st, pred)  & pred为真或st请求停止时,最多等待时间为dud \\ \hline
cv.wait\_until(lg, tp, st, pred) & 等待时间点tp,等待pred为真或st请求停止的通知   \\ \hline
\end{longtable}

\begin{center}
表12.1 condition\_variable\_any成员函数用于停止令牌
\end{center}

目前还不支持其他阻塞函数的停止令牌。






