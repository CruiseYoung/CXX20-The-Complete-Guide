
再来详细了解一下简化函数模板。

\mySubsubsection{2.3.1}{auto参数的限制}

使用auto声明函数形参遵循与声明Lambda表达式形参相同的规则:

\begin{itemize}
\item
对于用auto声明的每个形参,函数都有隐式模板形参。

\item
参数可以是参数包:
\begin{cpp}
void foo(auto... args);
\end{cpp}

这相当于以下内容(不引入类型):
\begin{cpp}
template<typename... Types>
void foo(Types... args);
\end{cpp}

\item
不允许使用decltype(auto)。
\end{itemize}

简化的函数模板可以用(部分地)显式指定的模板参数调用,模板参数的顺序与调用参数相同。

例如:

\begin{cpp}
void foo(auto x, auto y)
{
	...
}

foo("hello", 42); // x has type const char*, y has type int
foo<std::string>("hello", 42); // x has type std::string, y has type int
foo<std::string, long>("hello", 42); // x has type std::string, y has type long
\end{cpp}

\mySubsubsection{2.3.2}{模板参数和auto参数的组合}

简化的函数模板仍然可以显式指定模板参数,为占位符类型生成的模板参数可添加到指定参数之后:

\begin{cpp}
template<typename T>
void foo(auto x, T y, auto z)
{
	...
}

foo("hello", 42, '?'); // x has type const char*, T and y are int, z is char
foo<long>("hello", 42, '?'); // x has type const char*, T and y are long, z is char
\end{cpp}

因此,以下声明是等价的(除了没有为使用auto命名的类型):

\begin{cpp}
template<typename T>
void foo(auto x, T y, auto z);

template<typename T, typename T2, typename T3>
void foo(T2 x, T y, T3 z);
\end{cpp}

稍后会介绍,通过使用概念作为类型约束,可以约束占位符参数和模板参数,可再使用模板参数来进行这样的限定。

例如,下面的声明确保第二个形参y是整型,而第三个形参z的类型可以转换为y的类型:

\begin{cpp}
template<std::integral T>
void foo(auto x, T y, std::convertible_to<T> auto z)
{
	...
}

foo(64, 65, 'c'); // OK, x is int, T and y are int, z is char
foo(64, 65, "c"); // ERROR: "c" cannot be converted to type int (type of 65)
foo<long,char>(64, 65, 'c'); // NOTE: x is char, T and y are long, z is char
\end{cpp}

注意,最后一条语句以错误的顺序指定了参数类型。

模板参数的顺序不像预期的那样的话,可能会导致错误:

\filename{lang/tmplauto.cpp}

\begin{cpp}
#include <vector>
#include <ranges>

void addValInto(const auto& val, auto& coll)
{
	coll.insert(val);
}

template<typename Coll> // Note: different order of template parameters
requires std::ranges::random_access_range<Coll>
void addValInto(const auto& val, Coll& coll)
{
	coll.push_back(val);
}

int main()
{
	std::vector<int> coll;
	addValInto(42, coll); // ERROR: ambiguous
}
\end{cpp}

addValInto()的第二次声明中只对第一个参数使用auto,所以模板参数的顺序不同。由于C++20接受\url{http://wg21.link/p2113r0},因为重载解析不会优先于第二个声明,而非第一个声明,所以会得到一个歧义错误。[并不是所有的编译器都能正确处理这个问题。]

因此,在混合模板参数和自动参数时要谨慎。理想情况下,与声明保持一致。











