

本节列出一般适用于对象和类型的概念。

\mySubsubsection{5.2.1}{算法概念}

\mySamllsection{std::integral<T>}

\begin{itemize}
\item
保证类型T是整型(包括bool和所有字符类型)。

\item
要求:
\begin{itemize}
\item
类型特性std::is\_integral\_v<T>为true
\end{itemize}
\end{itemize}


\mySamllsection{std::signed\_integral<T>}

\begin{itemize}
\item
保证类型T是有符号整型(包括有符号字符类型,可能是char)。

\item
要求:
\begin{itemize}
\item
需要满足std::integral<T>

\item
类型特性std::is\_signed\_v<T>为true
\end{itemize}
\end{itemize}

\mySamllsection{std::unsigned\_integral<T>}

\begin{itemize}
\item
保证类型T是无符号整型(包括bool和无符号字符类型,可能是char)。

\item
要求:
\begin{itemize}
\item
需要满足std::integral<T>

\item
类型特性std::is\_signed\_v<T>为true
\end{itemize}
\end{itemize}

\mySamllsection{std::floating\_point<T>}

\begin{itemize}
\item
保证类型T是浮点类型(float、double或long double)。

\item
引入这个概念是为了能够定义数学常数。

\item
要求:
\begin{itemize}
\item
类型特性std::is\_floating\_point\_v<T>为true
\end{itemize}
\end{itemize}

\mySubsubsection{5.2.2}{对象概念}

对于对象(既不是引用,也不是函数,也不是void的类型),有一个所需基本操作的层次结构。

\mySamllsection{std::movable<T>}

\begin{itemize}
\item
保证类型T是可移动和可交换的。可以移动构造、移动赋值,以及与该类型的另一个对象交换。

\item
要求:
\begin{itemize}
\item
需要满足std::move\_constructible<T>

\item
需要满足std::assignable\_from<T\&, T>

\item
需要满足std::swappable<T>

\item
T既不是引用,也不是函数,也不是void
\end{itemize}
\end{itemize}

\mySamllsection{std::copyable<T>}

\begin{itemize}
\item
保证类型T是可复制的(可移动和可交换)。

\item
要求:
\begin{itemize}
\item
需要满足std::movable<T>

\item
需要满足std::copy\_constructible<T>

\item
std::assignable\_from用于任意T类型,T\&,const T,以及const T\&转换为T\&

\item
需要满足std::swappable<T>

\item
T既不是引用,也不是函数,也不是void
\end{itemize}
\end{itemize}

\mySamllsection{std::semiregular<T>}

\begin{itemize}
\item
保证类型T是半常规的类型(可以默认初始化、复制、移动和交换)。

\item
要求:
\begin{itemize}
\item
需要满足std::copyable<T>

\item
需要满足std::default\_initializable<T>

\item
需要满足std::movable<T>

\item
需要满足std::copy\_constructible<T>

\item
std::assignable\_from用于任意T类型,T\&,const T,以及const T\&转换为T\&

\item
需要满足std::swappable<T>

\item
T既不是引用,也不是函数,也不是void
\end{itemize}
\end{itemize}

\mySamllsection{std::regular<T>}

\begin{itemize}
\item
保证类型T是常规的(可以默认初始化、复制、移动、交换和检查相等性)。

\item
要求:
\begin{itemize}
\item
需要满足std::semiregular<T>

\item
需要满足std::equality\_comparable<T>

\item
需要满足std::copyable<T>

\item
需要满足std::default\_initializable<T>

\item
需要满足std::movable<T>

\item
需要满足std::copy\_constructible<T>

\item
std::assignable\_from用于任意T类型,T\&,const T,以及const T\&转换为T\&

\item
需要满足std::swappable<T>

\item
T既不是引用,也不是函数,也不是void
\end{itemize}
\end{itemize}

\mySubsubsection{5.2.3}{概念间的关系类型}

\mySamllsection{std::same\_as<T1, T2>}

\begin{itemize}
\item
保证T1和T2类型相同。

\item
概念两次调用std::is\_same\_v类型特性,以确保与参数的顺序无关。

\item
要求:
\begin{itemize}
\item
类型特性std::is\_same\_v<T1, T2>为true

\item
与T1和T2的顺序无关
\end{itemize}
\end{itemize}

\mySamllsection{std::convertible\_to<From, To>}

\begin{itemize}
\item
保证From类型的对象隐式和显式可转换为To类型的对象。

\item
要求:
\begin{itemize}
\item
类型特性std::is\_convertible\_v<From, To>为true

\item
支持从From到To的static\_cast

\item
可以将From类型的对象作为To返回
\end{itemize}
\end{itemize}

\mySamllsection{std::derived\_from<D, B>}

\begin{itemize}
\item
保证D类型是从B类型公开派生的(或者D和B是相同的),D类型的指针都可以转换为B类型的指针。换句话说:保证了D引用/指针可以用作B引用/指针。

\item
要求:
\begin{itemize}
\item
类型特性std::is\_base\_of\_v<B, D>为true

\item
对于类型D到B的const指针,类型特性std::is\_convertible\_v为true
\end{itemize}
\end{itemize}

\mySamllsection{std::constructible\_from<T, Args...>}

\begin{itemize}
\item
保证可以用Args…类型的参数初始化一个T类型的对象。

\item
要求:
\begin{itemize}
\item
需要满足std::destructible<T>

\item
类型特性std::is\_constructible\_v<T, Args...>为true
\end{itemize}
\end{itemize}

\mySamllsection{std::assignable\_from<To, From>}

\begin{itemize}
\item
保证可以将From类型的值移动或复制赋值给To类型的值。

赋值还必须产生原始To对象。

\item
要求:
\begin{itemize}
\item
To必须是一个左值引用

\item
std::common\_reference\_with<To, From>满足类型的const左值引用

\item
必须支持赋值运算符,并产生与To相同的类型
\end{itemize}
\end{itemize}

\mySamllsection{std::swappable\_with<T1, T2>}

\begin{itemize}
\item
保证类型T1和T2的值可以交换。

\item
要求:
\begin{itemize}
\item
需要满足std::common\_reference\_with<T1, T2>

\item
任意两个T1和T2类型的对象,都可以通过使用std::ranges::swap()交换值。
\end{itemize}
\end{itemize}

\mySamllsection{std::common\_with<T1, T2>}

\begin{itemize}
\item
保证类型T1和T2共享一个可以显式转换为的公共类型。

\item
要求:
\begin{itemize}
\item
类型特性std::common\_type\_t<T1, T2>会产生一个类型

\item
其通用类型支持static\_cast

\item
这两种类型的引用共享一个common\_reference类型

\item
与T1和T2的顺序无关
\end{itemize}
\end{itemize}

\mySamllsection{std::common\_reference\_with<T1, T2>}

\begin{itemize}
\item
保证类型T1和T2共享一个common\_reference类型,可以显式和隐式地转换为该类型。

\item
要求:
\begin{itemize}
\item
类型特性std::common\_reference\_t<T1, T2>会产生一个类型

\item
这两种类型可通过std::convertible\_to转换为公共引用类型

\item
与T1和T2的顺序无关
\end{itemize}
\end{itemize}

\mySubsubsection{5.2.4}{比较概念}

\mySamllsection{std::equality\_comparable<T>}

\begin{itemize}
\item
保证类型T的对象可使用==和!=操作符比较,与顺序无关。

\item
T的==操作符应该是对称的和可传递的:
\begin{itemize}
\item
当t2==t1时,t1==t2为true

\item
若t1==t2和t2==t3为true,则t1==t3为true
\end{itemize}

但该语义约束在编译时无法检查。

\item
要求:
\begin{itemize}
\item
==和!=操作符都受支持,并产生可转换为bool的值
\end{itemize}
\end{itemize}

\mySamllsection{std::equality\_comparable\_with<T1, T2>}

\begin{itemize}
\item
保证类型T1和T2的对象可以使用==和!=操作符进行比较。

\item
要求:
\begin{itemize}
\item
对于涉及T1和/或T2对象的所有比较,都支持==和!=,并产生可转换为bool的相同类型的值
\end{itemize}
\end{itemize}

\mySamllsection{std::totally\_ordered<T>}

\begin{itemize}
\item
保证类型T的对象可与==、!=、<、<=、>和>=操作符进行比较,以便两个值总是相互等于、小于或大于对方。

\item
这个概念并没有声称类型T具有所有值的总顺序,即使浮点值没有顺序,但表达式std::totally\_ordered<double>的结果仍为true:

\begin{cpp}
std::totally_ordered<double> // true
std::totally_ordered<std::pair<double, double>> // true
std::totally_ordered<std::complex<int>> // false
\end{cpp}

提供这个概念是为了检查能够对元素进行排序的正式需求,因使用了std::ranges::less,这是排序算法的默认排序条件。若类型没有所有值的总顺序,排序算法就不会编译失败,支持所有六个基本比较运算符就足够了。要排序的值应该是全序或弱序,但这是在编译时无法检查的语义约束。

\item
要求:
\begin{itemize}
\item
需要满足std::equality\_comparable<T>

\item
与==、!=、<、<=、>和>=操作符的所有比较都会产生可转换为bool的值
\end{itemize}
\end{itemize}

\mySamllsection{std::totally\_ordered\_with<T>}

\begin{itemize}
\item
保证类型T1和T2的对象与==、!=、<、<=、>和>=操作符具有可比性,以便两个值总是相互等于、小于或大于对方。

\item
要求:
\begin{itemize}
\item
对于涉及T1和/或T2对象的所有比较,都支持==,!=,<,<=,>和>=操作符,并产生可转换为bool的相同类型的值
\end{itemize}
\end{itemize}


\mySamllsection{std::three\_way\_comparable<T>, std::three\_way\_comparable<T, Cat>}


\begin{itemize}
\item
保证类型T的对象与==、!=、<、<=、>、>=和<=>操作符可比较(并且至少具有比较类别类型Cat)。若没有传递Cat,则需要std::partial\_ordering。

\item
这个概念在头文件<compare>中定义。

\item
这个概念并不包含std::equality\_comparable,因为后者要求==操作符只对两个相等的对象产生真值。对于弱序或偏序,情况可能不是这样。

\item
这个概念并不暗示和包含std::totally\_ordered,因为后者要求比较类别是std::strong\_ordering或std::weak\_ordering。

\item
要求:
\begin{itemize}
\item
与==、!=、<、<=、>和>=操作符的所有比较都会产生可转换为bool的值

\item
与<=>操作符的比较都会产生一个比较类别(至少是Cat)。
\end{itemize}
\end{itemize}

\mySamllsection{std::three\_way\_comparable\_with<T1, T2>, 	std::three\_way\_comparable\_with<T1, T2, Cat>}

\begin{itemize}
\item
保证任意两个T1和T2类型的对象都可以与==、!=、<、<=、>、>=和<=>操作符进行比较(并且至少具有比较类别类型Cat)。若没有传递Cat,则需要std::partial\_ordering。

\item
这个概念在头文件<compare>中定义。

\item
这个概念并不包含std::equality\_comparable\_with,因为后者要求==操作符只对两个相等的对象产生真值。对于弱序或偏序,情况可能并不是这样。

\item
这个概念并不暗示和包含std::totally\_ordered\_with,因为后者要求比较类别是std::strong\_ordering或std::weak\_ordering。

\item
要求:
\begin{itemize}
\item
std::three\_way\_comparable满足T1和T2(以及Cat)的值和公共引用

\item
与==、!=、<、<=、>和>=操作符的所有比较都会产生可转换为bool的值

\item
与<=>操作符的比较都会产生一个比较类别(至少是Cat)。

\item
与T1和T2的顺序无关
\end{itemize}
\end{itemize}




