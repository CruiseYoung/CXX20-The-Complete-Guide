

假设有一个限定范围的枚举类型(用枚举类声明):

\begin{cpp}
enum class Status{open, progress, done = 9};
\end{cpp}

与未限定作用域的枚举类型(不带类的枚举)不同,此类型的值需要带有类型名的限定符:

\begin{cpp}
auto x = Status::open; // OK
auto x = open; // ERROR
\end{cpp}

然而,在某些明显没有冲突的上下文中,一直限定每个值可能会变得有点乏味。为了更方便地使用作用域枚举类型,现在可以使用using枚举声明。

一个典型的例子是切换所有可能的枚举值。现在,可以这样实现:

\begin{cpp}
void print(Status s)
{
	switch (s) {
	using enum Status; // make enum values available in current scope
	case open:
		std::cout << "open";
		break;
	case progress:
		std::cout << "in progress";
		break;
	case done:
		std::cout << "done";
		break;
	}
}
\end{cpp}

只要在print()的作用域中没有声明其他名为open、progress或done的符号,这段代码就可以正常工作。

现在,也可以为特定的枚举值使用多个using声明:

\begin{cpp}
void print(Status s)
{
	switch (s) {
	using Status::open, Status::progress, Status::done;
	case open:
		std::cout << "open";
		break;
	case progress:
		std::cout << "in progress";
		break;
	case done:
		std::cout << "done";
		break;
	}
}
\end{cpp}

这样,就确切地知道哪些名称在当前作用域中可用。

注意,也可以对无作用域的枚举类型使用using声明。这不是必需的,但这样,就不必知道枚举类型是如何定义的:

\begin{cpp}
enum Status{open, progress, done = 9}; // unscoped enum

auto s1 = open; // OK
auto s2 = Status::open; // OK

using enum Status; // OK, but no effect
auto s3 = open; // OK
auto s4 = Status::open; // OK
\end{cpp}





