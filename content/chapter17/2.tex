Lambda提供了一种简单的方法来定义函数对象,若定义

\begin{cpp}
auto cmp = [] (const auto& x, const auto& y) {
				return x > y;
			};
\end{cpp}

这相当于定义一个类(闭包类型)并创建一个该类的对象:

\begin{cpp}
class NameChosenByCompiler {
	public:
	template<typename T1, T2>
	auto operator() (const T1& x, const T2& y) const {
		return x > y;
	}
};

auto cmp = NameChosenByCompiler{};
\end{cpp}

生成的闭包类型定义了函数操作符,则可以将Lambda对象cmp作为函数使用:

\begin{cpp}
cmp(val1, val); // yields the result of 42 > obj2
\end{cpp}

C++20之前,生成的闭包类型没有可调用的默认构造函数和赋值操作符。生成类的对象最初只能由编译器创建,只可复制:

\begin{cpp}
auto cmp1 = [] (const auto& x, const auto& y) {
				return x > y;
			};

auto cmp2 = cmp1; // OK, copy constructor supported since C++11
decltype(cmp1) cmp3; // ERROR until C++20: no default constructor provided
cmp1 = cmp2; // ERROR until C++20: no assignment operator provided
\end{cpp}

因为容器需要辅助函数的类型,所以不好将Lambda作为排序标准或哈希函数传递给容器。考虑一个具有以下接口的Customer类:

\begin{cpp}
class Customer
{
	public:
	...
	std::string getName() const;
};
\end{cpp}

要使用getName()返回的名称作为哈希函数的排序标准或值,必须将类型和Lambda作为模板和调用参数传递:

\begin{cpp}
// create balanced binary tree with user-defined ordering criterion:
auto lessName = [] (const Customer& c1, const Customer& c2) {
	return c1.getName() < c2.getName();
};
std::set<Customer, decltype(lessName)> coll1{lessName};

// create hash table with user-defined hash function:
auto hashName = [] (const Customer& c) {
	return std::hash<std::string>{}(c.getName());
};
std::unordered_set<Customer, decltype(hashName)> coll2{0, hashName};
\end{cpp}

容器在初始化时获得Lambda,以便可以使用Lambda的内部副本(对于无序容器,必须先传递最小桶大小)。要进行编译,容器的类型需要Lambda的类型。

自C++20起,没有捕获的Lambda有一个默认构造函数和一个赋值操作符:

\begin{cpp}
auto cmp1 = [] (const auto& x, const auto& y) {
				return x > y;
			};

auto cmp2 = cmp1; // OK, copy constructor supported
decltype(cmp1) cmp3; // OK since C++20
cmp1 = cmp2; // OK since C++20
\end{cpp}

出于这个原因,可以分别为排序条件或哈希函数传递Lambda的类型:

\begin{cpp}
// create balanced binary tree with user-defined ordering criterion:
auto lessName = [] (const Customer& c1, const Customer& c2) {
					return c1.getName() < c2.getName();
				};
std::set<Customer, decltype(lessName)> coll1; // OK since C++20

// create hash table with user-defined hash function:
auto hashName = [] (const Customer& c) {
					return std::hash<std::string>{}(c.getName());
				};

std::unordered_set<Customer, decltype(hashName)> coll2; // OK since C++20
\end{cpp}

这是有效的,因为排序条件或散列函数的参数有一个默认值,该值是排序条件或哈希列函数类型的默认构造对象。而且由于自C++20以来,没有捕获的Lambda有默认构造函数,因此使用Lambda类型的默认构造对象初始化排序条件现在可以编译了。

甚至可以在容器的声明中定义Lambda,并使用decltype传递其类型。例如,可以声明一个关联容器,并在声明中定义排序:

\begin{cpp}
// create balanced binary tree with user-defined ordering criterion:
std::set<Customer,
		decltype([] (const Customer& c1, const Customer& c2) {
			return c1.getName() < c2.getName();
		})> coll3; // OK since C++20
\end{cpp}

以同样的方式,可以用哈希函数声明一个无序容器:

\begin{cpp}
// create hash table with user-defined hash function:
std::unordered_set<Customer,
					decltype([] (const Customer& c) {
						return std::hash<std::string>{}(c.getName());
					})> coll; // OK since C++20
\end{cpp}

请参阅lang/lambdahash.cpp获取完整示例。





