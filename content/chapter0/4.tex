
请注意以下关于我在本书中编写代码和注释的提示

\subsubsection*{\zihao{3} 初始化}
\addcontentsline{toc}{subsubsection}{初始化}

我通常使用现代形式的初始化(C++11中作为统一初始化引入),通常情况下使用花括号或=:

\begin{cpp}
int i = 42;
std::string s{"hello"};
\end{cpp}

大括号初始化有以下优点:

\begin{itemize}
\item
可以与基本类型、类类型、聚合、枚举类型和auto一起使用

\item
可用于初始化具有多个值的容器

\item
可以检测窄化错误(例如,用浮点值初始化int)。

\item
不可与函数声明或函数调用混淆
\end{itemize}

若大括号为空,则调用(子)对象的默认构造函数,并保证基本数据类型用0/false/nullptr初始化。


\subsubsection*{\zihao{3} 术语“错误”}
\addcontentsline{toc}{subsubsection}{术语“错误”}

我经常谈论编程错误,若没有特殊提示,则使用术语“错误”或注释,例如

\begin{cpp}
... // ERROR
\end{cpp}

表示编译时错误。相应的代码不应该编译(使用符合标准的编译器)。

若使用术语“运行时错误”,则程序可能会编译,但行为不正确或导致未定义的行为(因此,可能会或可能不会执行预期的操作)。

\subsubsection*{\zihao{3} 代码简化}
\addcontentsline{toc}{subsubsection}{代码简化}

我会用例子来解释所有的特性,但为了专注于教授的关键方面,可能会跳过代码的其他细节。

\begin{itemize}
\item
大多数情况下,我使用省略号(“…”)来表示缺少额外的代码。注意,在这里没有使用代码字体。若看到带有代码字体的省略号,则代码必须将这三个点作为语言特性(例如“typename…”)。

\item
头文件中,我通常跳过预处理器保护。所有的头文件都应该像下面这样写:

\begin{cpp}
#ifndef MYFILE_HPP
#define MYFILE_HPP
...
#endif // MYFILE_HPP
\end{cpp}

因此,在项目中使用这些头文件时,请补全代码。
\end{itemize}














