

Both operator<=> and operator== can be defined for your data types:

\begin{itemize}
\item
Either as a member function taking one parameter

\item
Or as a free-standing function taking two parameters
\end{itemize}


\begin{table}[H]
\begin{tabular}{|l|l|}
	\hline
	\textbf{Function Object in std::} & \textbf{Effect}                                             \\ \hline
	strong\_order()                   & Maps to a strong order value also for floating-point values \\ \hline
	weak\_order()                     & Maps to a weak order value also for floating-point values   \\ \hline
	partial\_order()                  & Maps to a partial order value                               \\ \hline
	compare\_strong\_order\_fallback()  & Maps to a strong order value even if only == and \textless are defined  \\ \hline
	compare\_weak\_order\_fallback()    & Maps to a weak order value even if only == and \textless are defined    \\ \hline
	compare\_partial\_order\_fallback() & Maps to a partial order value even if only == and \textless are defined \\ \hline
\end{tabular}
\end{table}

\begin{center}
Table 1.1. Function objects for mapping comparison category types
\end{center}


\mySubsubsection{1.3.1}{Defaulted operator== and operator<=>}

Inside a class or data structure (as a member or friend function), all comparison operators can be declared as defaulted with =default. However, this usually only makes sense for operator== and operator<=>. The member functions have to take the second parameter as const lvalue reference (const \&). Friend functions might alternatively take both parameters by value.

The defaulted operators require the support of the members and possible base classes:

\begin{itemize}
\item
Defaulted operators == require the support of == in the members and base classes.

\item
Defaulted operators <=> require the support of == and either an implemented operator < or a defaulted operator <=> in the members and base classes (for details, see below).
\end{itemize}

For the generated defaulted operators, the following then applies:

\begin{itemize}
\item
The operator is noexcept if comparing the members guarantees not to throw.

\item
The operator is constexpr if comparing the members is possible at compile time.
\end{itemize}

For empty classes, the defaulted operators compare all objects as equal: operators ==, <=, and >= yield true, operators !=, <, and > yield false, and <=> yields std::strong\_ordering::equal.


\mySubsubsection{1.3.2}{Defaulted operator<=> Implies Defaulted operator==}

If and only if an operator<=> member is defined as defaulted, then by definition a corresponding operator== member is also defined if no defaulted operator== is provided. All aspects (visibility, virtual, attributes, requirements, etc.) are adopted. For example:

\begin{cpp}
template<typename T>
class Type {
	...
	public:
		[[nodiscard]] virtual std::strong_ordering
			operator<=>(const Type&) const requires(!std::same_as<T,bool>) = default;
};
\end{cpp}

is equivalent to the following:

\begin{cpp}
template<typename T>
class Type {
	...
	public:
	[[nodiscard]] virtual std::strong_ordering
		operator<=> (const Type&) const requires(!std::same_as<T,bool>) = default;
	
	[[nodiscard]] virtual bool
		operator== (const Type&) const requires(!std::same_as<T,bool>) = default;
};
\end{cpp}

For example, the following is enough to support all six comparison operators for objects of the type Coord:

\noindent
\hspace*{\fill} \\ %插入空行
\textit{lang/coord.hpp}

\begin{cpp}
#include <compare>
struct Coord {
	double x{};
	double y{};
	double z{};
	auto operator<=>(const Coord&) const = default;
};
\end{cpp}

Note again that the member function must be const and that the parameter must be declared to be a const lvalue reference (const \&).

You can use this data structure as follows:

\filename{lang/coord.cpp}

\begin{cpp}
#include "coord.hpp"
#include <iostream>
#include <algorithm>
int main()
{
	std::vector<Coord> coll{ {0, 5, 5}, {5, 0, 0}, {3, 5, 5},
							 {3, 0, 0}, {3, 5, 7} };
							 
	std::sort(coll.begin(), coll.end());
	for (const auto& elem : coll) {
		std::cout << elem.x << '/' << elem.y << '/' << elem.z << '\n';
	}
}
\end{cpp}

The program has the following output:

\begin{shell}
0/5/5
3/0/0
3/5/5
3/5/7
5/0/0
\end{shell}

\mySubsubsection{1.3.3}{Implementation of the Defaulted operator<=>}

If operator<=> is defaulted and you have members or base classes and you call one of the relational operators, then the following happens:

\begin{itemize}
\item
If operator<=> is defined for a member or base class, that operator is called.

\item
Otherwise, operator== and operator< are called to decide whether (from the point of view of the members or base classes)

\begin{itemize}
\item
The objects are equal/equivalent (operator== yields true)

\item
The objects are less or greater

\item
The objects are unordered (only when partial ordering is checked)
\end{itemize}

In that case, the return type of the defaulted operator<=> calling these operators cannot be auto.
\end{itemize}

For example, consider the following declarations:

\begin{cpp}
struct B {
	bool operator==(const B&) const;
	bool operator<(const B&) const;
};

struct D : public B {
	std::strong_ordering operator<=> (const D&) const = default;
};
\end{cpp}

Then:

\begin{cpp}
D d1, d2;
d1 > d2; // calls B::operator== and possibly B::operator<
\end{cpp}

If operator== yields true, we know that the result of > is false and that is it. Otherwise, operator< is called to find out whether the expression is true or false.

With

\begin{cpp}
struct D : public B {
	std::partial_ordering operator<=> (const D&) const = default;
};
\end{cpp}

the compiler might even call operator< twice to find out whether there is any order at all.

With

\begin{cpp}
struct B {
	bool operator==(const B&) const;
	bool operator<(const B&) const;
};

struct D : public B {
	auto operator<=> (const D&) const = default;
};
\end{cpp}

the compiler does not compile any call with relational operators because it cannot decide which ordering category the base class has. In that case, you need operator<=> in the base class too.

However, checks for equality work, because in D, operator== is automatically declared equivalent to the following:

\begin{cpp}
struct D : public B {
	auto operator<=> (const D&) const = default;
	bool operator== (const D&) const = default;
};
\end{cpp}

This means that we have the following behavior:

\begin{cpp}
D d1, d2;
d1 > d2; // ERROR: cannot deduce comparison category of operator<=>
d1 != d2; // OK (note: only tries operator<=> and B::operator== of a base class)
\end{cpp}

Equality checks always use only operator== of a base class (which might be generated according to a defaulted operator<=> though). Any operator< or operator!= in the base class is ignored.

The same applies, if D has a member of type B.
