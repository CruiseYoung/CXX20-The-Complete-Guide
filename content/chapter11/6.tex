

大国或国际交流中,仅仅说“中午见面吧”是不够的,因为我们有不同的时区。新的chrono库支持这一事实,它有一个API来处理地球上不同的时区,包括处理标准时间(“冬季”)和夏令时(“夏季”)。

\mySubsubsection{11.6.1}{时区的特点}

处理时区有点棘手,因为这是一个非常复杂的主题。例如,必须考虑以下几点:

\begin{itemize}
\item 
时区差异不一定是数小时。事实上,也有30分钟甚至15/45分钟的差异。例如,澳大利亚的很大一部分(北领地和南澳大利亚与阿德莱德)的标准时区是UTC+9:30,尼泊尔的时区是UTC+5:45。两个时区也可能相差0分钟(例如:北美和南美的时区)。

\item 
时区缩写可能指不同的时区。例如,CST可能代表中央标准时间(芝加哥、墨西哥城和哥斯达黎加的标准时区)或中国标准时间(北京和上海时区的国际名称),或古巴标准时间(哈瓦那的标准时区)。

\item 
时区变化。例如,当国家决定改变他们的时区或夏令时开始时,这种情况就会发生,所以在处理时区时,可能不得不考虑每年的多次更新。
\end{itemize}

\mySubsubsection{11.6.2}{IANA时区数据库}

为了处理时区,C++标准库使用IANA时区数据库,该数据库可在\url{http://www.iana.org/time-zones}上获得。再次注意,时区数据库可能不是在所有平台上都可用。

时区数据库的关键项是时区名称,通常是

\begin{itemize}
\item 
代表时区的某一地区或国家的城市。

例如: America/Chicago, Asia/Hong\_Kong, Europe/Berlin, Pacific/Honolulu

\item 
UTC时间的负偏移GMT。

例如: Etc/GMT或Etc/GMT+6表示UTC-6, 或Etc/GMT-8表示UTC+8 

是的,UTC时区偏移量有意地反转为GMT,不能搜索像UTC+6或UTC+5:45这样的东西。
\end{itemize}

支持一些额外的规范和别名(例如,UTC或GMT),也可以使用一些已弃用的时区(例如,PST8PDT、US/Hawaii、Canada/Central或仅日本)。但是,不能搜索单个时区缩写项,如CST或PST。我们将在后面讨论如何处理时区缩写。

另请参见\url{http://en.wikipedia.org/wiki/List_of_tz_database_time_zones}获取时区名称列表。

\mySamllsection{访问时区数据库}

IANA时区数据库每年都会进行多次更新,以确保在时区更改时数据库是最新的,以便程序能够做出相应的反应。

系统处理时区数据库的方式是特定于实现的,操作系统必须决定如何提供必要的数据以及如何使其保持最新。时区数据库的更新通常作为操作系统更新的一部分进行,此类更新通常需要重新启动计算机,因此在更新期间不会运行C++应用程序。

C++20标准提供了对该数据库的低级支持,用于处理时区的高级函数:

\begin{itemize}
\item 
std::chrono::get\_tzdb\_list()产生对时区数据库的引用,该数据库是std::chrono::tzdb\_list类型的单例。

它是一个时区数据库列表,能够并行支持多个版本的时区数据库。

\item 
std::chrono::get\_tzdb()产生对Std::chrono::tzdb类型的当前时区数据库的引用。该类型有以下多个成员:

\begin{itemize}
\item 
version, 数据库版本的字符串

\item 
zones, 一个类型为std::chrono::time\_zone的时区信息向量
\end{itemize}

处理时区的标准函数(例如std::chrono::current\_zone()或std::chrono::zoned\_time类型的构造函数)在内部使用这些调用。

例如,std::chrono::current\_zone()是以下方式的快捷方式:

\begin{cpp}
std::chrono::get_tzdb().current_zone()
\end{cpp}
\end{itemize}

当平台不提供时区数据库时,这些函数将抛出std::runtime\_error类型的异常。

对于支持更新IANA时区数据库而不需要重新启动的系统,提供了std::chrono::reload\_tzdb()。更新不会从旧数据库中删除内存,应用可能仍然有指向它的(time\_zone)指针。相反,新数据库被自动推到时区数据库列表的前面。

可以使用get\_tzdb()返回的tzdb类型的字符串成员版本来检查时区数据库的当前版本。例如:

\begin{cpp}
std::cout << "tzdb version: " << chr::get_tzdb().version << '\n';
\end{cpp}

通常,输出是年份和该年份更新的升序字母字符(例如,“2021b”)。remote\_version()提供最新可用时区数据库的版本,可以使用它来决定是否调用reload\_tzdb()。

若一个长时间运行的程序正在使用chrono时区数据库,但从不调用reload\_tzdb(),那么该程序将不会意识到数据库的任何更新,其将继续使用程序第一次访问数据库时存在的任何版本的数据库。

有关为长时间运行的程序重新加载IANA时区数据库的详细信息和示例,请参阅\url{http://github.com/HowardHinnant/date/wiki/Examples-and-Recipes#tzdb_manage}。

\mySubsubsection{11.6.3}{使用时区}

要处理时区,要使用两种类型:

\begin{itemize}
\item 
std::chrono::time\_zone,表示特定时区的类型。

\item 
std::chrono::zoned\_time, 表示与特定时区相关联的特定时间点的类型。
\end{itemize}

详细地了解一下。

\mySamllsection{time\_zone类型}

所有可能的时区值都由IANA时区数据库预定义,所以不能仅仅通过声明time\_zone对象来创建。这些值来自时区数据库,通常处理的是指向这些对象的指针:

\begin{itemize}
\item 
std::chrono::current\_zone() 生成指向当前时区的指针。

\item 
std::chrono::locate\_zone(name) 生成一个指向时区名称的指针。

\item 
由std::chrono::get\_tzdb()返回的时区数据库具有包含所有时区的非指针集合:

\begin{itemize}
\item 
成员区域拥有所有规范。

\item 
成员链接拥有所有带有链接的别名。
\end{itemize}

可以使用它根据诸如缩写名称之类的特征来查找时区。
\end{itemize}

例如:

\begin{cpp}
auto tzHere = std::chrono::current_zone(); // type const time_zone*
auto tzUTC = std::chrono::locate_zone("UTC"); // type const time_zone*
...
std::cout << tzHere->name() << '\n';
std::cout << tzUTC->name() << '\n';
\end{cpp}

输出取决于当前的时区。对我来说,在德国是这样的:

\begin{shell}
Europe/Berlin
Etc/UTC
\end{shell}

可以搜索时区数据库中的任何条目(如“UTC”),但得到的是它的规范。例如,"UTC"只是指向"Etc/UTC"的时区链接。若locate\_zone()没有找到与该名称对应的信息,则抛出std::runtime\_error异常。

time\_zone不能做很多事情,但最重要的是将它与系统时间点或本地时间点结合起来使用。

若输出一个time\_zone,会得到一些特定于实现的输出,仅用于调试:

\begin{cpp}
std::cout << *tzHere << '\n'; // some implementation-specific debugging output
\end{cpp}

chrono库还允许定义和使用自定义时区的类型。

\mySamllsection{zoned\_time类型}

std::chrono::zoned\_time类型的对象将时间点应用于时区。执行此转换有两个选项:

\begin{itemize}
\item 
将系统时间点(属于系统时钟的时间点)应用于时区,将事件的时间点转换为与另一个时区的本地时间同时发生。

\item
将本地时间点(属于伪时钟local\_t的时间点)应用于时区,将一个时间点作为本地时间应用到另一个时区。
\end{itemize}

此外,可以通过用另一个带zoned\_time对象初始化新的zoned\_time对象,将zoned\_time的时间点转换为不同的时区。

例如,将每个办公室安排18:00的本地派对,并在2021年9月底安排一个跨越多个时区的公司派对。

\begin{cpp}
auto day = 2021y/9/chr::Friday[chr::last]; // last Friday of month
chr::local_seconds tpOfficeParty{chr::local_days{day} - 6h}; // 18:00 the day before
chr::sys_seconds tpCompanyParty{chr::sys_days{day} + 17h}; // 17:00 that day

std::cout << "Berlin Office and Company Party:\n";
std::cout << " " << chr::zoned_time{"Europe/Berlin", tpOfficeParty} << '\n';
std::cout << " " << chr::zoned_time{"Europe/Berlin", tpCompanyParty} << '\n';

std::cout << "New York Office and Company Party:\n";
std::cout << " " << chr::zoned_time{"America/New_York", tpOfficeParty} << '\n';
std::cout << " " << chr::zoned_time{"America/New_York", tpCompanyParty} << '\n';
\end{cpp}

这段代码有如下输出:

\begin{shell}
Berlin Office and Company Party:
  2021-09-23 18:00:00 CEST
  2021-09-24 19:00:00 CEST
New York Office and Company Party:
  2021-09-23 18:00:00 EDT
  2021-09-24 13:00:00 EDT
\end{shell}

组合时间点和时区时,细节很重要。例如,考虑下面的代码:

\begin{cpp}
auto sysTp = chr::floor<chr::seconds>(chr::system_clock::now()); // system timepoint
auto locTime = chr::zoned_time{chr::current_zone(), sysTp}; // local time
...
std::cout << "sysTp:         " << sysTp << '\n';
std::cout << "locTime:       " << locTime << '\n';
\end{cpp}

首先,将sysTp初始化为以秒为单位的当前系统时间点,并将该时间点与当前时区结合起来。输出显示的是同一时间点的系统和本地时间点:

\begin{shell}
sysTp:      2021-04-13 13:40:02
locTime:    2021-04-13 15:40:02 CEST
\end{shell}

现在初始化一个本地时间点。一种方法是将系统时间点转换为本地时间点,现在需要一个时区。若使用当前时区,本地时间将被转换为UTC:

\begin{cpp}
auto sysTp = chr::floor<chr::seconds>(chr::system_clock::now()); // system timepoint
auto curTp = chr::current_zone()->to_local(sysTp); // local timepoint
std::cout << "sysTp:            " << sysTp << '\n';
std::cout << "locTp:            " << locTp << '\n';
\end{cpp}

输出结果如下:

\begin{shell}
sysTp:     2021-04-13 13:40:02
curTp:     2021-04-13 13:40:02
\end{shell}

但若使用UTC作为时区,则本地时间用作本地时间,并不会关联时区:

\begin{cpp}
auto sysTp = chr::floor<chr::seconds>(chr::system_clock::now()); // system timepoint
auto locTp = std::chrono::locate_zone("UTC")->to_local(sysTp); // use local time as is
std::cout << "sysTp:              " << sysTp << '\n';
std::cout << "locTp:              " << locTp << '\n';
\end{cpp}

根据输出,两个时间点看起来是一样的:

\begin{shell}
sysTp:     2021-04-13 13:40:02
locTp:     2021-04-13 13:40:02
\end{shell}

然而,它们是不一样的。sysTp有一个相关的UTC epoch,而locTp没有。若现在将本地时间点应用于时区,则不会进行转换。我们指定时区,保持时间不变:

\begin{cpp}
auto timeFromSys = chr::zoned_time{chr::current_zone(), sysTp}; // converted time
auto timeFromLoc = chr::zoned_time{chr::current_zone(), locTp}; // applied time
std::cout << "timeFromSys: " << timeFromSys << '\n';
std::cout << "timeFromLoc: " << timeFromLoc << '\n';
\end{cpp}

因此,输出如下:

\begin{shell}
timeFromSys: 2021-04-13 15:40:02 CEST
timeFromLoc: 2021-04-13 13:40:02 CEST
\end{shell}

现在让我们将这四个对象与纽约的时区结合起来:

\begin{cpp}
std::cout << "NY sysTp: "
		  << std::chrono::zoned_time{"America/New_York", sysTp} << '\n';
std::cout << "NY locTP: "
		  << std::chrono::zoned_time{"America/New_York", locTp} << '\n';
std::cout << "NY timeFromSys: "
		  << std::chrono::zoned_time{"America/New_York", timeFromSys} << '\n';
std::cout << "NY timeFromLoc: "
		  << std::chrono::zoned_time{"America/New_York", timeFromLoc} << '\n';
\end{cpp}

输出结果如下:

\begin{shell}
Y sysTp:        2021-04-13 09:40:02 EDT
NY locTP:       2021-04-13 13:40:02 EDT
NY timeFromSys: 2021-04-13 09:40:02 EDT
NY timeFromLoc: 2021-04-13 07:40:02 EDT
\end{shell}

系统时间点和由它导出的本地时间,都将现在的时间转换为纽约时区。与往常一样,将本地时间点应用于纽约,因此现在拥有与删除时区后的原始时间相同的值。timeFromLoc是中欧的初始当地时间13:40:02,适用于纽约时区。

\mySubsubsection{11.6.4}{处理时区的缩写}

由于时区缩写可能引用不同的时区,因此不能通过缩写定义唯一的时区。相反,必须将缩写映射到多个IANA时区项中的一个。

下面的程序演示了时区缩写CST的用法:

\filename{lib/chronocst.cpp}

\begin{cpp}
#include <iostream>
#include <chrono>
using namespace std::literals;

int main(int argc, char** argv)
{
	auto abbrev = argc > 1 ? argv[1] : "CST";
	
	auto day = std::chrono::sys_days{2021y/1/1};
	auto& db = std::chrono::get_tzdb();
	
	// print time and name of all timezones with abbrev:
	std::cout << std::chrono::zoned_time{"UTC", day}
			  << " maps to these '" << abbrev << "' entries:\n";
	// iterate over all timezone entries:
	for (const auto& z : db.zones) {
		// and map to those using my passed (or default) abbreviation:
		if (z.get_info(day).abbrev == abbrev) {
			std::chrono::zoned_time zt{&z, day};
			std::cout << " " << zt << " " << z.name() << '\n';
		}
	}
}
\end{cpp}

不传递命令行参数或“CST”的程序可能具有以下内容

\begin{shell}
2021-01-01 00:00:00 UTC maps these ’CST’ entries:
  2020-12-31 18:00:00 CST America/Bahia_Banderas
  2020-12-31 18:00:00 CST America/Belize
  2020-12-31 18:00:00 CST America/Chicago
  2020-12-31 18:00:00 CST America/Costa_Rica
  2020-12-31 18:00:00 CST America/El_Salvador
  2020-12-31 18:00:00 CST America/Guatemala
  2020-12-31 19:00:00 CST America/Havana
  2020-12-31 18:00:00 CST America/Indiana/Knox
  2020-12-31 18:00:00 CST America/Indiana/Tell_City
  2020-12-31 18:00:00 CST America/Managua
  2020-12-31 18:00:00 CST America/Matamoros
  2020-12-31 18:00:00 CST America/Menominee
  2020-12-31 18:00:00 CST America/Merida
  2020-12-31 18:00:00 CST America/Mexico_City
  ...
  2020-12-31 18:00:00 CST America/Winnipeg
  2021-01-01 08:00:00 CST Asia/Macau
  2021-01-01 08:00:00 CST Asia/Shanghai
  2021-01-01 08:00:00 CST Asia/Taipei
  2020-12-31 18:00:00 CST CST6CDT
\end{shell}

因为CST可能代表中央标准时间、中国标准时间或古巴标准时间,所以可以看到美国和中国的大多数时区之间有14个小时的差异。此外,古巴的哈瓦那与这些时区有1或13个小时的时差。

注意,当我们在夏季的某一天搜索“CST”时,输出要小得多,因为美国条目和古巴的时区会切换到“CDT”(相应的夏令时)。然而,我们仍然有一些信息,因为中国和哥斯达黎加没有夏令时。

还要注意,对于CST,可能根本找不到任何时区,因为时区数据库不可用,或者使用GMT-6之类的东西代替了CST。

\mySubsubsection{11.6.5}{自定义的时区}

chrono库允许使用自定义时区。一个常见的例子是需要有一个时区,它与UTC有一个直到运行时才知道的偏移量。

下面是一个提供自定义时区OffsetZone的示例,该时区可以保存以分钟为精度的UTC偏移量:

\filename{lib/offsetzone.hpp}

\begin{cpp}
#include <chrono>
#include <iostream>
#include <type_traits>

class OffsetZone
{
	private:
	std::chrono::minutes offset; // UTC offset
	public:
	explicit OffsetZone(std::chrono::minutes offs)
	: offset{offs} {
	}
	
	template<typename Duration>
	auto to_local(std::chrono::sys_time<Duration> tp) const {
		// define helper type for local time:
		using LT
		= std::chrono::local_time<std::common_type_t<Duration,
													std::chrono::minutes>>;
		// convert to local time:
		return LT{(tp + offset).time_since_epoch()};
	}
	
	template<typename Duration>
	auto to_sys(std::chrono::local_time<Duration> tp) const {
		// define helper type for system time:
		using ST
		= std::chrono::sys_time<std::common_type_t<Duration,
													std::chrono::minutes>>;
		// convert to system time:
		return ST{(tp - offset).time_since_epoch()};
	}
	
	template<typename Duration>
	auto get_info(const std::chrono::sys_time<Duration>& tp) const {
		return std::chrono::sys_info{};
	}
};
\end{cpp}

要定义的只是本地时间和系统时间之间的转换。

可以像使用任何其他time\_zone指针一样使用timezone:

\filename{lib/offsetzone.cpp}

\begin{cpp}
#include "offsetzone.hpp"
#include <iostream>

int main()
{
	using namespace std::literals; // for h and min suffix
	// timezone with 3:45 offset:
	OffsetZone p3_45{3h + 45min};
	
	// convert now to timezone with offset:
	auto now = std::chrono::system_clock::now();
	std::chrono::zoned_time<decltype(now)::duration, OffsetZone*> zt{&p3_45, now};
	
	std::cout << "UTC: " << zt.get_sys_time() << '\n';
	std::cout << "+3:45: " << zt.get_local_time() << '\n';
	std::cout << zt << '\n';
}
\end{cpp}

程序可能有以下输出:

\begin{shell}
UTC:    2021-05-31 13:01:19.0938339
+3:45:  2021-05-31 16:46:19.0938339
\end{shell}










