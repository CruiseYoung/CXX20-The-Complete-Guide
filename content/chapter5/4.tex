
本节列出了可调用对象的所有概念

\begin{itemize}
\item
函数

\item
函数对象

\item
Lambda表达式

\item
指向成员的指针(带对象类型作为附加参数)
\end{itemize}

\mySubsubsection{5.4.1}{可调用对象的基本概念}

\mySamllsection{std::invocable<Op, ArgTypes...>}

\begin{itemize}
\item
保证可以为ArgTypes....类型的参数调用Op

\item
Op可以是函数、函数对象、Lambda或指向成员的指针。

\item
要证明操作和传递的参数都没有修改,可以使用std::regular\_invocable代替。但请注意,这两个概念之间只有语义上的差异,不能在编译时进行检查,所以概念上的差异只说明了意图。

\item
例如:

\begin{cpp}
struct S {
	int member;
	int mfunc(int);
	void operator()(int i) const;
};

void testCallable()
{
	std::invocable<decltype(testCallable)> // satisfied
	std::invocable<decltype([](int){}), char> // satisfied (char converts to int)
	std::invocable<decltype(&S::mfunc), S, int> // satisfied (member-pointer, object, args)
	std::invocable<decltype(&S::member), S>; // satisfied (member-pointer and object)
	std::invocable<S, int>; // satisfied due to operator()
}
\end{cpp}

注意,作为类型约束,只需要指定参数类型:

\begin{cpp}
void callWithIntAndString(std::invocable<int, std::string> auto op);
\end{cpp}

完整示例,请参见使用Lambda作为非类型模板参数。

\item
要求:
\begin{itemize}
\item
std::invoke(op, args)对Op类型和ArgTypes...类型的args有效
\end{itemize}
\end{itemize}

\mySamllsection{std::regular\_invocable<Op, ArgTypes...>}

\begin{itemize}
\item
保证可以为ArgTypes…类型的参数调用Op。并且调用既不改变所传递操作的状态,也不改变所传递参数的状态。

\item
Op可以是函数、函数对象、Lambda或指向成员的指针。

\item
注意,std::invocable与std::invocable概念的区别纯粹是语义上的,不能在编译时检查,所以概念差异只是证明了意图。

\item
要求:
\begin{itemize}
\item
std::invoke(op, args)对Op类型和ArgTypes...类型的args有效
\end{itemize}
\end{itemize}

\mySamllsection{std::predicate<Op, ArgTypes...>}

\begin{itemize}
\item
保证可调用的(函数,函数对象,Lambda)Op是一个谓词,可以为ArgTypes...类型的参数所调用。

\item
若依调用既不改变所传递操作的状态,也不改变所传递参数的状态。

\item
要求:
\begin{itemize}
\item
满足std::regular\_invocable<Op>

\item
所有使用ArgTypes...的Op调用,都会生成一个可当作布尔值的值
\end{itemize}
\end{itemize}

\mySamllsection{std::relation<Pred, T1, T2>}

\begin{itemize}
\item
保证任意两个类型T1和T2的对象具有二元关系,可以作为参数传递给二元谓词Pred。

\item
调用既不改变所传递操作的状态,也不改变所传递参数的状态。

\item
注意,std::equivalence\_relation和std::strict\_weak\_order的区别纯粹是语义上的,不能在编译时检查,所以概念差异只能说明意图。

\item
要求:
\begin{itemize}
\item
对于pred以及两个对象类型T1和T2的任意组合,都满足std::predicate
\end{itemize}
\end{itemize}

\mySamllsection{std::equivalence\_relation<Pred, T1, T2}

\begin{itemize}
\item
保证任意两个类型T1和T2的对象与Pred相比具有等价关系。

可以作为参数传递给二进制谓词Pred,并且这种关系是自反的、对称的和可传递的。

\item
调用既不改变所传递操作的状态,也不改变所传递参数的状态。

\item
注意,std::relation和std::strict\_weak\_order的区别纯粹是语义上的,不能在编译时检查,所以概念差异只能说明意图。

\item
要求:
\begin{itemize}
\item
对于pred以及两个对象类型T1和T2的任意组合,都满足std::predicate
\end{itemize}
\end{itemize}

\mySamllsection{std::strict\_weak\_order<Pred, T1, T2>}

\begin{itemize}
\item
与Pred相比,任意两个类型T1和T2的对象具有严格的弱序关系。

可以作为参数传递给二元谓词Pred,并且这种关系是非自反的和传递的。

\item
调用既不改变所传递操作的状态,也不改变所传递参数的状态。

\item
注意,std::relation和std::equivalence\_relation的区别纯粹是语义上的,不能在编译时检查,所以概念差异只能说明意图。

\item
要求:
\begin{itemize}
\item
对于pred以及两个对象类型T1和T2的任意组合,都满足std::predicate
\end{itemize}
\end{itemize}

\mySamllsection{std::uniform\_random\_bit\_generator<Op>}

\begin{itemize}
\item
保证Op可以用作返回(理想情况下)相等概率的无符号整数值的随机数生成器。

\item
这个概念在头文件<random>中定义。

\item
要求:
\begin{itemize}
\item
满足std::invocable<Op\&>

\item
结果满足std::unsigned\_integral

\item
支持表达式Op::min()和Op::max(),并产生与生成器调用相同的类型

\item
Op::min()小于Op::max()
\end{itemize}
\end{itemize}

\mySubsubsection{5.4.2}{迭代器使用可调用对象的概念}

\mySamllsection{std::indirectly\_unary\_invocable<Op, Pos>}

\begin{itemize}
\item
保证可以用Pos所引用的值调用Op。

\item
请注意,与indirectly\_regular\_unary\_invocable概念的区别纯粹是语义上的,不能在编译时检查,所以概念差异只能说明意图。

\item
要求:
\begin{itemize}
\item
满足std::indirectly\_readable<Pos>

\item
满足std::copy\_constructible<Op>

\item
满足Op\&和Pos的值和(常用)引用类型的std::invocable

\item
同时使用值和引用调用Op的结果,具有共同的引用类型
\end{itemize}
\end{itemize}

\mySamllsection{std::indirectly\_regular\_unary\_invocable<Op, Pos>}

\begin{itemize}
\item
保证可以使用Pos所引用的值调用Op,并且调用不会改变Op的状态。

\item
请注意,与std::indirect\_unary\_invocable概念的区别是纯语义的,不能在编译时检查,所以概念差异只能说明意图。

\item
要求:
\begin{itemize}
\item
满足std::indirectly\_readable<Pos>

\item
满足std::copy\_constructible<Op>

\item
Op\&和Pos的值,以及(常见)引用类型都满足std::regular\ _invocation

\item
同时使用值和引用调用Op的结果具有共同的引用类型
\end{itemize}
\end{itemize}

\mySamllsection{std::indirect\_unary\_predicate<Pred, Pos>}

\begin{itemize}
\item
保证可以用Pos引用的值调用一元谓词Pred。

\item
要求:
\begin{itemize}
\item
满足std::indirectly\_readable<Pos>

\item
满足std::copy\_constructible<Pred>

\item
Pred\&和Pos值,以及(常见)引用类型都满足std::predicate

\item
所有这些对Pred的调用,都会产生一个可以当作布尔值的值
\end{itemize}
\end{itemize}

\mySamllsection{std::indirect\_binary\_predicate<Pred, Pos1, Pos2>}

\begin{itemize}
\item
确保可以用Pos1和Pos2引用的值调用二元谓词Pred。

\item
要求:
\begin{itemize}
\item
Pos1和Pos2满足了std::indirect\_readable

\item
满足std::copy\_constructible<Pred>

\item
若Pred\&是Pos1的值或(通用)引用类型,以及Pos2的值或(通用)引用类型,则满足std::predicate

\item
所有这些对Pred的调用都会产生一个可作为布尔值使用的值
\end{itemize}
\end{itemize}

\mySamllsection{std::indirect\_equivalence\_relation<Pred, Pos1>, \\ std::indirect\_equivalence\_relation<Pred, Pos1, Pos2>}

\begin{itemize}
\item
确保可以调用二元谓词Pred来检查Pos1和Pos1/Pos2的两个值是否相等。

\item
注意,与std::indirectly\_strict\_weak\_order概念的区别纯粹是语义上的,不能在编译时检查,所以概念差异只能说明意图。

\item
要求:
\begin{itemize}
\item
Pos1和Pos2满足了std::indirect\_readable

\item
满足std::copy\_constructible<Pred>

\item
若Pred\&是Pos1的值或(通用)引用类型,以及Pos2的值或(通用)引用类型,则满足std::predicate

\item
所有这些对Pred的调用都会产生一个可以作为布尔值使用的值
\end{itemize}
\end{itemize}

\mySamllsection{std::indirect\_strict\_weak\_order<Pred, Pos1>, std::indirect\_strict\_weak\_order<Pred, Pos1, Pos2>}

\begin{itemize}
\item
保证可以调用二元谓词Pred来检查Pos1和Pos1/Pos2的两个值是否具有严格的弱序。

\item
注意,与std::indirect\_equivalence\_relation概念的区别纯粹是语义上的,不能在编译时检查,所以概念差异只能说明意图。

\item
要求:
\begin{itemize}
\item
Pos1和Pos2满足了std::indirect\_readable

\item
满足std::copy\_constructible<Pred>

\item
若Pred\&是Pos1的值或(通用)引用类型,以及Pos2的值或(通用)引用类型,则满足std::predicate

\item
所有这些对Pred的调用都会产生一个可以作为布尔值使用的值
\end{itemize}
\end{itemize}










