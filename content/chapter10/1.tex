
在了解细节之前,先看一些例子。

\mySubsubsection{10.1.1}{使用std::format()}

用于应用程序程序员的格式化库的基本功能是std::format(),其允许它们根据一对大括号内指定的格式,将要用传入参数的值填充的格式化字符串组合在一起。一个简单的例子是使用每个传入参数的值:

\begin{cpp}
#include <format>

std::string str{"hello"};
...
std::cout << std::format("String '{}' has {} chars\n", str, str.size());
\end{cpp}

头文件<format>中定义的函数std::format()接受一个格式字符串(在编译时已知的字符串字面值、字符串或字符串视图),其中\{…\}表示下一个参数的值(这里使用其类型的默认格式),会生成一个std::string,并为其分配内存。

第一个示例的输出如下:

\begin{shell}
String ’hello’ has 5 chars
\end{shell}

右大括号后面的可选整数值指定参数的索引,以便可以以不同的顺序处理参数或多次使用:

\begin{cpp}
std::cout << std::format("{1} is the size of string '{0}'\n", str, str.size());
\end{cpp}

输出如下:

\begin{shell}
5 is the size of string ’hello’
\end{shell}

不必显式指定参数的类型,可以很容易地在泛型代码中使用std::format()。考虑下面的例子:

\begin{cpp}
void print2(const auto& arg1, const auto& arg2)
{
	std::cout << std::format("args: {} and {}\n", arg1, arg2);
}
\end{cpp}

若按如下方式调用此函数:

\begin{cpp}
print2(7.7, true);
print2("character:", '?');
\end{cpp}

输出如下:

\begin{shell}
args: 7.7 and true
args: character: and ?
\end{shell}

若支持格式化输出,格式化甚至可以用于用户定义类型。chrono库的格式化输出就是一个例子:

\begin{cpp}
print2(std::chrono::system_clock::now(), std::chrono::seconds{13});
\end{cpp}

可能有以下输出:

\begin{shell}
args: 2022-06-19 08:46:45.3881410 and 13s
\end{shell}

对于自定义类型,需要一个格式化器,稍后将对此进行描述。

在冒号后面格式化的占位符中,可以指定所传递参数格式化的详细信息。例如,定义一个字段宽度:

\begin{cpp}
std::format("{:7}", 42) // yields " 42"
std::format("{:7}", 42.0) // yields " 42"
std::format("{:7}", 'x') // yields "x "
std::format("{:7}", true) // yields "true "
\end{cpp}

不同的类型有不同的默认对齐方式。对于bool类型,输出值false和true,而不是像使用<{}<操作符的iostream输出那样输出0和1。

也可以显式地指定对齐方式(<为左,\^{}为中,>为右)并指定一个填充字符:

\begin{cpp}
std::format("{:*<7}", 42) // yields "42*****"
std::format("{:*>7}", 42) // yields "*****42"
std::format("{:*^7}", 42) // yields "**42***"
\end{cpp}

其他的格式化规范可以强制使用特定的表示法、特定的精度(或将字符串限制为一定的大小)、填充字符或正号:

\begin{cpp}
std::format("{:7.2f} Euro", 42.0) // yields " 42.00 Euro"
std::format("{:7.4}", "corner") // yields "corn "
\end{cpp}

通过使用实参的位置,可以以多种形式输出值。例如:

\begin{cpp}
std::cout << std::format("'{0}' has value {0:02X} {0:+4d} {0:03o}\n", '?');
std::cout << std::format("'{0}' has value {0:02X} {0:+4d} {0:03o}\n", 'y');
\end{cpp}

输出'?'的十六进制、十进制和八进制值和'y'使用实际的字符集,可能有如下输出:

\begin{shell}
’?’ has value 3F +63 077
’y’ has value 79 +121 171
\end{shell}

\mySubsubsection{10.1.2}{使用std::format\_to\_n()}

与其他格式化方式相比,std::format()的实现具有相当好的性能,但必须为结果字符串分配内存。为了节省时间,可以使用std::format\_to\_n(),将写入预分配的字符数组。还必须指定要写入的缓冲区及其大小:

\begin{cpp}
char buffer[64];
...
auto ret = std::format_to_n(buffer, std::size(buffer) - 1,
							"String '{}' has {} chars\n", str, str.size());
*(ret.out) = '\0';
\end{cpp}

或:

\begin{cpp}
std::array<char, 64> buffer;
...
auto ret = std::format_to_n(buffer.begin(), buffer.size() - 1,
							"String '{}' has {} chars\n", str, str.size());
*(ret.out) = '\0'; // write trailing null terminator
\end{cpp}

注意,std::format\_to\_n()不会写入尾随的空结束符,但返回值保存了处理此问题的所有信息。其是一个std::format\_to\_n\_result类型的数据结构,有两个成员:

\begin{itemize}
\item
out 对于第一个未写字符的位置

\item
size 在不将其截断为传递的大小的情况下将写入的字符数。
\end{itemize}

因此,在ret.out指向的末尾存储一个空终止符。注意,我们只将buffer.size()-1传递给std::format\_to\_n(),以确保为末尾的空结束符提供内存:

\begin{cpp}
auto ret = std::format_to_n(buffer.begin(), buffer.size() - 1, ... );
*(ret.out) = '\0';
\end{cpp}

或者,可以用\{\}初始化缓冲区,以确保所有字符都用空终止符初始化。

若大小不适合该值,则不是错误,写入的值会删除。例如:

\begin{cpp}
std::array<char, 5> mem{};
std::format_to_n(mem.data(), mem.size()-1, "{}", 123456.78);
std::cout << mem.data() << '\n';
\end{cpp}

输出如下:

\begin{shell}
1234
\end{shell}

\mySubsubsection{10.1.3}{使用std::format\_to()}

格式化库还提供了std::format\_to(),为格式化的输出写入无限数量的字符。在有限的内存中使用这个函数有风险,因为如果值需要太多的内存,就会有未定义行为,但通过使用输出流缓冲区迭代器,也可以安全地使用它写入流:

\begin{cpp}
std::format_to(std::ostreambuf_iterator<char>{std::cout},
				"String '{}' has {} chars\n", str, str.size());
\end{cpp}

通常,std::format\_to()接受任意字符的输出迭代器。例如,也可以使用尾插入器将字符追加到字符串中:

\begin{cpp}
std::string s;
std::format_to(std::back_inserter(s),
				"String '{}' has {} chars\n", str, str.size());
\end{cpp}

辅助函数std::back\_inserter()创建一个对象,为每个字符调用push\_back()。std::format\_to()的实现可以识别传递的回插入迭代器,并一次为某些容器写入多个字符,这样仍然具有良好的性能。

\mySubsubsection{10.1.4}{使用std::formatted\_size()}

要提前知道格式化输出将写入多少字符(不写入任何字符),可以使用std::formatted\_size()。例如:

\begin{cpp}
auto sz = std::formatted_size("String '{}' has {} chars\n", str, str.size());
\end{cpp}

这将允许保留足够的内存,或再次检查所保留的内存是否足够。






