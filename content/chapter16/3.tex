

本节讨论在实践中使用模块的一些其他方面。


\mySubsubsection{16.3.1}{用不同的编译器处理模块文件}

C++中,扩展是不标准化的。实践中,会使用不同的文件扩展名(通常是.cpp和.hpp,但也使用.cc、.cxx、.c、.hh、.hxx、.h,甚至是.h)。

所以也没有模块的标准后缀。更糟糕的是,甚至不同意是否有必要(目前)进行新的后缀。原则上,有两种判断方法:

\begin{itemize}
\item
编译器应该将各种模块文件视为普通的C++源文件,并根据其内容找出如何处理方法。使用这种方法,所有文件的扩展名仍然是.cpp。gcc/g++遵循这个策略。

\item
编译器对(某些)模块文件的处理方式不同,因为它们既可以是声明文件(传统的头文件),也可以是定义文件(传统的源文件)。由于这个原因,使用不同的后缀非常有帮助,编译器甚至可能间接地要求使用不同的后缀,以避免对相同的扩展使用不同的命令行选项。Visual C++遵循这种方法。
\end{itemize}

实践中,不同的编译器会为模块文件推荐不同的文件扩展名(.cppm、.ixx和.cpp),这就是为什么在实际使用模块仍然具有挑战性的原因之一。

我想了一会儿,尝试了一下,并与标准委员会的人讨论了当前的情况,但到目前为止,似乎没有令人信服的解决方案。在形式上,C++标准并没有标准化处理源代码的方式(代码甚至可能没有存储在文件中),所以没有一种简单的方法可以使用模块编写一个简单的可移植的第一个示例。

因此,我在这里提出一些建议,以便读者们可以在不同的平台上尝试使用模块:

为什么不同的文件扩展名似乎是必要的有多种原因:

\begin{itemize}
\item
编译器需要不同的命令行选项来处理不同类型的模块文件。

\item
与头文件类似,必须向客户和第三方代码提供一些模块文件。

\item
不同的模块文件创建不同的工件,可能必须处理这些工件(例如,在删除生成的工件时)。
\end{itemize}

就个人而言,我还没有最终的决定和建议模块文件的文件扩展名。然而,鉴于目前的情况,我的建议如下:

\begin{itemize}
\item
对于接口文件(主接口和接口分区),使用文件扩展名.cppm。原因是:

\begin{itemize}
\item
最好的自解释文件扩展名(远好于Visual C++目前推荐的.ixx)。

\item
这是Clang目前所需要的。

\item
也可以在gcc中使用。

\item
Visual C++无论如何都需要特殊处理,除非使用扩展名.ixx。
\end{itemize}


\item
对于模块实现文件(但不是分区实现文件),使用通常的文件扩展名.cpp。原因是:

\begin{itemize}
\item
没有生成特殊的工件。

\item
不需要特殊的命令行选项。
\end{itemize}

\item
对于内部分区文件(分区实现文件),使用文件扩展名.cppp。原因是:

\begin{itemize}
\item
Visual C++需要命令行选项/internalPartition来存放这些文件。文件后缀不重要,所以必须使用特殊后缀,以便在不想解析文件内容的构建系统中有机会使用通用规则。

\item
也可以在gcc中使用。

\item
目前,Clang根本不支持这些文件(2021年9月)
\end{itemize}

微软处理内部分区的方式对模块的成功不利,我希望他们能尽快解决对特定后缀的需求。

\end{itemize}

因此,必须(预)编译模块文件如下:

\begin{itemize}
\item
Visual C++:

Visual C++需要特定的命令行扩展名,并且倾向使用与我建议的不同的文件扩展名.ixx。出于这个原因:

\begin{itemize}
\item
编译一个接口文件file.cppm的命令行如下所示:

\begin{shell}
cl /TP /interface /c file.cppm
\end{shell}

选项/TP指定以下所有文件都包含C++源代码。或者,可以使用/Tpfile.cppm。选项/interface指定以下所有文件都是接口文件(在一个命令行中同时拥有接口和非接口文件可能无法正常工作)。

若使用文件扩展名.ixx,编译器会自动将该文件识别为接口文件。

\item
编译一个内部分区file.cppp的命令行如下所示:

\begin{shell}
cl /Tp /internalPartition /c file.cppp
\end{shell}

选项/internalPartition指定以下所有文件为内部分区,不支持在一个命令行中同时使用内部分区和接口文件。没有特定的后缀可以代替,内部分区需要这个选项。
\end{itemize}

事实上,Visual C++目前推荐不同的文件后缀,并要求针对特定模块单元使用特定的命令行选项,这使得模块的使用既麻烦又不可移植。为了规避Visual C++的限制(至少在通过命令行编译时),我提供了Python脚本clmod.py,可以在\url{http://github.com/josuttis/cppmodules}上找到它。我希望微软能修复他们的缺陷,这样就不再需要这种变通的方法了。

\item
gcc/g++:

gcc根本不需要任何特殊的文件扩展名或命令行选项。通过使用特殊的文件扩展名,只需要使用命令行选项-xc++指定文件包含C++代码:

\begin{itemize}
\item
编译一个接口文件file.cppm的命令行如下所示:

\begin{shell}
g++ -xc++ -c file.cppm
\end{shell}

\item
编译一个内部分区文件file.cppp的命令行如下所示:

\begin{shell}
g++ -xc++ -c file.cppp
\end{shell}
\end{itemize}

\item
Clang:

Clang目前只支持接口文件。由于建议的扩展名.cppm无论如何都是必需的,因此应该可以工作。

但不能使用内部分区文件。
\end{itemize}

\mySubsubsection{16.3.2}{处理头文件}

虽然理论上模块可以取代所有传统的头文件,但实际上这永远不会发生。将会有一些不需要使用模块的头文件,其来自为C++(和C)开发的代码和库。因为使用预编译器会使编译和链接C++程序变得更加复杂,所以情况尤其如此,所以模块应该能够处理传统的头文件。

使用传统头文件的基本方法是使用全局模块。

\begin{itemize}
\item
用module开启模块;

\item
然后,在进行模块声明之前放置所有必要的预处理器命令
\end{itemize}

这种情况下:

\begin{itemize}
\item
包含的头文件中未使用的所有内容都将丢弃。

\item
使用的所有内容都将获得模块链接,所以只在整个模块单元中可见,而在其他模块单元和模块外部都不可见。

\item
在\#include之前使用\#define。
\end{itemize}

例如:

\begin{cpp}
module;

#include <string>
#define NDEBUG
#include <cassert>

export module ModTest;
...
void foo(std::string s) {
	assert(s.empty()); // valid but not checked
	...
}
\end{cpp}

这个全局模块中,预处理器符号NDEBUG和宏assert()都是在这个模块单元中定义的。但由于NDEBUG,使用assert()的运行时检查都是禁用的。

NDEBUG和assert()在该模块或导入模块的其他单元中都不可见。

声明模块之后,不再支持\#include。可以使用其他预处理器命令,如\#define和\#ifdef。

\mySamllsection{头文件的导入}

未来的目标是使整个C++标准库作为模块可用。然而,对于标准的C++头文件,已经可以使用import,然后可以在模块中使用。例如:

\begin{cpp}
export module ModTest;

import <chrono>;
\end{cpp}

该命令是声明和导入模块的快捷方式,并导出相应头文件中的所有内容。通过此导入,甚至可以在该模块中看到宏(对于所有其他导入,都不可见)。

但在使用\#define导入之前定义的常量不会传递给导入的头文件。这样,就可以保证导入的头文件的内容总是相同的,这样头文件就可以预编译了。

注意,此功能只保证在标准C++头文件上工作,不适用于C++使用的标准C头文件:

\begin{cpp}
export module ModTest;

import <chrono>; // OK
import <cassert>; // ERROR (or at least not portable)
\end{cpp}

平台也允许为其他头文件支持此功能,但使用此特性的代码不可移植。

\mySamllsection{标准模块}

C++20只是介绍了这种技术,没有引入任何标准模块(原因是标准委员会希望重新组织不同模块上的符号,以清理头文件中一些历史上的混乱)。

看起来在C++23,将有两个标准模块(参见\url{http://wg21.link/p2465})

\begin{itemize}
\item
std将提供来自C++头文件的命名空间std中的所有内容,包括那些包装C的内容(例如std::sort(), std::ranges::sort(),std::fopen()和::operator new)。

不提供宏和特性测试宏。对于它们,必须手动包含<cassert>或<version>。

\item
std.compat将提供模块std中的所有内容,以及C头文件中的C符号对应(例如,::fopen())。
\end{itemize}

std和每个以std开头的模块名,都由C++标准为其标准模块进行保留。













