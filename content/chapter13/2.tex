
C++20引入了处理信号量的新类型。信号量是轻量级同步原语,允许同步或限制对一个或一组资源的访问。

可以像使用互斥锁一样使用,这样做的好处是,授予对资源访问权限的线程,不必是获得对资源访问权限的线程。还可以使用它们来限制资源的可用性,例如:启用和禁用线程池中线程的使用。

C++标准库提供了两种信号量类型:

\begin{itemize}
\item
std::counting\_semaphore<>将多个资源的使用限制在最大值

\item
std::binary\_semaphore<>限制对单一资源的使用
\end{itemize}

\mySubsubsection{13.2.1}{使用计数信号量}

下面的代码演示了信号量的常规操作方式:

\filename{lib/semaphore.cpp}

\begin{cpp}
#include <iostream>
#include <queue>
#include <chrono>
#include <thread>
#include <mutex>
#include <semaphore>
using namespace std::literals; // for duration literals

int main()
{
	std::queue<char> values; // queue of values
	std::mutex valuesMx; // mutex to protect access to the queue

	// initialize a queue with multiple sequences from ’a’ to ’z’:
	// - no mutex because no other thread is running yet
	for (int i = 0; i < 1000; ++i) {
		values.push(static_cast<char>('a' + (i % ('z'-'a' + 1))));
	}

	// create a pool of numThreads threads:
	// - limit their availability with a semaphore (initially none available):
	constexpr int numThreads = 10;
	std::counting_semaphore<numThreads> enabled{0};

	// create and start all threads of the pool:
	std::vector<std::jthread> pool;
	for (int idx = 0; idx < numThreads; ++idx) {
		std::jthread t{[idx, &enabled, &values, &valuesMx] (std::stop_token st) {
				while (!st.stop_requested()) {
					// request thread to become one of the enabled threads:
					enabled.acquire();
					// get next value from the queue:
					char val;
					{
						std::lock_guard lg{valuesMx};
						val = values.front();
						values.pop();
					}
					// print the value 10 times:
					for (int i = 0; i < 10; ++i) {
						std::cout.put(val).flush();
						auto dur = 130ms * ((idx % 3) + 1);
						std::this_thread::sleep_for(dur);
					}
					// remove thread from the set of enabled threads:
					enabled.release();
					}
				}};
		pool.push_back(std::move(t));
	}

	std::cout << "== wait 2 seconds (no thread enabled)\n" << std::flush;
	std::this_thread::sleep_for(2s);

	// enable 3 concurrent threads:
	std::cout << "== enable 3 parallel threads\n" << std::flush;
	enabled.release(3);
	std::this_thread::sleep_for(2s);

	// enable 2 more concurrent threads:
	std::cout << "\n== enable 2 more parallel threads\n" << std::flush;
	enabled.release(2);
	std::this_thread::sleep_for(2s);

	// Normally we would run forever, but let’s end the program here:
	std::cout << "\n== stop processing\n" << std::flush;
	for (auto& t : pool) {
	t.request_stop();
	}
}
\end{cpp}

代码中,启动了10个线程,但限制了允许其中多少线程主动运行和处理数据,因此将信号量初始化为最大数量(10)和初始资源数量(0):

\begin{cpp}
constexpr int numThreads = 10;
std::counting_semaphore<numThreads> enabled{0};
\end{cpp}

读者们可能想知道,为什么必须指定最大值作为模板参数。原因是,有了这个编译时值,库可以决定切换到最有效的实现(本机支持可能只能达到某个值,或者若最大值为1,我们可以使用简化版本)。

每个线程中,使用信号量来请求运行许可。可尝试“获取”一个可用资源来启动任务,并在任务完成后“释放”该资源以供其他使用:

\begin{cpp}
std::jthread{[&, idx] (std::stop_token st) {
				while (!st.stop_requested()) {
					// request thread to become one of the enabled threads:
					enabled.acquire();
					...
					// remove thread from the set of enabled threads:
					enabled.release();
				}
		}}
\end{cpp}

因为信号量初始化为零,所以最初的情况是阻塞,因此没有可用的资源。

可以使用信号量来允许三个线程进行并发操作:

\begin{cpp}
// enable 3 concurrent threads:
enabled.release(3);
\end{cpp}

之后,允许两个线程并发运行:

\begin{cpp}
// enable 2 more concurrent threads:
enabled.release(2);
\end{cpp}

若不能获得资源,就休眠或做其他事情,可以使用try\_acquire():

\begin{cpp}
std::jthread{[&, idx] (std::stop_token st) {
				while (!st.stop_requested()) {
					// request thread to become one of the enabled threads:
					if (enabled.try_acquire()) {
						...
						// remove thread from the set of enabled threads:
						enabled.release();
					}
					else {
						...
					}
				}
		}}
\end{cpp}

还可以使用try\_acquire\_for()或try\_acquire\_until()尝试在有限的时间内获取资源:

\begin{cpp}
std::jthread{[&, idx] (std::stop_token st) {
		while (!st.stop_requested()) {
			// request thread to become one of the enabled threads:
			if (enabled.try_acquire_for(100ms)) {
				...
				// remove thread from the set of enabled threads:
				enabled.release();
				}
			}
		}}
\end{cpp}

可不时地再次检查停止令牌的状态。

\mySamllsection{线程调度不公平}

注意,线程不一定是公平调度的。因此,当主线程想要结束程序时,可能需要一些时间(甚至永远)来调度。调用线程的析构函数之前请求所有线程停止也是一种很好的方法;否则,当前面的线程完成后,会在稍后请求停止。

线程没有公平调度的另一个后果是,不能保证等待时间最长的线程是首选。通常情况正好相反:若线程调度器已经有一个正在运行的线程调用release(),并且立即使用acquire(),则调度器会保持线程运行(“太好了,不需要上下文切换”),所以无法保证acquire()会唤醒等待的多个线程中的哪一个,可能总是使用相同的线程(伪唤醒)。

因此,在请求运行权限之前,不应该从队列中取出下一个值。

\begin{cpp}
// BAD if order of processing matters:
{
	std::lock_guard lg{valuesMx};
	val = values.front();
	values.pop();
}
enabled.acquire();
...
\end{cpp}

可能是在读取其他几个值之后才处理值val,或者甚至从未处理过。在读取下一个值之前请求许可:

\begin{cpp}
enabled.acquire();
{
	std::lock_guard lg{valuesMx};
	val = values.front();
	values.pop();
}
...
\end{cpp}

出于同样的原因,不能轻易地减少已启用线程的数量。可以试着调用:

\begin{cpp}
// reduce the number of enabled concurrent threads by one:
enabled.acquire();
\end{cpp}

但不知道这个报表是什么时候处理的。由于线程没有得到公平的调度,减少启用的线程数量的反应可能会花费很长时间,甚至可能会饿死(饥饿)。

为了公平地处理队列和对资源限制的即时响应,应该使用原子的新wait()和通知机制。

\mySubsubsection{13.2.2}{使用二值信号量的例子}

对于信号量,定义了一个特殊类型std::binary\_semaphore,这只是std::counting\_semaphore<1>的快捷方式,因此它只能启用或禁用单个资源的使用。

还可以将其用作互斥锁,这样做的好处是释放资源的线程不必是以前获取资源的线程,但更常用的应用程序是一种从另一个线程发出信号/通知线程的机制。与条件变量不同,可以多次执行此操作。

考虑下面的例子:

\filename{lib/semaphorenotify.cpp}

\begin{cpp}
#include <iostream>
#include <chrono>
#include <thread>
#include <semaphore>
using namespace std::literals; // for duration literals

int main()
{
	int sharedData;
	std::binary_semaphore dataReady{0}; // signal there is data to process
	std::binary_semaphore dataDone{0}; // signal processing is done

	// start threads to read and process values by value:
	std::jthread process{[&] (std::stop_token st) {
						while(!st.stop_requested()) {
							// wait until the next value is ready:
							// - timeout after 1s to check stop_token
							if (dataReady.try_acquire_for(1s)) {
								int data = sharedData;

								// process it:
								std::cout << "[process] read " << data << std::endl;
								std::this_thread::sleep_for(data * 0.5s);
								std::cout << "[process] done" << std::endl;

								// signal processing done:
								dataDone.release();
							}
							else {
								std::cout << "[process] timeout" << std::endl;
							}
							}
						}};

	// generate a couple of values:
	for (int i = 0; i < 10; ++i) {
		// store next value:
		std::cout << "[main] store " << i << std::endl;
		sharedData = i;

		// signal to start processing:
		dataReady.release();

		// wait until processing is done:
		dataDone.acquire();
		std::cout << "[main] processing done\n" << std::endl;
	}
	// end of loop signals stop
}
\end{cpp}

使用两个二值信号量让一个线程通知另一个线程:

\begin{itemize}
\item
使用dataReady,主线程通知线程进程sharedData中有新数据需要处理。

\item
使用dataDone,处理线程通知主线程数据已处理完成。
\end{itemize}

两个信号量都初始化为零,因此在默认情况下,获取线程阻塞。发出通知的线程调用release()的那一刻,获取消息的线程解除阻塞,以便程序做出反应。

程序的输出如下所示:

\begin{shell}
[main] store 0
[process] read 0
[process]      done
[main] processing done

[main] store 1
[process] read 1
[process]      done
[main] processing done

[main] store 2
[process] read 2
[process]      done
[main] processing done

[main] store 3
[process] read 3
[process]      done
[main] processing done
...

[main] store 9
[process] read 9
[process]      done
[main] processing done

[process] timeout
\end{shell}

处理线程只能使用try\_acquire\_for()获取有限的时间,返回值表示是否通知了它(访问资源)。这允许线程不时地检查是否发出了停止信号(就像主线程结束后那样)。

二值信号量的另一个应用是,等待在不同线程中运行的协程结束。

\mySamllsection{信号量的详情}

类模板std::counting\_semaphore<>与std::binary\_semaphore<1>的快捷方式std::binary\_semaphore也声明在头文件<semaphore>中:

\begin{cpp}
namespace std {
	template<ptrdiff_t least_max_value = implementation-defined >
	class counting_semaphore;
	using binary_semaphore = counting_semaphore<1>;
}
\end{cpp}

“类counting\_semaphore<>和binary\_semaphore对象的操作”表列出了信号量的API。

注意,不能复制或移动(分配)信号量。

将容器的大小(std::array除外)作为计数器的初始值是错误的。构造函数接受一个带符号的std::ptrdiff\_t,会看到以下结果:

\begin{cpp}
std::counting_semaphore s1{10}; // OK
std::counting_semaphore s2{10u}; // warnings may occur

std::vector<int> coll{ ... };
...
std::counting_semaphore s3{coll.size()}; // ERROR
std::counting_semaphore s4 = coll.size(); // ERROR
std::counting_semaphore s4(coll.size()); // OK (no narrowing checked)
std::counting_semaphore s6{int(coll.size())}; // OK
std::counting_semaphore s7{std::ssize(coll)}; // OK (see std::ssize())
\end{cpp}

% Please add the following required packages to your document preamble:
% \usepackage[normalem]{ulem}
% \useunder{\uline}{\ul}{}
% \usepackage{longtable}
% Note: It may be necessary to compile the document several times to get a multi-page table to line up properly
\begin{longtable}[c]{|l|l|}
\hline
\textbf{操作} & \textbf{效果}                                                                     \\ \hline
\endfirsthead
%
\endhead
%
semaphore s\{num\} & 创建一个用num初始化计数器的信号量                           \\ \hline
s.acquire()        & 阻塞,直到可以自动减少计数器(请求更多的资源)。 \\ \hline
s.try\_acquire()          & 尝试立即自动减少计数器(请求更多资源),若成功则返回true      \\ \hline
s.try\_acquire\_for(dur)  & 尝试在时间段内自动减少计数器(请求更多资源),若成功则返回true   \\ \hline
s.tre\_acquire\_until(tp) & 尝试直到时间点tp自动减少计数器(请求更多资源),若成功则返回true \\ \hline
s.release()        & 自动增加计数器(多启用一个资源)                      \\ \hline
s.release(num)     & 自动将num添加到计数器(启用更多资源)                     \\ \hline
max()              & 产生计数器最大可能值的静态函数               \\ \hline
\end{longtable}

\begin{center}
表13.3 类counting\_semaphore<>和binary\_semaphore对象的操作
\end{center}








