Each and every C++ version introduces various language and library features that are supported by compilers step by step. For that reason, knowing which C++ version a compiler supports in general is often not enough; for portable code, it might be important to know whether a specific feature is available.

For that purpose, C++20 officially introduces feature test macros. For each new language and library feature, a macro can be used to signal whether the feature is available. The macro can even provide information about which version of a feature is supported.

For example, the following source code will use different code depending on whether (and in which form) generic lambdas are available:

\begin{cpp}
#ifdef __cpp_generic_lambdas
#if __cpp_generic_lambdas >= 201707
... // generic lambdas with template parameters can be used
#else
... // generic lambdas can be used
#else
... // no generic lambdas can be used
#endif
\end{cpp}

All feature test macros for language features start with \_\_cpp.

As another example, the following code provides and uses a workaround if std::as\_const() is not available yet:

\begin{cpp}
#ifndef __cpp_lib_as_const
template<typename T>
const T& asConst(T& t) {
	return t;
} 
#endif

#ifdef __cpp_lib_as_const
	auto printColl = [&coll = std::as_const(coll)] {
#else
	auto printColl = [&coll = asConst(coll)] {
#endif
	...
	};
\end{cpp}

All feature test macros for library features start with \_\_cpp\_lib.

Feature test macros for language features are defined by the compilers. Feature test macros for library features are provided by the new header <version>.

See the use of \_\_cpp\_char8\_t as another example of making code that uses UTF-8 characters portable before and after C++20.





















