
视图有一个公共类型,提供了视图的大多数成员函数,这将在本节中讨论。

本节还讨论为什么范围适配器和工厂使用特殊的命名空间。

\mySubsubsection{8.2.1}{视图的基类}

所有标准视图都派生自类std::ranges::view\_interface<viewType>。已发布的C++20标准中,view\_interface<>也派生自一个空基类std::ranges::view\_base,但这个问题通过\url{http://wg21.link/lwg3549}解决了。]

类模板std::ranges::view\_interface<>基于派生视图类型的begin()和end()的定义引入了几个基本成员函数,这些函数必须作为模板参数传递给这个基类。std::ranges::view\_interface<>的操作表列出了类为视图提供的API。

\begin{longtable}[c]{|l|l|l|}
\hline
\textbf{操作} & \textbf{效果}                             & \textbf{是否提供}                       \\ \hline
\endfirsthead
%
\endhead
%
r.empty()          & 返回r是否为空(begin() == end()) & 前向迭代器                 \\ \hline
if (r)             & 若r不为空,则为True                      & 前向迭代器                 \\ \hline
r.size()           & 产生元素的数目               & 是否可以计算和end之间的差值 \\ \hline
r.front()          & 生成的第一个元素                    & 前向迭代器                 \\ \hline
r.back() & 生成的最后一个元素                            & 双向迭代器和end()产生与begin()相同的类型 \\ \hline
r{[}idx{]}         & 生成的第n个元素                     & 随机访问迭代器           \\ \hline
r.data() & 生成指向元素内存的原始指针 & 元素位于连续内存中                                          \\ \hline
\end{longtable}

\begin{center}
表8.3 std::ranges::view\_interface<>的操作
\end{center}

类view\_interface<>还为派生自它的每个类型初始化了std::ranges::enable\_view<>,这些类型满足std::ranges::view概念。

当自定义视图类型时,应该从view\_interface<>派生,并将自己的类型作为参数传递。例如:

\begin{cpp}
template<typename T>
class MyView : public std::ranges::view_interface<MyView<T>> {
	public:
	... begin() ... ;
	... end() ... ;
	...
};
\end{cpp}

基于begin()和end()的返回类型,类型自动提供std::ranges::view\_interface<>的操作表中列出的成员函数,若可用性的先决条件合适。这些成员函数的const版本要求,视图类型的const版本是一个有效范围。

C++23将添加成员cbegin()和cend(),将其映射到std::ranges::cbegin()和std::ranges::cend()(将与\url{http://wg21.link/p2278r4}一起添加)。

\mySubsubsection{8.2.2}{为什么范围适配器/工厂有自己的命名空间}

范围适配器和工厂有自己的命名空间std::ranges::views,为其定义了命名空间别名std::views:

\begin{cpp}
namespace std {
	namespace views = ranges::views;
}
\end{cpp}

这样,就可以为可能在其他命名空间(甚至在另一个标准命名空间)中使用的视图使用名称。

可以(也必须)在使用视图时进行限定:

\begin{cpp}
std::ranges::views::reverse // full qualification
std::views::reverse // shortcut
\end{cpp}

通常,不加限定地使用视图是不可能的。ADL不会启动,因为在命名空间std::views中没有定义范围(容器或视图)。

\begin{cpp}
std::vector<int> v;
...
take(v, 3) | drop(2); // ERROR: can’t find views (may find different symbol)
\end{cpp}

即使视图也会产生不属于视图命名空间的对象(std::ranges中),使用视图时仍然需要对视图进行限定:

\begin{cpp}
std::vector<int> values{ 0, 1, 2, 3, 4 };
auto v1 = std::views::all(values);
auto v2 = take(v1, 3); // ERROR
auto v3 = std::views::take(v1, 3); // OK
\end{cpp}

有一件重要的事情需要注意:永远不要使用using声明跳过范围适配器的限定:

\begin{cpp}
using namespace std::views; // do not do this
\end{cpp}

以作composers为例,若只是用定义的局部值对象过滤掉值,就会遇到麻烦:

\begin{cpp}
std::vector<int> values;
...
std::map<std::string, int> composers{ ... };

using namespace std::views; // do not do this

for (const auto& elem : composers | values) { // OOPS: finds wrong values
	...
}
\end{cpp}

本例中,将使用本地vector值,而非视图。若很幸运,会得到了一个编译时错误。若不够幸运,非限定视图会找到不同的符号,这可能会导致遇到,甚至出现覆盖其他对象内存的未定义行为。
