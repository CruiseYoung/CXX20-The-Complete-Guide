
C++20引入的一个新关键字是constinit,可用于强制并确保在编译时初始化可变静态或全局变量。大致效果为:

\begin{cpp}
constinit = constexpr - const
\end{cpp}

是的,constinit变量不是const(最好将关键字命名为compiletimeinit)。这个名称来源于这样一个事实,即这些初始化通常在编译时常量初始化时发生。

无论何时声明静态或全局变量,都可以使用constinit。例如:

\begin{cpp}
// outside any function:
constinit auto i = 42;
int getNextClassId() {
	static constinit int maxId = 0;
	return ++maxId;
}

class MyType {
	static inline constinit long max = sizeof(int) * 1000;
	...
};

constexpr std::array<int, 5> getColl() {
	return {1, 2, 3, 4, 5};
}
constinit auto globalColl = getColl();
\end{cpp}

仍然可以修改声明的值。下面的代码是第一次使用上面的声明:

\begin{cpp}
std::cout << i << " " << coll[0] << '\n'; // prints 42 1
i *= 2;
coll = {};
std::cout << i << " " << coll[0] << '\n'; // prints 84 0
\end{cpp}

输出如下:

\begin{shell}
42 1
84 0
\end{shell}

使用constinit的效果是,只有当初始值是编译时已知的常数值时,初始化才会编译,与以下声明相反:

\begin{cpp}
auto x = f(); // f() might be a runtime function
\end{cpp}

与constinit相对应的声明需要编译时初始化,则必须能够在编译时调用f()(f()必须是constexpr或consteval)。

\begin{cpp}
constinit auto x = f(); // f() must be a compile-time function
\end{cpp}

若用constinit初始化一个对象,必须能够在编译时使用构造函数:

\begin{cpp}
constinit std::pair p{42, "ok"}; // OK constinit std::list l; // ERROR: default constructor not constexpr
\end{cpp}

使用constinit的原因如下所示:

\begin{itemize}
\item
可以在编译时要求初始化可变全局/静态对象。就可以避免在运行时进行初始化,当使用thread\_local变量时,这可以提高性能。

\item
可以确保全局/静态对象在使用时总是初始化。实际上,constinit可以用来修复静态初始化顺序的错误,当一个静态/全局对象的初始值依赖于另一个静态/全局对象时,就会出现这种错误。
\end{itemize}

使用constinit永远不会改变程序的功能行为(除非我们遇到静态初始化顺序的惨败),只能导致代码不再可编译。

\mySubsubsection{18.1.1}{实践constinit}

使用constinit时有几件事需要注意。

首先,不能用另一个constinit值初始化一个constinit值:

\begin{cpp}
constinit auto x = f(); // f() must be a compile-time function
constinit auto y = x; // ERROR: x is not a constant initializer
\end{cpp}

原因是初始值必须是编译时已知的常量值,但constinit不是常量。只编译以下代码:

\begin{cpp}
constexpr auto x = f(); // f() must be a compile-time function
constinit auto y = x; // OK
\end{cpp}

初始化对象时,需要编译时构造函数,但编译时析构函数不是必需的。出于这个原因,可以对智能指针使用constinit:

\begin{cpp}
constinit std::unique_ptr<int> up; // OK
constinit std::shared_ptr<int> sp; // OK
\end{cpp}

constinit并不意味着内联(这与constexpr不同)。例如:

\begin{cpp}
class Type {
	constinit static int val1 = 42; // ERROR
	inline static constinit int val2 = 42; // OK
	...
};
\end{cpp}

可以和extern一起使用constinit:

\begin{cpp}
// header:
extern constinit int max;

// translation unit:
constinit int max = 42;
\end{cpp}

为了达到同样的效果,也可以跳过声明中的constinit,但在定义中跳过constinit,将不再强制进行编译时初始化。

也可以将constinit与static和thread\_local一起使用:

\begin{cpp}
static thread_local constinit int numCalls = 0;
\end{cpp}

constinit、static和thread\_local的顺序无所谓。

对于thread\_local变量,使用constinit可能会提高性能,因为其可以避免生成的代码(需要内部保护)来指示变量是否初始化:

\begin{cpp}
extern thread_local int x1 = 0;
extern thread_local constinit int x2 = 0; // better (might avoid an internal guard)
\end{cpp}

使用constinit声明引用是可能的,但是没有意义,因为引用引用的是一个常量对象,这里应该使用constexpr。

\mySubsubsection{18.1.2}{constinit如何解决静态初始化顺序的问题}

C++中,有一个叫做静态初始化顺序的问题,constinit可以解决这个问题。问题是没有定义不同翻译单元中静态和全局初始化的顺序。出于这个原因,下面的代码可能有问题:

\begin{itemize}
\item
假设有一个带构造函数的类型来初始化对象,并引入该类型的外部全局对象:

\filename{comptime/truth.hpp}

\begin{cpp}
#ifndef TRUTH_HPP
#define TRUTH_HPP

struct Truth {
	int value;
	Truth() : value{42} { // ensure all objects are initialized with 42
	}
};
extern Truth theTruth; // declare global object

#endif // TRUTH_HPP
\end{cpp}

\item
在对象自己的转换单元中初始化该对象:

\filename{comptime/truth.cpp}

\begin{cpp}
#include "truth.hpp"

Truth theTruth; // define global object (should have value 42)
\end{cpp}

\item
然后,在另一个转换单元中,用theTruth初始化另一个全局/静态对象:

\filename{comptime/fiasco.cpp}

\begin{cpp}
#include "truth.hpp"
#include <iostream>

int val = theTruth.value; // may be initialized before theTruth is initialized

int main()
{
	std::cout << val << '\n'; // OOPS: may be 0 or 42
	++val;
	std::cout << val << '\n'; // OOPS: may be 1 or 43
}
\end{cpp}
\end{itemize}

有一个很好的机会,val初始化在theTruth初始化之前。程序可能有以下输出:[例如,在使用gcc编译器时,若传递在fiasco.o前传递了truth.o给连接器,就会出现这个问题。]

\begin{shell}
0
1
\end{shell}

当使用constinit声明val时,就不会出现这个问题。constinit确保对象总是在使用之前进行初始化,因为初始化是在编译时进行的。若不能给出保证,代码将无法编译。若val用constexpr声明,初始化也会得到保证,但这种情况下,将无法再修改该值。

例子中,仅仅使用constinit首先会导致一个编译时错误(表示在编译时无法保证初始化):

\begin{cpp}
// truth.hpp:
struct Truth {
	int value;
	Truth() : value{42} {
	}
};

extern Truth theTruth;
// main translation unit:
constinit int val = theTruth.value ; // ERROR: no constant initializer
\end{cpp}

该错误消息表明,在编译时不能初始化val。为了使初始化有效,必须修改类Truth和theTruth的声明,以便theTruth可以在编译时使用:

\filename{comptime/truthc.hpp}

\begin{cpp}
#ifndef TRUTH_HPP
#define TRUTH_HPP

struct Truth {
	int value;
	constexpr Truth() : value{42} { // enable compile-time initialization
	}
};

constexpr Truth theTruth; // force compile-time initialization

#endif // TRUTH_HPP
\end{cpp}

现在,编译程序,val可保证用theTruth的初始值进行初始化:

\filename{comptime/constinit.hpp}

\begin{cpp}
#include "truthc.hpp"
#include <iostream>

constinit int val = theTruth.value ; // initialized after theTruth is initialized

int main()
{
	std::cout << val << '\n'; // guaranteed to be 42
	++val;
	std::cout << val << '\n'; // guaranteed to be 43
}
\end{cpp}

程序的输出现在可以保证为:

\begin{shell}
42
43
\end{shell}

还有其他方法可以解决静态初始化顺序的问题(使用静态函数获取值或使用内联)。然而,若初始化不需要运行时值/特性,则可能需要仔细考虑遵循总是使用constinit声明全局变量和静态变量的编程风格。在这样的函数中使用,至少没有坏处:

\begin{cpp}
long nextId()
{
	constinit static long id = 0;
	return ++id;
}
\end{cpp}








