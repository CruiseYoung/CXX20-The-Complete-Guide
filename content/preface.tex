
C++20 is the next evolution in modern C++ programming, and it is already (partially) supported by the latest version of GCC, Clang, and Visual C++. The move to C++20 is at least as big a step as the move to C++11. C++20 contains a significant number of new language features and libraries that will again change the way we program in C++. This applies to both application programmers and programmers who provide foundation libraries.

\subsection*{An Experiment}
\addcontentsline{toc}{subsection}{An Experiment}

This book is an experiment in two ways:

\begin{itemize}
\item
I am writing an in-depth book that covers complex new features invented and provided by different programmers and C++ working groups. However, I can ask questions and I do.

\item
I am publishing the book myself on Leanpub and for printing on demand. That is, this book is written step by step and I will publish new versions as soon there is a significant improvement that makes the publication of a new version worthwhile.
\end{itemize}

The good thing is:

\begin{itemize}
\item
You get the view of the language features from an experienced application programmer—somebody who feels the pain a feature might cause and asks the relevant questions to be able to explain the motivation for a feature, its design, and all consequences for using it in practice.

\item
You can benefit from my experience with C++20 while I am still learning and writing.

\item
This book and all readers can benefit from your early feedback.
\end{itemize}

This means that you are also part of the experiment. So help me out: give feedback about flaws, errors, features that are not explained well, or gaps, so that we all can benefit from these improvements.

\subsection*{Acknowledgments}
\addcontentsline{toc}{subsection}{Acknowledgments}

This book would not have been possible without the help and support of a huge number of people.

First of all, I would like to thank you, the C++ community, for making this book possible. The incredible design of all the features of C++20, the helpful feedback, and your curiosity are the basis for the evolution of a successful language. In particular, thanks for all the issues you told me about and explained and for the feedback you gave.

I would especially like to thank everyone who reviewed drafts of this book or corresponding slides and provided valuable feedback and clarification. These reviews increased the quality of the book significantly, again proving that good things are usually the result of collaboration between many people. Therefore, so far (this list is still growing) huge thanks to Carlos Buchart, Javier Estrada, Howard Hinnant, Yung-Hsiang Huang, Daniel Krugler, Dietmar K ¨ uhl, Jens Maurer, Paul Ranson, Thomas Symalla, Steve Vinoski, Ville ¨ Voutilainen. Andreas Weis, Hui Xie, Leor Zolman, and Victor Zverovich.

In addition, I would like to thank everyone on the C++ standards committee. In addition to all the work involved in adding new language and library features, these experts spent many, many hours explaining and discussing their work with me, and they did so with patience and enthusiasm. Special thanks here go to Howard Hinnant, Tom Honermann, Tomasz Kaminski, Peter Sommerlad, Tim Song, Barry Revzin, Ville Voutilainen, and Jonathan Wakely.

Special thanks go to the LaTeX community for a great text system and to Frank Mittelbach for solving my \LaTeX{} issues (it was almost always my fault).

And finally, many thanks go to my proofreader, Tracey Duffy, who has again done a tremendous job of converting my “German English” into native English.


















