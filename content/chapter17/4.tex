通过对Lambda使用新的consteval关键字,现在可以要求lambdas成为直接函数,以便“函数调用”必须在编译时求值。例如:

\begin{cpp}
auto hashed = [] (const char* str) consteval {
					...
				};
auto hashWine = hashed("wine"); // hash() called at compile time
\end{cpp}

由于在Lambda的定义中使用了consteval,调用都必须在编译时使用编译时已知的值进行。传递运行时值是错误的:

\begin{cpp}
const char* s = "beer";
auto hashBeer = hashed(s); // ERROR

constexpr const char* cs = "water";
auto hashWater = hashed(cs); // OK
\end{cpp}

注意,哈希本身并不一定是constexpr。它是Lambda的运行时对象,在编译时对其执行“函数调用”。

关于新的consteval关键字一节中对Lambda的consteval进行更加细致的讨论。

还可以将新的模板语法用于具有consteval的泛型Lambda。这使开发者能够在另一个函数中定义编译时函数的初始化。例如:

\begin{cpp}
// local compile-time computation of Num prime numbers:
auto primeNumbers = [] <int Num> () consteval {
					std::array<int, Num> primes;
					int idx = 0;
					for (int val = 1; idx < Num; ++val) {
						if (isPrime(val)) {
							primes[idx++] = val;
						}
					}
					return primes;
				};
\end{cpp}

请参阅lang/lambdaconsteval.cpp获取使用该Lambda的完整示例。

注意,这时,模板参数没有推导出来,所以显式指定模板参数的语法会有点难看:

\begin{cpp}
auto primes = primeNumbers.operator()<100>();
\end{cpp}

还要注意的是,必须在consteval之前提供参数列表(指定constexpr时也适用)。即使没有声明参数,也不能没有括号。




