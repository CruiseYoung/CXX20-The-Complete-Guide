

自定义数据类型都可以定义operator<=>和operator==:

\begin{itemize}
\item
要么作为带一个参数的成员函数

\item
或者一个独立的函数,有两个参数
\end{itemize}


\begin{longtable}[c]{|l|l|}
\hline
\textbf{std::中的函数对象} & \textbf{效果}                                             \\ \hline
\endfirsthead
%
\endhead
%
strong\_order()                   & 映射到强序值也适用于浮点值 \\ \hline
weak\_order()                     & 映射到弱序值,也适用于浮点值   \\ \hline
partial\_order()                  & 映射到偏序值                               \\ \hline
compare\_strong\_order\_fallback()  & 映射到强序值,即使只定义了==和<  \\ \hline
compare\_weak\_order\_fallback()    & 映射到弱序值,即使只定义了==和<    \\ \hline
compare\_partial\_order\_fallback() & 映射到偏序值,即使只定义了==和< \\ \hline
\end{longtable}

\begin{center}
表1.1 用于映射比较类别类型的函数对象
\end{center}


\mySubsubsection{1.3.1}{默认操作符==和<=>}

在类或数据结构中(作为成员或友元函数),可以使用=default将所有比较操作符声明为默认值,但这通常只对operator==和operator<=>有意义。成员函数必须接受第二个形参作为const左值引用(const \&),友元函数也可以按值接受这两个形参。

默认操作符需要成员和可能的基类的支持:

\begin{itemize}
\item
默认操作符==要求在成员和基类中支持==。

\item
默认操作符<=>要求在成员和基类中支持==和已实现的操作符<或默认操作符<=>(详细信息请参见下文)。
\end{itemize}

对于生成的默认操作符,有两点要提一下:

\begin{itemize}
\item
若比较成员保证不抛出异常,则操作符为noexcept型。

\item
若可以在编译时比较成员,则操作符为constexpr型。
\end{itemize}

对于空类,默认操作符比较所有对象为相等:operator ==、<=和>=生成true, operator !=、<和>生成false,而<=>生成std::strong\_ordering::equal。


\mySubsubsection{1.3.2}{默认<=>操作符实现默认==操作符}

当操作符<=>成员定义为默认值时,若没有提供默认操作符==,则根据定义也定义相应的操作符==成员。可采用了所有方式(可见性、虚拟、属性、需求等)。例如:

\begin{cpp}
template<typename T>
class Type {
	...
	public:
		[[nodiscard]] virtual std::strong_ordering
			operator<=>(const Type&) const requires(!std::same_as<T,bool>) = default;
};
\end{cpp}

等价于:

\begin{cpp}
template<typename T>
class Type {
	...
	public:
	[[nodiscard]] virtual std::strong_ordering
		operator<=> (const Type&) const requires(!std::same_as<T,bool>) = default;
	
	[[nodiscard]] virtual bool
		operator== (const Type&) const requires(!std::same_as<T,bool>) = default;
};
\end{cpp}

例如,下面的代码足以支持Coord类型对象的所有六种比较操作符:

\filename{lang/coord.hpp}

\begin{cpp}
#include <compare>
struct Coord {
	double x{};
	double y{};
	double z{};
	auto operator<=>(const Coord&) const = default;
};
\end{cpp}

成员函数必须是const,形参必须声明为const左值引用(const \&)。

可以这样使用这个数据结构:

\filename{lang/coord.cpp}

\begin{cpp}
#include "coord.hpp"
#include <iostream>
#include <algorithm>
int main()
{
	std::vector<Coord> coll{ {0, 5, 5}, {5, 0, 0}, {3, 5, 5},
							 {3, 0, 0}, {3, 5, 7} };
							 
	std::sort(coll.begin(), coll.end());
	for (const auto& elem : coll) {
		std::cout << elem.x << '/' << elem.y << '/' << elem.z << '\n';
	}
}
\end{cpp}

该程序应有如下的输出:

\begin{shell}
0/5/5
3/0/0
3/5/5
3/5/7
5/0/0
\end{shell}

\mySubsubsection{1.3.3}{默认操作符<=>的实现}

若operator<=>是默认的,并且有成员或基类,并且调用关系操作符之一,则会发生以下情况:

\begin{itemize}
\item
若为成员或基类定义了操作符<=>,则调用该操作符。

\item
否则,使用operator==和operator<来决定是否(从成员或基类的角度)

\begin{itemize}
\item
对象是相等/等价的(运算符==产生true)

\item
对象有“大”有“小”

\item
对象是无序的(仅在检查偏排序时)
\end{itemize}

这种情况下,调用这些操作符的默认操作符<=>的返回类型不能为auto。
\end{itemize}

例如,以下声明:

\begin{cpp}
struct B {
	bool operator==(const B&) const;
	bool operator<(const B&) const;
};

struct D : public B {
	std::strong_ordering operator<=> (const D&) const = default;
};
\end{cpp}

然后:

\begin{cpp}
D d1, d2;
d1 > d2; // calls B::operator== and possibly B::operator<
\end{cpp}

若操作符==的结果为true,则知道>的结果为false,仅此而已。否则,调用操作符<来判断表达式是true还是false。

\begin{cpp}
struct D : public B {
	std::partial_ordering operator<=> (const D&) const = default;
};
\end{cpp}

编译器甚至可能两次调用operator<来确定是否有顺序。

\begin{cpp}
struct B {
	bool operator==(const B&) const;
	bool operator<(const B&) const;
};

struct D : public B {
	auto operator<=> (const D&) const = default;
};
\end{cpp}

编译器不会编译带有关系操作符的调用,它无法确定基类具有哪个排序类别在这种情况下,在基类中也需要操作符<=>。

然而,检查相等性是有效的,因为在D中,operator==自动声明为等价于以下内容的代码:

\begin{cpp}
struct D : public B {
	auto operator<=> (const D&) const = default;
	bool operator== (const D&) const = default;
};
\end{cpp}

当有以下行为:

\begin{cpp}
D d1, d2;
d1 > d2; // ERROR: cannot deduce comparison category of operator<=>
d1 != d2; // OK (note: only tries operator<=> and B::operator== of a base class)
\end{cpp}

相等性检查总是只使用基类的操作符==(但可以根据默认操作符<=>生成),可忽略基类中的<或!=运算符。

同样地,若D中有B类型的成员。
