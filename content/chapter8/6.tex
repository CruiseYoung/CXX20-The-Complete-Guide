

本节讨论产生所遍历元素修改值的视图。

\mySubsubsection{8.6.1}{转换视图}

% Please add the following required packages to your document preamble:
% \usepackage{longtable}
% Note: It may be necessary to compile the document several times to get a multi-page table to line up properly
\begin{longtable}[c]{|l|l|}
\hline
\textbf{类型:}                 & std::ranges::transform\_view\textless{}\textgreater{} \\ \hline
\endfirsthead
%
\endhead
%
\textbf{内容:}              & 范围内所有元素的转换值      \\ \hline
\textbf{适配器:}              & std::views::transform()                               \\ \hline
\textbf{元素类型:}         & 转换的返回类型               \\ \hline
\textbf{要求:}             & 至少为输入范围                                  \\ \hline
\textbf{类别:}             & 与传递的相同,但至多是随机访问              \\ \hline
\textbf{是否是长度范围:}       & 若在长度范围内                                     \\ \hline
\textbf{是否是通用范围:}      & 若在通用范围内                                    \\ \hline
\textbf{是否是租借范围:}    & 从不                                                 \\ \hline
\textbf{缓存:}               & 无                                               \\ \hline
\textbf{常量可迭代:} & 若有常量可迭代范围和转换作用于const值,则可用 \\ \hline
\textbf{传播常量性:} & 只有在右值范围内                               \\ \hline
\end{longtable}

类模板std::ranges::transform\_view<>定义了一个视图,该视图在传递转换后生成底层范围内的所有元素。例如:

\begin{cpp}
std::vector<int> rg{1, 2, 3, 4, 5, 6, 7, 8, 9, 10, 11, 12, 13};
...
// print all elements squared:
for (const auto& elem : std::ranges::transform_view{rg, [](auto x) {
							return x * x;
					}}) {
	std::cout << elem << ' ';
}
\end{cpp}

循环输出:

\begin{shell}
1 4 9 16 25 36 49 64 81 100 121 144 169
\end{shell}

\mySamllsection{转换视图的范围适配器}

转换视图也可以(通常应该)使用范围适配器创建,适配器只是将其参数传递给std::ranges::transform\_view构造函数。

\begin{cpp}
std::views::transform(rg, func)
\end{cpp}

例如:

\begin{cpp}
for (const auto& elem : std::views::transform(rg, [](auto x) {
						return x * x;
					})) {
	std::cout << elem << ' ';
}
\end{cpp}

或:

\begin{cpp}
for (const auto& elem : rg | std::views::transform([](auto x) {
						x * x;
					})) {
	std::cout << elem << ' ';
}
\end{cpp}

下面是一个使用转换视图的完整示例:

\filename{ranges/transformview.cpp}

\begin{cpp}
#include <iostream>
#include <string>
#include <vector>
#include <ranges>
#include <cmath> // for std::sqrt()

void print(std::ranges::input_range auto&& coll)
{
	for (const auto& elem : coll) {
		std::cout << elem << ' ';
	}
	std::cout << '\n';
}

int main()
{
	std::vector coll{1, 2, 3, 4, 1, 2, 3, 4, 1};
	
	print(coll); // 1 2 3 4 1
	auto sqrt = [] (auto v) { return std::sqrt(v); };
	print(std::ranges::transform_view{coll, sqrt}); // 1 1.41421 1.73205 2 1
	print(coll | std::views::transform(sqrt)); // 1 1.41421 1.73205 2 1
}
\end{cpp}

该程序有以下输出:

\begin{shell}
1 2 3 4 1 2 3 4 1
1 1.41421 1.73205 2 1 1.41421 1.73205 2 1
1 1.41421 1.73205 2 1 1.41421 1.73205 2 1
\end{shell}

转换必须是满足std::regular\_invocable概念的可调用对象,所以转换永远不应该修改传递的基础范围的值。但不修改值是语义约束,在编译时不能总是检查这一点,所以至少应该将可调用对象声明为通过值或const引用接受实参。

转换视图产生具有转换返回类型的值,所以在示例中,转换视图产生的元素类型为double。

转换的返回类型,甚至可以是引用。例如:

\filename{ranges/transformref.cpp}

\begin{cpp}
#include <iostream>
#include <vector>
#include <utility>
#include <ranges>

void printPairs(const auto& collOfPairs)
{
	for (const auto& elem : collOfPairs) {
		std::cout << elem.first << '/' << elem.second << ' ';
	}
	std::cout << '\n';
}

int main()
{
	// initialize collection with pairs of int as elements:
	std::vector<std::pair<int,int>> coll{{1,9}, {9,1}, {2,2}, {4,1}, {2,7}};
	printPairs(coll);
	
	// function that yields the smaller of the two values in a pair:
	auto minMember = [] (std::pair<int,int>& elem) -> int& {
		return elem.second < elem.first ? elem.second : elem.first;
	};
	
	// increment the smaller of the two values in each pair element:
	for (auto&& member : coll | std::views::transform(minMember)) {
		++member;
	}
	printPairs(coll);
}
\end{cpp}

该程序有以下输出:

\begin{shell}
1/9 9/1 2/2 4/1 2/7
2/9 9/2 3/2 4/2 3/7
\end{shell}

注意,这里传递给transform的Lambda必须返回一个引用:

\begin{cpp}
[] (std::pair<int,int>& elem) -> int& {
	return ...
}
\end{cpp}

否则,Lambda将返回每个元素的副本,以便这些副本的成员递增,而coll中的元素保持不变。

转换视图在内部存储传递的范围(可选择以与all()相同的方式转换为视图),所以只有在传递的范围有效的情况下才有效(除非传递了右值,则在内部使用了归属视图)。

\mySamllsection{转换视图的特殊特性}

与通常的标准视图一样,转换视图不会将常量性委托给元素。所以即使视图是const,传递给转换函数的元素也不是const。

例如,使用上面的transform.cpp示例,可以完成以下实现:

\begin{cpp}
std::vector<std::pair<int,int>> coll{{1,9}, {9,1}, {2,2}, {4,1}, {2,7}};

// function that yields the smaller of the two values in a pair:
auto minMember = [] (std::pair<int,int>& elem) -> int& {
	return elem.second < elem.first ? elem.second : elem.first;
};

// increment the smaller of the two values in each pair element:
const auto v = coll | std::views::transform(minMember);
for (auto&& member : v) {
	++member;
}
\end{cpp}

注意,若底层范围是长度范围和通用范围,则获取视图仅是通用的(迭代器和哨兵具有相同的类型)。为了协调类型,可能必须使用通用视图。

开始迭代器和哨兵(结束迭代器)都是特殊的内部辅助类型。

\mySamllsection{转换视图的接口}

“类std::ranges::transform\_view<>的操作”表列出了转换视图的API。

% Please add the following required packages to your document preamble:
% \usepackage{longtable}
% Note: It may be necessary to compile the document several times to get a multi-page table to line up properly
\begin{longtable}[c]{|l|l|}
\hline
\textbf{操作} & \textbf{效果}                                                \\ \hline
\endfirsthead
%
\endhead
%
transform\_view r\{\}         & 创建一个transform\_view,它引用一个默认构造的范围                               \\ \hline
transform\_view r\{rg, func\} & 创建一个transform\_view,用函数变换rg范围内所有元素的值          \\ \hline
r.begin()          & 生成begin迭代器                                      \\ \hline
r.end()            & 生成哨兵(end迭代器)                             \\ \hline
r.empty()          & 生成的r是否为空(若范围支持,则可用) \\ \hline
if (r)             & 若r不为空,则为true(若定义了empty(),则可用)         \\ \hline
r.size()           & 生成元素的数量(若引用了一个长度范围,则可用)                            \\ \hline
r.front()          & 生成的第一个元素(若转发可用)              \\ \hline
r.back()           & 生成的最后一个元素(若迭代器为双向且通用,则可用) \\ \hline
r{[}idx{]}         & 生成的第n个元素(若果随机访问,则可用)            \\ \hline
r.data()           & 生成一个指向元素内存的原始指针(若元素在连续内存中可用) \\ \hline
r.base()           & 产生对r所引用或归属范围的引用       \\ \hline
\end{longtable}

\begin{center}
表8.19 类std::ranges::transform\_view<>的操作
\end{center}

\mySubsubsection{8.6.2}{元素视图}

% Please add the following required packages to your document preamble:
% \usepackage{longtable}
% Note: It may be necessary to compile the document several times to get a multi-page table to line up properly
\begin{longtable}[c]{|l|l|}
\hline
\textbf{类型:}              & std::ranges::elements\_view\textless{}\textgreater{}            \\ \hline
\endfirsthead
%
\endhead
%
\textbf{内容:}           & 范围中所有类元组元素的第n个成员/属性 \\ \hline
\textbf{适配器:}           & std::views::elements\textless{}\textgreater{}                   \\ \hline
\textbf{元素类型:}         & 成员/属性的类型 \\ \hline
\textbf{要求:}             & 至少为输入范围        \\ \hline
\textbf{类别:}          & 与传递的相同,但至多是随机访问                        \\ \hline
\textbf{是否是长度范围:}       & 若在长度范围内           \\ \hline
\textbf{是否是通用范围:}      & 若在通用范围内          \\ \hline
\textbf{是否是租借范围:} & 若是租借视图或左值非视图                       \\ \hline
\textbf{缓存:}               & 无                     \\ \hline
\textbf{常量可迭代:}       & 若在可迭代范围内  \\ \hline
\textbf{传播常量性:} & 只有在右值范围内     \\ \hline
\end{longtable}

类模板std::ranges::elements\_view<>定义了一个视图,该视图选择传递的范围中所有元素的第idx个成员/属性/元素。视图为每个元素调用get<idx>(elem),并且可以使用

\begin{itemize}
\item
获取所有std::tuple元素的第idx个成员

\item
获取所有std::variant元素的第idx个成员

\item
获取std::pair元素的第一个或第二个成员(不过,对于关联容器和无序容器的元素,使用keys\_view和values\_view更方便)
\end{itemize}

注意,对于这个视图,类模板参数推导不起作用,因为必须显式地将索引指定为参数,并且类模板不支持部分参数推导。出于这个原因,必须这样书写:

\begin{cpp}
std::vector<std::tuple<std::string, std::string, int>> rg{
	{"Bach", "Johann Sebastian", 1685}, {"Mozart", "Wolfgang Amadeus", 1756},
	{"Beethoven", "Ludwig van", 1770}, {"Chopin", "Frederic", 1810},
};

for (const auto& elem
	: std::ranges::elements_view<decltype(std::views::all(rg)), 2>{rg}) {
	std::cout << elem << ' ';
}
\end{cpp}

循环输出:

\begin{shell}
1685 1756 1770 1810
\end{shell}

请注意,若传递的范围不是视图,则必须使用范围适配器all()指定底层范围的类型:

\begin{cpp}
std::ranges::elements_view<decltype(std::views::all(rg)), 2>{rg} // OK
\end{cpp}

也可以直接使用std::views::all\_t<>指定类型:

\begin{cpp}
std::ranges::elements_view<std::views::all_t<decltype(rg)&>, 2>{rg} // OK
std::ranges::elements_view<std::views::all_t<decltype((rg))>, 2>{rg} // OK
\end{cpp}

然而,这里的细节很重要。若范围还不是视图,那么all\_t<>的参数必须是左值引用。因此,需要在rg的类型后面加上\&,或者在rg周围加上双括号(根据规则,若传递的表达式是左值,则decltype会产生一个左值引用)。没有\&的单圆括号不起作用:

\begin{cpp}
std::ranges::elements_view<std::views::all_t<decltype(rg)>, 2>{rg} // ERROR
\end{cpp}

因此,声明视图更简单的方法是使用相应的范围适配器。

\mySamllsection{元素视图的范围适配器}

元素视图也可以(通常应该)使用范围适配器创建。适配器使视图的使用更容易,不必指定底层范围的类型:

\begin{cpp}
std::views::elements<idx>(rg)
\end{cpp}

例如:

\begin{cpp}
for (const auto& elem : std::views::elements<2>(rg)) {
	std::cout << elem << ' ';
}
\end{cpp}

或:

\begin{cpp}
for (const auto& elem : rg | std::views::elements<2>) {
	std::cout << elem << ' ';
}
\end{cpp}

使用范围适配器,elements<idx>(rg)等价于:

\begin{cpp}
std::ranges::elements_view<std::views::all_t<decltype(rg)&>, idx>{rg}
\end{cpp}

元素视图在内部存储传递的范围(可选择以与all()相同的方式转换为视图),所以只有在传递的范围有效的情况下才有效(除非传递了右值,则在内部使用了归属视图)。

begin迭代器和哨兵(end迭代器)都是特殊的内部辅助类型,若传递的是通用范围,则是通用的。

下面是一个使用元素视图的完整示例:

\filename{ranges/elementsview.cpp}

\begin{cpp}
#include <iostream>
#include <string>
#include <vector>
#include <ranges>
#include <numbers> // for math constants
#include <algorithm> // for sort()

void print(std::ranges::input_range auto&& coll)
{
	for (const auto& elem : coll) {
		std::cout << elem << ' ';
	}
	std::cout << '\n';
}

int main()
{
	std::vector<std::tuple<int, std::string, double>> coll{
		{1, "pi", std::numbers::pi},
		{2, "e", std::numbers::e},
		{3, "golden-ratio", std::numbers::egamma},
		{4, "euler-constant", std::numbers::phi},
	};
	
	std::ranges::sort(coll, std::less{},
					  [](const auto& e) {return std::get<2>(e);});
	print(std::ranges::elements_view<decltype(std::views::all(coll)), 1>{coll});
	print(coll | std::views::elements<2>);
}
\end{cpp}

该程序有以下输出:

\begin{shell}
golden-ratio euler-constant e pi
0.577216 1.61803 2.71828 3.14159
\end{shell}

请注意,不应该按以下方式对coll的元素进行排序:

\begin{cpp}
std::ranges::sort(coll | std::views::elements<2>); // OOPS
\end{cpp}

这将只对元素的值进行排序,而不是对整个元素进行排序,并将出现以下输出:

\begin{shell}
pi e golden-ratio euler-constant
0.577216 1.61803 2.71828 3.14159
\end{shell}

\mySamllsection{其他类元组类型的元素视图}

为了能够将此视图用于用户定义的类型,需要为它们指定一个类似元组的API。但就目前的规范而言,这实际上并不是因为一般的类元组api的设计和相应概念的定义方式存在问题。严格地说,只能对std::pair<>和std::tuple<>使用std::ranges::elements\_view类或适配器std::views::elements<>。

但若确保在包含头文件<ranges>之前定义了类似元组的API,就可以工作。

例如:

\filename{ranges/elementsviewhack.cpp}

\begin{cpp}
#include <iostream>
#include <string>
#include <tuple>
// don’t include <ranges> yet !!

struct Data {
	int id;
	std::string value;
};

std::ostream& operator<< (std::ostream& strm, const Data& d) {
	return strm << '[' << d.id << ": " << d.value << ']';
}

// tuple-like access to Data:
namespace std {
	template<>
	struct tuple_size<Data> : integral_constant<size_t, 2> {
	};
	
	template<>
	struct tuple_element<0, Data> {
		using type = int;
	};
	template<>
	struct tuple_element<1, Data> {
		using type = std::string;
	};
	
	template<size_t Idx> auto get(const Data& d) {
		if constexpr (Idx == 0) {
			return d.id;
		}
		else {
			return d.value;
		}
	}
} // namespace std
\end{cpp}

可以这样使用:

\filename{ranges/elementsviewhack.cpp}

\begin{cpp}
// don’t include <ranges> before "elementsview.hpp"
#include "elementsviewhack.hpp"
#include <iostream>
#include <vector>
#include <ranges>

void print(const auto& coll)
{
	std::cout << "coll:\n";
	for (const auto& elem : coll) {
		std::cout << "- " << elem << '\n';
	}
}

int main()
{
	Data d1{42, "truth"};
	std::vector<Data> coll{d1, Data{0, "null"}, d1};
	print(coll);
	print(coll | std::views::take(2));
	print(coll | std::views::elements<1>);
}
\end{cpp}

输出如下:

\begin{shell}
coll:
- [42: truth]
- [0: null]
- [42: truth]
coll:
- [42: truth]
- [0: null]
coll:
- truth
- null
- truth
\end{shell}

\mySamllsection{元素视图的接口}

“类std::ranges::elements\_view<>的操作”表列出了元素视图的API。

% Please add the following required packages to your document preamble:
% \usepackage{longtable}
% Note: It may be necessary to compile the document several times to get a multi-page table to line up properly
\begin{longtable}[c]{|l|l|}
\hline
\textbf{操作}     & \textbf{效果}                                                \\ \hline
\endfirsthead
%
\endhead
%
elements\_view r\{\} & 创建一个elements\_view,它引用一个默认构造的范围                               \\ \hline
elements\_view r\{rg\} & 创建一个引用范围rg的elements\_view              \\ \hline
r.begin()              & 生成begin迭代器                                      \\ \hline
r.end()                & 生成哨兵(end迭代器)                             \\ \hline
r.empty()              & 生成的r是否为空(若范围支持,则可用) \\ \hline
if (r)                 & 若r不为空,则为true(若定义了empty(),则可用)        \\ \hline
r.size()             & 生成元素的数量(若引用了一个长度范围,则可用)                             \\ \hline
r.front()              & 生成的第一个元素(若转发可用)              \\ \hline
r.back()               & 生成的最后一个元素(若迭代器为双向且通用,则可用) \\ \hline
r{[}idx{]}             & 生成的第n个元素(若随机访问可用)            \\ \hline
r.data()             & 生成一个指向元素内存的原始指针(若元素在连续内存中可用)。 \\ \hline
r.base()               & 生成对r所引用或归属范围的引用       \\ \hline
\end{longtable}

\begin{center}
表8.20 类std::ranges::elements\_view<>的操作
\end{center}

	
\mySubsubsection{8.6.3}{键和值的视图}

% Please add the following required packages to your document preamble:
% \usepackage{longtable}
% Note: It may be necessary to compile the document several times to get a multi-page table to line up properly
\begin{longtable}[c]{|l|ll|}
\hline
\textbf{类型:}    & \multicolumn{1}{l|}{std::ranges::keys\_view\textless{}\textgreater{}} & std::ranges::values\_view\textless{}\textgreater{} \\ \hline
\endfirsthead
%
\endhead
%
\textbf{内容:} & \multicolumn{2}{l|}{范围中所有类元组元素的第一个/第二个成员或属性}                            \\ \hline
\textbf{适配器:}              & \multicolumn{1}{l|}{std::views::keys}           & std::views::values         \\ \hline
\textbf{元素类型:}         & \multicolumn{1}{l|}{第一个成员的类型}   & 第二个成员的类型  \\ \hline
\textbf{要求:}             & \multicolumn{1}{l|}{至少为输入范围}       & 至少为输入范围       \\ \hline
\textbf{类别:}             & \multicolumn{2}{l|}{与传递的相同,但至多是随机访问}                \\ \hline
\textbf{是否是长度范围:}       & \multicolumn{1}{l|}{若在长度范围内}          & 若在长度范围内          \\ \hline
\textbf{是否是通用范围:}      & \multicolumn{1}{l|}{若在通用范围内}         & 若在通用范围内         \\ \hline
\textbf{是否是租借范围:}    & \multicolumn{2}{l|}{若为租借视图或左值非视图}               \\ \hline
\textbf{缓存:}               & \multicolumn{1}{l|}{无}                    & Nothing                    \\ \hline
\textbf{常量可迭代:}       & \multicolumn{1}{l|}{若在可迭代范围内} & 若在可迭代范围内 \\ \hline
\textbf{传播常量性:} & \multicolumn{1}{l|}{只有在右值范围内}    & 只有在右值范围内    \\ \hline
\end{longtable}

类模板std::ranges::keys\_view<>定义了一个视图,该视图从传递的范围的元素中选择第一个成员/属性/元素。它只是使用索引为0的元素视图的快捷方式,为每个元素调用get<0>(elem)。

类模板std::ranges::values\_view<>定义了一个视图,该视图从传递的范围的元素中选择第二个成员/属性/元素。它只是使用索引为1的元素视图的快捷方式,为每个元素调用get<1>(elem)。

这些视图可以这样使用:

\begin{itemize}
\item
获取std::pair元素的第一/第二成员,这对于选择map、unordered\_map、multimap和unordered\_multimap的元素的键/值特别有用

\item
获取std::元组元素的第一/第二个成员

\item
获取std::variant元素的第一/第二个的选项
\end{itemize}

但对于最后两个应用程序,直接使用索引为0的elements\_view元素可能更具可读性。

注意,类模板参数推导对这些视图还不起作用出于这个原因,必须显式地指定模板参数。例如:

\begin{cpp}
std::map<std::string, int> rg{
	{"Bach", 1685}, {"Mozart", 1756}, {"Beethoven", 1770},
	{"Tchaikovsky", 1840}, {"Chopin", 1810}, {"Vivaldi", 1678},
};

for (const auto& e : std::ranges::keys_view<decltype(std::views::all(rg))>{rg}) {
	std::cout << e << ' ';
}
\end{cpp}

循环输出:

\begin{shell}
Bach Beethoven Chopin Mozart Tchaikovsky Vivaldi
\end{shell}

\mySamllsection{键/值视图的范围适配器}

键和值视图也可以(通常应该)使用范围适配器创建:适配器使视图的使用更容易,不必指定底层范围的类型:

\begin{cpp}
std::views::keys(rg)
std::views::values(rg)
\end{cpp}

例如:

\begin{cpp}
for (const auto& elem : rg | std::views::keys) {
	std::cout << elem << ' ';
}
\end{cpp}

或:

\begin{cpp}
for (const auto& elem : rg | std::views::values) {
	std::cout << elem << ' ';
}
\end{cpp}

所有其他方面都与元素视图相匹配。

下面是一个使用键和值视图的完整示例:

\filename{ranges/keysvaluesview.cpp}

\begin{cpp}
#include <iostream>
#include <string>
#include <unordered_map>
#include <ranges>
#include <numbers> // for math constants
#include <algorithm> // for sort()

void print(std::ranges::input_range auto&& coll)
{
	for (const auto& elem : coll) {
		std::cout << elem << ' ';
	}
	std::cout << '\n';
}

int main()
{
	std::unordered_map<std::string, double> coll{
		{"pi", std::numbers::pi},
		{"e", std::numbers::e},
		{"golden-ratio", std::numbers::egamma},
		{"euler-constant", std::numbers::phi},
	};
	
	print(std::ranges::keys_view<decltype(std::views::all(coll))>{coll});
	print(std::ranges::values_view<decltype(std::views::all(coll))>{coll});
	
	print(coll | std::views::keys);
	print(coll | std::views::values);
}
\end{cpp}

该程序有以下输出:

\begin{shell}
euler-constant golden-ratio e pi
1.61803 0.577216 2.71828 3.14159
euler-constant golden-ratio e pi
1.61803 0.577216 2.71828 3.14159
\end{shell}










