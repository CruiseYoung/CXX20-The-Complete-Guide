
详细讨论如何使用chrono库之前,先了解一下所有的基本类型和符号。

\mySubsubsection{11.3.1}{时间段类型}

C++20引入了更多的时间段类型,包括天、周、月和年。表“C++20的标准时间段类型”列出了C++现在提供的所有时间段类型,以及可用于创建值的标准字面符后缀和打印值时使用的默认输出后缀。

使用std::chrono::months和std::chrono::years时要小心。月和年表示一个月或一年的平均时间段,这是一个小数天。平均年的时间段是通过考虑闰年来计算的:

\begin{itemize}
\item 
每四年多一天(366天,而非365天)

\item 
每隔100年,就会多一天

\item 
每隔400年,又会多一天
\end{itemize}

因此,其值为400 * 365 + 100−4 + 1除以400,平均一个月的时间是它的十二分之一。

% Please add the following required packages to your document preamble:
% \usepackage{longtable}
% Note: It may be necessary to compile the document several times to get a multi-page table to line up properly
\begin{longtable}[c]{|l|l|l|l|}
\hline
\textbf{类型}         & \textbf{定义为}     & \textbf{字面} & \textbf{输出后缀}   \\ \hline
\endfirsthead
%
\endhead
%
\textbf{nanoseconds}  & 1/1,000,000,000 seconds & \textbf{ns}      & \textbf{ns}              \\ \hline
\textbf{microseconds} & 1,000 nanoseconds       & \textbf{us}      & \textbf{μs or us}        \\ \hline
\textbf{milliseconds} & 1,000 microseconds      & \textbf{ms}      & \textbf{ms}              \\ \hline
\textbf{seconds}      & 1,000 milliseconds      & \textbf{s}       & \textbf{s}               \\ \hline
\textbf{minutes}      & 60 seconds              & \textbf{min}     & \textbf{min}             \\ \hline
\textbf{hours}        & 60 minutes              & \textbf{h}       & \textbf{h}               \\ \hline
\textbf{days}         & 24 hours                & \textbf{d}       & \textbf{d}               \\ \hline
\textbf{weeks}        & 7 days                  & \textbf{}        & \textbf{{[}604800{]}s}   \\ \hline
\textbf{months}       & 30.436875 days          & \textbf{}        & \textbf{{[}2629746{]}s}  \\ \hline
\textbf{years}        & 365.2425 days           & \textbf{y}       & \textbf{{[}31556952{]}s} \\ \hline
\end{longtable}

\begin{center}
表11.1 C++20的标准时间段类型
\end{center}

\mySubsubsection{11.3.2}{时钟}

C++20引入了两个新的时钟(记住,时钟定义了时间点的origin/epoch)。

表“C++20以后的标准时钟类型”,描述了C++标准库现在提供的所有时钟的名称和含义。

% Please add the following required packages to your document preamble:
% \usepackage{longtable}
% Note: It may be necessary to compile the document several times to get a multi-page table to line up properly
\begin{longtable}[c]{|l|l|l|}
\hline
\textbf{类型}                    & \textbf{含义}                           & \textbf{Epoch} \\ \hline
\endfirsthead
%
\endhead
%
\textbf{system\_clock} & 与系统时钟相关联(C++11起) & UTC    \\ \hline
\textbf{utc\_clock}              & 用于UTC时间的时钟                  & UTC        \\ \hline
\textbf{gps\_clock}              & GPS的时钟                       & GPS        \\ \hline
\textbf{tai\_clock}              & 用于国际原子时值的时钟 & TAI        \\ \hline
\textbf{file\_clock}   & 文件系统库时间点的时钟       & 由实现指定 \\ \hline
\textbf{local\_t}                & 本地时间点的伪时钟          & 无           \\ \hline
\textbf{steady\_clock}           & 用于测量的时钟(C++11起)        & 由实现指定     \\ \hline
\textbf{high\_resolution\_clock} & (见文本内容)                                 &                \\ \hline
\end{longtable}

\begin{center}
表11.2 C++20后的标准时钟类型
\end{center}

Epoch列指定时钟是否指定唯一的时间点,以便始终定义特定的UTC时间。这需要一个稳定的指定纪元。对于file\_clock, epoch是特定于系统的,但在程序的多次运行中是稳定的。steady\_clock的epoch可能会随着应用程序的每次运行而改变(例如,当系统重新启动时)。对于伪时钟local\_t,,epoch解释为“本地时间”,所以必须将它与时区结合起来才能知道其代表的是哪个时间点。

C++标准也从C++11开始提供了一个高分辨率的时钟,但在将代码从一个平台移植到另一个平台时,使用它可能会带来一些微妙的问题。实践中,high\_resolution\_clock可能是system\_clock或steady\_clock的别名,所以时钟有时稳定,有时不稳定,并且这个时钟可能支持或不支持转换到其他时钟或time\_t。

因为在任何平台上,high\_resolution\_clock不比steady\_clock好,所以应该使用steady\_clock。

稍后将讨论时钟的以下细节:

\begin{itemize}
\item 
时钟之间的差异

\item 
时钟之间的转换

\item 
如何处理闰秒
\end{itemize}

\mySubsubsection{11.3.3}{时间点类型}

C++20引入了两个新的时间点类型,基于C++11以来的通用定义:

\begin{cpp}
template<typename Clock, typename Duration = typename Clock::duration>
class time_point;
\end{cpp}

新的类型提供了一种更方便的方式来使用不同时钟的时间点。

表“C++20后的标准时间点类型”描述了时间点类型的名称和含义。

% Please add the following required packages to your document preamble:
% \usepackage{longtable}
% Note: It may be necessary to compile the document several times to get a multi-page table to line up properly
\begin{longtable}[c]{|l|l|l|}
\hline
\textbf{类型}                                  & \textbf{含义}            & \textbf{定义为}                                         \\ \hline
\endfirsthead
%
\endhead
%
\textbf{local\_time\textless{}Dur\textgreater{}} & 本地时间点      & time\_point\textless{}LocalTime, Dur\textgreater{}      \\ \hline
\textbf{local\_seconds}                        & 本地时间点,单位为秒  & time\_point\textless{}LocalTime, seconds\textgreater{}      \\ \hline
\textbf{local\_days}                           & 本地时间点,单位为天     & time\_point\textless{}LocalTime, days\textgreater{}         \\ \hline
\textbf{sys\_time\textless{}Dur\textgreater{}}   & 系统时间点     & time\_point\textless{}systime\_clock, Dur\textgreater{} \\ \hline
\textbf{sys\_seconds}                          & 系统时间点,单位为秒 & time\_point\textless{}systime\_clock, seconds\textgreater{} \\ \hline
\textbf{sys\_days}                             & 系统时间点,单位为天    & time\_point\textless{}systime\_clodk, days\textgreater{}    \\ \hline
\textbf{utc\_time\textless{}Dur\textgreater{}} & UTC时间点               & time\_point\textless{}utc\_clock, Dur\textgreater{}         \\ \hline
\textbf{utc\_seconds}                          & UTC时间点,单位为秒    & time\_point\textless{}utc\_clock, seconds\textgreater{}     \\ \hline
\textbf{tai\_time\textless{}Dur\textgreater{}} & TAI时间点               & time\_point\textless{}tai\_clock, Dur\textgreater{}         \\ \hline
\textbf{tai\_seconds}                          & TAI时间点,单位为秒    & time\_point\textless{}tai\_clock, seconds\textgreater{}     \\ \hline
\textbf{gps\_time\textless{}Dur\textgreater{}} & GPS时间点               & time\_point\textless{}gps\_clock, Dur\textgreater{}         \\ \hline
\textbf{gps\_seconds}                          & GPS时间点,单位为秒    & time\_point\textless{}gps\_clock, seconds\textgreater{}     \\ \hline
\textbf{file\_time\textless{}Dur\textgreater{}}  & 文件系统时间点 & time\_point\textless{}file\_clock, Dur\textgreater{}    \\ \hline
\end{longtable}

\begin{center}
表11.3 C++20后的标准时间点类型
\end{center}

对于每个标准时钟(稳定时钟除外),都有一个相应的\_time<>类型,允许声明表示时钟日期或时间点的对象,所以…\_seconds类型允许以秒为粒度定义相应类型的日期/时间对象。对于系统时间和当地时间,…\_days类型允许以天为粒度定义相应类型的日期/时间对象。例如:

\begin{cpp}
std::chrono::sys_days x; // time_point<system_clock, days>
std::chrono::local_seconds y; // time_point<local_t, seconds>
std::chrono::file_time<std::chrono::seconds> z; // time_point<file_clock, seconds>
\end{cpp}

注意,此类型的对象仍然表示一个时间点,尽管不幸的是该类型有一个复数名称。例如,sys\_days表示定义为系统时间点的一天(名称来自“粒度天的系统时间点”)。

\mySubsubsection{11.3.4}{日历类型}

作为时间点类型的扩展,C++20在chrono库中引入了民用(公历)日历类型。

虽然时间点指定为从epoch开始的时间段,但日历类型具有不同的组合类型和值,适用于年、月、工作日和一个月中的天数。这两种用法都很有用:

\begin{itemize}
\item 
时间点类型很好,只要我们只计算秒、小时和天(比如在一年中的每一天做一些事情)。

\item 
日历类型对于日期计算非常有用,因为要考虑到月和年有不同的天数。此外,还可以处理诸如“本月的第三个星期一”或“本月的最后一天”之类的事情。
\end{itemize}

表“标准日历类型”列出了新的日历类型及其默认输出格式。

% Please add the following required packages to your document preamble:
% \usepackage{longtable}
% Note: It may be necessary to compile the document several times to get a multi-page table to line up properly
\begin{longtable}[c]{|l|l|l|}
\hline
\textbf{类型}                   & \textbf{含义}                       & \textbf{输出格式} \\ \hline
\endfirsthead
%
\endhead
%
\textbf{day}                    & 日                                    & 05                     \\ \hline
\textbf{month}                  & 月                                  & Feb                    \\ \hline
\textbf{year}                   & 年                                   & 1999                   \\ \hline
\textbf{weekday}                & 工作日                                & Mon                    \\ \hline
\textbf{weekday\_indexed}       & 第n个工作日                            & Mon{[}2{]}             \\ \hline
\textbf{weekday\_last}          & 最后一个工作日                         & Mon{[}last{]}          \\ \hline
\textbf{month\_day}             & 一个月中的一天                         & Feb/05                 \\ \hline
\textbf{month\_day\_last}       & 一个月的最后一天                    & Feb/last               \\ \hline
\textbf{month\_weekday}         & 一个月的第n个工作日                 & Feb/Mon{[}2{]}         \\ \hline
\textbf{month\_weekday\_last}   & 一个月最后一个工作日                & Feb/Mon{[}last{]}      \\ \hline
\textbf{year\_month}            & 一年中的月份                        & 1999/Feb               \\ \hline
\textbf{year\_month\_day}       & 一个完整的日期(一年中的一个月的一天) & 1999-02-05             \\ \hline
\textbf{year\_month\_day\_last} & 一年中一个月的最后一天          & 1999/Feb/last          \\ \hline
\textbf{year\_month\_weekday}       & 一年中一个月的第n个工作日  & 1999/Feb/Mon{[}2{]}    \\ \hline
\textbf{year\_month\_weekday\_last} & 一年中一个月的最后一个工作日 & 1999/Feb/Mon{[}last{]} \\ \hline
\end{longtable}

\begin{center}
表11.4 标准日历类型
\end{center}

请记住,这些类型名称并不传递日期元素进行初始化或写入格式化输出的顺序。例如,类型std::chrono::year\_month\_day可以这样使用:

\begin{cpp}
using namespace std::chrono;

year_month_day d = January/31/2021; // January 31, 2021
std::cout << std::format("{:%D}", d); // writes 01/31/21
\end{cpp}

像year\_month\_day这样的日历类型可精确地计算下个月同一天的日期:

\begin{cpp}
std::chrono::year_month_day start = ... ; // 2021/2/5
auto end = start + std::chrono::months{1}; // 2021/3/5
\end{cpp}

若使用像sys\_days这样的时间点,相应的代码将无法工作,因为它使用了一个月的平均时间段:

\begin{cpp}
std::chrono::sys_days start = ... ; // 2021/2/5
auto end = start + std::chrono::months{1}; // 2021/3/7 10:29:06
\end{cpp}

请注意,在本例中,end具有不同的类型,因为对于月份,小数天数也会起作用。

当添加4周或28天时,时间点类型更好,这是一个简单的算术运算,不必考虑月或年的不同长度:

\begin{cpp}
std::chrono::sys_days start = ... ; // 2021/1/5
auto end = start + std::chrono::weeks{4}; // 2021/2/2
\end{cpp}

后面将讨论使用月份和年份的细节。

有特定的类型用于处理工作日以及一个月内的第n个和最后一个工作日,可以迭代所有第二个星期一或跳转到下个月的最后一天:

\begin{cpp}
std::chrono::year_month_day_last start = ... ; // 2021/2/28
auto end = start + std::chrono::months{1}; // 2021/3/31
\end{cpp}

每种类型都有一个默认的输出格式,是一个固定的英文字符序列。对于其他格式,请使用格式化的chrono输出。注意,在默认输出格式中,只有year\_month\_day使用减号作为分隔符。默认情况下,所有其他类型的都用斜杠分隔。

为了处理日历类型的值,定义了两个日历常量:

\begin{itemize}
\item 
std::chrono::last指定一个月的最后一天/最后一个工作日,该常量的类型为std::chrono::last\_spec。

\item 
std::chrono::Sunday,std::chrono::Monday,…,std::chrono::Saturday指定工作日(Sunday的值为0,Saturday的值为6)。

这些常量的类型为std::chrono::weekday。

\item 
std::chrono::January、std::chrono::February…、std::chrono::December指定月份(January的值为1,December的值为12)。

这些常量的类型为std::chrono::month。
\end{itemize}

\mySamllsection{日历类型的限制操作}

日历类型的设计目的是在编译时检测操作是否有用或性能不佳,所以一些“显而易见”的操作不会按照表中列出的日历类型的标准操作进行编译:

\begin{itemize}
\item 
日和月的计算取决于年份。不能添加天数/月份,也不能计算没有年份的所有月份类型的差值。

\item 
完全指定日期的日运算需要一些时间来处理不同长度的月份和闰年,可以添加/减去天数或计算这些类型之间的差异。

\item 
因为chrono不对一周的第一天做任何假设,所以无法在工作日之间获得顺序。当与工作日比较类型时,只支持==和!=。
\end{itemize}

% Please add the following required packages to your document preamble:
% \usepackage{longtable}
% Note: It may be necessary to compile the document several times to get a multi-page table to line up properly
\begin{longtable}[c]{|l|l|l|l|l|l|}
\hline
\textbf{类型} & \textbf{++/--} & \textbf{增/减} & \textbf{-(差.)} & \textbf{==} & \textbf{\textless{}/\textless{}=\textgreater{}} \\ \hline
\endfirsthead
%
\endhead
%
\textbf{day}                        & yes & days         & yes & yes & yes \\ \hline
\textbf{month}                      & yes & months,years & yes & yes & yes \\ \hline
\textbf{year}                       & yes & years        & yes & yes & yes \\ \hline
\textbf{weekday}                    & yes & days         & yes & yes & -   \\ \hline
\textbf{weekday\_indexed}           & -   & -            & -   & yes & -   \\ \hline
\textbf{weekday\_last}              & -   & -            & -   & yes & -   \\ \hline
\textbf{month\_day}                 & -   & -            & -   & yes & yes \\ \hline
\textbf{month\_day\_last}           & -   & -            & -   & yes & yes \\ \hline
\textbf{month\_weekday}             & -   & -            & -   & yes & -   \\ \hline
\textbf{month\_weekday\_last}       & -   & -            & -   & yes & -   \\ \hline
\textbf{year\_month}                & -   & months,years & yes & yes & yes \\ \hline
\textbf{year\_month\_day}           & -   & months,years & -   & yes & yes \\ \hline
\textbf{year\_month\_day\_last}     & -   & months,years & -   & yes & yes \\ \hline
\textbf{year\_month\_weekday}       & -   & months,years & -   & yes & -   \\ \hline
\textbf{year\_month\_weekday\_last} & -   & months,years & -   & yes & -   \\ \hline
\end{longtable}

\begin{center}
表11.5 日历类型的标准操作
\end{center}

工作日的运算是对7取模,所以哪一天是一周的第一天并不重要。也可以计算任意两个工作日之间的差,结果总是介于0和6之间的值。将差额加到第一个工作日总是第二个工作日。例如:

\begin{cpp}
std::cout << chr::Friday - chr::Tuesday << '\n'; // 3d (Tuesday thru Friday)
std::cout << chr::Tuesday - chr::Friday << '\n'; // 4d (Friday thru Tuesday)

auto d1 = chr::February / 25 / 2021;
auto d2 = chr::March / 3 / 2021;
std::cout << chr::sys_days{d1} - chr::sys_days{d2} << '\n'; // -6d (date diff)
std::cout << chr::weekday(d1) - chr::weekday(d2) << '\n'; // 3d (weekday diff)
\end{cpp}

这样,就可以很容易地计算出“下周一”与“周一减去当前工作日”的差值:

\begin{cpp}
d1 = chr::sys_days{d1} + (chr::Monday - chr::weekday(d1)); // set d1 to next Monday
\end{cpp}

请注意,若d1是日历日期类型,首先必须将其转换为类型std::chrono::sys\_days,以便支持日期算术(使用该类型声明d1可能会更好)。

同样,月的运算以12为模,则12月之后的下一个月是1月。若年份是类型的一部分,则相应地进行调整:

\begin{cpp}
auto m = chr::December;
std::cout << m + chr::months{10} << '\n'; // Oct
std::cout << 2021y/m + chr::months{10} << '\n'; // 2022/Oct
\end{cpp}

还请注意,接受整数值的时间日历类型的构造函数是显式的,所以使用整数值的显式初始化失败:

\begin{cpp}
std::chrono::day d1{3}; // OK
std::chrono::day d2 = 3; // ERROR

d1 = 3; // ERROR
d1 = std::chrono::day{3}; // OK

passDay(3); // ERROR
passDay(std::chrono::day{3}); // OK
\end{cpp}

\mySubsubsection{11.3.5}{时间类型hh\_mm\_ss}

与日历类型一致,C++20引入了一个新的时间类型std::chrono::hh\_mm\_ss,将时间段转换为具有相应时间字段的数据结构。表“std::chrono::hh\_mm\_ss members”描述了hh\_mm\_ss成员的名称和含义。

% Please add the following required packages to your document preamble:
% \usepackage{longtable}
% Note: It may be necessary to compile the document several times to get a multi-page table to line up properly
\begin{longtable}[c]{|l|l|}
\hline
\textbf{成员函数}               & \textbf{含义}                                       \\ \hline
\endfirsthead
%
\endhead
%
\textbf{hours()}              & 小时数                                            \\ \hline
\textbf{minutes()}            & 分钟数                                          \\ \hline
\textbf{seconds()}            & 秒数                                          \\ \hline
\textbf{subseconds()}         & 具有适当粒度的部分秒数 \\ \hline
\textbf{is\_negative()}       & 若值为负值,则为true                          \\ \hline
\textbf{to\_duration()}       & 转换(返回)到时间段                         \\ \hline
\textbf{precision()}          & 亚秒的时间段类型                        \\ \hline
\textbf{operator precision()} & 转换为具有相应精度的值       \\ \hline
\textbf{fractional\_width}    & 亚秒精度(静态成员)            \\ \hline
\end{longtable}

\begin{center}
表11.6 std::chrono::hh\_mm\_ss members
\end{center}

这种类型对于处理时间段和时间点的不同属性非常有用,允许将时间段拆分为其属性,并作为格式化辅助。例如,可以检查特定的小时或将小时和分钟作为整数值传递给另一个函数:

\begin{cpp}
auto dur = measure(); // process and yield some duration
std::chrono::hh_mm_ss hms{dur}; // convert to data structure for attributes
process(hms.hours(), hms.minutes()); // pass hours and minutes
\end{cpp}

若有一个时间点,必须首先将其转换为持续时间。要做到这一点,通常只需计算时间点和当天午夜之间的差值(通过将其舍入到天数粒度来计算)。例如:

\begin{cpp}
auto tp = getStartTime(); // process and yield some timepoint
// convert time to data structure for attributes:
std::chrono::hh_mm_ss hms{tp - std::chrono::floor<std::chrono:days>(tp)};
process(hms.hours(), hms.minutes()); // pass hours and minutes
\end{cpp}

另一个例子,可以使用hh\_mm\_ss以不同的形式打印时间段的属性:

\begin{cpp}
auto t0 = std::chrono::system_clock::now();
...
auto t1 = std::chrono::system_clock::now();
std::chrono::hh_mm_ss hms{t1 - t0};
std::cout << "minutes: " << hms.hours() + hms.minutes() << '\n';
std::cout << "seconds: " << hms.seconds() << '\n';
std::cout << "subsecs: " << hms.subseconds() << '\n';
\end{cpp}

可能会输出:

\begin{shell}
minutes: 63min
seconds: 19s
subsecs: 502998000ns
\end{shell}

没有办法用特定的值直接初始化hh\_mm\_ss对象的不同属性。一般来说,应该使用duration类型来处理时间:

\begin{cpp}
using namespace std::literals;
...
auto t1 = 18h + 30min; // 18 hours and 30 minutes
\end{cpp}

hh\_mm\_ss最强大的功能是,其接受任何精度时间段,并将其转换为通常的属性和持续时间类型小时、分钟和秒。此外,subseconds()将以适当的持续时间类型(如毫秒或纳秒)生成值的其余部分。即使单位不是10的幂(例如,三分之一秒),hh\_mm\_ss将其转换为最多18位的10次方次秒。若不能表示准确的值,则使用六位数的精度,可以使用标准输出操作符将结果作为一个整体输出。例如:

\begin{cpp}
std::chrono::duration<int, std::ratio<1,3>> third{1};
auto manysecs = 10000s;
auto dblsecs = 10000.0s;

std::cout << "third: " << third << '\n';
std::cout << " " << std::chrono::hh_mm_ss{third} << '\n';
std::cout << "manysecs: " << manysecs << '\n';
std::cout << " " << std::chrono::hh_mm_ss{manysecs} << '\n';
std::cout << "dblsecs: " << dblsecs << '\n';
std::cout << " " << std::chrono::hh_mm_ss{dblsecs} << '\n';
\end{cpp}

这段代码有以下输出:

\begin{shell}
third:    1[1/3]s
          00:00:00.333333
manysecs: 10000s
          02:46:40
dblsecs:  10000.000000s
          02:46:40.000000
\end{shell}

要输出特定属性,还可以使用带有特定转换说明符的格式化输出:

\begin{cpp}
auto manysecs = 10000s;
std::cout << "manysecs: " << std::format("{:%T}", manysecs) << '\n';
\end{cpp}

输出为:

\begin{shell}
manysecs: 02:46:40
\end{shell}

\mySubsubsection{11.3.6}{时间工具}

chrono库现在还提供了一些辅助函数来处理12小时和24小时格式。表“时间工具”列出了这些函数。

% Please add the following required packages to your document preamble:
% \usepackage{longtable}
% Note: It may be necessary to compile the document several times to get a multi-page table to line up properly
\begin{longtable}[c]{|l|l|}
\hline
std::chrono::is\_am(h)       & h是否为0到11之间的小时值  \\ \hline
\endfirsthead
%
\endhead
%
std::chrono::is\_pm(h)       & h是否为12到23之间的小时值 \\ \hline
std::chrono::make12(h)       & 生成小时值h的12小时等价值   \\ \hline
std::chrono::make24(h, toPM) & 生成小时值h的24小时等价值   \\ \hline
\end{longtable}

\begin{center}
表11.7 时间工具
\end{center}

std::chrono::make12()和std::chrono::make24()都要求小时值在0到23之间。std::chrono::make24()的第二个参数指定是否将传递的小时值解释为PM值。若为true,则传递的值在0到11之间,则该函数将12小时添加到该值中。

例如:

\begin{cpp}
for (int hourValue : {9, 17}) {
	std::chrono::hours h{hourValue};
	if (std::chrono::is_am(h)) {
		h = std::chrono::make24(h, true); // assume a PM hour is meant
	}
	std::cout << "Tea at " << std::chrono::make12(h).count() << "pm" << '\n';
}
\end{cpp}

代码有以下输出:

\begin{shell}
Tea at 9pm
Tea at 5pm
\end{shell}










