
范围库在命名空间std::ranges和std中提供了几个通用的辅助函数。

其中一些在C++20之前就已经存在了,在命名空间std中具有相同的名称或略有不同的名称(并且仍然提供向后兼容性)。但范围工具通常为特定的功能提供更好的支持,可能会修复旧版本的缺陷,或者使用一些概念来限制旧版本的使用。

这些“函数”需要是函数对象或函子,甚至是定制点对象(CPO是要求是半常规的,并保证所有实例相等的函数对象)。

它们不支持依赖参数的查找(ADL),所以只调用函数对象不正确。但因为它们像函数一样使用,所以我仍然推荐于这样做,其余的就是实现细节了。]

出于这个原因,应该选择std::ranges中的工具,而不是std中的工具。

\mySubsubsection{7.4.1}{处理范围(和数组)元素的函数}

“处理范围元素的泛型函数”表列出了处理范围及其元素的新的独立函数,它们也适用于原始数组。

C++20之前,几乎所有相应的工具函数都可以直接在命名空间std中使用。唯一的例外是:

\begin{itemize}
\item
ssize(),C++20中引入了std::ssize()

\item
cdata(),不存在于命名空间std中
\end{itemize}

读者们可能想知道为什么命名空间std::ranges中会有新函数,而不是修复命名空间std中的现有函数。好吧,问题是若修复现有的函数,可能会破坏现有的代码,而向后兼容是C++的一个重要目标。

下一个问题是什么时候使用哪个函数。这里的指导原则非常简单:优先使用命名空间std::ranges中的函数/工具,而不是命名空间std中的函数/工具。

原因不仅在于命名空间std::ranges中的函数/工具使用了概念,这有助于在编译时发现问题和bug;更推荐命名空间std::ranges中的新函数的另一个原因是,命名空间std::ranges中的函数有时存在缺陷,std::ranges中的新实现实则修复了这些缺陷:

\begin{itemize}
\item
一个问题是参数相关查找(ADL)。

\item
另一个问题是const的正确性。
\end{itemize}

\begin{longtable}[c]{|l|l|}
\hline
\textbf{函数} &
\textbf{意义} \\ \hline
\endfirsthead
%
\endhead
%
\begin{tabular}[c]{@{}l@{}}std::ranges::empty(rg)\\ std::ranges::size(rg)\\ std::ranges::ssize(rg)\\ std::ranges::begin(rg)\\ std::ranges::end(rg)\\ std::ranges::cbegin(rg)\\ std::ranges::cend(rg)\\ std::ranges::rbegin(rg)\\ std::ranges::rend(rg)\\ std::ranges::crbegin(rg)\\ std::ranges::crend(rg)\\ std::ranges::data(rg)\\ std::ranges::cdata(rg)\end{tabular} &
\begin{tabular}[c]{@{}l@{}}生成的范围是否为空\\ 生成范围的大小\\ 将范围的大小作为signed类型的值\\ 生成指向该范围的第一个元素的迭代器\\ 生成范围哨兵(一个迭代器)\\ 生成指向范围第一个元素的常量迭代器\\ 生成范围的一个常量哨兵(一个到末尾的常量迭代器)\\ 生成范围内第一个元素的反向迭代器\\ 生成范围的反向前哨(一个迭代器到末尾)\\ 生成指向范围第一个元素的反向常量迭代器\\ 生成范围的反向常量哨兵(一个到末尾的常量迭代器)\\ 生成该范围的原始数据\\ 生成具有const元素的范围的原始数据\end{tabular} \\ \hline
\end{longtable}

\begin{center}
表7.3 用于处理范围元素的泛型函数
\end{center}

\mySamllsection{std::ranges::begin()如何解决ADL问题}

来看看为什么在处理泛型代码中的范围时,使用std::ranges::begin()比使用std::begin()更好。问题是,若不巧妙地使用std::begin(),依赖于参数的查找并不总是有效。

假设要编写泛型代码,调用为范围对象obj定义的begin()函数。std::begin()的问题如下:

\begin{itemize}
\item
要支持具有begin()成员函数的容器等范围类型,可以始终调用成员函数begin():

\begin{cpp}
obj.begin(); // only works if a member function begin() is provided
\end{cpp}

但对于只有独立begin()的类型(例如原始数组),这不起作用。

\item
但若使用独立的begin(),则有一个问题:

\begin{itemize}
\item
用于原始数组的标准独立std::begin()需要使用std::进行限定。

\item
其他非标准范围不能处理完整的限定条件std::。例如:

\begin{cpp}
class MyColl {
	...
};
... begin(MyColl); // declare free-standing begin() for MyColl

MyColl obj;
std::begin(obj); // std::begin() does not find ::begin(MyType)
\end{cpp}
\end{itemize}

\item
必要的解决方法是在begin()前添加一个额外的using声明,并且不限定调用本身:

\begin{cpp}
using std::begin;
begin(obj); // OK, works in all these cases
\end{cpp}
\end{itemize}

std::ranges::begin()则没有这个问题:

\begin{cpp}
std::ranges::begin(obj); // OK, works in all these cases
\end{cpp}

诀窍在于begin不是一个使用依赖参数查找的函数,而是一个实现所有可能查找的函数对象。参见lib/begin.cpp获得完整的示例。

这适用于std::ranges中定义的所有工具,所以使用std::ranges工具的代码通常支持更多类型和更复杂的用例。

\mySubsubsection{7.4.2}{用于处理迭代器的函数}

“处理迭代器的泛型函数”表列出了所有用于移动迭代器、向前或向后查找以及计算迭代器之间距离的泛型函数。注意,下一小节描述了交换和移动迭代器引用的元素的函数。

\begin{longtable}[c]{|l|l|}
\hline
\textbf{函数} &
\textbf{意义} \\ \hline
\endfirsthead
%
\endhead
%
\begin{tabular}[c]{@{}l@{}}std::ranges::distance(from, to)\\ std::ranges::distance(rg)\\ std::ranges::next(pos)\\ std::ranges::next(pos, n)\\ std::ranges::next(pos, to)\\ std::ranges::next(pos, n, maxpos)\\ std::ranges::prev(pos)\\ std::ranges::prev(pos, n)\\ std::ranges::prev(pos, n, minpos)\\ std::ranges::advance(pos, n)\\ std::ranges::advance(pos, to)\\ std::ranges::advance(pos, n, maxpos)\end{tabular} &
\begin{tabular}[c]{@{}l@{}}生成from和to之间的距离(元素数量)\\ 生成rg中元素的数量(对于没有size()的范围也是如此)\\ 生成pos后面的下一个元素的位置\\ 生成pos后第n个元素的位置\\ 生成pos后的位置\\ 返回第n个元素在pos之后但不在maxpos之后的位置\\ 生成元素在pos之前的位置\\ 生成第n个元素在pos之前的位置\\ 返回第n个元素在pos之前的位置,但不在minpos之前\\ 向前/向后推进n个元素\\ 将pos向前推进到\\ 将pos向前/向后移动n个元素,但不超过maxpos\end{tabular} \\ \hline
\end{longtable}

\begin{center}
表7.4 处理迭代器的泛型函数
\end{center}

同样,比起命名空间std中对应的传统工具,更推荐喜欢这些工具。例如,与std::ranges::next()相比,旧的工具std::next()要求为传递的迭代器It提供std::iterator\_traits<It>::difference\_type。但某些视图类型的内部迭代器不支持迭代器特性,因此若使用std::next(),代码可能无法编译。

\mySubsubsection{7.4.3}{用于交换和移动元素/值的函数}

范围库还提供了交换和移动值的函数。这些函数列在表中,用于交换和移动元素/值的通用函数,其不能只用于范围。

\begin{longtable}[c]{|l|l|}
\hline
\textbf{函数} &
\textbf{意义} \\ \hline
\endfirsthead
%
\endhead
%
\begin{tabular}[c]{@{}l@{}}std::ranges::swap(val1, val2)\\ std::ranges::iter\_swap(pos1, pos2)\\ std::ranges::iter\_move(pos)\end{tabular} &
\begin{tabular}[c]{@{}l@{}}交换值val1和val2(使用移动语义)\\ 交换迭代器pos1和pos2所引用的值(使用移动语义)\\ 产生迭代器pos在移动时所引用的值\end{tabular} \\ \hline
\end{longtable}

\begin{center}
表7.5 用于交换和移动元素/值的通用函数
\end{center}

因为标准概念std::swappable和std::swappable\_with使用了它,所以函数std::ranges::swap()在<concepts>中定义(std::swap()在<utility>中定义)。函数std::ranges::iter\_swap()和std::ranges::iter\_move()在<iterator>中定义。

std::ranges::swap()修复了在泛型代码中,std::swap()可能找不到为某些类型提供的最佳交换函数的问题(类似于begin()和cbegin()所存在的问题)。

但由于存在泛型回退,所以此修复不会修复无法编译的代码,而可能只会有更好的性能。

考虑下面的例子:

\filename{lib/swap.cpp}

\begin{cpp}
#include <iostream>
#include <utility> // for std::swap()
#include <concepts> // for std::ranges::swap()

struct Foo {
	Foo() = default;
	Foo(const Foo&) {
		std::cout << " COPY constructor\n";
	}
	Foo& operator=(const Foo&) {
		std::cout << " COPY assignment\n";
		return *this;
	}
	void swap(Foo&) {
		std::cout << " efficient swap()\n"; // swaps pointers, no data
	}
};

void swap(Foo& a, Foo& b) {
	a.swap(b);
}

int main()
{
	Foo a, b;
	std::cout << "--- std::swap()\n";
	std::swap(a, b); // generic swap called
	
	std::cout << "--- swap() after using std::swap\n";
	using std::swap;
	swap(a, b); // efficient swap called
	
	std::cout << "--- std::ranges::swap()\n";
	std::ranges::swap(a, b); // efficient swap called
}
\end{cpp}

该程序有以下输出:

\begin{shell}
--- std::swap()
COPY constructor
COPY assignment
COPY assignment
--- swap() after using std::swap
efficient swap()
--- std::ranges::swap()
efficient swap()
\end{shell}

注意,当swap函数在命名空间std中定义为未在std中定义的类型时(这在形式上是不允许的),使用std::ranges::swap()会适得其反:

\begin{cpp}
class Foo {
	...
};

namespace std {
	void swap(Foo& a, Foo& b) { // not found by std::ranges::swap()
		...
	}
}
\end{cpp}

应该更改此代码。最好的选择是使用“隐藏友元”:

\begin{cpp}
class Foo {
	...
	friend void swap(Foo& a, Foo& b) { // found by std::ranges::swap()
		...
	}
};
\end{cpp}

函数std::ranges::iter\_move()和std::ranges::iter\_swap()与解引用迭代器和调用std::move()或swap()相比有一个好处:它们与代理迭代器一起工作得很好,所以要在泛型代码中支持代理迭代器:

\begin{cpp}
auto val = std::ranges::iter_move(it);
std::ranges::iter_swap(it1, it2);
\end{cpp}

而不是:

\begin{cpp}
auto val = std::move(*it);
using std::swap;
swap(*it1, *it2);
\end{cpp}

\mySubsubsection{7.4.4}{用于值比较的函数}

“用于比较标准的通用函数”表列出了现在可以用作比较标准的所有范围工具,其在<functional>中定义。


\begin{longtable}[c]{|l|l|}
\hline
\textbf{函数} &
\textbf{意义} \\ \hline
\endfirsthead
%
\endhead
%
\begin{tabular}[c]{@{}l@{}}std::ranges::equal\_to(val1, val2)\\ std::ranges::not\_equal\_to(val1, val2)\\ std::ranges::less(val1, val2)\\ std::ranges::greater(val1, val2)\\ std::ranges::less\_equal(val1, val2)\\ std::ranges::greater\_equal(val1, val2)\end{tabular} &
\begin{tabular}[c]{@{}l@{}}生成的val1是否等于val2\\ 生成的val1是否不等于val2\\ 生成的val1是否小于val2\\生成的val1是否大于val2\\ 生成的val1是否小于或等于val2\\ 生成的val1是否大于或等于val2\end{tabular} \\ \hline
\end{longtable}

\begin{center}
表7.6 用于比较标准的通用函数
\end{center}

检查相等性时,可以使用概念std::equality\_comparable\_with。

在检查顺序时,使用std::totally\_ordered\_with概念。注意,这并不要求必须支持<=>操作符,并且该类型不必支持全序。若可以用<、>、<=和>=比较这些值就足够了,但这是一个在编译时无法检查的语义约束。

注意,C++20还引入了函数对象类型std::compare\_three\_way,可以用它来使用新的操作符<=>。








