
泛型代码中,使用<=>会有一些挑战。可能存在提供操作符<=>的类型,也可能存在提供偏或全基本比较操作符的类型。

\mySubsubsection{1.5.1}{compare\_three\_way}

std::compare\_three\_way是用于调用<=>的函数对象类型,就像std::less是用于调用小于操作符的函数对象类型一样。

可以这样使用:

\begin{itemize}
\item
比较泛型类型的值

\item
必须指定函数对象的类型时作为默认类型
\end{itemize}

例如:

\begin{cpp}
template<typename T>
struct Value {
	T val{};
	...
	auto operator<=> (const Value& v) const noexcept(noexcept(val<=>val)) {
		return std::compare_three_way{}(val<=>v.val);
	}
};
\end{cpp}

使用std::compare\_three\_way(与std::less一样)的好处是,定义了原始指针的总序(操作符<=>或<不是这种情况),所以当使用是原始指针类型的泛型类型时可以使用。

允许开发者前向声明operator<=>(),C++20还引入了类型特性std::compare\_three\_way\_result和别名模板std::compare\_three\_way\_result\_t:

\begin{cpp}
template<typename T>
struct Value {
	T val{};
	...
	std::compare_three_way_result_t<T,T>
		operator<=> (const Value& v) const noexcept(noexcept(val<=>val));
};
\end{cpp}

\mySubsubsection{1.5.2}{算法lexicographical\_compare\_three\_way()}

为了能够比较两个范围并产生匹配比较类别的值,C++20还引入了lexicographical\_compare\_three\_way()算法。该算法对于为集合成员实现<=>操作符特别有帮助。

例如:

\filename{lib/lexicothreeway.cpp}

\begin{cpp}
#include <iostream>
#include <vector>
#include <algorithm>

std::ostream& operator<< (std::ostream& strm, std::strong_ordering val)
{
	if (val < 0) return strm << "less";
	if (val > 0) return strm << "greater";
	return strm << "equal";
}

int main()
{
	std::vector v1{0, 8, 15, 47, 11};
	std::vector v2{0, 15, 8};

	auto r1 = std::lexicographical_compare(v1.begin(), v1.end(),
										   v2.begin(), v2.end());

	auto r2 = std::lexicographical_compare_three_way(v1.begin(), v1.end(),
													 v2.begin(), v2.end());

	std::cout << "r1: " << r1 << '\n';
	std::cout << "r2: " << r2 << '\n';
}
\end{cpp}

该程序应有以下输出:

\begin{shell}
r1: 1
r2: less
\end{shell}

注意,lexicographical\_compare\_three\_way()还没有支持范围。既不能将范围作为单个参数传递,也不能传递投射形参:

\begin{cpp}
auto r3 = std::ranges::lexicographical_compare(v1, v2); // OK
auto r4 = std::ranges::lexicographical_compare_three_way(v1, v2); // ERROR
\end{cpp}




