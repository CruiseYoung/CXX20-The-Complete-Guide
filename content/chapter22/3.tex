
要禁用隐式类型转换,可以将构造函数声明为显式。但对于泛型代码,当且仅当类型参数具有显式构造函数时,可能希望显式地使用构造函数。

这样,就可以完美地将类型转换支持委托给包装器类型,指定显式条件的方法类似于指定条件noexcept。直接在explicit之后,就可以在括号中指定布尔编译时表达式。下面是一个例子:

\filename{lang/wrapper.hpp}

\begin{cpp}
#include <type_traits> // for std::is_convertible_v<>

template<typename T>
class Wrapper {
	T value;
public:
	template<typename U>
	explicit(!std::is_convertible_v<U, T>)
	Wrapper(const U& val)
	:value{val} {
	}
	...
};
\end{cpp}

保存T类型值的类模板Wrapper<>有一个泛型构造函数,则可以用可以隐式转换为T的类型初始化该值。若没有从U到T的隐式转换,则构造函数是显式的。

只有在启用了隐式类型转换的情况下,才能初始化类型的Wrapper。例如:

\filename{lang/explicitwrapper.cpp}

\begin{cpp}
#include "wrapper.hpp"
#include <string>
#include <vector>

void printStringWrapper(Wrapper<std::string>) {
}
void printVectorWrapper(Wrapper<std::vector<std::string>>) {
}

int main()
{
	// implicit conversion from string literal to string:
	std::string s1{"hello"};
	std::string s2 = "hello"; // OK
	Wrapper<std::string> ws1{"hello"};
	Wrapper<std::string> ws2 = "hello"; // OK
	printStringWrapper("hello"); // OK

	// NO implicit conversion from size to vector<string>:
	std::vector<std::string> v1{42u};
	std::vector<std::string> v2 = 42u; // ERROR: explicit
	Wrapper<std::vector<std::string>> wv1{42u};
	Wrapper<std::vector<std::string>> wv2 = 4u2; // ERROR: explicit
	printVectorWrapper(42u); // ERROR: explicit
}
\end{cpp}

为了演示条件显式的效果,对字符串和字符串的vector使用Wrapper<>:

\begin{itemize}
\item
对于std::string类型,构造函数支持从字符串字面值进行隐式转换,所以Wrapper<>类型的构造函数不是显式的,其作用是传递字符串字面值来复制初始化字符串包装器,或者将字符串字面值传递给接受字符串包装器的函数。[记住,复制初始化(initialization with =)和参数传递需要隐式转换。]

\item
对于std::vector<std::string>类型,接受无符号大小的构造函数在C++标准库中声明为显式,所以std::is\_convertible<>从大小转换为向量为false,Wrapper<>构造函数变为显式。也不能传递size来初始化string类型vector的包装器,也不能将size传递给接受该包装器类型的函数。
\end{itemize}

可以使用explicit的地方都可以使用条件显式,也可以使用它使转换操作符显式条件化:

\begin{cpp}
template<typename T>
class MyType {
	public:
	...
	explicit(!std::is_convertible_v<T, bool>) operator bool();
};
\end{cpp}

然而,到bool的转换是迄今为止最有用的转换操作符的应用,应该始终是显式的,这样就不会意外地将MyType对象传递给期望布尔值的函数。

\mySubsubsection{22.3.1}{标准库中的条件显式}

C++标准库在几个地方使用了条件显式。例如,std::pair<>和std::tuple<>用它来支持稍微不同类型的对和元组的赋值,前提是存在可用的隐式转换。

例如:

\begin{cpp}
std::pair<int, int> p1{11, 11};

std::pair<long, long> p2{};
p2 = p1; // OK: implicit conversion from int to long

std::pair<std::chrono::day, std::chrono::month> p3{};
p3 = p1; // ERROR
p3 = std::pair<std::chrono::day, std::chrono::month>{p1}; // OK
\end{cpp}

因为接受整数值的chrono日期类型的构造函数是显式的,所以将一对int赋值给一对day和month会失败,所以必须使用显式转换。

这也适用于在映射中插入元素时(因为元素是键/值对):

\begin{cpp}
std::map<std::string, std::string> coll1;
coll1.insert({"hi", "ho"}); // OK: uses implicit conversions to strings

std::map<std::string, std::chrono::month> coll2;
coll2.insert({"XI", 11}); // ERROR: no implicit conversion that fits
coll2.insert({"XI", std::chrono::month{11}}); // OK (inserts elem with string and month)
\end{cpp}

std::pair<>的这种行为并不新鲜。在C++20前,标准库的实现必须使用SFINAE来实现explicit的条件行为(声明两个构造函数,若不满足条件则禁用其中一个)。

另一个例子,std::span<>的构造函数是条件显式的:

\begin{cpp}
namespace std {
	template<typename ElementType, size_t Extent = dynamic_extent>
	class span {
	public:
		static constexpr size_type extent = Extent;
		...
		constexpr span() noexcept;
		template<typename It>
			constexpr explicit(extent != dynamic_extent)
			span(It first, size_type count);
		template<typename It, typename End>
			constexpr explicit(extent != dynamic_extent)
			span(It first, End last);
		...
	};
}
\end{cpp}

因此,只有当span具有动态范围时,才允许span的隐式类型转换。











