
比较不同类型的整数值比预期的要复杂。本节介绍C++20处理此问题的两个新特性:

\begin{itemize}
\item 
比较整型的工具

\item 
std::ssize()
\end{itemize}

\mySubsubsection{23.3.1}{整数值的安全比较}

比较和转换数字,即使是不同的数字类型,也应该是一项微不足道的任务。不幸的是,事实并非如此。几乎所有的开发者都看到过关于代码这样做的警告。这些警告有一个很好的理由:由于隐式转换,可能会编写不安全的代码而没有注意到它。

例如,大多数时候我们期望一个简单的x < y就可以了。考虑下面的例子:

\begin{cpp}
int x = -7;
unsigned y = 42;

if (x < y) ... // OOPS: false
\end{cpp}

这种行为的原因是,当根据规则(来自编程语言C)比较有符号值和无符号值时,有符号值被转换为无符号类型,这使其成为一个大的正整数值。

要修复此代码,必须编写以下代码:

\begin{cpp}
if (x < static_cast<int>(y)) ... // true
\end{cpp}

若将小整数值与大整数值进行比较,可能会出现其他问题。

C++20在<utility>中提供了安全的整数比较函数。“安全整数比较函数”表列出了这些函数,可以这样使用它们:

\begin{cpp}
if (std::cmp_less(x, y)) ... // true
\end{cpp}

这种整数比较总是安全的。

% Please add the following required packages to your document preamble:
% \usepackage{longtable}
% Note: It may be necessary to compile the document several times to get a multi-page table to line up properly
\begin{longtable}[c]{|l|l|}
\hline
\textbf{常数}          & \textbf{模板}                  \\ \hline
\endfirsthead
%
\endhead
%
std::cmp\_equal(x, y)      & 表示x是否等于y     \\ \hline
std::cmp\_not\_equal(x, y) & 表示x是否等于y \\ \hline
std::cmp\_less(x, y)       & 表示x是否小于y    \\ \hline
std::cmp\_less\_equal(x, y)                 & 表示x是否小于等于y    \\ \hline
std::cmp\_greater(x, y)    & 表示x是否大于y \\ \hline
std::cmp\_greater\_equal(x, y)              & 表示x是否大于等于y \\ \hline
std::in\_range\textless{}T\textgreater{}(x) & 生成x是否是类型T的有效值   \\ \hline
\end{longtable}

\begin{center}
表23.1 安全整数比较函数
\end{center}

若传递的值是可以由传递的类型表示的值,则函数std::在\_range()中产生true:

\begin{cpp}
bool b = std::in_range<int>(x); // true if x has a valid int value
\end{cpp}

只是生成x是否大于或等于所传递类型的最小值,是否小于或等于最大值。

注意,这些函数不能用于比较bool、字符类型或std::byte的值。

\mySubsubsection{23.3.2}{ssize()}

我们经常需要将集合、数组或范围的大小作为带符号值。例如,为了避免在这里出现警告:

\begin{cpp}
for (int i = 0; i < coll.size(); ++coll) { // possible warning
	...
}
\end{cpp}

问题是size()产生一个无符号值,如果值非常大,有符号值和无符号值之间的比较可能会失败。

虽然可以将i声明为unsigned int或std::size\_t,但可能有一个很好的理由将其用作int。

引入了一个辅助函数std::ssize(),允许以下用法:

\begin{cpp}
for (int i = 0; i < std::ssize(coll); ++coll) { // usually no warning
	...
}
\end{cpp}

多亏了ADL,当使用标准容器或其他标准类型时,编写以下代码就足够了:

\begin{cpp}
for (int i = 0; i < ssize(coll); ++coll) { // OK for std types
	...
}
\end{cpp}

请注意,这有时甚至是避免编译时错误所必需的。一个例子是锁存器、屏障和信号量的初始化。

还要注意,ranges库在命名空间std::ranges中提供了ssize()。















