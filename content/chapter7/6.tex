
由于支持将range作为一个参数传递,并且能够处理不同类型的哨兵(end迭代器),C++20中现在提供了调用算法的新方法。

然而,其有一些限制,并不是所有的算法都支持传递范围。

\mySubsubsection{7.6.1}{范围算法的优点和限制}

对于某些算法,还没有允许开发者将范围作为单个参数传递的版本:

\begin{itemize}
\item
范围算法不支持并行执行(C++17引入),没有API将范围参数作为单个对象和执行策略。

\item
目前还没有针对单对象范围的数值算法。要调用像std::accumulate()这样的算法,仍然必须传递范围的开始和结束。C++23可能会提供数值范围算法。
\end{itemize}

若算法有范围支持,则使用概念在编译时查找可能的错误:

\begin{itemize}
\item
迭代器和范围的概念确保传递有效的迭代器和范围。

\item
可调用对象的概念确保传递有效的辅助函数。
\end{itemize}

此外,返回类型可能会有所不同,因为:

\begin{itemize}
\item
支持并可能返回不同类型的迭代器,为这些迭代器定义了特殊的返回类型。这些类型列在“范围算法的新返回类型”表中。

\item
可能返回借来的迭代器,表明该迭代器不是有效的迭代器,因为传递了一个临时范围(右值)。
\end{itemize}


\begin{longtable}[c]{|l|l|l|}
\hline
\textbf{类型} &
\textbf{意义} &
\textbf{成员} \\ \hline
\endfirsthead
%
\endhead
%
\begin{tabular}[c]{@{}l@{}}std::ranges::in\_in\_result\\ std::ranges::in\_out\_result\\ std::ranges::in\_in\_out\_result\\ std::ranges::in\_out\_out\_result\\ std::ranges::in\_fun\_result\\ std::ranges::min\_max\_result\\ std::ranges::in\_found\_result\end{tabular} &
\begin{tabular}[c]{@{}l@{}}对于两个输入范围的位置\\ 对于一个输入范围的位置和一个输出范围的位置\\ 用于两个输入范围的位置和一个输出范围的位置\\ 对于输入范围的一个位置和两个输出范围的位置\\ 对于一个输入范围和一个函数的位置\\ 对于一个最大和一个最小的位置/值\\ 对于输入范围和布尔值的一个位置\end{tabular} &
\begin{tabular}[c]{@{}l@{}}in1, in2\\ in, out\\ in1, in2, out\\ in, out1, out2\\ in, out\\ min, max\\ in, found\end{tabular} \\ \hline
\end{longtable}

\begin{center}
表7.11 范围算法的新返回类型
\end{center}

下面的程序演示了如何使用这些返回类型:

\filename{ranges/results.cpp}

\begin{cpp}
#include <iostream>
#include <string_view>
#include <vector>
#include <algorithm>

void print(std::string_view msg, auto beg, auto end)
{
	std::cout << msg;
	for(auto pos = beg; pos != end; ++pos) {
		std::cout << ' ' << *pos;
	}
	std::cout << '\n';
}

int main()
{
	std::vector inColl{1, 2, 3, 4, 5, 6, 7};
	std::vector outColl{1, 2, 3, 4, 5, 6, 7, 8, 9, 10};
	
	auto result = std::ranges::transform(inColl, outColl.begin(),
	[] (auto val) {
		return val*val;
	});
	
	print("processed in:", inColl.begin(), result.in);
	print("rest of in: ", result.in, inColl.end());
	print("written out: ", outColl.begin(), result.out);
	print("rest of out: ", result.out, outColl.end());
}
\end{cpp}

该程序有以下输出:

\begin{shell}
processed in: 1 2 3 4 5 6 7
rest of in:
written out: 1 4 9 16 25 36 49
rest of out: 8 9 10
\end{shell}

\mySubsubsection{7.6.2}{算法概述}

各种算法的情况变得越来越复杂。其中有些可以并行使用,有些则不能。其中一些是ranges库支持的,而另一些则不支持。本节概述了哪些算法以哪种形式可用(不详细介绍每个算法的功能)。

后三栏含义如下所示:

\begin{itemize}
\item
Ranges表示该算法是否被范围库支持

\item
\_result指示是否使用以及使用哪些\_result类型(例如,in\_out表示in\_out\_result)

\item
Borrowed指示算法是否返回借来的迭代器
\end{itemize}

表“不可修改的算法”列出了C++20中可用的不可修改标准算法。

\begin{longtable}[c]{|l|l|l|l|l|l|}
\hline
\textbf{名称}                          & \textbf{启用标准} & \textbf{并行性} & \textbf{Ranges} & \textbf{\_result} & \textbf{Borrowed} \\ \hline
\endfirsthead
%
\endhead
%
for\_each()                & C++98 & yes & yes       & in\_fun  & yes \\ \hline
for\_each\_n()             & C++17 & yes & yes       & in\_fun  &     \\ \hline
count()                    & C++98 & yes & yes       &          &     \\ \hline
count\_if()                & C++98 & yes & yes       &          &     \\ \hline
min\_element()             & C++98 & yes & yes       & yes      &     \\ \hline
max\_element()             & C++98 & yes & yes       & yes      &     \\ \hline
minmax\_element()          & C++11 & yes & yes       & min\_max & yes \\ \hline
min()                      & C++20 & no  & yes(only) &          &     \\ \hline
max()                      & C++20 & no  & yes(only) &          &     \\ \hline
minmax()                   & C++20 & no  & yes(only) & min\_max &     \\ \hline
find()                     & C++98 & yes & yes       &          & yes \\ \hline
find\_if()                 & C++98 & yes & yes       &          & yes \\ \hline
find\_if\_not()            & C++11 & yes & yes       &          & yes \\ \hline
search()                   & C++98 & yes & yes       &          & yes \\ \hline
search\_n()                & C++98 & yes & yes       &          & yes \\ \hline
find\_end()                & C++98 & yes & yes       &          & yes \\ \hline
find\_first\_of()          & C++98 & yes & yes       &          & yes \\ \hline
adjacent\_find()           & C++98 & yes & yes       &          & yes \\ \hline
equal()                    & C++98 & yes & yes       &          & yes \\ \hline
is\_permutation()          & C++11 & no  & yes       &          &     \\ \hline
mismatch()                 & C++98 & yes & yes       & in\_in   & yes \\ \hline
lexicographical\_compare() & C++98 & yes & yes       &          &     \\ \hline
lexicographical\_compare\_three\_way() & C++20          & no                & no              &                   &                   \\ \hline
is\_sorted()               & C++11 & yes & yes       &          &     \\ \hline
is\_sorted\_until()        & C++11 & yes & yes       &          & yes \\ \hline
is\_partitioned()          & C++11 & yes & yes       &          &     \\ \hline
partition\_point()         & C++11 & no  & yes       &          &     \\ \hline
is\_heap()                 & C++11 & yes & yes       &          &     \\ \hline
is\_heap\_until()          & C++11 & yes & yes       &          & yes \\ \hline
all\_of()                  & C++11 & yes & yes       &          &     \\ \hline
any\_of()                  & C++11 & yes & yes       &          &     \\ \hline
none\_of()                 & C++11 & yes & yes       &          &     \\ \hline
\end{longtable}

\begin{center}
表7.12 不可修改的算法
\end{center}

注意,std::ranges::min()、std::ranges::max()和std::ranges::minmax()算法在std中只有两个值或一个std::initializer\_list<>的对应项。

表“可修改算法”列出了C++20中可用的修改标准算法。

\begin{longtable}[c]{|l|l|l|l|l|l|}
\hline
\textbf{名称}              & \textbf{启用标准} & \textbf{并行性} & \textbf{Ranges} & \textbf{\_result} & \textbf{Borrowed} \\ \hline
\endfirsthead
%
\endhead
%
for\_each()         & C++98 & yes & yes & in\_fun & yes \\ \hline
for\_each\_n()      & C++17 & yes & yes & in\_fun &     \\ \hline
copy()              & C++98 & yes & yes & in\_out & yes \\ \hline
copy\_if()          & C++11 & yes & yes & in\_out &     \\ \hline
copy\_n()           & C++11 & yes & yes & in\_out & yes \\ \hline
copy\_backward()    & C++11 & no  & yes & in\_out & yes \\ \hline
move()              & C++11 & yes & yes & in\_out & yes \\ \hline
move\_backward()    & C++11 & no  & yes & in\_out & yes \\ \hline
transform() for one range  & C++98          & yes               & yes             & in\_out           & yes               \\ \hline
transform() for two ranges & C++98          & yes               & yes             & in\_in\_out       & yes               \\ \hline
merge()             & C++98 & yes & yes &         &     \\ \hline
swap\_ranges()      & C++98 & yes & yes & in\_in  & yes \\ \hline
fill()              & C++98 & yes & yes &         & yes \\ \hline
fill\_n()           & C++98 & yes & yes &         &     \\ \hline
generate()          & C++98 & yes & yes &         & yes \\ \hline
generate\_n()       & C++98 & yes & yes &         &     \\ \hline
iota()              & C++11 & no  & no  &         &     \\ \hline
replace()           & C++98 & yes & yes &         & yes \\ \hline
replace\_if()       & C++98 & yes & yes &         & yes \\ \hline
replcae\_copy()     & C++98 & yes & yes & in\_out & yes \\ \hline
replace\_copy\_if() & C++98 & yes & yes & in\_out & yes \\ \hline
\end{longtable}

\begin{center}
表7.13 可修改算法
\end{center}

表“可移除算法”列出了C++20中可用的“移除”标准算法。注意,算法从来没有真正地删除元素。当它们将整个范围作为单个参数时,这仍然适用。改变顺序,使未删除的元素被打包到前面并返回新端。


\begin{longtable}[c]{|l|l|l|l|l|l|}
\hline
\textbf{名称} & \textbf{启用标准} & \textbf{并行性} & \textbf{Ranges} & \textbf{\_result} & \textbf{Borrowed} \\ \hline
\endfirsthead
%
\endhead
%
remove()           & C++98 & yes & yes &         & yes \\ \hline
remove\_if()       & C++98 & yes & yes &         & yes \\ \hline
remove\_copy()     & C++98 & yes & yes & in\_out & yes \\ \hline
remove\_copy\_if() & C++98 & yes & yes & in\_out & yes \\ \hline
unique()           & C++98 & yes & yes &         & yes \\ \hline
unique\_copy()     & C++98 & yes & yes & in\_out & yes \\ \hline
\end{longtable}

\begin{center}
表7.14 可移除算法
\end{center}

表“可变异算法”列出了C++20中可用的变异标准算法。

\begin{longtable}[c]{|l|l|l|l|l|l|}
\hline
\textbf{名称} & \textbf{启用标准} & \textbf{并行性} & \textbf{Ranges} & \textbf{\_result} & \textbf{Borrowed} \\ \hline
\endfirsthead
%
\endhead
%
reverse()           & C++98 & yes & yes &           & yes \\ \hline
reverse\_copy()     & C++98 & yes & yes & in\_out   & yes \\ \hline
rotate()            & C++98 & yes & yes &           & yes \\ \hline
rotate\_copy()      & C++98 & yes & yes & in\_out   & yes \\ \hline
shift\_left()       & C++20 & yes & no  &           &     \\ \hline
shift\_right()      & C++20 & yes & no  &           &     \\ \hline
sample()            & C++17 & no  & yes &           &     \\ \hline
next\_permutation() & C++98 & no  & yes & in\_found & yes \\ \hline
prev\_permutation() & C++98 & no  & yes & in\_found & yes \\ \hline
shuffle()           & C++11 & no  & yes &           & yes \\ \hline
random\_shuffle()   & C++98 & no  & no  &           &     \\ \hline
partition()         & C++98 & yes & yes &           &     \\ \hline
stable\_partition() & C++98 & yes & yes &           &     \\ \hline
partition\_copy()   & C++11 & yes & yes &           &     \\ \hline
\end{longtable}


\begin{center}
表7.15 可变异算法
\end{center}

表“排序算法”列出了C++20中可用的排序标准算法。


\begin{longtable}[c]{|l|l|l|l|l|l|}
\hline
\textbf{名称} & \textbf{启用标准} & \textbf{并行性} & \textbf{Ranges} & \textbf{\_result} & \textbf{Borrowed} \\ \hline
\endfirsthead
%
\endhead
%
sort()                & C++98 & yes & yes &              & yes \\ \hline
stable\_sort()        & C++98 & yes & yes &              & yes \\ \hline
partial\_sort()       & C++98 & yes & yes &              & yes \\ \hline
partial\_sort\_copy() & C++98 & yes & yes & in\_out      & yes \\ \hline
nth\_element()        & C++98 & yes & yes &              & yes \\ \hline
partition()           & C++98 & yes & yes &              & yes \\ \hline
stable\_partition()   & C++98 & yes & yes &              & yes \\ \hline
partition\_copy()     & C++11 & yes & yes & in\_out\_out & yes \\ \hline
make\_heap()          & C++98 & no  & yes &              & yes \\ \hline
push\_heap()          & C++98 & no  & yes &              & yes \\ \hline
pop\_heap()           & C++98 & no  & yes &              & yes \\ \hline
sort\_heap()          & C++98 & no  & yes &              & yes \\ \hline
\end{longtable}

\begin{center}
表7.16 排序算法
\end{center}

表中列出了C++20中用于排序范围的可用标准算法。

\begin{longtable}[c]{|l|l|l|l|l|l|}
\hline
\textbf{名称}                & \textbf{启用标准} & \textbf{并行性} & \textbf{Ranges} & \textbf{\_result} & \textbf{Borrowed} \\ \hline
\endfirsthead
%
\endhead
%
binary\_search()    & C++98 & no  & yes &             &     \\ \hline
includes()          & C++98 & yes & yes &             &     \\ \hline
lower\_bound()      & C++98 & no  & yes &             & yes \\ \hline
upper\_bound()      & C++98 & no  & yes &             & yes \\ \hline
equal\_range()      & C++98 & no  & yes &             & yes \\ \hline
merge()             & C++98 & yes & yes & in\_in\_out & yes \\ \hline
inplace\_merge()    & C++98 & yes & yes &             & yes \\ \hline
set\_union()        & C++98 & yes & yes & in\_in\_out & yes \\ \hline
set\_intersection() & C++98 & yes & yes & in\_in\_out & yes \\ \hline
set\_difference()   & C++98 & yes & yes & in\_out     & yes \\ \hline
set\_symmetric\_difference() & C++98          & yes               & yes             & in\_in\_out       & yes               \\ \hline
partition\_point()  & C++11 & no  & yes &             & yes \\ \hline
\end{longtable}

\begin{center}
表7.17 可排序范围的算法
\end{center}

表“数学算法”列出了C++20中可用的数学标准算法。


\begin{longtable}[c]{|l|l|l|l|l|l|}
\hline
\textbf{名称} & \textbf{启用标准} & \textbf{并行性} & \textbf{Ranges} & \textbf{\_result} & \textbf{Borrowed} \\ \hline
\endfirsthead
%
\endhead
%
accumulate()                 & C++98 & no  & no &  &  \\ \hline
reduce()                     & C++17 & yes & no &  &  \\ \hline
transform\_reduce()          & C++17 & yes & no &  &  \\ \hline
inner\_product()             & C++98 & no  & no &  &  \\ \hline
adjacent\_difference()       & C++98 & no  & no &  &  \\ \hline
partial\_sum()               & C++98 & no  & no &  &  \\ \hline
inclusive\_scan()            & C++17 & yes & no &  &  \\ \hline
exclusive\_scan()            & C++17 & yes & no &  &  \\ \hline
transform\_inclusive\_scan() & C++17 & yes & no &  &  \\ \hline
transform\_exclusive\_scan() & C++17 & yes & no &  &  \\ \hline
iota                         & C++11 & no  & no &  &  \\ \hline
\end{longtable}

\begin{center}
表7.18 数学算法
\end{center}



