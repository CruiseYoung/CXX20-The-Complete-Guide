
C++20开始,编译时函数可以分配内存,前提是这些内存在编译时也被释放。由于这个原因,现在可以在编译时使用字符串或vector,但有一个重要的限制:在编译时创建的字符串或vector不能在运行时使用,原因是在编译时分配的内存也必须在编译时释放。

\mySubsubsection{18.5.1}{在编译时使用vector}

下面是第一个例子,在编译时使用std::vector<>:

\filename{comptime/vector.hpp}

\begin{cpp}
#include <vector>
#include <ranges>
#include <algorithm>
#include <numeric>

template<std::ranges::input_range T>
constexpr auto modifiedAvg(const T& rg)
{
	using elemType = std::ranges::range_value_t<T>;
	
	// initialize compile-time vector with passed elements:
	std::vector<elemType> v{std::ranges::begin(rg),
							std::ranges::end(rg)};
		
	// perform several modifications:
	v.push_back(elemType{});
	std::ranges::sort(v);
	auto newEnd = std::unique(v.begin(), v.end());
	...
	
	// return average of modified vector:
	auto sum = std::accumulate(v.begin(), newEnd,
								elemType{});
	return sum / static_cast<double>(v.size());
}
\end{cpp}

这里,用constexpr定义modifiedAvg(),以便在编译时调用。在内部,使用std::vector<> v,用传递的范围的元素初始化,可以使用vector的全部API(特别是插入和删除元素)。例如,插入一个元素,对元素排序,并使用unique()删除连续的重复项。所有这些算法现在都是constexpr的,因此可以在编译时使用。

但在最后,不返回vector,只返回在编译时向量的帮助下计算出来的值。

可以在编译时调用这个函数:

\filename{comptime/vector.cpp}

\begin{cpp}
#include "vector.hpp"
#include <iostream>
#include <array>

int main()
{
	constexpr std::array orig{0, 8, 15, 132, 4, 77};
	
	constexpr auto avg = modifiedAvg(orig);
	std::cout << "average: " << avg << '\n';
}
\end{cpp}

由于avg是用constexpr声明的,所以modiedavg()是在编译时求值的。

也可以用consteval声明modifiedAvg(),因为不会在运行时复制元素,所以可以按值传递参数:

\begin{cpp}
template<std::ranges::input_range T>
consteval auto modifiedAvg(T rg)
{
	using elemType = std::ranges::range_value_t<T>;
	
	// initialize compile-time vector with passed elements:
	std::vector<elemType> v{std::ranges::begin(rg),
							std::ranges::end(rg)};
	...
}
\end{cpp}

然而,若vector可以在运行时使用,则仍然不能在编译时声明和初始化:

\begin{cpp}
int main()
{
	constexpr std::vector orig{0, 8, 15, 132, 4, 77}; // ERROR
	...
}
\end{cpp}

出于同样的原因,编译时函数只能在编译时使用返回值时向调用者返回一个vector:

\filename{comptime/returnvector.cpp}

\begin{cpp}
#include <vector>

constexpr auto returnVector()
{
	std::vector<int> v{0, 8, 15};
	v.push_back(42);
	...
	
	return v;
}

constexpr auto returnVectorSize()
{
	auto coll = returnVector();
	return coll.size();
}
	
int main()
{
	// constexpr auto coll = returnVector(); // ERROR
	constexpr auto tmp = returnVectorSize(); // OK
	...
}
\end{cpp}

\mySubsubsection{18.5.2}{编译时返回集合}

虽然不能返回编译时的vector,以便在运行时使用它,但有一种方法可以返回编译时计算的元素集合:可以返回std::array<>。唯一的问题是需要知道数组的大小,因为大小不能通过vector的大小初始化:

\begin{cpp}
std::vector v;
...
std::array<int, v.size()> arr; // ERROR
\end{cpp}

这样的代码永远不会编译,因为size()是一个运行时值,而arr的声明需要一个编译时值。这段代码是否在编译时求值并不重要,所以必须返回一个固定大小的数组。例如:

\filename{comptime/mergevalues.hpp}

\begin{cpp}
#include <vector>
#include <ranges>
#include <algorithm>
#include <array>

template<std::ranges::input_range T>
consteval auto mergeValues(T rg, auto... vals)
{
	// create compile-time vector from passed range:
	std::vector<std::ranges::range_value_t<T>> v{std::ranges::begin(rg),
												 std::ranges::end(rg)};
	(... , v.push_back(vals)); // merge all passed parameters
	std::ranges::sort(v); // sort all elements

	// return extended collection as array:
	constexpr auto sz = std::ranges::size(rg) + sizeof...(vals);
	std::array<std::ranges::range_value_t<T>, sz> arr{};
	std::ranges::copy(v, arr.begin());
	return arr;
}
\end{cpp}

在consteval函数中使用vector将传入的可变数量的实参与传入范围的元素合并,并对它们全排序。但可将结果集合作为std::array返回,以便将其传递给运行时上下文。例如,下面的程序可以很好地使用这个函数:

\filename{comptime/mergevalues.cpp}

\begin{cpp}
#include "mergevalues.hpp"
#include <iostream>
#include <array>

int main()
{
	// compile-time initialization of array:
	constexpr std::array orig{0, 8, 15, 132, 4, 77, 3};
	
	// initialization of sorted extended array:
	auto merged = mergeValues(orig, 42, 4);
	
	// print elements:
	for(const auto& i : merged) {
		std::cout << i << ' ';
	}
}
\end{cpp}

其输出如下:

\begin{shell}
0 3 4 4 8 15 42 77 132
\end{shell}

若在编译时不知道数组的结果大小,则必须声明具有最大大小的返回数组,并返回结果大小。合并这些值的函数现在可能如下所示:

\filename{comptime/mergevaluessz.hpp}

\begin{cpp}
#include <vector>
#include <ranges>
#include <algorithm>
#include <array>

template<std::ranges::input_range T>
consteval auto mergeValuesSz(T rg, auto... vals)
{
	// create compile-time vector from passed range:
	std::vector<std::ranges::range_value_t<T>> v{std::ranges::begin(rg),
												 std::ranges::end(rg)};
												 
	(... , v.push_back(vals)); // merge all passed parameters
	
	std::ranges::sort(v); // sort all elements
	
	// return extended collection as array and its size:
	constexpr auto maxSz = std::ranges::size(rg) + sizeof...(vals);
	std::array<std::ranges::range_value_t<T>, maxSz> arr{};
	auto res = std::ranges::unique_copy(v, arr.begin());
	return std::pair{arr, res.out - arr.begin()};
}
\end{cpp}

此外,可使用std::ranges::unique\_copy()来删除排序后的连续重复项,并返回数组和结果元素的数量。

注意,应该用\{\}声明arr,以确保数组中的所有值都初始化。编译时函数不允许产生未初始化的内存。

现在可以这样使用返回值:

\filename{comptime/mergevaluessz.cpp}

\begin{cpp}
#include "mergevaluessz.hpp"
#include <iostream>
#include <array>
#include <ranges>

int main()
{
	// compile-time initialization of array:
	constexpr std::array orig{0, 8, 15, 132, 4, 77, 3};
	
	// initialization of sorted extended array:
	auto tmp = mergeValuesSz(orig, 42, 4);
	auto merged = std::views::counted(tmp.first.begin(), tmp.second);
	
	// print elements:
	for(const auto& i : merged) {
		std::cout << i << ' ';
	}
}
\end{cpp}

通过使用视图适配器std::views:: counts(),可以很容易地将返回的数组和返回的元素大小结合起来,作为一个范围在数组中使用。

现在程序的输出是:

\begin{shell}
0 3 4 8 15 42 77 132
\end{shell}

\mySubsubsection{18.5.3}{编译时使用字符串}

对于编译时字符串,现在所有的操作都是constexpr,所以现在可以在编译时使用std::string,也可以使用任何其他字符串类型,例如std::u8string。

但也有一个限制,即不能在运行时使用编译时字符串。例如:

\begin{cpp}
consteval std::string returnString()
{
	std::string s = "Some string from compile time";
	...
	return s;
}

void useString()
{
	constexpr auto s = returnString(); // ERROR
	...
}

constexpr void useStringInConstexpr()
{
	std::string s = returnString(); // ERROR
	...
}

consteval void useStringInConsteval()
{
	std::string s = returnString(); // OK
	...
}
\end{cpp}

不能通过使用data()或c\_str()或std::string\_view返回编译时字符串来解决这个问题,可将返回在编译时分配的内存地址。若发生这种情况,编译器会引发编译时错误。

然而,可以使用与上面描述的向量相同的技巧。可以将string对象转换为固定大小的数组,并返回数组和vector对象的大小。

下面是一个完整的例子:

\filename{comptime/comptimestring.cpp}

\begin{cpp}
#include <iostream>
#include <string>
#include <array>
#include <cassert>
#include "asstring.hpp"

// function template to export a compile-time string to runtime:
template<int MaxSize>
consteval auto toRuntimeString(std::string s)
{
	// ensure the size of the exported array is large enough:
	assert(s.size() <= MaxSize);
	
	// create a compile-time array and copy all characters into it:
	std::array<char, MaxSize+1> arr{}; // ensure all elems are initialized
	for (int i = 0; i < s.size(); ++i) {
		arr[i] = s[i];
	}
	
	// return the compile-time array and the string size:
	return std::pair{arr, s.size()};
}

// function to import an exported compile-time string at runtime:
std::string fromComptimeString(const auto& dataAndSize)
{
	// init string with exported array of chars and size:
	return std::string{dataAndSize.first.data(),
					   dataAndSize.second};
}

// test the functions:
consteval auto comptimeMaxStr()
{
	std::string s = "max int is " + asString(std::numeric_limits<int>::max())
					+ " (" + asString(std::numeric_limits<int>::digits + 1)
					+ " bits)";
	return toRuntimeString<100>(s);
}

int main()
{
	std::string s = fromComptimeString(comptimeMaxStr());
	std::cout << s << '\n';
}
\end{cpp}

同样,这里定义了两个辅助函数来将编译时字符串导出为运行时字符串:

\begin{itemize}
\item 
编译时函数toRuntimeString()将字符串转换为std::array<>,并返回该数组和字符串的大小:

\begin{cpp}
template<int MaxSize>
consteval auto toRuntimeString(std::string s)
{
	assert(s.size() <= MaxSize); // ensure array size fits
	
	// create a compile-time array and copy all characters into it:
	std::array<char, MaxSize+1> arr{}; // ensure all elems are initialized
	for (int i = 0; i < s.size(); ++i) {
		arr[i] = s[i];
	}
	
	return std::pair{arr, s.size()}; // return array and size
}
\end{cpp}

通过使用assert(),再次检查数组的大小是否足够大。以同样的方式,可以在编译时确定有没有浪费太多的内存。

\item 
运行时函数fromComptimeString()接受返回的数组和大小,初始化运行时字符串并返回:

\begin{cpp}
std::string fromComptimeString(const auto& dataAndSize)
{
	return std::string{dataAndSize.first.data(),
					   dataAndSize.second};
}
\end{cpp}

该函数的测试用例使用辅助函数asString(),可以在编译时和运行时使用它将整型值转换为字符串。

该程序有以下输出:

\begin{shell}
max int is 2147483647 (32 bits)
\end{shell}

\end{itemize}






