
先来看一些例子。

\mySubsubsection{11.1.1}{每月5号安排一次会议}

考虑这样一个程序,希望遍历一年中的所有月份,以便在每个月的5号安排一次会议:

\filename{lib/chrono1.cpp}

\begin{cpp}
#include <chrono>
#include <iostream>

int main()
{
	namespace chr = std::chrono; // shortcut for std::chrono
	using namespace std::literals; // for h, min, y suffixes

	// for each 5th of all months of 2021:
	chr::year_month_day first = 2021y / 1 / 5;
	for (auto d = first; d.year() == first.year(); d += chr::months{1}) {
		// print out the date:
		std::cout << d << ":\n";

		// init and print 18:30 UTC of those days:
		auto tp{chr::sys_days{d} + 18h + 30min};
		std::cout << std::format(" We meet on {:%A %D at %R}\n", tp);
	}
}
\end{cpp}

程序的输出如下所示:

\begin{shell}
2021-01-05:
 We meet on Tuesday 01/05/21 at 18:30
2021-02-05:
 We meet on Friday 02/05/21 at 18:30
2021-03-05:
 We meet on Friday 03/05/21 at 18:30
2021-04-05:
 We meet on Monday 04/05/21 at 18:30
2021-05-05:
 We meet on Wednesday 05/05/21 at 18:30
...
2021-11-05:
 We meet on Friday 11/05/21 at 18:30
2021-12-05:
 We meet on Sunday 12/05/21 at 18:30
\end{shell}

让我们一步步地完成这个程序。

\mySamllsection{声明命名空间}

从命名空间开始,并使用声明使chrono库更方便使用:

\begin{itemize}
\item
首先,引入chr::作为chrono库的标准命名空间std::chrono的快捷方式:

\begin{cpp}
namespace chr = std::chrono; // shortcut for std::chrono
\end{cpp}

为了使示例更具可读性,只使用chr::,而非std::chrono::。

\item
然后,需要确保可以使用字面后缀,如s、min、h和y (y是C++20添加的新特性,其他由C++14添加)。

\begin{cpp}
using namespace std::literals; // for min, h, y suffixes
\end{cpp}
\end{itemize}

可以直接使用using声明:

\begin{cpp}
using namespace std::chrono; // skip chrono namespace qualification
\end{cpp}

不过,应该尽量避免使用这种using声明的作用域,以避免不必要的副作用。

\mySamllsection{日历类型year\_month\_day}

\begin{cpp}
chr::year_month_day first = 2021y / 1 / 5;
\end{cpp}

数据流首先初始化开始日期,类型为std::chrono::year\_month\_day:

year\_month\_day是一个日历类型,为日期的所有三个字段提供属性,以便于处理特定日期的年、月和日。

因为要想遍历每个月的所有第五天,所以用2021年1月5日初始化对象,使用运算符/将年与月和日的值组合在一起:

\begin{itemize}
\item
首先,将年份创建为std::chrono::year类型的对象,使用新的标准字面符\textbf{y}:

\begin{cpp}
2021y
\end{cpp}

要使这个字面符可用,必须提供以下using声明之一:

\begin{cpp}
using std::literals; // enable all standard literals
using std::chrono::literals; // enable all standard chrono literals
using namespace std::chrono; // enable all standard chrono literals
using namespace std; // enable all standard literals
\end{cpp}

若没有这些字面符,则可使用以下等价的方式:

\begin{cpp}
std::chrono::year{2021}
\end{cpp}

\item
使用操作符/将std::chrono::year与整型值组合,创建一个std::chrono::year\_month类型的对象。

第一个操作数是年,所以很明显第二个操作数必须是月,不能在这里指定日期。

\item
再次使用操作符/将std::chrono::year\_month对象与整数值组合,以创建std::chrono::year\_month\_day。
\end{itemize}

日历类型的初始化是类型安全的,只需要指定第一个操作数的类型。

操作符已经产生了正确的类型,所以可以先用auto声明。若没有命名空间声明,则需要这样写:

\begin{cpp}
auto first = std::chrono::year{2021} / 1 / 5;
\end{cpp}

有了时间字面值,就简单了:

\begin{cpp}
auto first = 2021y/1/5;
\end{cpp}

\mySamllsection{初始化日历类型的其他方法}

还有其他方法可以初始化日历类型,如year\_month\_day:

\begin{cpp}
auto d1 = std::chrono::year{2021}/1/5; // January 5, 2021
auto d2 = std::chrono::month{1}/5/2021; // January 5, 2021
auto d3 = std::chrono::day{5}/1/2021; // January 5, 2021
\end{cpp}

操作符/的第一个参数的类型指定了如何解释其他参数。

有了时间字面符,可以简单地写为:

\begin{cpp}
using namespace std::literals;
auto d4 = 2021y/1/5; // January 5, 2021
auto d5 = 5d/1/2021; // January 5, 2021
\end{cpp}

没有月的标准后缀,但有预定义的标准对象:

\begin{cpp}
auto d6 = std::chrono::January / 5 / 2021; // January 5, 2021
\end{cpp}

有了相应的using声明,可以这样写:

\begin{cpp}
using namespace std::chrono;
auto d6 = January/5/2021; // January 5, 2021
\end{cpp}

所有情况下,都会初始化std::chrono::year\_month\_day类型的对象。

\mySamllsection{新持续时间类型}

例子中,使用一个循环来迭代一年中的所有月份:

\begin{cpp}
for (auto d = first; d.year() == first.year(); d += chr::months{1}) {
	...
}
\end{cpp}

std::chrono::months是一个新的持续时间类型,表示一个月。可以将它用于所有日历类型来处理日期的特定月份,像在这里添加一个月那样:

\begin{cpp}
std::chrono::year_month_day d = ... ;
d += chr::months{1}; // add one month
\end{cpp}

注意,当将其用于普通持续时间和时间点时,类型months表示一个月的平均持续时间,即30.436875天。所以对于普通的时间点,应该小心使用月和年。

\mySamllsection{所有Chrono类型的输出操作符}

循环中,输出当前日期:

\begin{cpp}
std::cout << d << '\n';
\end{cpp}

C++20起,为几乎所有可能的chrono类型定义了一个输出操作符。

\begin{cpp}
std::chrono::year_month_day d = ... ;
std::cout << "d: " << d << '\n';
\end{cpp}

这使得输出任何计时值变得很容易,但输出并不总是适合特定的需要。对于类型year\_month\_day,输出格式是我们作为开发者所期望的类型:year-month-day。例如:

\begin{shell}
2021-01-05
\end{shell}

其他默认输出格式通常使用斜杠作为分隔符,这里有详细说明。

对于用户定义的输出,chrono库还支持用于格式化输出的新库。稍后会使用它来输出时间点tp:

\begin{cpp}
std::cout << std::format("We meet on {:%A %D at %R}\n", tp);
\end{cpp}

格式化说明符,\%A表示工作日,\%D表示日期,\%R表示时间(小时和分钟),与C函数strftime()和POSIX date命令的说明符对应,因此输出可能如下所示:

\begin{shell}
We meet on Saturday 05/01/21 at 18:30
\end{shell}

还支持特定于语言环境的输出,稍后将详细描述chrono类型的格式化输出。

\mySamllsection{合并日期和时间}

为了初始化tp,将循环的天数与一天中的特定时间结合起来:

\begin{cpp}
auto tp{sys_days{d} + 18h + 30min};
\end{cpp}

为了组合日期和时间,必须使用时间点和持续时间。std::chrono::year\_month\_day这样的日历类型不是时间点,所以首先将year\_month\_day值转换为time\_point<>对象:

\begin{cpp}
std::chrono::year_month_day d = ... ;
std::chrono::sys_days{d} // convert to a time_point
\end{cpp}

类型std::chrono::sys\_days是以天为粒度的系统时间点的新快捷方式,相当于:std::chrono::time\_point<std::chrono::system\_clock, std::chrono::days>。

通过添加一些持续时间(18小时和30分钟),可以计算一个新值,该值(chrono库中通常如此)具有一个粒度足以满足计算结果的类型。将天与小时和分钟结合起来,结果类型是一个以分钟为粒度的系统时间点,但不必知道类型,只要使用auto就好。

为了更明确地指定tp的类型,还可以这样声明:

\begin{itemize}
\item
作为系统时间点,但不指定其粒度:

\begin{cpp}
chr::sys_time tp{chr::sys_days{d} + 18h + 30min};
\end{cpp}

通过类模板参数推导,可以推导出粒度的模板参数。

\item
作为具有指定粒度的系统时间点:

\begin{cpp}
chr::sys_time<chr::minutes> tp{chr::sys_days{d} + 18h + 30min};
\end{cpp}

这时,初始值不能比指定类型更细粒度,否则必须使用舍入函数。

\item
作为系统时间点,方便以秒为粒度的类型:

\begin{cpp}
chr::sys_seconds tp{chr::sys_days{d} + 18h + 30min};
\end{cpp}

初始值不能比指定类型更细粒度,否则必须使用舍入函数。
\end{itemize}

默认输出操作符根据上述指定的格式输出时间点。例如:

\begin{shell}
We meet on Saturday 05/01/21 at 18:30
\end{shell}

注意,处理系统时间时,默认输出是UTC。

更细粒度的时间点,还可以使用输出精确值所需的任何参数(例如毫秒)。

\mySubsubsection{11.1.2}{把会议安排在每个月的最后一天}

通过迭代每个月的最后几天来修改第一个示例程序,可以这样做:

\filename{lib/chrono2.cpp}

\begin{cpp}
#include <chrono>
#include <iostream>

int main()
{
	namespace chr = std::chrono; // shortcut for std::chrono
	using namespace std::literals; // for h, min, y suffixes

	// for each last of all months of 2021:
	auto first = 2021y / 1 / chr::last;
	for (auto d = first; d.year() == first.year(); d += chr::months{1}) {
		// print out the date:
		std::cout << d << ":\n";

		// init and print 18:30 UTC of those days:
		auto tp{chr::sys_days{d} + 18h + 30min};
		std::cout << std::format(" We meet on {:%A %D at %R}\n", tp);
	}
}
\end{cpp}

只修改了first的初始化。用auto类型声明,用std::chrono::last作为day进行初始化:

\begin{cpp}
auto first = 2021y / 1 / chr::last;
\end{cpp}

类型std::chrono::last不仅表示一个月的最后一天,还表示first具有不同的类型std::chrono::year\_month\_day\_last。通过增加一个月,日期的日得到了调整。实际上,该类型总是保存最后一天。当输出日期并指定相应的输出格式时,就会发生向数字日期的转换。

结果,输出为以下内容:

\begin{shell}
2021/Jan/last:
 We meet on Sunday 01/31/21 at 18:30
2021/Feb/last:
 We meet on Sunday 02/28/21 at 18:30
2021/Mar/last:
 We meet on Wednesday 03/31/21 at 18:30
2021/Apr/last:
 We meet on Friday 04/30/21 at 18:30
2021/May/last:
 We meet on Monday 05/31/21 at 18:30
...
2021/Nov/last:
 We meet on Tuesday 11/30/21 at 18:30
2021/Dec/last:
 We meet on Friday 12/31/21 at 18:30
\end{shell}

year\_month\_day\_last的默认输出格式使用last和斜杠,而非减号作为分隔符(只有year\_month\_day在其默认输出格式中使用连字符)。例如:

\begin{shell}
2021/Jan/last
\end{shell}

可以将first声明为std::chrono::year\_month\_day:

\begin{cpp}
// for each last days of all months of 2021:
std::chrono::year_month_day first = 2021y / 1 / chr::last;
for (auto d = first; d.year() == first.year(); d += chr::months{1}) {
	// print out the date:
	std::cout << d << ":\n";

	// init and print 18:30 UTC of those days:
	auto tp{chr::sys_days{d} + 18h + 30min};
	std::cout << std::format(" We meet on {:%A %D at %R}\n", tp);
}
\end{cpp}

会有以下输出:

\begin{shell}
2021-01-31:
 We meet on Sunday 01/31/21 at 18:30
2021-02-31 is not a valid date:
 We meet on Wednesday 03/03/21 at 18:30
2021-03-31:
 We meet on Wednesday 03/31/21 at 18:30
2021-04-31 is not a valid date:
 We meet on Saturday 05/01/21 at 18:30
2021-05-31:
 We meet on Monday 05/31/21 at 18:30
...
\end{shell}

first类型存储的是日期的数值,初始化为1月的最后一天,所以可以迭代每个月的第31天。若这样的日期不存在,则默认输出格式打印为无效日期,而std::format()甚至执行错误的计算。

处理这种情况的一种方法是检查日期是否有效,再执行要做的事情。例如:

\filename{lib/chrono3.cpp}

\begin{cpp}
#include <chrono>
#include <iostream>

int main()
{
	namespace chr = std::chrono; // shortcut for std::chrono
	using namespace std::literals; // for h, min, y suffixes

	// for each last of all months of 2021:
	chr::year_month_day first = 2021y / 1 / 31;
	for (auto d = first; d.year() == first.year(); d += chr::months{1}) {
		// print out the date:
		if (d.ok()) {
			std::cout << d << ":\n";
		}
		else {
			// for months not having the 31st use the 1st of the next month:
			auto d1 = d.year() / d.month() / 1 + chr::months{1};
			std::cout << d << ":\n";
		}

		// init and print 18:30 UTC of those days:
		auto tp{chr::sys_days{d} + 18h + 30min};
		std::cout << std::format(" We meet on {:%A %D at %R}\n", tp);
	}
}
\end{cpp}

通过使用成员函数ok(),将无效日期调整为下个月的第一天。可以得到以下输出:

\begin{shell}
2021-01-31:
 We meet on Sunday 01/31/21 at 18:30
2021-02-31 is not a valid date:
 We meet on Wednesday 03/03/21 at 18:30
2021-03-31:
 We meet on Wednesday 03/31/21 at 18:30
2021-04-31 is not a valid date:
 We meet on Saturday 05/01/21 at 18:30
2021-05-31:
 We meet on Monday 05/31/21 at 18:30
...
2021-11-31 is not a valid date:
 We meet on Wednesday 12/01/21 at 18:30
2021-12-31:
 We meet on Friday 12/31/21 at 18:30
\end{shell}

\mySubsubsection{11.1.3}{每一个第一个星期一安排会议}

类似地,可以在每个月的第一个星期一安排一次会议:

\filename{lib/chrono4.cpp}

\begin{cpp}
#include <chrono>
#include <iostream>

int main()
{
	namespace chr = std::chrono; // shortcut for std::chrono
	using namespace std::literals; // for min, h, y suffixes

	// for each first Monday of all months of 2021:
	auto first = 2021y / 1 / chr::Monday[1];
	for (auto d = first; d.year() == first.year(); d += chr::months{1}) {
		// print out the date:
		std::cout << d << '\n';

		// init and print 18:30 UTC of those days:
		auto tp{chr::sys_days{d} + 18h + 30min};
		std::cout << std::format(" We meet on {:%A %D at %R}\n", tp);
	}
}
\end{cpp}

first有一个特殊类型std::chrono::year\_month\_weekday,表示一年中一个月中的工作日:

默认输出格式表示使用特定格式,但格式化的输出工作良好

\begin{shell}
2021/Jan/Mon[1]
 We meet on Monday 01/04/21 at 18:30
2021/Feb/Mon[1]
 We meet on Monday 02/01/21 at 18:30
2021/Mar/Mon[1]
 We meet on Monday 03/01/21 at 18:30
2021/Apr/Mon[1]
 We meet on Monday 04/05/21 at 18:30
2021/May/Mon[1]
 We meet on Monday 05/03/21 at 18:30
...
2021/Nov/Mon[1]
 We meet on Monday 11/01/21 at 18:30
2021/Dec/Mon[1]
 We meet on Monday 12/06/21 at 18:30
\end{shell}

\mySamllsection{工作日的日历类型}

这一次,首先初始化为一个日历类型std::chrono::year\_month\_weekday的对象,并将其初始化为2021年1月的第一个星期一:

\begin{shell}
auto first = 2021y / 1 / chr::Monday[1];
\end{shell}

使用操作符/来组合不同的日期字段,但这一次,工作日的类型就要发挥作用了:

\begin{itemize}
\item
首先,调用2021y / 1,将std::chrono::year与整型值结合,创建std::chrono::year\_month。

\item
然后,调用std::chrono::Monday(std::chrono::weekday类型的标准对象)的[]操作符创建一个std::chrono::weekday\_indexed类型的对象,表示第n个工作日。

\item
操作符/用于将std::chrono::year\_month与std::chrono::weekday\_indexed类型的对象组合在一起,创建一个std::chrono::year\_month\_weekday对象。
\end{itemize}

因此,完整的声明如下所示:

\begin{cpp}
std::chrono::year_month_weekday first = 2021y / 1 / std::chrono::Monday[1];
\end{cpp}

year\_month\_weekday的默认输出格式使用斜杠,而不是减号作为分隔符(只有year\_month\_day在其默认输出格式中使用连字符)。例如:

\begin{shell}
2021/Jan/Mon[1]
\end{shell}

\mySubsubsection{11.1.4}{不同的时区}

让我们修改第一个示例程序,以使时区发挥作用。实际上,我们希望程序迭代一年中每个月的所有第一个星期一,并在不同的时区安排会议。

为此需要做以下修改:

\begin{itemize}
\item
迭代当前年份的所有月份

\item
会议安排在当地时间18:30

\item
使用其他时区输出会议时间
\end{itemize}

可以这样编写代码:

\filename{lib/chronotz.cpp}

\begin{cpp}
#include <chrono>
#include <iostream>

int main()
{
	namespace chr = std::chrono; // shortcut for std::chrono
	using namespace std::literals; // for min, h, y suffixes

	try {
		// initialize today as current local date:
		auto localNow = chr::current_zone()->to_local(chr::system_clock::now());
		chr::year_month_day today{chr::floor<chr::days>(localNow)};
		std::cout << "today: " << today << '\n';
		// for each first Monday of all months of the current year:
		auto first = today.year() / 1 / chr::Monday[1];
		for (auto d = first; d.year() == first.year(); d += chr::months{1}) {
			// print out the date:
			std::cout << d << '\n';

			// init and print 18:30 local time of those days:
			auto tp{chr::local_days{d} + 18h + 30min}; // no timezone
			std::cout << " tp: " << tp << '\n';

			// apply this local time to the current timezone:
			chr::zoned_time timeLocal{chr::current_zone(), tp}; // local time
			std::cout << " local: " << timeLocal << '\n';

			// print out date with other timezones:
			chr::zoned_time timeUkraine{"Europe/Kiev", timeLocal}; // Ukraine time
			chr::zoned_time timeUSWest{"America/Los_Angeles", timeLocal}; // Pacific time
			std::cout << " Ukraine: " << timeUkraine << '\n';
			std::cout << " Pacific: " << timeUSWest << '\n';
		}
	}
	catch (const std::exception& e) {
		std::cerr << "EXCEPTION: " << e.what() << '\n';
	}
}
\end{cpp}

该计划将于2021年在欧洲启动,结果如下:

\begin{shell}
today: 2021-03-29
2021/Jan/Mon[1]
    tp:      2021-01-04 18:30:00
    local:   2021-01-04 18:30:00 CET
    Ukraine: 2021-01-04 19:30:00 EET
    Pacific: 2021-01-04 09:30:00 PST
2021/Feb/Mon[1]
    tp:      2021-02-01 18:30:00
    local:   2021-02-01 18:30:00 CET
    Ukraine: 2021-02-01 19:30:00 EET
    Pacific: 2021-02-01 09:30:00 PST
2021/Mar/Mon[1]
    tp:      2021-03-01 18:30:00
    local:   2021-03-01 18:30:00 CET
    Ukraine: 2021-03-01 19:30:00 EET
    Pacific: 2021-03-01 09:30:00 PST
2021/Apr/Mon[1]
    tp:      2021-04-05 18:30:00
    local:   2021-04-05 18:30:00 CEST
    Ukraine: 2021-04-05 19:30:00 EEST
    Pacific: 2021-04-05 09:30:00 PDT
2021/May/Mon[1]
    tp:      2021-05-03 18:30:00
    local:   2021-05-03 18:30:00 CEST
    Ukraine: 2021-05-03 19:30:00 EEST
    Pacific: 2021-05-03 09:30:00 PDT
...
2021/Oct/Mon[1]
    tp:      2021-10-04 18:30:00
    local:   2021-10-04 18:30:00 CEST
    Ukraine: 2021-10-04 19:30:00 EEST
    Pacific: 2021-10-04 09:30:00 PDT
2021/Nov/Mon[1]
    tp:      2021-11-01 18:30:00
    local:   2021-11-01 18:30:00 CET
    Ukraine: 2021-11-01 19:30:00 EET
    Pacific: 2021-11-01 10:30:00 PDT
2021/Dec/Mon[1]
    tp:      2021-12-06 18:30:00
    local:   2021-12-06 18:30:00 CET
    Ukraine: 2021-12-06 19:30:00 EET
    Pacific: 2021-12-06 09:30:00 PST
\end{shell}

看看10月和11月的输出:在洛杉矶,会议现在安排在不同的时间,尽管使用了相同的时区PDT。这是因为会议时间的来源(中欧)从夏季改为冬季/标准时间。

再一次,让我们一步步地了解一下这个程序的改进。

\mySamllsection{应对今天}

第一个新语句是将today初始化为year\_month\_day类型的对象:

\begin{cpp}
auto localNow = chr::current_zone()->to_local(chr::system_clock::now());
chr::year_month_day today = chr::floor<chr::days>(localNow);
\end{cpp}

从C++11开始就提供了对std::chrono::system\_clock::now()的支持,会产生一个具有系统时钟粒度的std::chrono::time\_point<>。这个系统时钟使用UTC(从C++20开始,system\_clock保证使用Unix时间,基于UTC)。所以首先必须将当前的UTC时间和日期调整为当前/本地时区的时间和日期,current\_zone()->to\_local()就是这样做的。否则,本地日期可能与UTC日期不匹配(因为已经过了午夜,而UTC还没有过,或者相反)。

直接使用localNow初始化year\_month\_day值不会编译,因为这会“窄化”值(丢失值的小时、分钟、秒和次秒部分)。通过使用floor()这样的函数(C++17起可用),可以根据所要求的粒度将值向下舍入。

若需要根据UTC的当前日期,以下内容就足够了:

\begin{cpp}
chr::year_month_day today = chr::floor<chr::days>(chr::system_clock::now());
\end{cpp}

\mySamllsection{本地日期及时间}

同样,将迭代的天数与一天中的特定时间结合起来。但这一次,首先会将每一天转换为类型std::chrono::local\_days:

\begin{cpp}
auto tp{chr::local_days{d} + 18h + 30min};
\end{cpp}

std::chrono::local\_days是time\_point<local\_t, days>的快捷方式。这里使用了伪时钟std::chrono::local\_t,所以有一个本地时间点,这个时间点还没有关联时区(甚至还没有UTC)。

下一个语句将本地时间点与当前时区结合起来,这样就得到了一个特定于时区的时间点,类型为std::chrono::zoned\_time<>:

\begin{cpp}
chr::zoned_time timeLocal{chr::current_zone(), tp}; // local time
\end{cpp}

时间点已经与系统时钟相关联,则已经将时间与UTC相关联。将这样的时间点与不同的时区组合将时间转换为特定的时区。

默认输出操作符演示了时间点和分区时间之间的差异:

\begin{cpp}
auto tpLocal{chr::local_days{d} + 18h + 30min}; // local timepoint
std::cout << "timepoint: " << tpLocal << '\n';

chr::zoned_time timeLocal{chr::current_zone(), tpLocal}; // apply to local timezone
std::cout << "zonedtime: " << timeLocal << '\n';
\end{cpp}

这段代码可能的输出为:

\begin{shell}
timepoint: 2021-01-04 18:30:00
zonedtime: 2021-01-04 18:30:00 CET
\end{shell}

输出的时间点没有时区,而经过时区处理的时间有时区。但两个输出的时间相同,因为我们将本地时间点使用的是本地时区。

事实上,时间点和分区时间的区别如下:

\begin{itemize}
\item
一个时间点可能与一个已定义的纪元相关联,可以定义一个独特的时间点,但它也可能与一个未定义或伪epoch相关联,直到将其与时区结合起来,其含义才清楚。

\item
分区时间总是与时区相关联,以便epoch最终具有已定义的含义。epoch总是代表一个唯一的时间点。
\end{itemize}

使用std::chrono::sys\_days,而非std::chrono::local\_days时会发生什么:

\begin{cpp}
auto tpSys{chr::sys_days{d} + 18h + 30min}; // system timepoint
std::cout << "timepoint: " << tpSys << '\n';

chr::zoned_time timeSys{chr::current_zone(), tpSys}; // convert to local timezone
std::cout << "zonedtime: " << timeSys << '\n';
\end{cpp}

这里使用一个系统时间点,有一个相关联的时区UTC。当本地时间点应用于时区时,将系统时间点转换为时区。所以当在一个时差一小时的时区运行程序时,可得到以下输出:

\begin{shell}
timepoint: 2021-01-04 18:30:00
zonedtime: 2021-01-04 19:30:00 CET
\end{shell}

\mySamllsection{使用其他时区}

最后,使用不同的时区打印时间点,一个用于乌克兰经过基辅,另一个用于北美太平洋时区经过洛杉矶:

\begin{cpp}
chr::zoned_time timeUkraine{"Europe/Kiev", timeLocal}; // Ukraine time
chr::zoned_time timeUSWest{"America/Los_Angeles", timeLocal}; // Pacific time
std::cout << " Ukraine: " << timeUkraine << '\n';
std::cout << " Pacific: " << timeUSWest << '\n';
\end{cpp}

要指定时区,必须使用IANA时区数据库的官方时区名称,该名称通常基于代表时区的城市。时区缩写(如PST)可能会在一年中发生变化,或者适用于不同的时区。

使用这些对象的默认输出操作符添加相应的时区缩写,无论它在“冬令时”:

\begin{shell}
local:   2021-01-04 18:30:00 CET
Ukraine: 2021-01-04 19:30:00 EET
Pacific: 2021-01-04 09:30:00 PST
\end{shell}

或者在“夏令时”:

\begin{shell}
local:   2021-07-05 18:30:00 CEST
Ukraine: 2021-07-05 19:30:00 EEST
Pacific: 2021-07-05 09:30:00 PDT
\end{shell}

有时候,有些时区是夏令时,而有些则不是。例如,在11月初,美国有夏令时,但在乌克兰没有。

\begin{shell}
local:   2021-11-01 18:30:00 CET
Ukraine: 2021-11-01 19:30:00 EET
Pacific: 2021-11-01 10:30:00 PDT
\end{shell}

\mySamllsection{不支持相应时区时}

C++可以在小型系统上存在并发挥作用,甚至可以在烤面包机上,IANA时区数据库的可用性将耗费太多的资源,所以不需要存在时区数据库中。

若不支持时区数据库,则所有特定于时区的调用都会抛出异常,所以在可移植的C++程序中使用时区时,应该可以捕捉到异常:

\begin{cpp}
try {
	// initialize today as current local date:
	auto localNow = chr::current_zone()->to_local(chr::system_clock::now());
	...
}
catch (const std::exception& e) {
	std::cerr << "EXCEPTION: " << e.what() << '\n'; // IANA timezone DB missing
}
\end{cpp}

注意,对于Visual C++,这也适用于较旧的Windows系统,所需的操作系统支持不能低于Windows 10。此外,平台可以在不使用所有IANA时区数据库的情况下,支持时区API。

例如,在德语版本的Windows 11安装中,我得到的输出是GMT-8和GMT-7,而不是PST和PDT。















