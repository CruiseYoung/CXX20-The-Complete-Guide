
本节列出迭代器和范围的所有基本概念,这些概念在算法和类似函数中很有用。

注意,范围的概念是在命名空间std::ranges中提供的,不是在std中。并在头文件<ranges>中声明。

迭代器的概念在头文件<iterator>中声明。

\mySubsubsection{5.3.1}{范围和视图的概念}

定义了几个概念来约束范围,其与迭代器的概念相对应,并处理了自C++20以来新的迭代器类别。

\mySamllsection{std::ranges::range<Rg>}

\begin{itemize}
\item
保证Rg是一个有效的范围。

\item
这个概念在头文件<compare>中定义。

\item
Rg类型的对象支持通过使用std::ranges::begin()和std::ranges::end()对元素进行迭代。

若范围是一个数组,或者提供begin()和end()成员,或者可以与独立的begin()和end()函数一起使用,就会出现这种情况。

\item
此外,对于std::ranges::begin()和std::ranges::end(),适用以下约束:

\begin{itemize}
\item
必须在(平摊的)常数时间内运行。

\item
不修改范围。

\item
begin()在多次调用时产生相同的位置(除非该范围至少不提供前向迭代器)。
\end{itemize}

可以以良好的性能迭代所有元素(甚至多次迭代,除非有纯输入迭代器)。

\item
要求:

\begin{itemize}
\item
对于Rg类型的对象rg,支持std::ranges::begin(rg)和std::ranges::end(rg)
\end{itemize}
\end{itemize}

\mySamllsection{std::ranges::output\_range<Rg, T>}

\begin{itemize}
\item
保证Rg至少可提供接受T类型值的输出迭代器(可以用来编写的迭代器)的范围。

\item
要求:

\begin{itemize}
\item
需要满足std::range<Rg>

\item
迭代器类型满足std::output\_iterator和T
\end{itemize}
\end{itemize}


\mySamllsection{std::ranges::input\_range<Rg>}

\begin{itemize}
\item
保证Rg是一个至少提供输入迭代器(可用于读取的迭代器)的范围。

\item
要求:

\begin{itemize}
\item
需要满足std::range<Rg>

\item
迭代器类型满足std::input\_iterator
\end{itemize}
\end{itemize}

\mySamllsection{std::ranges::forward\_range<Rg>}

\begin{itemize}
\item
保证Rg是一个至少提供前向迭代器(可用于读写和多次迭代的迭代器)的范围。

\item
注意,迭代器的iterator\_category成员可能不匹配。对于产生左值的迭代器,它是std::input\_iterator\_tag(若可用)。

\item
要求:

\begin{itemize}
\item
需要满足std::input\_range<Rg>

\item
迭代器类型满足std::forward\_iterator
\end{itemize}
\end{itemize}

\mySamllsection{std::ranges::bidirectional\_range<Rg>}

\begin{itemize}
\item
保证Rg是一个至少提供双向迭代器的范围(可以用于读写和向后迭代的迭代器)。

\item
注意,迭代器的iterator\_category成员可能不匹配。对于产生左值的迭代器,它是std::input\_iterator\_tag(若可用)。

\item
要求:

\begin{itemize}
\item
需要满足std::forward\_range<Rg>

\item
迭代器类型满足std::bidirectional\_iterator
\end{itemize}
\end{itemize}

\mySamllsection{std::ranges::random\_access\_range<Rg>}

\begin{itemize}
\item
保证Rg是一个提供随机访问迭代器(可用于读写、前后跳转和计算距离的迭代器)的范围。

\item
注意,迭代器的iterator\_category成员可能不匹配。对于产生左值的迭代器,它是std::input\_iterator\_tag(若可用)。

\item
要求:

\begin{itemize}
\item
需要满足std::bidirectional\_range<Rg>

\item
迭代器类型满足std::random\_access\_iterator
\end{itemize}
\end{itemize}

\mySamllsection{std::ranges::contiguous\_range<Rg>}

\begin{itemize}
\item
保证Rg是一个范围,该范围提供随机访问迭代器,并附加约束元素存储在连续内存中。

\item
注意,迭代器的iterator\_category成员不匹配。若定义了类别成员,则只有std::random\_access\_iterator\_tag,若值是右值,甚至只有std::input\_iterator\_tag。

\item
要求:

\begin{itemize}
\item
需要满足std::random\_access\_range<Rg>

\item
迭代器类型满足std::contiguous\_iterator

\item
调用std::ranges::data()将生成指向第一个元素的裸指针
\end{itemize}
\end{itemize}

\mySamllsection{std::ranges::sized\_range<Rg>}

\begin{itemize}
\item
保证Rg是可以在常量时间内计算出元素数量的范围(通过调用成员size()或通过计算开始和结束之间的差值)。

\item
若满足这个概念,std::ranges::size()对于Rg类型的对象是快速且定义良好的。

\item
注意,这个概念的性能方面是一个语义约束,不能在编译时检查。为了表明一个类型即使提供了size()也不满足这个概念,可以定义std::disable\_size\_range<Rg>产生true[disable\_size\_range的真正含义是“忽略size\_range的size()”。]。

\item
要求:

\begin{itemize}
\item
需要满足std::range<Rg>

\item
支持std::ranges::size()
\end{itemize}
\end{itemize}

\mySamllsection{std::ranges::common\_range<Rg>}

\begin{itemize}
\item
保证Rg是一个范围,其中开始迭代器和哨兵(结束迭代器)具有相同的类型。

\item
保证总是由:

\begin{itemize}
\item
所有标准容器(vector, list等)

\item
empty\_view

\item
single\_view

\item
common\_view
\end{itemize}

保证不是由:

\begin{itemize}
\item
查看视图

\item
删除const视图

\item
没有最终值或最终值为不同类型的iota视图
\end{itemize}

对于其他视图,其取决于底层范围的类型。

\item
要求:

\begin{itemize}
\item
需要满足std::range<Rg>

\item
std::ranges::iterator\_t<Rg>和std::ranges::sentinel\_t<Rg>具有相同类型
\end{itemize}
\end{itemize}

\mySamllsection{std::ranges::borrowed\_range<Rg>}

\begin{itemize}
\item
保证在当前上下文中传递的范围Rg,产生即使该范围不再存在也可以使用的迭代器。若传递左值或传递的范围始终是借来的范围,则满足该概念。

\item
若满足这个概念,则范围的迭代器与范围的生命周期无关。所以当创建迭代器的范围被销毁时,迭代器不能挂起。但若范围的迭代器指向基础范围,并且基础范围不再存在,则其仍然可以悬挂。

\item
形式上,若Rg是左值(例如有名称的对象),或者变量模板std::ranges::enable\_borrow\_range<Rg>为真,则满足该概念,以下视图都是如此:subrange、ref\_view、string\_view、span、iota\_view和empty\_view。

\item
要求:

\begin{itemize}
\item
需要满足std::range<Rg>

\item
Rg是左值或enable\_borrow\_range<Rg>产生true
\end{itemize}
\end{itemize}

\mySamllsection{std::ranges::view<Rg>}

\begin{itemize}
\item
保证Rg是一个视图(复制、移动、赋值和销毁代价很低的范围)。

\item
视图有以下要求:[最初,C++20要求视图也应该是默认可构造的,并且销毁应该是低开销的,但\url{http://wg21.link/p2325r3}和\url{http://wg21.link/p2415r2}删除了这些要求。]

\begin{itemize}
\item
必须是一个范围(支持对元素的迭代)。

\item
必须可移动。

\item
移动构造函数和复制构造函数(若可用)必须具有恒定的复杂性。

\item
移动赋值运算符和复制赋值运算符(若可用的话)都必须是低开销的(保持不变的复杂度,或者不比销毁加创建的复杂度更低)。
\end{itemize}

除最后一个要求外,所有要求都通过相应的概念进行检查。最后一个要求是语义约束,必须由类型的实现者通过从std::ranges::view\_interface公开派生或特化std::ranges::enable\_view<Rg>生成true来保证。

\item
要求:

\begin{itemize}
\item
需要满足std::range<Rg>

\item
需要满足std::movable<Rg>

\item
变量模板std::ranges::enable\_view<Rg>为true
\end{itemize}
\end{itemize}

\mySamllsection{std::ranges::viewable\_range<Rg>}

\begin{itemize}
\item
保证Rg是一个可以使用std::views::all()适配器安全地转换为视图的范围。

\item
若Rg已经是一个视图或一个范围的左值,或一个可移动的右值,或一个范围但不是初始化列表,则满足这个概念。[对于仅移动视图类型的左值,即使满足viewable\_range,也存在all()格式错误的问题。]

\item
要求:

\begin{itemize}
\item
需要满足std::range<Rg>

\end{itemize}
\end{itemize}

\mySubsubsection{5.3.2}{类指针对象的概念}

本节列出了对象的所有标准概念,可以使用操作符*来处理它们所指向的值。这通常适用于裸指针、智能指针和迭代器,所以这些概念通常用作处理迭代器和算法的概念的基本约束。

注意,这些概念不需要支持操作符->。

这些概念在头文件<iterator>中声明。


\mySamllsection{基本概念的间接支持}

\mySamllsection{std::indirectly\_writable<P, Val>}

\begin{itemize}
\item
保证P是一个类似指针的对象,支持操作符*来赋值Val。
	
\item
满足非const裸指针、智能指针和迭代器的要求,提供Val可以赋值给P所指的位置。
\end{itemize}


\mySamllsection{std::indirectly\_readable<P>}

\begin{itemize}
\item
保证P是一个类似指针的对象,支持*的读访问方式。

\item
满足裸指针,智能指针和迭代器。


\item
要求:
\begin{itemize}
\item
结果值对于const和非const对象具有相同的引用类型(排除了std::optional<>),这确保了P的constness不会传播到所指向的位置(当操作符返回对成员的引用时通常不会出现这种情况)。

\item
std::iter\_value\_t<P>必须有效,该类型不必支持操作符->。
\end{itemize}
\end{itemize}

\mySamllsection{间接可读对象的概念}

对于间接可读的类指针概念,可以检查其他的约束:

\mySamllsection{std::indirectly\_movable<InP,OutP>}

\begin{itemize}
\item
保证InP的值可以直接赋值给OutP。

\item
有了这个概念,以下代码是有效的:

\begin{cpp}
void foo(InP inPos, OutP, outPos) {
	*outPos = std::move(*inPos);
}
\end{cpp}

\item
要求:
\begin{itemize}
\item
需要满足std::indirectly\_readable<InP>

\item
std::indirectly\_writable满足从InP值到OutP的右值引用
\end{itemize}
\end{itemize}

\mySamllsection{std::indirectly\_movable\_storable<InP,OutP>}

\begin{itemize}
\item
保证InP的值可以间接赋值给OutP,即使在使用OutP指向的类型的(临时)对象时也是如此。

\item
有了这个概念,以下代码是有效的:

\begin{cpp}
void foo(InP inPos, OutP, outPos) {
	OutP::value_type tmp = std::move(*inPos);
	*outPos = std::move(tmp);
}
\end{cpp}

\item
要求:
\begin{itemize}
\item
需要满足std::indirectly\_movable<InP, OutP>

\item
std::indirectly\_writable满足OutP所引用对象的InP值

\item
std::movable满足InP所引用的值

\item
InP引用的右值是可复制/移动的、可构造的和可赋值的
\end{itemize}
\end{itemize}

\mySamllsection{std::indirectly\_copyable<InP, OutP>}

\begin{itemize}
\item
保证InP的值可以直接赋值给OutP。

\item
有了这个概念,以下代码是有效的:

\begin{cpp}
void foo(InP inPos, OutP outPos) {
	*outPos = *inPos;
}
\end{cpp}

\item
要求:
\begin{itemize}
\item
需要满足std::indirectly\_readable<InP>

\item
std::indirectly\_writable满足于对InP值的引用(OutP值)可赋值
\end{itemize}
\end{itemize}

\mySamllsection{std::indirectly\_copyable\_storable<InP, OutP>}

\begin{itemize}
\item
保证InP的值可以间接赋值给OutP,即使在使用OutP指向类型的(临时)对象时也是如此。

\item
有了这个概念,以下代码是有效的:

\begin{cpp}
void foo(InP inPos, OutP outPos) {
	OutP::value_type tmp = *inPos;
	*outPos = tmp;
}
\end{cpp}

\item
要求:
\begin{itemize}
\item
需要满足std::indirectly\_copyable<InP, OutP> 

\item
std::indirectly\_writable满足InP值对OutP所引用对象的const左值和右值引用

\item
std::copyable满足InP引用的值

\item
InP引用的值是可复制/移动的、可构造的和可赋值的
\end{itemize}
\end{itemize}

\mySamllsection{std::indirectly\_swappable<P>, std::indirectly\_swappable<P1, P2>}

\begin{itemize}
\item
保证P或P1和P2的值可以交换(使用std::ranges::iter\_swap())。

\item
要求:
\begin{itemize}
\item
满足std::indirect\_readable<P1>(和std::indirect\_readable<P2>)

\item
对于任意两个类型为P1和P2的对象,支持std::ranges::iter\_swap()
\end{itemize}
\end{itemize}

\mySamllsection{std::indirectly\_comparable<P1, P2, Comp>, std::indirectly\_comparable<P1, P2, Comp, Proj1>, std::indirectly\_comparable<P1, P2, Comp, Proj1, Proj2>}

\begin{itemize}
\item
保证可以将元素(可选地用Proj1和Proj2转换)与P1和P2所引用的位置进行比较。

\item
要求:
\begin{itemize}
\item
Comp满足std::indirect\_binary\_predicate

\item
std::projected<P1, Proj1>和std::projected<P2, Proj2>满足std::identity为默认投影
\end{itemize}
\end{itemize}

\mySubsubsection{5.3.3}{迭代器的概念}

本节列出了需要不同类型迭代器的概念。其处理的是自C++20起的新迭代器类别,它们与范围的概念相对应。

这些概念在头文件<iterator>中提供。

\mySamllsection{std::input\_output\_iterator<Pos>}

\begin{itemize}
\item
保证Pos支持所有迭代器的基本接口:operator++和operator*,其中operator*必须引用一个值。

这个概念并不要求迭代器可复制(所以,这低于算法使用的迭代器的基本要求)。

\item
要求:
\begin{itemize}
\item
满足std::weakly\_incrementable<pos>

\item
操作符*产生了一个引用
\end{itemize}
\end{itemize}

\mySamllsection{std::output\_iterator<Pos, T>}

\begin{itemize}
\item
保证Pos是一个输出迭代器(可以为元素赋值的迭代器),可以为其赋类型为typeT的值。

\item
Pos类型的迭代器可用于赋值T类型的值val:

\begin{cpp}
*i++ = val;
\end{cpp}

\item
这些迭代器只适用于单遍迭代。

\item
要求:
\begin{itemize}
\item
满足std::input\_or\_output\_iterator<Pos>

\item
满足std::indirectly\_writable<Pos, I>
\end{itemize}
\end{itemize}

\mySamllsection{std::input\_iterator<Pos>}

\begin{itemize}
\item
保证Pos是一个输入迭代器(可以从元素中读取值的迭代器)。

\item
若std::forward\_iterator也不满足,这些迭代器只适用于单通迭代。

\item
要求:
\begin{itemize}
\item
满足std::input\_or\_output\_iterator<Pos>

\item
满足std::indirectly\_readable<Pos>

\item
Pos有一个派生自std::input\_iterator\_tag的迭代器类别
\end{itemize}
\end{itemize}

\mySamllsection{std::forward\_iterator<Pos>}

\begin{itemize}
\item
确保Pos是一个前向迭代器(一个读取迭代器,可以对元素进行多次前向迭代)。

\item
注意,迭代器的iterator\_category成员可能不匹配。对于产生左值的迭代器,是std::input\_iterator\_tag(若可用)。

\item
要求:
\begin{itemize}
\item
满足std::input\_iterator<Pos>

\item
满足std::incrementable<Pos>

\item
Pos有一个派生自std::forward\_iterator\_tag的迭代器类别
\end{itemize}
\end{itemize}

\mySamllsection{std::bidirectional\_iterator<Pos>}

\begin{itemize}
\item
确保Pos是一个双向迭代器(一个读取迭代器,可以使用它在元素上向前或向后迭代多次)。

\item
注意,迭代器的iterator\_category成员可能不匹配。对于产生左值的迭代器,是std::input\_iterator\_tag(若可用)。

\item
要求:
\begin{itemize}
\item
满足std::forward\_iterator<Pos>

\item
支持使用操作符向后迭代-{}-

\item
Pos有一个派生自std::bidirectional\_iterator\_tag的迭代器类别
\end{itemize}
\end{itemize}

\mySamllsection{std::random\_access\_iterator<Pos>}

\begin{itemize}
\item
确保Pos是一个随机访问迭代器(可以在元素上来回跳转的读取迭代器)。

\item
注意,迭代器的iterator\_category成员可能不匹配。对于产生左值的迭代器,是std::input\_iterator\_tag(若可用)。

\item
要求:
\begin{itemize}
\item
满足std::bidirectional\_iterator<Pos> 

\item
满足std::totally\_ordered<Pos> 

\item
满足std::sized\_sentinel\_for<Pos, Pos>

\item
支持+, +=, -, -=, [] 操作符

\item
Pos有一个派生自std::random\_access\_iterator\_tag的迭代器类别
\end{itemize}
\end{itemize}

\mySamllsection{std::contiguous\_iterator<Pos>}

\begin{itemize}
\item
保证Pos是迭代器,迭代连续内存中的元素。

\item
注意,迭代器的iterator\_category成员不匹配。若定义了类别成员,则只有std::random\_access\_iterator\_tag,若值是右值,甚至只有std::input\_iterator\_tag。

\item
要求:
\begin{itemize}
\item
满足std::random\_access\_iterator<Pos> 

\item
Pos有一个派生自std::consecuous\_iterator\_tag的迭代器类别

\item
元素的to\_address()是指向该元素的裸指针
\end{itemize}
\end{itemize}

\mySamllsection{std::sentinel\_for<S, Pos>}

\begin{itemize}
\item
保证S可以用作Pos的哨兵(可能是不同类型的结束迭代器)。

\item
要求:
\begin{itemize}
\item
满足std::semiregular<S>

\item
满足std::input\_or\_output\_iterator<Pos> 

\item
可以使用操作符==和!=比较Pos和S
\end{itemize}
\end{itemize}

\mySamllsection{std::sized\_sentinel\_for<S, Pos>}

\begin{itemize}
\item
保证S可以用作Pos的哨兵(可能是不同类型的结束迭代器),并且可以在常数时间内计算它们之间的距离。

\item
要表示不能在常数时间内计算距离(尽管可以计算距离),可以定义std::disable\_size\_sentinel\_for<Rg>的结果为true。

\item
要求:
\begin{itemize}
\item
满足std::sentinel\_for<S, Pos> 

\item
对Pos和S调用操作符减号,将产生迭代器不同类型的值

\item
std::disable\_size\_sentinel\_for<S, Pos>未定义为返回true
\end{itemize}
\end{itemize}

\mySubsubsection{5.3.4}{算法的迭代器概念}

\mySamllsection{std::permutable<Pos>}

\begin{itemize}
\item
保证您可以使用操作符++向前迭代,并通过移动和交换元素来重新排序

\item
要求:
\begin{itemize}
\item
满足std::forward\_iterator<Pos> 

\item
满足std::indirectly\_movable\_storable<Pos>

\item
满足std::indirectly\_swappable<Pos>
\end{itemize}
\end{itemize}

\mySamllsection{std::mergeable<Pos1, Pos2, ToPos>, \\
std::mergeable<Pos1, Pos2, ToPos, Comp>, \\
std::mergeable<Pos1, Pos2, ToPos, Comp, Proj1>, \\
std::mergeable<Pos1, Pos2, ToPos, Comp, Proj1, Proj2>,}

\begin{itemize}
\item
保证可以通过将两个排序序列的元素复制到ToPos所引用的序列中,从而将其合并到Pos1和Pos2所引用的位置。顺序由运算符<或Comp定义(应用于可选地用投影Proj1和Proj2转换的值)。

\item
要求:
\begin{itemize}
\item
Pos1和Pos2满足std::input\_iterator

\item
满足std::weakly\_incrementable<ToPos>

\item
Pos1和Pos2满足std::indirect\_copyable<PosN, ToPos>

\item
Comp满足std::indirect\_strict\_weak\_order,以及满足std::projected<Pos1, Proj1>和std::projected<Pos2, Proj2>(以<作为默认比较,std::identity作为默认投影):

\begin{itemize}
\item
Pos1和Pos2满足std::indirectly\_readable

\item
满足std::copy\_constructible<Comp>

\item
Comp\&和(投影的)值/引用类型满足std::strict\_weak\_order<Comp>
\end{itemize}
\end{itemize}
\end{itemize}

\mySamllsection{std::sortable<Pos>
	std::sortable<Pos, Comp>
	std::sortable<Pos, Comp, Proj>}

\begin{itemize}
\item
保证可以使用操作符<或Comp对迭代器Pos引用的元素进行排序(在可选地对值应用投影Proj之后)。

\item
要求:
\begin{itemize}
\item
满足std::permutable<Pos>

\item
std::indirect\_strict\_weak\_order的Comp和(投影)值满足(以<作为默认比较,std::identity作为默认投影):

\begin{itemize}
\item
满足std::indirectly\_readable<Pos>

\item
满足std::copy\_constructible<Comp>

\item
Comp\&和(投影的)值/引用类型满足std::strict\_weak\_order<Comp>
\end{itemize}
\end{itemize}
\end{itemize}

























