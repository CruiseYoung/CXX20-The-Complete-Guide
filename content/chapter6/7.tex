第一个C++标准采用Standard Template Llibrary(该标准引入了一对迭代器作为容器/集合的抽象,以便在算法中处理)之后,一直有关于如何处理单个范围对象,而不是传递begin迭代器和end迭代器的讨论。Boost的Range库和Adobe源库(ASL)是最早提出具体库的两种方法。

2005年,Thorsten Ottosen在\url{http://wg21.link/n1871}上提出了将“范围”作为C++标准的第一个建议。Eric Niebler花了数年时间(得到了许多人的支持)推动这一进程,并于2014年由Eric Niebler、Sean Parent和Andrew Sutton在\url{http://wg21.link/n4128}上提出了另一项提案(该文档包含了许多关键设计决策的基本原理)。因此,2015年10月,范围技术规范以\url{http://wg21.link/n4560}起点。

范围库最终通过,并将范围的TS合并到由Eric Niebler、Casey Carter和Christopher Di Bella在\url{http://wg21.link/p0896r4}中提出的C++标准中。

采用之后,针对C++20的一些建议、论文,甚至是缺陷改变了重要的方面,特别是视图方面。例如,由Barry Revzin, Tim Song和Nicolai Josuttis提出的\url{http://wg21.link/p2210r2}(修复分割视图),\url{http://wg21.link/p2325r3}(修复视图的定义),\url{http://wg21.link/p2415r2}(引入右值范围的拥有视图),和\url{http://wg21.link/p2432r1}(修复istream视图)。

此外,视图的几个const问题中的一些可能会在C++23中通过\url{http://wg21.link/p2278r4}修复。希望厂商能在C++23之前提供这个修复。否则,C++20的代码可能与C++23不兼容。