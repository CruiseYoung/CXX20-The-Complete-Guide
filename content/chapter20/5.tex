
一些情况下,了解两个类型或指向类型的指针是否可以安全地相互转换很重要。C++标准使用术语“布局兼容”来表示这一点。

为了检查类成员之间的布局兼容关系,C++20还引入了两个新的普通函数,可以在运行时上下文中使用。

\mySubsubsection{20.5.1}{is\_layout\_compatible\_v<>}

\mySamllsection{std::is\_layout\_compatible\_v<T1, T2>}

生成类型T1和T2是否与布局兼容,以便可以使用reinterpret\_cast安全地转换指向它们的指针。

例如:

\begin{cpp}
struct Data {
	int i;
	const std::string s;
};

class Type {
	private:
	const int id = nextId();
	std::string name;
	public:
	...
};

std::is_layout_compatible_v<Data, Type> // true
\end{cpp}

对于布局兼容性来说,仅仅是类型和位的粗略匹配完全不够。根据语言规则:

\begin{itemize}
\item
有符号类型永远不会与无符号类型的布局兼容。

\item
char甚至不兼容有符号char和无符号char的布局。

\item
引用与非引用的布局永不兼容。

\item
不同(甚至是布局兼容)类型的数组永远不与布局兼容。

\item
枚举类型永远不会与其基础类型的布局兼容。
\end{itemize}

例如:

\begin{cpp}
enum class E {};
enum class F : int {};

std::is_layout_compatible_v<E, F> // true
std::is_layout_compatible_v<E[2], F[2]> // false
std::is_layout_compatible_v<E, int> // false

std::is_layout_compatible_v<char, char> // true
std::is_layout_compatible_v<char, signed char> // false
std::is_layout_compatible_v<char, unsigned char> // false
std::is_layout_compatible_v<char, char&> // false
\end{cpp}

\mySubsubsection{20.5.2}{is\_pointer\_interconvertible\_base\_of\_v<>}

\mySamllsection{std::is\_pointer\_interconvertible\_base\_of<Base, Der>}

若使用reinterpret\_cast可以安全地将指向Der类型的指针,并转换为指向其基类型base的指针,则返回true。

若两者具有相同的类型,则该特性返回true。

例如:

\begin{cpp}
struct B1 { };
struct D1 : B1 { int x; };

struct B2 { int x; };
struct D2 : B2 { int y; }; // no standard-layout type

std::is_pointer_interconvertible_base_of_v<B1, D1> // true
std::is_pointer_interconvertible_base_of_v<B2, D2> // false
\end{cpp}

指向D2的指针不能安全地转换为指向B2的指针,原因是它不是标准布局类型,因为并非所有成员都定义在具有相同访问权限的同一类中。

\mySubsubsection{20.5.3}{is\_corresponding\_member()}

\begin{cpp}
template<typename S1, typename S2, typename M1, typename M2>
constexpr bool is_corresponding_member(M1 S1::*m1, M2 S2::*m2) noexcept;
\end{cpp}

返回m1和m2是否指向S1和S2的布局兼容成员,所以这些成员和这些成员前面的所有成员必须布局兼容。当且仅当S1和S2是标准布局类型,M1和M2是对象类型,且M1和M2不为空时,函数返回true。

例如:

\begin{cpp}
struct Point2D { int a; int b; };
struct Point3D { int x; int y; int z; };
struct Type1 { const int id; int val; std::string name; };
struct Type2 { unsigned int id; int val; };

std::is_corresponding_member(&Point2D::b, &Point3D::y) // true
std::is_corresponding_member(&Point2D::b, &Point3D::z) // false (2nd versus 3rd int)
std::is_corresponding_member(&Point2D::b, &Type1::val) // true
std::is_corresponding_member(&Point2D::b, &Type2::val) // false (signed vs. unsigned)
\end{cpp}

\mySubsubsection{20.5.4}{is\_pointer\_interconvertible\_with\_class()}

\begin{cpp}
template<typename S, typename M>
constexpr bool is_pointer_interconvertible_with_class(M S::*m) noexcept;
\end{cpp}

返回类型s的每个对象,s是否与其子对象的s指针可互换。当且仅当S是标准布局类型,M是对象类型,且M不为空时,该函数返回true。

例如:

\begin{cpp}
struct B1 { int x; };
struct B2 { int y; };

struct DB1 : B1 {};
struct DB1B2 : B1, B2 {}; // not a standard-layout type

std::is_pointer_interconvertible_with_class<B1, int>(&DB1::x) // true
std::is_pointer_interconvertible_with_class<DB1, int>(&B1::x) // true
std::is_pointer_interconvertible_with_class<DB1, int>(&DB1::x) // true

std::is_pointer_interconvertible_with_class<B1, int>(&DB1B2::x) // true
std::is_pointer_interconvertible_with_class<DB1B2, int>(&B1::x) // false
std::is_pointer_interconvertible_with_class<DB1B2, int>(&DB1B2::x) // false
\end{cpp}

C++20标准在注释中解释了后两个语句为false的原因:

指针到成员表达式\&C::b的类型,并不总是指向C成员的指针。当将这些函数与继承结合使用时,会导致令人惊讶的结果:

\begin{cpp}
struct A { int a; }; // a standard-layout class
struct B { int b; }; // a standard-layout class
struct C: public A, public B { }; // not a standard-layout class

std::is_pointer_interconvertible_with_class(&C::b) // true
// true because, despite its appearance, &C::b has type
// “pointer to member of B of type int”

std::is_pointer_interconvertible_with_class<C>(&C::b) // false
// false because it forces the use of class C and fails
\end{cpp}





