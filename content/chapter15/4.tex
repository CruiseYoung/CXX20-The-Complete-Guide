
类型std::coroutine\_handle<>是协程句柄的泛型类型,可用于引用正在执行或挂起的协程。其模板参数是协程的promise类型,可用于放置额外的数据成员和行为。

“std::coroutine\_handle<>的API”表列出了为协程句柄提供的所有操作。

% Please add the following required packages to your document preamble:
% \usepackage{longtable}
% Note: It may be necessary to compile the document several times to get a multi-page table to line up properly
\begin{longtable}[c]{|l|l|}
\hline
\textbf{操作}                                                                     & \textbf{效果}                                                                        \\ \hline
\endfirsthead
%
\endhead
%
coroutine\_handle\textless{}PrmT\textgreater{}::from\_promise(prm)                     & 用promise prm创建一个句柄                                                  \\ \hline
CoroHandleType\{\}                                                                     & 创建无协程的句柄                                                       \\ \hline
CoroHandleType\{nullptr\}                                                              & 创建无协程的句柄                                                      \\ \hline
CoroHandleType\{hdl\}                                                                  & Copies 句柄hdl(两者都引用相同的协程)                               \\ \hline
hdl = hdl2                                                                             & 分配句柄hdl2(两者都引用相同的协程)                             \\ \hline
if (hdl)                                                                               & Yields 句柄是否引用协程                                    \\ \hline
==, !=                                                                                 & Checks 两个句柄是否引用同一个协程                                 \\ \hline
\textless{}, \textless{}=, \textgreater{}, \textgreater{}=, \textless{}=\textgreater{} & 创建协程句柄之间的顺序                                             \\ \hline
hdl.resume()                                                                           & 恢复协程                                                                 \\ \hline
hdl()                                                                                  & 恢复协程                                                               \\ \hline
hdl.done()                                                                             & Yields 挂起的协程是否已结束,并且不再允许使用resume() \\ \hline
hdl.destroy()                                                                          & 销毁协程                                                                \\ \hline
hdl.promise()                                                                          & Yields 协程的promise                                                    \\ \hline
hdl.address()                                                                          & Yields 协程数据的内部地址                                     \\ \hline
coroutine\_handle\textless{}PrmT\textgreater{}::from\_address(addr)                    & Yields 句柄的地址addr                                             \\ \hline
\end{longtable}

\begin{center}
表15.2 std::coroutine\_handle<>的API
\end{center}

静态成员函数from\_promise()方法初始化一个协同程序处理提供了协同程序。该函数只是将promise的地址存储在句柄中,若有一个promise prm:

\begin{cpp}
auto hdl = std::coroutine_handle<decltype(prm)>::from_promise(prm);
\end{cpp}

from\_promise()通常在get\_return\_object()中调用,用于协程框架创建的promise:

\begin{cpp}
class CoroIf {
	public:
	struct promise_type {
		auto get_return_object() { // init and return the coroutine interface
			return CoroIf{std::coroutine_handle<promise_type>::from_promise(*this)};
		}
		...
	};
...
};
\end{cpp}

当句柄被默认初始化或nullptr用作初始值或赋值时,协程句柄不引用任何协程。这种情况下,对布尔值的转换都会产生false:

\begin{cpp}
std::coroutine_handle<PrmType> hdl = nullptr;

if (hdl) ... // false
\end{cpp}

复制和分配协程句柄的成本很低,只是复制并赋值内部指针。协程句柄通常是按值传递的,所以多个协程句柄可以引用同一个协程。开发者必须确保当另一个协程句柄恢复或销毁该协程时,该句柄永远不会调用resume()或destroy()。

address()接口产生协程的内部指针为void*。这允许开发者将协程句柄导出到某个地方,然后使用from\_address()的静态函数重新创建句柄:

\begin{cpp}
om_address():
auto hdl = std::coroutine_handle<decltype(prm)>::from_promise(prm);
...
void* hdlPtr = hdl.address();
...
auto hdl2 = std::coroutine_handle<decltype(prm)>::from_address(hdlPtr);
hdl == hdl2 // true
\end{cpp}

只要协程存在,就可以使用该地址。在一个协程销毁后,该地址可以让另一个协程句柄重用。

\mySubsubsection{15.4.1}{std::coroutine\_handle<void>}

所有协程句柄类型都有到类std::coroutine<void>的隐式类型转换(可以跳过此声明中的void)

[原始的C++标准指定所有协程句柄类型都派生自std::coroutine<void>,但\url{http://wg21.link/lwg3460}改变了这一点。感谢黄永祥指出这一点。]

\begin{cpp}
namespace std {
	template<typename Promise>
	struct coroutine_handle {
		...
		// implicit conversion to coroutine_handle<void>:
		constexpr operator coroutine_handle<>() const noexcept;
		...
	};
}
\end{cpp}

若不需要promise,可以使用类型std::coroutine\_handle<void>或类型std::coroutine\_handle<>,而不需要模板参数。注意std::coroutine\_handle<>没有提供promise()成员函数:

\begin{cpp}
void callResume(std::coroutine_handle<> h)
{
	h.resume(); // OK
	h.promise(); // ERROR: no member promise() provided for h
}

auto hdl = std::coroutine_handle<decltype(prm)>::from_promise(prm);
...
callResume(hdl); // OK: hdl converts to std::coroutine_handle<void>
\end{cpp}








