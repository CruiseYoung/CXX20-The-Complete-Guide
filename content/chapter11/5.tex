
已经了解了chrono库的新特性和新类型,本节将讨论如何在实践中使用它们。

更多的例子请参见\url{http://github.com/HowardHinnant/date/wiki/Examples-and-Recipes}。

\mySubsubsection{11.5.1}{无效日期}

日历类型的值可能无效。这有两种情况:

\begin{itemize}
\item 
使用无效值进行初始化。例如:

\begin{cpp}
std::chrono::day d{0}; // invalid day
std::chrono::year_month ym{2021y/13}; // invalid year_month
std::chrono::year_month_day ymd{2021y/2/31}; // invalid year_month_day
\end{cpp}

\item 
通过导致无效日期的计算。例如:

\begin{cpp}
auto ymd1 = std::chrono::year{2021}/1/31; // January 31, 2021
ymd1 += std::chrono::months{1}; // February 31, 2021 (invalid)

auto ymd0 = std::chrono::year{2020}/2/29; // February 29, 2020
ymd1 += std::chrono::years{1}; // February 29, 2021 (invalid)
\end{cpp}
\end{itemize}

表“标准日期属性的有效值”列出了日期的内部类型和不同属性的可能值。

% Please add the following required packages to your document preamble:
% \usepackage{longtable}
% Note: It may be necessary to compile the document several times to get a multi-page table to line up properly
\begin{longtable}[c]{|l|l|l|}
\hline
\textbf{属性} & \textbf{内部类型} & \textbf{有效值}                                                  \\ \hline
\endfirsthead
%
\endhead
%
Day           & unsigned char & 1 到 31         \\ \hline
Month         & unsigned char & 1 到 12         \\ \hline
Year          & short         & -32767 到 32767 \\ \hline
Weekday            & unsigned char          & 0(周日) 到 6(周六) 和 7 (还是周日),转换为0\\ \hline
Weekday index & unsigned char & 1 到 5          \\ \hline
\end{longtable}

\begin{center}
表11.10 标准日期属性的有效值
\end{center}

若单个组件有效,则所有组合类型都有效(例如,若月份和工作日都有效,则有效的month\_weekday有效),但需要进行以下检查:

\begin{itemize}
\item 
必须存在类型为year\_month\_day的完整日期,这里考虑了闰年。例如:

\begin{cpp}
2020y/2/29; // valid (there is a February 29 in 2020)
2021y/2/29; // invalid (there is no February 29 in 2021)
\end{cpp}

\item 
month\_day只有当该日在当月有效时才有效。2月29是有效的,30是无效的。

例如:

\begin{cpp}
February/29; // valid (a February can have 29 days)
February/30; // invalid (no February can have 30 days)
\end{cpp}

\item 
year\_month\_weekday仅当weekday索引能够在一年中的指定月份中存在时才有效。

例如:

\begin{cpp}
2020y/1/Thursday[5]; // valid (there is a fifth Thursday in January 2020)
2020y/1/Sunday[5]; // invalid (there is no fifth Sunday in January 2020)
\end{cpp}
\end{itemize}

每个日历类型都提供一个成员函数ok()来检查值是否有效,默认输出操作符表示无效日期。

处理无效日期的方式取决于编程逻辑。对于一个月中创建的天数过高的典型场景,有以下选项:

\begin{itemize}
\item 
四舍五入到每月的最后一天:

\begin{cpp}
auto ymd = std::chrono::year{2021}/1/31;
ymd += std::chrono::months{1};
if (!ymd.ok()) {
	ymd = ymd.year()/ymd.month()/std::chrono::last; // February 28, 2021
}
\end{cpp}

注意,右边的表达式创建了year\_month\_last,然后将其转换为year\_month\_day类型。

\item 
四舍五入到下个月的第一天:

\begin{cpp}
auto ymd = std::chrono::year{2021}/1/31;
ymd += std::chrono::months{1};
if (!ymd.ok()) {
	ymd = ymd.year()/ymd.month()/1 + std::chrono::months{1}; // March 1, 2021
}
\end{cpp}

不要只在月份上加1,因为对于12月,则创建了一个无效的月份。

\item
根据所有超过的天数进行四舍五入:

\begin{cpp}
auto ymd = std::chrono::year{2021}/1/31;
ymd += std::chrono::months{1}; // March 3, 2021
if (!ymd.ok()) {
	ymd = std::chrono::sys_days(ymd);
}
\end{cpp}

这使用了year\_month\_day转换的特殊功能,其中所有溢出的天数都被逻辑地添加到下一个月。然而,这只适用于几天(不能以这种方式添加1000天)。
\end{itemize}

若日期不是一个有效值,默认输出格式将用“不是一个有效类型”发出信号。例如:

\begin{cpp}
std::chrono::day d{0}; // invalid day
std::chrono::year_month_day ymd{2021y/2/31}; // invalid year_month_day
std::cout << "day: " << d << '\n';
std::cout << "ymd: " << ymd << '\n';
\end{cpp}

这段代码将输出:[注意,C++20标准中的规范在这里有点不一致,目前正在修复。例如,它声明对于无效的year\_month\_days“不是一个有效日期”是输出]。

\begin{shell}
day: 00 is not a valid day
ymd: 2021-02-31 is not a valid year_month_day
\end{shell}

当使用格式化输出的默认格式(仅为\{\})时,也会发生同样的情况。通过使用特定的转换说明符,可以禁用“not a valid”输出(在银行或季度处理软件中,有时甚至使用6月31日之类的日期):

\begin{cpp}
std::chrono::year_month_day ymd{2021y/2/31};
std::cout << ymd << '\n'; // “2021-02-31 is not a valid year_month_day”
std::cout << std::format("{:%F}\n", ymd); // “2021-02-31”
std::cout << std::format("{:%Y-%m-%d}\n", ymd); // “2021-02-31”
\end{cpp}

\mySubsubsection{11.5.2}{处理月份和年份}

因为std::chrono::months和std::chrono::years不是天数的整数倍,所以在使用时必须非常小心。

\begin{itemize}
\item 
对于仅具有年份值的标准类型(month, year, year\_month, year\_month\_day, year\_month\_weekday\_last,…),向日期添加特定的月份或年份数时可以正常工作。

\item
对于整个日期只有一个值的标准时间点类型(time\_point、sys\_time、sys\_seconds和sys\_days),将相应的平均分数周期添加到日期中,这可能不会产生期望的日期。 
\end{itemize}

例如,看看在2020年12月31日之前增加四个月或四年的不同影响:

\begin{itemize}
\item 
当处理year\_month\_day时:

\begin{cpp}
chr::year_month_day ymd0 = chr::year{2020}/12/31;
auto ymd1 = ymd0 + chr::months{4}; // OOPS: April 31, 2021
auto ymd2 = ymd0 + chr::years{4}; // OK: December 31, 2024

std::cout << "ymd: " << ymd0 << '\n'; // 2020-12-31
std::cout << " +4months: " << ymd1 << '\n'; // 2021-04-31 is not a valid ...
std::cout << " +4years: " << ymd2 << '\n'; // 2024-12-31
\end{cpp}

\item 
当处理year\_month\_day\_last时:

\begin{cpp}
chr::year_month_day_last yml0 = chr::year{2020}/12/chr::last;
auto yml1 = yml0 + chr::months{4}; // OK: last day of April 2021
auto yml2 = yml0 + chr::years{4}; // OK: last day of Dec. 2024

std::cout << "yml: " << yml0 << '\n'; // 2020/Dec/last
std::cout << " +4months: " << yml1 << '\n'; // 2021/Apr/last
std::cout << " as date: "
		  << chr::sys_days{yml1} << '\n'; // 2021-04-30
std::cout << " +4years: " << yml2 << '\n'; // 2024/Dec/last
\end{cpp}

\item 
当处理sys\_days时:

\begin{cpp}
chr::sys_days day0 = chr::year{2020}/12/31;
auto day1 = day0 + chr::months{4}; // OOPS: May 1, 2021 17:56:24
auto day2 = day0 + chr::years{4}; // OOPS: Dec. 30, 2024 23:16:48

std::cout << "day: " << day0 << '\n'; // 2020-12-31
std::cout << " with time: "
		  << chr::sys_seconds{day0} << '\n'; // 2020-12-31 00:00:00
std::cout << " +4months: " << day1 << '\n'; // 2021-05-01 17:56:24
std::cout << " +4years: " << day2 << '\n'; // 2024-12-30 23:16:48
\end{cpp}

支持此特性的唯一目的是,允许对诸如物理或生物过程之类的事物进行建模,这些过程不关心人类日历(天气、妊娠期等)的复杂性,并且不应将此特性用于其他目的。
\end{itemize}

注意,输出值和默认输出格式都是不同的。这是一个明显的迹象,表明使用了不同的类型:

\begin{itemize}
\item 
当向日历类型添加月份或年份时,这些类型处理正确的逻辑日期(同一天或下个月或年份的最后一天)。请注意,这可能会导致无效日期,例如4月31日,默认输出操作符甚至在其输出中发出如下信号:[由于一些不一致,无效日期的确切格式可能与C++20标准不匹配,该标准在这里指定“不是有效日期”。]

\begin{shell}
2021-04-31 is not a valid year_month_day
\end{shell}

\item
当在类型为std::chrono::sys\_days的日期上添加月或年时,结果不是类型为sys\_days。与往常一样,在chrono库中,结果具有能够表示结果的最佳类型:

\begin{itemize}
\item 
加上月份会得到一个单位为54秒的类型。

\item 
加上年份将得到一个单位为216秒的类型。
\end{itemize}

这两个单位都是秒的倍数,们可以用作std::chrono::sys\_seconds。默认情况下,相应的输出操作符以秒为单位打印日期和时间,这不是相同的日期和后一个月或年的时间。两个时间点都不再是一个月的31号或最后一天,时间也不再是午夜:

\begin{shell}
2021-05-01 17:56:24
2024-12-30 23:16:48
\end{shell}

\end{itemize}

两者都是有用的,但使用带有时间点的月份和年份通常只适用于计算许多月份和/或年份之后的大致日期。

\mySubsubsection{11.5.3}{解析时间点和时间段}

若有一个时间点或时间段,可以访问不同的字段,如下程序所示:

\filename{lib/chronoattr.cpp}

\begin{cpp}
#include <chrono>
#include <iostream>

int main()
{
	auto now = std::chrono::system_clock::now(); // type is sys_time<>
	auto today = std::chrono::floor<std::chrono::days>(now); // type is sys_days
	std::chrono::year_month_day ymd{today};
	std::chrono::hh_mm_ss hms{now - today};
	std::chrono::weekday wd{today};
	
	std::cout << "now: " << now << '\n';
	std::cout << "today: " << today << '\n';
	std::cout << "ymd: " << ymd << '\n';
	std::cout << "hms: " << hms << '\n';
	std::cout << "year: " << ymd.year() << '\n';
	std::cout << "month: " << ymd.month() << '\n';
	std::cout << "day: " << ymd.day() << '\n';
	std::cout << "hours: " << hms.hours() << '\n';
	std::cout << "minutes: " << hms.minutes() << '\n';
	std::cout << "seconds: " << hms.seconds() << '\n';
	std::cout << "subsecs: " << hms.subseconds() << '\n';
	std::cout << "weekday: " << wd << '\n';
	
	try {
		std::chrono::sys_info info{std::chrono::current_zone()->get_info(now)};
		std::cout << "timezone: " << info.abbrev << '\n';
	}
	catch (const std::exception& e) {
		std::cerr << "no timezone database: (" << e.what() << ")\n";
	}
}
\end{cpp}

The output of the program might be, for example, as follows:

\begin{shell}
now:      2021-04-02 13:37:34.059858000
today:    2021-04-02
ymd:      2021-04-02
hms:      13:37:34.059858000
year:     2021
month:    Apr
day:      02
hours:    13h
minutes:  37min
seconds:  34s
subsecs:  59858000ns
weekday:  Fri
timezone: CEST
\end{shell}

函数now()产生一个与系统时钟粒度相同的时间点:

\begin{cpp}
auto now = std::chrono::system_clock::now();
\end{cpp}

结果具有std::chrono::sys\_time<>类型和一些特定于实现的解析时间段类型。

若要处理本地时间,必须实现以下几点:

\begin{cpp}
auto tpLoc = std::chrono::zoned_time{std::chrono::current_zone(),
										std::chrono::system_clock::now()
									}.get_local_time();
\end{cpp}

为了处理时间点的日期部分,需要按天数粒度:

\begin{cpp}
auto today = std::chrono::floor<std::chrono::days>(now);
\end{cpp}

现在初始化的变量的类型是std::chrono::sys\_days,可以把它赋值给std::chrono::year\_month\_day类型的对象,这样就可以用相应的成员函数访问year、month和day:

\begin{cpp}
std::chrono::year_month_day ymd{today};
std::cout << "year: " << ymd.year() << '\n';
std::cout << "month: " << ymd.month() << '\n';
std::cout << "day: " << ymd.day() << '\n';
\end{cpp}

为了处理时间点的时间部分,需要一个时间段,当计算原始时间点和午夜值之间的差值时,就会得到这个时间段。我们使用duration来初始化hh\_mm\_ss对象:

\begin{cpp}
std::chrono::hh_mm_ss hms{now - today};
\end{cpp}

从这里,可以直接得到小时、分钟、秒和亚秒:

\begin{cpp}
std::cout << "hours: " << hms.hours() << '\n';
std::cout << "minutes: " << hms.minutes() << '\n';
std::cout << "seconds: " << hms.seconds() << '\n';
std::cout << "subsecs: " << hms.subseconds() << '\n';
\end{cpp}

对于亚秒,hh\_mm\_ss决定必要的粒度并使用适当的单位。例子中,其输出为纳秒:

\begin{shell}
hms:      13:37:34.059858000
hours:    13h
minutes:  37min
seconds:  34s
subsecs:  59858000ns
\end{shell}

对于工作日,只需要用粒度日类型初始化即可。这适用于sys\_days、local\_days和year\_month\_day(后者可隐式转换为std::sys\_days):

\begin{cpp}
std::chrono::weekday wd{today}; // OK (today has day granularity)
std::chrono::weekday wd{ymd}; // OK due to implicit conversion to sys_days
\end{cpp}

对于时区,必须将时间点(这里是现在)与时区(这里是当前时区)结合起来。生成的std::chrono::sys\_info对象包含如下信息:

\begin{cpp}
std::chrono::sys_info info{std::chrono::current_zone()->get_info(now)};
std::cout << "timezone: " << info.abbrev << '\n';
\end{cpp}

注意,并非每个C++平台都支持时区数据库。在某些系统上,这部分代码可能会抛出异常。




















