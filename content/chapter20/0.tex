本章将介绍C++20在其标准库中引入的几个类型特征(以及两个用于类型的底层函数)。

“新类型特征”表列出了C++20的这些新类型特征(都定义在命名空间std中)。


% Please add the following required packages to your document preamble:
% \usepackage{longtable}
% Note: It may be necessary to compile the document several times to get a multi-page table to line up properly
\begin{longtable}[c]{|l|l|}
\hline
\textbf{特征}                                                                                                                 & \textbf{效果}                                                                                                           \\ \hline
\endfirsthead
%
\endhead
%
is\_bounded\_array\_v\textless{}T\textgreater{}                                                                                & 若类型T是已知范围的数组类型,则返回true                                                                  \\ \hline
is\_unbounded\_array\_v\textless{}T\textgreater{}                                                                              & 若类型T是范围未知的数组类型,则返回true                                                                \\ \hline
is\_nothrow\_convertible\_v\textless{}T, T2\textgreater{}                                                                      & 若类型T可转换为类型T2而不抛出,则返回true                                                          \\ \hline
is\_layout\_compatible\_v\textless{}T1, T2\textgreater{}                                                                       & 若T1与类型T2的布局兼容,则返回true                                                                       \\ \hline
\begin{tabular}[c]{@{}l@{}}is\_pointer\_interconvertible...\\   \_base\_of\_v\textless{}BaseT, DerT\textgreater{}\end{tabular} & \begin{tabular}[c]{@{}l@{}} 若指向DerT的指针可以安全地转换为指向其基类型 \\ BaseT的指针,则返回true\end{tabular}                              \\ \hline
remove\_cvref\_t\textless{}T\textgreater{}                                                                                     & 产生类型T,没有引用,const, volatile                                                                \\ \hline
unwrap\_reference\_t\textless{}T\textgreater{}                                                                                 & \begin{tabular}[c]{@{}l@{}}若类型T是std::reference\_wrapper\textless{}\textgreater,则返回T的包 \\ 装类型,否则为T     \end{tabular}           \\ \hline
unwrap\_ref\_decay\_t\textless{}T\textgreater{}                                                                                & \begin{tabular}[c]{@{}l@{}}若类型T是std::reference\_wrapper\textless{}\textgreater,则返回T的包 \\ 装类型,否则返会T的衰退类型 \end{tabular} \\ \hline
common\_reference\_t\textless{}T...\textgreater{}                                                                              & 产生通用类型pf所有类型T…可以为其赋值                                                  \\ \hline
type\_identity\_t\textless{}T\textgreater{}                                                                                    & 生成类型T                                                                                                    \\ \hline
iter\_difference\_t\textless{}T\textgreater{}                                                                                  & 生成可增量/迭代器类型的差异类型 T                                                           \\ \hline
iter\_value\_t\textless{}T\textgreater{}                                                                                       & 返回指针/迭代器类型T的值/元素类型                                                             \\ \hline
iter\_reference\_t\textless{}T\textgreater{}                                                                                   & 生成指针/迭代器类型的引用类型 T                                                                  \\ \hline
iter\_rvalue\_reference\_t\textless{}T\textgreater{}                                                                           & 生成指针/迭代器类型T的右值引用类型                                                           \\ \hline
is\_clock\_v\textless{}T\textgreater{}                                                                                         & 若T是时钟类型,则返回true                                                                                          \\ \hline
compare\_three\_way\_result\_t\textless{}T\textgreater{}                                                                       & 使用操作符\textless{}=\textgreater{}比较两个值的类型。                                          \\ \hline
\end{longtable}

\begin{center}
表20.1 新类型特征
\end{center}

以下章节将详细讨论这些特征,除了以下特征:

\begin{itemize}
\item 
std::is\_clock\_v<>将在关于新计时特性的部分中讨论。

\item 
std::compare\_three\_way\_result\_t<>将在有关新三路比较的一节中讨论。
\end{itemize}

另外请注意,类型特性std::is\_pod<>在C++20中已弃用。












