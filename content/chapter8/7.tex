
本节讨论所有改变元素顺序的视图(到目前为止,只有一个视图)。

\mySubsubsection{8.7.1}{反向视图}

% Please add the following required packages to your document preamble:
% \usepackage{longtable}
% Note: It may be necessary to compile the document several times to get a multi-page table to line up properly
\begin{longtable}[c]{|l|l|}
\hline
\textbf{类型:}              & std::ranges::reverse\_view\textless{}\textgreater{}                 \\ \hline
\endfirsthead
%
\endhead
%
\textbf{内容:}              & 一个范围内的所有元素按倒序排列 \\ \hline
\textbf{适配器:}              & std::views::reverse                      \\ \hline
\textbf{元素类型:}         & 与传递的范围类型相同                \\ \hline
\textbf{要求:}             & 至少为双向范围             \\ \hline
\textbf{类别:}             & 与传递的相同,但至多是随机访问 \\ \hline
\textbf{是否是长度范围:}       & 若在常长度范围内                        \\ \hline
\textbf{是否是通用范围:}      & 总是                                   \\ \hline
\textbf{是否是租借范围:} & 若是租借视图或左值非视图                           \\ \hline
\textbf{缓存:}            & 缓存begin(),除非通用范围或随机访问和长度范围 \\ \hline
\textbf{常量可迭代:}       & 若在可重复的通用范围内        \\ \hline
\textbf{传播常量性:} & 只有在右值范围内                  \\ \hline
\end{longtable}

类模板std::ranges::reverse\_view<>定义了一个以相反顺序遍历基础范围元素的视图。

例如:

\begin{cpp}
std::vector<int> rg{1, 2, 3, 4, 5, 6, 7, 8, 9, 10, 11, 12, 13};
...
for (const auto& elem : std::ranges::reverse_view{rg}) {
	std::cout << elem << ' ';
}
\end{cpp}

循环输出:

\begin{shell}
13 12 11 10 9 8 7 6 5 4 3 2 1
\end{shell}

\mySamllsection{范围适配器的反向视图}

反向视图也可以(通常应该)使用范围适配器创建:

\begin{cpp}
std::views::reverse(rg)
\end{cpp}

例如:

\begin{cpp}
for (const auto& elem : std::views::reverse(rg)) {
	std::cout << elem << ' ';
}
\end{cpp}

或:

\begin{cpp}
for (const auto& elem : rg | std::views::reverse) {
	std::cout << elem << ' ';
}
\end{cpp}

注意,适配器可能并不总是产生reverse\_view:

\begin{itemize}
\item
反向反转的范围会生成原始范围。

\item
若传递反向范围,则返回原始子范围和相应的非反向迭代器。
\end{itemize}

反向视图在内部存储传递的范围(可选择以与all()相同的方式转换为视图),所以只有在传递的范围有效的情况下才有效(除非传递了右值,则在内部使用了归属视图)。

迭代器只是传递范围的反向迭代器。

下面是一个使用反向视图的完整示例:

\filename{ranges/reverseview.cpp}

\begin{cpp}
include <iostream>
#include <string>
#include <vector>
#include <ranges>

void print(std::ranges::input_range auto&& coll)
{
	for (const auto& elem : coll) {
		std::cout << elem << ' ';
	}
	std::cout << '\n';
}

int main()
{
	std::vector coll{1, 2, 3, 4, 1, 2, 3, 4};
	
	print(coll); // 1 2 3 4 1 2 3 4
	print(std::ranges::reverse_view{coll}); // 4 3 2 1 4 3 2 1
	print(coll | std::views::reverse); // 4 3 2 1 4 3 2 1
}
\end{cpp}

该程序有以下输出:

\begin{shell}
1 2 3 4 1 2 3 4
4 3 2 1 4 3 2 1
4 3 2 1 4 3 2 1
\end{shell}

\mySamllsection{反向视图和缓存}

为了获得更好的性能,反向视图会在视图中缓存begin()的结果(除非该范围只是一个输入范围)。

注意,对于具有随机访问的范围(例如,数组,向量和队列),缓存的开始偏移量与视图一起复制。否则,不复制缓存的开头。

当反向视图引用的范围被修改时,缓存可能会产生功能后果。

例如:

\filename{ranges/reversecache.cpp}

\begin{cpp}
#include <iostream>
#include <vector>
#include <list>
#include <ranges>

void print(auto&& coll)
{
	for (const auto& elem : coll) {
		std::cout << elem << ' ';
	}
	std::cout << '\n';
}

int main()
{
	std::vector vec{1, 2, 3, 4};
	std::list lst{1, 2, 3, 4};
	
	auto vVec = vec | std::views::take(3) | std::views::reverse;
	auto vLst = lst | std::views::take(3) | std::views::reverse;
	print(vVec); // OK: 3 2 1
	print(vLst); // OK: 3 2 1
	
	// insert a new element at the front (=> 0 1 2 3 4)
	vec.insert(vec.begin(), 0);
	lst.insert(lst.begin(), 0);
	print(vVec); // OK: 2 1 0
	print(vLst); // OOPS: 3 2 1
	
	// creating a copy heals:
	auto vVec2 = vVec;
	auto vLst2 = vLst;
	print(vVec2); // OK: 2 1 0
	print(vLst2); // OK: 2 1 0
}
\end{cpp}

请注意,从形式上讲,将视图复制到vector会创建未定义的行为,因为C++标准没有指定如何进行缓存。由于vector的重新分配会使所有迭代器失效,因此缓存的迭代器将失效。但对于随机访问范围,视图通常缓存偏移量,而不是迭代器。因此,对于vector,视图仍然有效。

还需要注意的是,对整个容器的反向视图可以正常工作,因为结束迭代器是缓存的。

根据经验,\textbf{在底层范围被修改后不要使用一个反向视图的begin()}。

\mySamllsection{反向视图和const}

注意,不能总是迭代const反向视图。实际上,引用的范围必须是通用范围。

例如:

\begin{cpp}
void printElems(const auto& coll) {
	for (const auto elem& e : coll) {
		std::cout << elem << '\n';
	}
}

std::vector vec{1, 2, 3, 4, 5};

// leading odd elements of vec:
auto vecFirstOdd = std::views::take_while(vec, [](auto x) {
						return x % 2 != 0;
					});
					
printElems(vec | std::views::reverse); // OK
printElems(vecFirstOdd); // OK
printElems(vecFirstOdd | std::views::reverse); // ERROR
\end{cpp}

要在泛型代码中支持这个视图,必须使用通用/转发引用:

\begin{cpp}
void printElems(auto&& coll) {
	...
}

std::vector vec{1, 2, 3, 4, 5};

// leading odd elements of vec:
auto vecFirstOdd = std::views::take_while(vec, [](auto x) {
							return x % 2 != 0;
						});

printElems(vecFirstOdd | std::views::reverse); // OK
\end{cpp}


\mySamllsection{反向视图的接口}

“类std::ranges::reverse\_view<>的操作”表列出了反向视图的API。

% Please add the following required packages to your document preamble:
% \usepackage{longtable}
% Note: It may be necessary to compile the document several times to get a multi-page table to line up properly
\begin{longtable}[c]{|l|l|}
\hline
\textbf{操作}    & \textbf{效果}                                                \\ \hline
\endfirsthead
%
\endhead
%
reverse\_view r\{\} & 创建一个指向默认构造范围的reverse\_view                                 \\ \hline
reverse\_view r\{rg\} & 创建一个指向范围rg的reverse\_view                \\ \hline
r.begin()             & 生成begin迭代器                                      \\ \hline
r.end()               & 生成哨兵(end迭代器)                             \\ \hline
r.empty()             & 生成的r是否为空(若范围支持,则可用) \\ \hline
if (r)                & 若r不为空,则为true(若定义了empty(),则可用)       \\ \hline
r.size()            & 生成元素的数量(若引用一个长度范围,则可用)                              \\ \hline
r.front()             & 生成的第一个元素(若转发可用)              \\ \hline
r.back()              & 生成的最后一个元素(若范围为双向且通用,则可用) \\ \hline
r{[}idx{]}            & 生成的第n个元素(若随机访问可用)            \\ \hline
r.data()            & 生成一个指向元素内存的原始指针(若元素在连续内存中可用) \\ \hline
r.base()              & 产生对r所引用或归属范围的引用       \\ \hline
\end{longtable}

\begin{center}
表8.21 类std::ranges::reverse\_view<>的操作
\end{center}

