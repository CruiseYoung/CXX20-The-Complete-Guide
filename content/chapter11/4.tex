
C++20提供了对所有计时类型的直接输出和解析的新支持。

\mySubsubsection{11.4.1}{默认输出格式}

对于所有的chrono类型,标准输出操作符从C++20开始定义。不仅输出值,而且使用适当的格式和单位,依赖于地区的格式化也是可能的。

所有日历类型都会输出带有日历类型输出格式的值。

所有时间段类型都以单位类型输出值,如表中时间段的输出单位所示,这适用于任何字面符操作符。

% Please add the following required packages to your document preamble:
% \usepackage{longtable}
% Note: It may be necessary to compile the document several times to get a multi-page table to line up properly
\begin{longtable}[c]{|l|l|}
\hline
\textbf{单位}                          & \textbf{输出后缀} \\ \hline
\endfirsthead
%
\endhead
%
atto                                   & as                     \\ \hline
femto                                  & fs                     \\ \hline
pico                                   & ps                     \\ \hline
nano                                   & ns                     \\ \hline
micro                                  & μs 或 us               \\ \hline
milli                                  & ms                     \\ \hline
centi                                  & cs                     \\ \hline
deci                                   & ds                     \\ \hline
ratio\textless{}1\textgreater{}        & s                      \\ \hline
deca                                   & das                    \\ \hline
hecto                                  & hs                     \\ \hline
kilo                                   & ks                     \\ \hline
mega                                   & Ms                     \\ \hline
giga                                   & Gs                     \\ \hline
tera                                   & Ts                     \\ \hline
peta                                   & Ps                     \\ \hline
exa                                    & Es                     \\ \hline
ratio\textless{}60\textgreater{}       & min                    \\ \hline
ratio\textless{}3600\textgreater{}     & h                      \\ \hline
ratio\textless{}86400\textgreater{}    & d                      \\ \hline
ratio\textless{}num, 1\textgreater{}   & {[}num{]}s             \\ \hline
ratio\textless{}num, den\textgreater{} & {[}num/den{]}s         \\ \hline
\end{longtable}

\begin{center}
表11.8 时间段的输出单位
\end{center}

对于所有标准时间点类型,输出操作符以以下格式输出日期和时间(可选):

\begin{itemize}
\item
用于可隐式转换为天数的整数粒度单位的“年-月-日”

\item
年-月-日\ 时:分:秒,表示小于等于天的整数粒度单位
\end{itemize}

若时间点值(成员rep)的类型为浮点类型,则不定义输出操作符。[这个漏洞有望在C++23中修复。]

对于时间部分,使用类型为hh\_mm\_ss的输出运算符,这对应于格式化输出的\%F \%T转换说明符。

例如:

\begin{cpp}
auto tpSys = std::chrono::system_clock::now();
std::cout << tpSys << '\n'; // 2021-04-25 13:37:02.936314000

auto tpL = chr::zoned_time{chr::current_zone(), tpSys}.get_local_time();
std::cout << tpL; // 2021-04-25 15:37:02.936314000
std::cout << chr::floor<chr::milliseconds>(tpL); // 2021-04-25 15:37:02.936
std::cout << chr::floor<chr::seconds>(tpL); // 2021-04-25 15:37:02
std::cout << chr::floor<chr::minutes>(tpL); // 2021-04-25 15:37:00
std::cout << chr::floor<chr::days>(tpL); // 2021-04-25
std::cout << chr::floor<chr::weeks>(tpL); // 2021-04-22

auto tp3 = std::chrono::floor<chr::duration<long long, std::ratio<1, 3>>>(tpSys);
std::cout << tp3 << '\n'; // 2021-04-25 13:37:02.666666

chr::sys_time<chr::duration<double, std::milli>> tpD{tpSys};
std::cout << tpD << '\n'; // ERROR: no output operator defined

std::chrono::gps_seconds tpGPS;
std::cout << tpGPS << '\n'; // 1980-01-06 00:00:00

auto tpStd = std::chrono::steady_clock::now();
std::cout << "tpStd: " << tpStd; // ERROR: no output operator defined
\end{cpp}

zoned\_time<>类型输出类似于timepoint类型,扩展为缩写的时区名称。以下时区缩写用于标准时钟:

\begin{itemize}
\item
UTC有sys\_clock, utc\_clock和file\_clock

\item
TAI有tai\_clock

\item
GPS有gps\_clock
\end{itemize}

sys\_info和local\_info具有未定义格式的输出操作符,只用于调试。

\mySubsubsection{11.4.2}{格式化输出}

chrono支持格式化输出的新库,所以可以对std::format()和std::format\_to()的参数使用日期/时间类型。

例如:

\begin{cpp}
auto t0 = std::chrono::system_clock::now();
...
auto t1 = std::chrono::system_clock::now();

std::cout << std::format("From {} to {}\nit took {}\n", t0, t1, t1-t0);
\end{cpp}

使用日期/时间类型的默认输出格式。例如,可能有以下输出:

\begin{shell}
From 2021-04-03 15:21:33.197859000 to 2021-04-03 15:21:34.686544000
it took 1488685000ns
\end{shell}

从20121-04-03 15:21:33.197859000到20121-04-03 15:21:34.686544000,耗时1488685000ns

\begin{cpp}
std::cout << std::format("From {:%T} to {:%T} it took {:%S}s\n", t0, t1, t1-t0);
\end{cpp}

这可能会有以下输出:

\begin{shell}
From 15:21:34.686544000 to 15:21:34.686544000 it took 01.488685000s
\end{shell}

chrono类型的格式说明符是标准格式说明符语法的一部分(每个说明符都可选):

\begin{shell}
fill align width .prec L spec
\end{shell}

\begin{itemize}
\item
fill, align, with 和 prec 表示与标准格式说明符相同。

\item
spec 指定格式化的通用符号,以\%开头。

\item
L 与往常一样,还为支持它的说明符打开语言环境相关的格式。
\end{itemize}

表“计时类型的转换”说明符列出了日期/时间类型格式化输出的所有转换说明符,示例基于Sunday, June 9, 2019 17:33:16和850毫秒。

若不使用特定的转换说明符,则使用默认输出操作符,该操作符标记无效日期。当使用特定的转换说明符时,情况并非如此:

\begin{cpp}
std::chrono::year_month_day ymd{2021y/2/31}; // February 31, 2021
std::cout << std::format("{}", ymd); // 2021-02-31 is not a valid ...
std::cout << std::format("{:%F}", ymd); // 2021-02-31
std::cout << std::format("{:%Y-%m-%d}", ymd); // 2021-02-31
\end{cpp}

若日期/时间值类型没有为转换说明符提供必要的信息,则抛出std::format\_error异常:

\begin{itemize}
\item
年份指定符用于month\_day

\item
工作日指定符用于指定时间段

\item
输出月份或工作日名称,并且该值无效

\item
时区说明符用于本地时间点
\end{itemize}

此外,请注意以下关于转换说明符的事项:

\begin{itemize}
\item
负值时间段或hh\_mm\_ss值在整个值前面打印负号。例如:

\begin{cpp}
std::cout << std::format("{:%H:%M:%S}", -10000s); // outputs: -02:46:40
\end{cpp}

% Please add the following required packages to your document preamble:
% \usepackage{longtable}
% Note: It may be necessary to compile the document several times to get a multi-page table to line up properly
\begin{longtable}[c]{|lll|}
\hline
\multicolumn{1}{|l|}{\textbf{Spec.}} & \multicolumn{1}{l|}{\textbf{例子}}  & \textbf{含义}                                                \\ \hline
\endfirsthead
%
\endhead
%
\multicolumn{1}{|l|}{\%c} & \multicolumn{1}{l|}{Sun Jun  9 17:33:16 2019} & 表示标准或地区的日期和时间              \\ \hline
\multicolumn{3}{|l|}{\textbf{日期:}}                                                                                                            \\ \hline
\multicolumn{1}{|l|}{\%x}            & \multicolumn{1}{l|}{06/09/19}          & 表示标准或地区的日期                        \\ \hline
\multicolumn{1}{|l|}{\%F}            & \multicolumn{1}{l|}{2019-06-09}        & 年-月-日,四位和两位数字                         \\ \hline
\multicolumn{1}{|l|}{\%D}            & \multicolumn{1}{l|}{06/09/19}          & 两位数的月/日/年                                  \\ \hline
\multicolumn{1}{|l|}{\%e}            & \multicolumn{1}{l|}{ 9}                & 个位数时,以空格开头的日期                          \\ \hline
\multicolumn{1}{|l|}{\%d}            & \multicolumn{1}{l|}{09}                & 两位数的日期                                                \\ \hline
\multicolumn{1}{|l|}{\%b}            & \multicolumn{1}{l|}{Jun}               & 标准或地区的缩写月名                     \\ \hline
\multicolumn{1}{|l|}{\%h}            & \multicolumn{1}{l|}{Jun}               & 同上                                                              \\ \hline
\multicolumn{1}{|l|}{\%B}            & \multicolumn{1}{l|}{June}              & 标准或地区的月份名称                                 \\ \hline
\multicolumn{1}{|l|}{\%m}            & \multicolumn{1}{l|}{06}                & 两位数的月份                                           \\ \hline
\multicolumn{1}{|l|}{\%Y}            & \multicolumn{1}{l|}{2019}              & 四位数的年份                                           \\ \hline
\multicolumn{1}{|l|}{\%y}            & \multicolumn{1}{l|}{19}                & 两位数的年份,不显示世纪                              \\ \hline
\multicolumn{1}{|l|}{\%G}            & \multicolumn{1}{l|}{2019}              & ISO以周为基础的年份为四位数字(按\%V表示周)      \\ \hline
\multicolumn{1}{|l|}{\%g}            & \multicolumn{1}{l|}{19}                & iso以周为基础的年份为两位数字(按\%V表示周)     \\ \hline
\multicolumn{1}{|l|}{\%C}            & \multicolumn{1}{l|}{20}                & 两位数的世纪                                           \\ \hline
\multicolumn{3}{|l|}{\textbf{工作日和周:}}                                                                                              \\ \hline
\multicolumn{1}{|l|}{\%a}            & \multicolumn{1}{l|}{Sun}               & 标准或地区的工作日名称缩写                  \\ \hline
\multicolumn{1}{|l|}{\%A}            & \multicolumn{1}{l|}{Sunday}            & 标准或地区的工作日名称                               \\ \hline
\multicolumn{1}{|l|}{\%w}            & \multicolumn{1}{l|}{0}                 & 工作日为十进制数(星期日为0,星期六为6)     \\ \hline
\multicolumn{1}{|l|}{\%u}            & \multicolumn{1}{l|}{7}                 & 工作日为十进制数(星期一为1,星期天为7)      \\ \hline
\multicolumn{1}{|l|}{\%W}            & \multicolumn{1}{l|}{22}                & 一年中的第几周(00…(第01周从第一个星期一开始) \\ \hline
\multicolumn{1}{|l|}{\%U}            & \multicolumn{1}{l|}{23}                & 一年中的第几周(00…(第01周从第一个星期日开始)  \\ \hline
\multicolumn{1}{|l|}{\%V}            & \multicolumn{1}{l|}{23}                & 一年中的ISO星期(01…53、第一周为1月4日)         \\ \hline
\multicolumn{3}{|l|}{\textbf{时间:}}                                                                                                           \\ \hline
\multicolumn{1}{|l|}{\%X}            & \multicolumn{1}{l|}{17:33:16}          & 表示标准或地区的时间                        \\ \hline
\multicolumn{1}{|l|}{\%r}            & \multicolumn{1}{l|}{05:33:16 PM}       & 标准或当地的12小时时钟时间                         \\ \hline
\multicolumn{1}{|l|}{\%T} & \multicolumn{1}{l|}{17:33:16.850}             & 时:分:秒(根据需要依赖于区域设置的亚秒)     \\ \hline
\multicolumn{1}{|l|}{\%R}            & \multicolumn{1}{l|}{17:33}             & 时:分,两位数                               \\ \hline
\multicolumn{1}{|l|}{\%H}            & \multicolumn{1}{l|}{17}                & 24小时时钟,时间为两位数                                \\ \hline
\multicolumn{1}{|l|}{\%I}            & \multicolumn{1}{l|}{05}                & 12小时时钟,时间为两位数                                \\ \hline
\multicolumn{1}{|l|}{\%p}            & \multicolumn{1}{l|}{PM}                &按照12小时制是AM还是PM                         \\ \hline
\multicolumn{1}{|l|}{\%M}            & \multicolumn{1}{l|}{33}                & 分钟,两位数表示                                          \\ \hline
\multicolumn{1}{|l|}{\%S} & \multicolumn{1}{l|}{16.850}                   & 秒为十进制数(根据需要特定于本地的亚秒) \\ \hline
\multicolumn{3}{|l|}{\textbf{其他:}}                                                                                                           \\ \hline
\multicolumn{1}{|l|}{\%Z}            & \multicolumn{1}{l|}{CEST}              & 时区缩写(也可以是UTC、TAI或GPS)           \\ \hline
\multicolumn{1}{|l|}{\%z}            & \multicolumn{1}{l|}{+0200}             & 与UTC(+02:00时\%Ez或\%Oz)的偏移(时和分))    \\ \hline
\multicolumn{1}{|l|}{\%j}            & \multicolumn{1}{l|}{160}               & 三位数年份中的第几天(1月1日是001)             \\ \hline
\multicolumn{1}{|l|}{\%q}            & \multicolumn{1}{l|}{ms}                & 时间段的后缀单位                    \\ \hline
\multicolumn{1}{|l|}{\%Q}            & \multicolumn{1}{l|}{6196850}           & 时间段的具体值                          \\ \hline
\multicolumn{1}{|l|}{\%n}            & \multicolumn{1}{l|}{\textbackslash{}n} & 换行符                                               \\ \hline
\multicolumn{1}{|l|}{\%t}            & \multicolumn{1}{l|}{\textbackslash{}t} & 制表符                                             \\ \hline
\multicolumn{1}{|l|}{\%\%}           & \multicolumn{1}{l|}{\%}                & \% 字符                                                    \\ \hline
\end{longtable}

\begin{center}
表11.9 计时类型的转换说明符
\end{center}

\item
不同的周数和年份格式可能导致不同的输出值。

例如,2023年1月1日,星期天,的结果如下:

\begin{itemize}
\item
第00周, \%W (第一个星期一之前的第一周)

\item
第01周, \%U(第一个星期一的周)

\item
第52周, \%V(ISO周:第01周周一之前的一周,即1月4日)
\end{itemize}

ISO周可能是上一年的最后一周,所以ISO年就是那一周所在的年份,可能会少一个:

\begin{itemize}
\item
Year 2023 ,\%Y

\item
Year 23 , \%y

\item
Year 2022 , \%G (ISO年的ISO周\%V,即前一个月的最后一周)

\item
Year 22 , \%g (ISO年的ISO周\%V,即前一个月的最后一周)
\end{itemize}

\item
以下时区缩写用于标准时钟:

\begin{itemize}
\item
UTC有sys\_clock, utc\_clock和 file\_clock

\item
TAI 有tai\_clock

\item
GPS 有gps\_clock
\end{itemize}
\end{itemize}

除了\%q和\%Q之外,的所有转换说明符也可用于格式化解析。

\mySubsubsection{11.4.3}{依赖于本地环境的输出}

若输出流是由具有自己格式的区域设置注入的,则各种类型的默认输出操作符使用与区域设置相关的格式。例如:

\begin{cpp}
using namespace std::literals;
auto dur = 42.2ms;

std::cout << dur << '\n'; // 42.2ms

#ifdef _MSC_VER
	std::locale locG("deu_deu.1252");
#else
	std::locale locG("de_DE");
#endif
std::cout.imbue(locG); // switch to German locale
std::cout << dur << '\n'; // 42,2ms
\end{cpp}

std::format()格式化的输出像往常一样处理:[此行为指定为C++20的一个错误修复,使用\url{http://wg21.link/p2372},这意味着C++20的原始措辞没有指定此行为。]

\begin{itemize}
\item
默认情况下,格式化输出使用独立于语言环境的“C”语言环境。

\item
通过指定L,可以切换到通过locale参数,或作为全局locale指定的依赖于locale的输出
\end{itemize}

则要使用依赖于语言环境的表示法,必须使用L说明符,并将语言环境作为第一个参数传递给std::format(),或者在调用之前设置全局语言环境。例如:

\begin{cpp}
using namespace std::literals;
auto dur = 42.2ms; // duration to print

#ifdef _MSC_VER
	std::locale locG("deu_deu.1252");
#else
	std::locale locG("de_DE");
#endif

std::string s1 = std::format("{:%S}", dur); // "00.042s" (not localized)
std::string s3 = std::format(locG, "{:%S}", dur); // "00.042s" (not localized)
std::string s2 = std::format(locG, "{:L%S}", dur); // "00,042s" (localized)

std::locale::global(locG); // set German locale globally
std::string s4 = std::format("{:L%S}", dur); // "00,042s" (localized)
\end{cpp}

某些情况下,甚至可以根据strftime()和ISO 8601:2004使用另一种语言环境的表示,可以在转换说明符前面使用前导O或E来指定:

\begin{itemize}
\item
E可以用作区域设置在c,C,x,X,y,Y和z前面的替代表示

\item
O可以用作区域设置前面的替代数字符号d, e, H, I, m, M, S, u, U V, w, W, y和z
\end{itemize}

\mySubsubsection{11.4.4}{格式化输入}

chrono库还支持格式化输入。这里有两个选择:

\begin{itemize}
\item
特定日期/时间类型提供了一个独立的函数std::chrono::from\_stream(),用于根据传入的格式字符串读取特定的值。

\item
操纵符std::chrono::parse()允们使用from\_stream()作为带有>{}>输入操作符解析的一部分。
\end{itemize}

\mySamllsection{使用from\_stream()}

下面的代码演示了,如何使用from\_stream()解析完整的时间点:

\begin{cpp}
std::chrono::sys_seconds tp;
std::istringstream sstrm{"2021-2-28 17:30:00"};

std::chrono::from_stream(sstrm, "%F %T", tp);
if (sstrm) {
	std::cout << "tp: " << tp << '\n';
}
else {
	std::cerr << "reading into tp failed\n";
}
\end{cpp}

代码会有以下输出:

\begin{shell}
tp: 2021-02-28 17:30:00
\end{shell}

另一个例子,可以从指定完整月份名称和年份的字符序列中解析year\_month,以及其他内容:

\begin{cpp}
std::chrono::year_month m;
std::istringstream sstrm{"Monday, April 5, 2021"};
std::chrono::from_stream(sstrm, "%A, %B %d, %Y", m);
if (sstrm) {
	std::cout << "month: " << m << '\n'; // prints: month: 2021/Apr
}
\end{cpp}

格式字符串接受除\%q和\%Q以外的格式化输出的所有转换说明符,从而提高了灵活性。例如:

\begin{itemize}
\item
\%d表示一个或两个字符来指定日期,使用\%4d,可以指定最多解析四个字符。

\item
\%n只表示一个空白字符。

\item
\%t表示零或一个空白字符。

\item
像空格这样的空白字符,可以表示任意数量的空白(包括零空白)。
\end{itemize}

from\_stream()可用于下列类型:

\begin{itemize}
\item
duration<>类型

\item
sys\_time<>, utc\_time<>, gps\_time<>, tai\_time<>, local\_time<>或file\_time<>类型

\item
day, month或year类型

\item
year\_month, month\_day, 或year\_month\_day类型

\item
weekday类型
\end{itemize}

格式必须是const char*类型的C字符串,必须匹配输入流中的字符和要解析的值。若出现以下情况,解析失败:

\begin{itemize}
\item
输入的字符序列与所需的格式不匹配

\item
该格式没有为值提供足够的信息

\item
解析的日期无效
\end{itemize}

这些情况下,设置了流的failbit,可以通过调用fail()或使用流作为布尔值进行测试。

\mySamllsection{解析日期/时间的通用函数}

实际上,日期和时间很少硬编码。但在测试代码时,通常需要一种简单的方法来指定日期/时间值。

下面是我用来测试本书示例的一个小辅助函数:

\filename{lib/chronoparse.hpp}

\begin{cpp}
#include <chrono>
#include <string>
#include <sstream>
#include <cassert>

// parse year-month-day with optional hour:minute and optional :sec
// - returns a time_point<> of the passed clock (default: system_clock)
// in seconds
template<typename Clock = std::chrono::system_clock>
auto parseDateTime(const std::string& s)
{
	// return value:
	std::chrono::time_point<Clock, std::chrono::seconds> tp;

	// string stream to read from:
	std::istringstream sstrm{s}; // no string_view support

	auto posColon = s.find(":");
	if (posColon != std::string::npos) {
		if (posColon != s.rfind(":")) {
			// multiple colons:
			std::chrono::from_stream(sstrm, "%F %T", tp);
		}
		else {
			// one colon:
			std::chrono::from_stream(sstrm, "%F %R", tp);
		}
	}
	else {
		// no colon:
		std::chrono::from_stream(sstrm, "%F", tp);
	}

	// handle invalid formats:
	assert((!sstrm.fail()));

	return tp;
}
\end{cpp}

可以这样使用parseDateTime():

\filename{lib/chronoparse.cpp}

\begin{cpp}
#include "chronoparse.hpp"
#include <iostream>

int main()
{
	auto tp1 = parseDateTime("2021-1-1");
	std::cout << std::format("{:%F %T %Z}\n", tp1);

	auto tp2 = parseDateTime<std::chrono::local_t>("2021-1-1");
	std::cout << std::format("{:%F %T}\n", tp2);

	auto tp3 = parseDateTime<std::chrono::utc_clock>("2015-6-30 23:59:60");
	std::cout << std::format("{:%F %T %Z}\n", tp3);

	auto tp4 = parseDateTime<std::chrono::gps_clock>("2021-1-1 18:30");
	std::cout << std::format("{:%F %T %Z}\n", tp4);
}
\end{cpp}

程序有以下输出:

\begin{shell}
2021-01-01 00:00:00 UTC
2021-01-01 00:00:00
2015-06-30 23:59:60 UTC
2021-01-01 18:30:00 GPS
\end{shell}

对于本地时间点,不能使用\%Z来输出其时区(这样做会引发异常)。

\mySamllsection{使用parse()操纵符}

等价from\_stream()的方式

\begin{cpp}
std::chrono::from_stream(sstrm, "%F %T", tp);
\end{cpp}

也可以这样:

\begin{cpp}
sstrm >> std::chrono::parse("%F %T", tp);
\end{cpp}

最初的C++20标准没有直接作为字符串字面值传递格式,所以必须使用

\begin{cpp}
sstrm >> std::chrono::parse(std::string{"%F %T"}, tp);
\end{cpp}

然而,这应该通过\url{http://wg21.link/lwg3554}来解决。

std::chrono::parse()是一个I/O流操纵符,允许在从输入流读取的一条语句中解析多个值。使用移动语义,可以传递一个临时输入流。例如:
\begin{cpp}
chr::sys_days tp;
chr::hours h;
chr::minutes m;
// parse date into tp, hour into h and minute into m:
std::istringstream{"12/24/21 18:00"} >> chr::parse("%D", tp)
									 >> chr::parse(" %H", h)
									 >> chr::parse(":%M", m);
std::cout << tp << " at " << h << ' ' << m << '\n';
\end{cpp}

这段代码输出为:

\begin{shell}
2021-12-24 at 18h 0min
\end{shell}

可以显式地将字符串字面量“\%D”、“\%H”和“:\%M”转换为字符串。

\mySamllsection{解析时区}

解析时区有点棘手,因为时区缩写不是唯一的:为了完成功能,from\_stream()具有以下格式:

\begin{shell}
istream from_stream(istream, format, value)
istream from_stream(istream, format, value, abbrevPtr)
istream from_stream(istream, format, value, abbrevPtr, offsetPtr)
\end{shell}

可以选择传递std::string的地址,将解析后的时区缩写存储到字符串中,也可以传递std::chrono::minutes对象的地址以将解析后的时区偏移存储到该字符串中。在这两种情况下都可以传递nullptr。

然而,仍然要谨慎:

\begin{itemize}
\item
以下代码可以工作:

\begin{cpp}
chr::sys_seconds tp;
std::istringstream sstrm{"2021-4-13 12:00 UTC"};
chr::from_stream(sstrm, "%F %R %Z", tp);
std::cout << std::format("{:%F %R %Z}\n", tp); // 2021-04-13 12:00 UTC
\end{cpp}

然而,工作原因只是因为系统时间点使用的是UTC。

\item
因为忽略了时区,下面的命令不起作用:

\begin{cpp}
chr::sys_seconds tp;
std::istringstream sstrm{"2021-4-13 12:00 MST"};
chr::from_stream(sstrm, "%F %R %Z", tp);
std::cout << std::format("{:%F %R %Z}", tp); // 2021-04-13 12:00 UTC
\end{cpp}

\%Z用于解析MST,但没有参数来存储该值。

\item
下面的方法似乎行得通:

\begin{cpp}
chr::sys_seconds tp;
std::string tzAbbrev;
std::istringstream sstrm{"2021-4-13 12:00 MST"};
chr::from_stream(sstrm, "%F %R %Z", tp, &tzAbbrev);
std::cout << tp << '\n'; // 2021-04-13 12:00
std::cout << tzAbbrev << '\n'; // MST
\end{cpp}

但若计算时区时间,会看到正在将UTC时间转换为不同的时区:

\begin{cpp}
chr::zoned_time zt{tzAbbrev, tp}; // OK: MST exists
std::cout << zt << '\n'; // 2021-04-13 05:00:00 MST
\end{cpp}

\item
下面的方法似乎确实有效:

\begin{cpp}
chr::local_seconds tp; // local time
std::string tzAbbrev;
std::istringstream sstrm{"2021-4-13 12:00 MST"};
chr::from_stream(sstrm, "%F %R %Z", tp, &tzAbbrev);
std::cout << tp << '\n'; // 2021-04-13 12:00
std::cout << tzAbbrev << '\n'; // MST
chr::zoned_time zt{tzAbbrev, tp}; // OK: MST exists
std::cout << zt << '\n'; // 2021-04-13 12:00:00 MST
\end{cpp}

但MST是时区数据库中为数不多的不推荐使用的缩写之一,将此代码与CEST或CST一起使用时,会在初始化zoned\_time时抛出异常。

\item
要么只使用tzAbbrev,而非zoned\_time和\%Z:

\begin{cpp}
chr::local_seconds tp; // local time
std::string tzAbbrev;
std::istringstream sstrm{"2021-4-13 12:00 CST"};
chr::from_stream(sstrm, "%F %R %Z", tp, &tzAbbrev);
std::cout << std::format("{:%F %R} {}", tp, tzAbbrev); // 2021-04-13 12:00 CST
\end{cpp}

或必须处理将时区缩写映射到时区的代码。
\end{itemize}

注意,\%Z不能解析伪时区GPS和TAI。





