
C++20中,字符串类型的某些方面发生了变化。这些更改影响字符串(类型std::basic\_string<>及其实例化,如std::string),字符串视图(std::basic\_string\_view<>及其实例化,如std::string\_view),或两者都有。

实际上,C++20为字符串类型引入了以下改进:

\begin{itemize}
\item
所有字符串类型现在都支持太空船操作符<=>,则现在只声明==和<=>操作符,不再声明!=、<、<=、>和>=。

\item
所有字符串类型现在都提供了新的成员函数,ends\_with(),ends\_with()。

\item
对于字符串,成员函数reserve()不能再用于请求缩小字符串的容量(为值分配的内存)。由于这个原因,不能再向reserve()传递参数。

\item
对于UTF-8字符,C++现在提供了字符串类型std::u8string和std::u8string\_view。它们定义为std::basic\_string<>和std::basic\_string\_view<>,用于新的UTF-8字符类型char8\_t,返回UTF-8字符串的库函数现在具有返回类型std::u8string。当切换到C++20时,此更改可能会破坏现有的代码。

\item
字符串(std::string和std::basic\_string<>的其他实例)现在是constexpr,则可以在编译时使用字符串。

不能在编译时和运行时同时使用std::string,但有几种方法可以将编译时字符串导出到运行时。

\item
字符串视图现在标记为视图和租借范围。

\item
为类型std::u8string和std::u8string\_view以及std::pmr::string、std::pmr::u8string、std::pmr::u16string、std::pmr::u32string和std::pmr::wstring添加了标准哈希函数。
\end{itemize}

下面几节将解释其他章节中没有介绍和解释的重要改进。

\mySubsubsection{23.1.1}{字符串成员starts\_with()和ends\_with()}

字符串和字符串视图现在都有新的成员函数,starts\_with()和ends\_with()。它们提供了一种简单的方法,根据特定的字符序列检查字符串的开头和结尾字符。可以对单个字符、字符数组、字符串或字符串视图进行比较。

例如:

\begin{cpp}
void foo(const std::string& s, std::string_view suffix)
{
	if (s.starts_with('.')) {
		...
	}
	if (s.ends_with(".tmp")) {
		...
	}
	if (s.ends_with(suffix)) {
		...
	}
}
\end{cpp}

\mySubsubsection{23.1.2}{受限字符串成员reserve()}

对于字符串,成员函数reserve()不能再用于请求缩小字符串的容量(为值分配的内存):

\begin{cpp}
void modifyString(std::string& s)
{
	if ( ... ) {
		s.clear();
		s.reserve(0); // may no longer shrink memory (as before on some platforms)
		return;
	}
	...
}
\end{cpp}

这样做的原因是释放内存可能需要一些时间,则在移植代码时,此调用的性能可能会有很大差异。

reserve()需要传递参数:

\begin{cpp}
s.reserve(); // ERROR since C++20
\end{cpp}

可以使用shrink\_to\_fit()代替:

\begin{cpp}
s.shrink_to_fit(); // still OK
\end{cpp}







