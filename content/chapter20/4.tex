

本节列出了迭代器的特征,这些特征在关于迭代器的基本类型函数一节中列出。

\mySubsubsection{20.4.1}{iter\_difference\_t<>}

\mySamllsection{std::iter\_difference\_t<T>}

生成与可递增类型/迭代器类型T对应的差值类型。

该特性特别用于处理间接可读类型的两个对象的值类型,与传统的迭代器特征(std::iterator\_traits<>)相比,这个特征可以正确地处理新的迭代器类别。

与名为type的成员没有相应的数据结构std::iter\_difference。相反,类型特性是通过尝试使用新辅助类型std::incrementable\_traits<>的成员difference\_type来定义的,该成员difference\_type已经定义如下:

\begin{itemize}
\item
若特化,则使用std::incrementable\_traits<T>::difference\_type。

\item
对于原始指针,使用std::ptrdiff\_t。

\item
否则,若定义了,则使用T::difference\_type。

\item
否则,使用两个T之间的有符号整数差分类型。

\item
对于const T,使用T的不同类型。
\end{itemize}

例如:

\begin{cpp}
using T1 = std::iter_difference_t<int*>; // std::ptrdiff_t
using T2 = std::iter_difference_t<std::string>; // std::ptrdiff_t
using T3 = std::iter_difference_t<std::vector<long>>; // std::ptrdiff_t
using T4 = std::iter_difference_t<int>; // int
using T5 = std::iter_difference_t<std::chrono::sys_seconds>; // ERROR
\end{cpp}

\mySubsubsection{20.4.2}{iter\_value\_t<>}

\mySamllsection{std::iter\_value\_t<T>}

生成非const值/元素类型,对应指针/迭代器类型T。

该特性特别用于处理间接可读类型的值类型,与传统的迭代器特性(std::iterator\_traits<>)相比,这个特性可以正确处理新迭代器的类别。

名为type的成员中没有相应的数据结构std::iter\_value。相反,类型特性是通过尝试使用新的辅助类型std::indirectly\_readable\_traits<>的成员value\_type来定义的,该成员值已经定义如下:

\begin{itemize}
\item
若特化,则使用std::indirectly\_readable\_traits<T>::value\_type。

\item
对于原始指针,使用引用的非const/volatile类型。

\item
否则,若定义了,则使用remove\_cv\_t<T::value\_type>。

\item
否则,若定义了,则使用remove\_cv\_t<T::element\_type>。

\item
对于const T,使用T的值类型。
\end{itemize}

例如:

\begin{cpp}
using T1 = std::iter_value_t<int*>; // int
using T2 = std::iter_value_t<const int* const>; // int
using T3 = std::iter_value_t<std::string>; // char
using T4 = std::iter_value_t<std::vector<long>>; // long
using T5 = std::iter_value_t<int>; // ERROR
\end{cpp}

\mySubsubsection{20.4.3}{iter\_reference\_t<> and iter\_rvalue\_reference\_t<>}

\mySamllsection{std::iter\_reference\_t<T>}

生成对应于可解引用指针/迭代器类型T的左值引用类型。等价于:

\begin{cpp}
decltype(*declval<T&>())
\end{cpp}

\mySamllsection{std::iter\_rvalue\_reference\_t<T>}

产生对应于可解引用指针/迭代器类型T的右值类型

\begin{cpp}
decltype(std::ranges::iter_move(declval<T&>()))
\end{cpp}

这些特性是专门用来处理间接可写类型的值类型的,与传统的迭代器特征(std::iterator\_traits<>)不同,这些特征可以正确处理新的迭代器类别。

没有带有类型成员的相应数据结构std::iter\_value。相反,这些特征是按照上面描述的那样直接定义。

例如:

\begin{cpp}
using T1 = std::iter_reference_t<int*>; // int&
using T2 = std::iter_reference_t<const int* const>; // const int&
using T3 = std::iter_reference_t<std::string>; // ERROR
using T4 = std::iter_reference_t<std::vector<long>>; // ERROR
using T5 = std::iter_reference_t<int>; // ERROR

using T6 = std::iter_rvalue_reference_t<int*>; // int&&
using T7 = std::iter_rvalue_reference_t<const int* const>; // const int&&
using T8 = std::iter_rvalue_reference_t<std::string>; // ERROR
\end{cpp}












