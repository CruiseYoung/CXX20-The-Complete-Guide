
参数使用auto有什么好处呢?

\mySubsubsection{2.2.1}{进行延迟类型检查}

使用auto参数,实现具有循环依赖关系的代码会容易得多。

考虑两个使用其他类的对象的类。要使用另一个类的对象,需要定义它们的类型,前置声明是不够的(除非只声明引用或指针):

\begin{cpp}
class C2; // forward declaration

class C1 {
	public:
	void foo(const C2& c2) const { // OK
		c2.print(); // ERROR: C2 is incomplete type
	}

	void print() const;
};

class C2 {
	public:
	void foo(const C1& c1) const {
		c1.print(); // OK
	}
	void print() const;
};
\end{cpp}

虽然可以在类定义中实现C2::foo(),但不能实现C1::foo(),因为要检查C2.print()的调用是否有效,编译器需要知道类C2的定义。

必须在声明两个类结构之后实现C2::foo():

\begin{cpp}
class C2;

class C1 {
public:
	void foo(const C2& c2) const; // forward declaration
	void print() const;
};

class C2 {
public:
	void foo(const C1& c1) const {
		c1.print(); // OK
	}
	void print() const;
};

inline void C1::foo(const C2& c2) const { // implementation (inline if in header)
	c2.print(); // OK
}
\end{cpp}

泛型函数在调用时检查泛型形参的成员,所以可以使用auto实现:

\begin{cpp}
class C1 {
	public:
	void foo(const auto& c2) const {
		c2.print(); // OK
	}

	void print() const;
};

class C2 {
	public:
	void foo(const auto& c1) const {
		c1.print(); // OK
	}
	void print() const;
};
\end{cpp}

将C1::foo()声明为成员函数模板时也会有同样的效果,但使用auto会更容易一些。

注意,auto允许调用者传递任何类型的参数,只要该类型提供成员函数print()。若不想这样,可以使用标准概念std::same\_as来约束这个成员函数只对C2类型的形参使用:

\begin{cpp}
#include <concepts>

class C2;

class C1 {
	public:
	void foo(const std::same_as<C2> auto& c2) const {
		c2.print(); // OK
	}
	void print() const;
};
...
\end{cpp}

对于概念来说,不完整的类型可以工作得很好。

\mySubsubsection{2.2.2}{auto函数与Lambda表达式}

带有auto参数的函数不同于Lambda表达式。例如,不能在不指定泛型参数的情况下,传递带有auto作为参数的函数:

\begin{cpp}
bool lessByNameFunc(const auto& c1, const auto& c2) { // sorting criterion
	return c1.getName() < c2.getName(); // - compare by name
}
...
std::sort(persons.begin(), persons.end(),
		  lessByNameFunc); // ERROR: can’t deduce type of parameters in sorting criterion
\end{cpp}

记住,lessByName()的声明相当于:

\begin{cpp}
template<typename T1, typename T2>
bool lessByNameFunc(const T1& c1, const T2& c2) { // sorting criterion
	return c1.getName() < c2.getName(); // - compare by name
}
\end{cpp}

因为没有直接调用函数模板,编译器不能推断模板参数来编译调用。所以在传递函数模板作为形参时,必须显式指定模板的形参:

\begin{cpp}
std::sort(persons.begin(), persons.end(),
		  lessByName<Customer, Customer>); // OK
\end{cpp}

使用Lambda表达式,可以按原样传递表达式:

\begin{cpp}
auto lessByNameLambda = [] (const auto& c1, const auto& c2) { // sorting criterion
							return c1.getName() < c2.getName(); // compare by name
						};
...
std::sort(persons.begin(), persons.end(),
		  lessByNameLambda); // OK
\end{cpp}

Lambda是一个没有泛型类型的对象,所以只有将对象作为函数使用才可通用。

另外,显式指定(缩写)函数模板参数更容易:

\begin{itemize}
\item
只需在函数名后传递指定的类型:

\begin{cpp}
void printFunc(const auto& arg) {
	...
}

printFunc<std::string>("hello"); // call function template compiled for std::string
\end{cpp}

\item
对于泛型Lambda表达式,必须完成以下操作:

\begin{cpp}
auto printFunc = [] (const auto& arg) {
	...
};

printFunc.operator()<std::string>("hello"); // call lambda compiled for std::string
\end{cpp}

对于泛型Lambda表达式,运算符operator()为模板。所以必须将请求的类型作为参数传递给operator(),以显式指定模板形参。
\end{itemize}






