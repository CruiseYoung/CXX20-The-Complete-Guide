
本节详细介绍span的类型和操作。

\mySubsubsection{9.4.1}{span的操作和成员类型概述}

表“span的操作”列出了为span提供的所有操作。

需要注意的是不支持的操作:

\begin{itemize}
\item
比较(不包括==)

\item
swap()

\item
assign()

\item
swap()

\item
at()

\item
I/O操作符

\item
cbegin(), cend(), crbegin(), crend()

\item
哈希

\item
用于结构化绑定的类元组API
\end{itemize}

所以,span既不是容器(传统的C++ STL意义上),也不是常规类型。

关于静态成员和成员类型,span提供了容器的常用成员(const\_iterator除外)以及两个特殊成员:element\_type和extent(参见表span的static和type成员)。

注意,std::value\_type不是指定的元素类型(std::array和其他几个类型的value\_type通常是),它是去掉了const和volatile的元素类型。

% Please add the following required packages to your document preamble:
% \usepackage{longtable}
% Note: It may be necessary to compile the document several times to get a multi-page table to line up properly
\begin{longtable}[c]{|l|l|}
\hline
\textbf{操作} & \textbf{效果}                                                      \\ \hline
\endfirsthead
%
\endhead
%
构造函数       & 创建或复制一个span                                             \\ \hline
析构函数         & 销毁一个span                                                      \\ \hline
=                  & 分配一个新的值序列                                    \\ \hline
empty()            & 返回span是否为空                                    \\ \hline
size()             & 返回元素的个数                                       \\ \hline
size\_bytes()      & 返回所有元素所使用的内存大小                     \\ \hline
{[}{]}             & 访问元素                                                  \\ \hline
front(),back()     & 访问第一个或最后一个元素                                   \\ \hline
begin(), end()     & 提供迭代器支持(不支持const\_iterator)               \\ \hline
rbegin(), rend()   & 提供常量反向迭代器支持                           \\ \hline
first(n)           & 返回具有前n个元素动态范围的子span       \\ \hline
first\textless{}n\textgreater{}()         & 返回前n个元素的固定范围的子span            \\ \hline
last(n)            & 返回最后n个元素的动态范围的子span        \\ \hline
last\textless{}n\textgreater{}()          & 返回最后n个元素的固定范围的子span             \\ \hline
subspan(offs)                             & 返回一个跳过第一个off元素的动态范围的子span \\ \hline
subspan(offs, n)   & 在off元素之后返回一个动态范围为n的子span      \\ \hline
subspan\textless{}offs\textgreater{}()    & 跳过第一个off元素,返回具有相同范围的子span    \\ \hline
subspan\textless{}offs, n\textgreater{}() & 在off元素之后返回一个固定范围为n的子span           \\ \hline
data()             & 返回指向元素的原始指针                                \\ \hline
as\_bytes()        & 以只读std::bytes的span返回元素的内存 \\ \hline
as\_writeable\_bytes()                    & 以可写std::bytes的span返回元素的内存     \\ \hline
\end{longtable}

\begin{center}
表9.1 span的操作
\end{center}

% Please add the following required packages to your document preamble:
% \usepackage{longtable}
% Note: It may be necessary to compile the document several times to get a multi-page table to line up properly
\begin{longtable}[c]{|l|l|}
\hline
\textbf{成员} & \textbf{效果}                      \\ \hline
\endfirsthead
%
\endhead
%
extent            & 若大小不同,则表示元素数或std::dynamic\_extent              \\ \hline
size\_type      & 范围类型(总是std::size\_t) \\ \hline
difference\_type  & 指向元素指针的不同类型(总是std::difference\_type) \\ \hline
element\_type   & 指定的元素类型       \\ \hline
pointer         & 指向元素的指针类型    \\ \hline
const\_pointer    & 用于对元素进行只读访问的指针类型                 \\ \hline
reference       & 元素引用的类型  \\ \hline
const\_reference  & 对元素进行只读访问的引用类型               \\ \hline
iterator        & 元素迭代器的类型  \\ \hline
reverse\_iterator & 元素的反向迭代器类型                             \\ \hline
value\_type       & 不带const或volatile的元素类型                            \\ \hline
\end{longtable}

\begin{center}
表9.2 span的静态和类型成员
\end{center}

\mySubsubsection{9.4.2}{构造函数}

只有当它具有动态区段或范围为0时,才提供默认构造函数:

\begin{cpp}
std::span<int> sp0a; // OK
std::span<int, 0> sp0b; // OK
std::span<int, 5> sp0c; // compile-time ERROR
\end{cpp}

若此初始化有效,则size()为0,data()为nullptr。

原则上,可以初始化数组的span,以哨兵(end迭代器)开始的span和以size开始的span。若beg指向连续内存中的元素,视图std::views::counted(beg, sz)会使用后者,还支持类模板参数推导。

当使用原始数组或std::array<>初始化span时,将推导出具有固定范围的span(除非指定了元素类型):

\begin{cpp}
int a[10] {};
std::array arr{1, 2, 3, 4, 5, 6, 7, 8, 9, 10, 11, 12, 13, 14, 15};

std::span sp1a{a}; // std::span<int, 10>
std::span sp1b{arr}; // std::span<int, 15>
std::span<int> sp1c{arr}; // std::span<int>
std::span sp1d{arr.begin() + 5, 5}; // std::span<int>
auto sp1e = std::views::counted(arr.data() + 5, 5); // std::span<int>
\end{cpp}

使用std::vector<>初始化span时,除非显式指定了span的大小,否则将推导出具有动态区段的span:

\begin{cpp}
std::vector vec{1, 2, 3, 4, 5};
std::span sp2a{vec}; // std::span<int>

std::span<int> sp2b{vec}; // std::span<int>
std::span<int, 2> sp2c{vec}; // std::span<int, 2>
std::span<int, std::dynamic_extent> sp2d{vec}; // std::span<int>
std::span<int, 2> sp2e{vec.data() + 2, 2}; // std::span<int, 2>
std::span sp2f{vec.begin() + 2, 2}; // std::span<int>
auto sp2g = std::views::counted(vec.data() + 2, 2); // std::span<int>
\end{cpp}

若是使用右值(临时对象)初始化span,则元素必须为const:

\begin{cpp}
std::span sp3a{getArrayOfInt()}; // ERROR: rvalue and not const
std::span<int> sp3b{getArrayOfInt()}; // ERROR: rvalue and not const
std::span<const int> sp3c{getArrayOfInt()}; // OK
std::span sp3d{getArrayOfConstInt()}; // OK

std::span sp3e{getVectorOfInt()}; // ERROR: rvalue and not const
std::span<int> sp3f{getVectorOfInt()}; // ERROR: rvalue and not const
std::span<const int> sp3g{getVectorOfInt()}; // OK
\end{cpp}

使用返回的临时集合初始化span可能会导致致命的运行时错误。例如,永远不该使用基于范围的for循环来迭代(直接初始化的)span:

\begin{cpp}
for (auto elem : std::span{getCollOfConst()}) ... // fatal runtime error

for (auto elem : std::span{getCollOfConst()}.last(2)) ... // fatal runtime error

for (auto elem : std::span<const Type>{getColl()}) ... // fatal runtime error
\end{cpp}

问题在于,在迭代为临时对象返回的引用时,基于范围的for循环会导致未定义的行为,因为临时对象在循环开始内部迭代之前销毁。这是一个C++标准委员会多年来一直不愿意修复的bug(参见\url{http://wg21.link/p2012})。

作为一种解决方案,可以使用初始化的新的基于范围的for循环:

\begin{cpp}
for (auto&& coll = getCollOfConst(); auto elem : std::span{coll}) ... // OK
\end{cpp}

无论用一个迭代器和一个长度初始化span,还是用两个定义有效范围的迭代器初始化span,迭代器都必须指向连续内存中的元素(满足std::consecuous\_iterator的概念):

\begin{cpp}
std::vector<int> vec{1, 2, 3, 4, 5, 6, 7, 8, 9, 10};

std::span<int> sp6a{vec}; // OK, refers to all elements
std::span<int> sp6b{vec.data(), vec.size()}; // OK, refers to all elements
std::span<int> sp6c{vec.begin(), vec.end()}; // OK, refers to all elements
std::span<int> sp6d{vec.data(), 5}; // OK, refers to first 5 elements
std::span<int> sp6e{vec.begin()+2, 5}; // OK, refers to elements 3 to 7 (including)
std::list<int> lst{ ... };
std::span<int> sp6f{lst.begin(), lst.end()}; // compile-time ERROR
\end{cpp}

若span具有固定的区段,则必须匹配所传递范围中的元素数量。所以,编译器不能在编译时检查:

\begin{cpp}
std::vector<int> vec{1, 2, 3, 4, 5, 6, 7, 8, 9, 10};

std::span<int, 10> sp7a{vec}; // OK, refers to all elements
std::span<int, 5> sp7b{vec}; // runtime ERROR (undefined behavior)
std::span<int, 20> sp7c{vec}; // runtime ERROR (undefined behavior)
std::span<int, 5> sp7d{vec, 5}; // compile-time ERROR
std::span<int, 5> sp7e{vec.begin(), 5}; // OK, refers to first 5 elements
std::span<int, 3> sp7f{vec.begin(), 5}; // runtime ERROR (undefined behavior)
std::span<int, 8> sp7g{vec.begin(), 5}; // runtime ERROR (undefined behavior)
std::span<int, 5> sp7h{vec.begin()}; // compile-time ERROR
\end{cpp}

也可以使用原始数组或std::array直接创建和初始化span,由于元素数量不合法而导致的一些运行时错误将成为编译时错误:

\begin{cpp}
int raw[10];
std::array arr{1, 2, 3, 4, 5, 6, 7, 8, 9, 10};

std::span<int> sp8a{raw}; // OK, refers to all elements
std::span<int> sp8b{arr}; // OK, refers to all elements
std::span<int, 5> sp8c{raw}; // compile-time ERROR
std::span<int, 5> sp8d{arr}; // compile-time ERROR
std::span<int, 5> sp8e{arr.data(), 5}; // OK
\end{cpp}

也就是说:要么传递一个整体上包含顺序元素的容器,要么传递两个参数来指定元素的初始范围。通常情况下,元素的数量必须匹配指定的固定范围。

\mySamllsection{带隐式转换的构造}

span必须具有它们引用的序列元素的元素类型。不支持转换(甚至是隐式的标准转换),但允许使用const等其他限定符,这也适用于复制构造函数:

\begin{cpp}
std::vector<int> vec{1, 2, 3, 4, 5, 6, 7, 8, 9, 10};

std::span<const int> sp9a{vec}; // OK: element type with const
std::span<long> sp9b{vec}; // compile-time ERROR: invalid element type
std::span<int> sp9c{sp9a}; // compile-time ERROR: removing constness
std::span<const long> sp9d{sp9a}; // compile-time ERROR: different element type
\end{cpp}

为了允许容器引用用户定义容器的元素,这些容器必须发出信号,表明其迭代器要求所有元素都位于连续内存中,其必须满足continuous\_iterator的概念。

构造函数还允许在span之间进行以下类型转换:

\begin{itemize}
\item
具有固定区段的span将转换为具有相同的固定区段和附加限定符的span。

\item
具有固定区段的span转换为具有动态区段的span。

\item
若当前区段适合,则具有动态区段的span将转换为具有固定区段的span。
\end{itemize}

使用条件显式时,只有固定范围span的构造函数是显式的。若必须转换初始值,则无法进行复制初始化(使用a =):

\begin{cpp}
std::vector<int> vec{1, 2, 3, 4, 5, 6, 7, 8, 9, 10};

std::span<int> spanDyn{vec.begin(), 5}; // OK
std::span<int> spanDyn2 = {vec.begin(), 5}; // OK
std::span<int, 5> spanFix{vec.begin(), 5}; // OK
std::span<int, 5> spanFix2 = {vec.begin(), 5}; // ERROR
\end{cpp}

因此,只有在转换为具有动态或相同固定范围的span时,才隐式支持转换:

\begin{cpp}
void fooDyn(std::span<int>);
void fooFix(std::span<int, 5>);

fooDyn({vec.begin(), 5}); // OK
fooDyn(spanDyn); // OK
fooDyn(spanFix); // OK
fooFix({vec.begin(), 5}); // ERROR
fooFix(spanDyn); // ERROR
fooFix(spanFix); // OK

spanDyn = spanDyn; // OK
spanDyn = spanFix; // OK
spanFix = spanFix; // OK
spanFix = spanDyn; // ERROR
\end{cpp}

\mySubsubsection{9.4.3}{子span的操作}

创建子span的成员函数可以创建动态或固定区段的span,将大小作为调用参数传递通常会产生具有动态区段的span。将大小作为模板参数传递通常会产生具有固定范围的span,成员函数first()和last()至少总是这样:

\begin{cpp}
std::vector vec{1.1, 2.2, 3.3, 4.4, 5.5};
std::span spDyn{vec};

auto sp1 = spDyn.first(2); // first 2 elems with dynamic extent
auto sp2 = spDyn.last(2); // last 2 elems with dynamic extent
auto sp3 = spDyn.first<2>(); // first 2 elems with fixed extent
auto sp4 = spDyn.last<2>(); // last 2 elems with fixed extent

std::array arr{1.1, 2.2, 3.3, 4.4, 5.5};
std::span spFix{arr};

auto sp5 = spFix.first(2); // first 2 elems with dynamic extent
auto sp6 = spFix.last(2); // last 2 elems with dynamic extent
auto sp7 = spFix.first<2>(); // first 2 elems with fixed extent
auto sp8 = spFix.last<2>(); // last 2 elems with fixed extent
\end{cpp}

但对于subspan(),结果有时可能令人惊讶。传递调用参数总是会产生具有动态区段的span:

\begin{cpp}
std::vector vec{1.1, 2.2, 3.3, 4.4, 5.5};
std::span spDyn{vec};

auto s1 = spDyn.subspan(2); // 3rd to last elem with dynamic extent
auto s2 = spDyn.subspan(2, 2); // 3rd to 4th elem with dynamic extent
auto s3 = spDyn.subspan(2, std::dynamic_extent); // 3rd to last with dynamic extent

std::array arr{1.1, 2.2, 3.3, 4.4, 5.5};
std::span spFix{arr};

auto s4 = spFix.subspan(2); // 3rd to last elem with dynamic extent
auto s5 = spFix.subspan(2, 2); // 3rd to 4th elem with dynamic extent
auto s6 = spFix.subspan(2, std::dynamic_extent); // 3rd to last with dynamic extent
\end{cpp}

然而,当传递模板参数时,结果可能与期望不同:

\begin{cpp}
std::vector vec{1.1, 2.2, 3.3, 4.4, 5.5};
std::span spDyn{vec};

auto s1 = spDyn.subspan<2>(); // 3rd to last with dynamic extent
auto s2 = spDyn.subspan<2, 2>(); // 3rd to 4th with fixed extent
auto s3 = spDyn.subspan<2, std::dynamic_extent>(); // 3rd to last with dynamic extent

std::array arr{1.1, 2.2, 3.3, 4.4, 5.5};
std::span spFix{arr};

auto s4 = spFix.subspan<2>(); // 3rd to last with fixed extent
auto s5 = spFix.subspan<2, 2>(); // 3rd to 4th with fixed extent
auto s6 = spFix.subspan<2, std::dynamic_extent>(); // 3rd to last with fixed extent
\end{cpp}


